%\renewcommand\arraystretch{1.9}% (MyValue=1.0 is for standard)

%\begin{figure}[h!]
\hspace{-3ex}
\begin{center}
\footnotesize
%\begin{tabular}{
\begin{landscape}
\begin{longtable}{
|>{\columncolor[gray]{.7}}m{0.1\columnwidth}
|>{\columncolor[gray]{.9}}m{0.1\columnwidth}
|>{\columncolor[gray]{.8}}m{0.1\columnwidth}
|>{\columncolor[gray]{.9}}m{0.1\columnwidth}
|>{\columncolor[gray]{.9}}m{0.1\columnwidth}
|>{\columncolor[gray]{.8}}m{0.1\columnwidth}
|>{\columncolor[gray]{.9}}m{0.1\columnwidth}
|}
\caption[Summary table for unstructured algorithms]{Summary table for unstructured algorithms.} \label{unstructured:table} \\
\hline
%%%%%%%%%%%%%%%%%%%%%%%%%%%%%%%%%%%%%%%%%%%%%%%%%%%%%%%%%%%%%%%%%%%%%%%%%%%%%%%%
% first head
\rowcolor[gray]{.5}
\textbf{Algorithm/Paper} &
\textbf{Topology Adaptation} &
\textbf{Forwarding Optimisation} &
\textbf{Caching/Replication} &
\textbf{Landmarking} &
\textbf{Highlights} &
\textbf{Pros/Cons}\\
\hline
\endfirsthead
%%%%%%%%%%%%%%%%%%%%%%%%%%%%%%%%%%%%%%%%%%%%%%%%%%%%%%%%%%%%%%%%%%%%%%%%%%%%%%%%
% subsequent heads
\multicolumn{7}{c}%
{\tablename\ \thetable\ -- \textit{Continued from previous page}} \\
\hline
\rowcolor[gray]{.5}
\textbf{Algorithm/Paper} &
\textbf{Topology Adaptation} &
\textbf{Forwarding Optimisation} &
\textbf{Caching/Replication} &
\textbf{Landmarking} &
\textbf{Highlights} &
\textbf{Pros/Cons}\\
\hline
\endhead
%%%%%%%%%%%%%%%%%%%%%%%%%%%%%%%%%%%%%%%%%%%%%%%%%%%%%%%%%%%%%%%%%%%%%%%%%%%%%%%%
% foot
\hline \multicolumn{7}{r}{\textit{Continued on next page}} \\
\endfoot
%%%%%%%%%%%%%%%%%%%%%%%%%%%%%%%%%%%%%%%%%%%%%%%%%%%%%%%%%%%%%%%%%%%%%%%%%%%%%%%%
% last foot
\hline
\endlastfoot
%%%%%%%%%%%%%%%%%%%%%%%%%%%%%%%%%%%%%%%%%%%%%%%%%%%%%%%%%%%%%%%%%%%%%%%%%%%%%%%%
% data
\textbf{Improving search in peer-to-peer networks} &
{\large \Square} &
{\large \CheckedBox} &
{\large \CheckedBox} &
{\large \Square} &
\begin{tabular}[l]{@{}l@{}l@{}}
Iterativie Depening (ID)\\
Directed BFS (DBFS)\\
Local Indices (LI)
\end{tabular} &
\begin{tabular}[l]{@{}l@{}l@{}l@{}l@{}l@{}}
+ ID reduces the messages especially in upper levels of the tree.\\
+ DBFS ??????\\
+ LI reduces aggregate bandwidth usage and improves query efficiency.\\
-- ID needs evaluation time between iterations.\\
-- DBFS uses heuristics so it depends on their efficient choise.\\
-- LI add index update overhead which might be heavy and might not work at all in systems with high churn.
\end{tabular}
\\
\hline
%%%%%%%%%%%%%%%%%%%%%%%%%%%%%%%%%%%%%%%%%%%%%%%%%%%%%%%%%%%%%%%%%%%%%%%%%%%%%%%%
\textbf{Gia} &
{\large \CheckedBox} &
{\large \CheckedBox} &
{\large \CheckedBox} &
{\large \Square} &
Random Walks (RW) &
\begin{tabular}[l]{@{}l@{}l@{}}
+ RWs issue one copy of the query thus not flooding the whole network.\\
-- RWs can reduce search scope.\\
-- Gia does not actually consider the underlying network.
\end{tabular}
\\
\hline
%%%%%%%%%%%%%%%%%%%%%%%%%%%%%%%%%%%%%%%%%%%%%%%%%%%%%%%%%%%%%%%%%%%%%%%%%%%%%%%%
\textbf{DCMP} &
{\large \CheckedBox} &
{\large \Square} &
{\large \Square} &
{\large \Square} &
Cycle detection &
\begin{tabular}[l]{@{}l@{}l@{}}
+ Drastically reduces duplicate messages.\\
-- Cannot detect cycles in distance bigger than the TTL value of the IC message.\\
-- DCMP does not consider the underline network.
\end{tabular}
\\
\hline
%%%%%%%%%%%%%%%%%%%%%%%%%%%%%%%%%%%%%%%%%%%%%%%%%%%%%%%%%%%%%%%%%%%%%%%%%%%%%%%%
\textbf{Replication Strategies in Unstructured Peer-to-Peer Networks} &
{\large \Square} &
{\large \Square} &
{\large \CheckedBox} &
{\large \Square} &
\begin{tabular}[l]{@{}l@{}l@{}}
Uniform replication\\
Proportional replication\\
Square root replication allocation
\end{tabular} &
\begin{tabular}[l]{@{}l@{}l@{}}
+ Uniform replication reduces time spend on unsuccessfull searches.\\
+ Reduces search time for frequent queries.\\
-- Proportional replication struggles in locating rare objects.
\end{tabular}
\\
\hline
%%%%%%%%%%%%%%%%%%%%%%%%%%%%%%%%%%%%%%%%%%%%%%%%%%%%%%%%%%%%%%%%%%%%%%%%%%%%%%%%
\textbf{Tracing a large-scale Peer to Peer System: an hour in the life of Gnutella} &
? &
? &
? &
? &
? &
?
\\
\hline
%%%%%%%%%%%%%%%%%%%%%%%%%%%%%%%%%%%%%%%%%%%%%%%%%%%%%%%%%%%%%%%%%%%%%%%%%%%%%%%%
\textbf{Narada} &
{\large \Square} &
{\large \CheckedBox} &
{\large \Square} &
{\large \Square} &
\begin{tabular}[l]{@{}l@{}}
Mess creation\\
Minimum spanning trees
\end{tabular} &
\begin{tabular}[l]{@{}l@{}}
+ Mess and trees are kept up to date in high churn environments.\\
-- Works well only for small groups of peers.
\end{tabular}
\\
\hline
%%%%%%%%%%%%%%%%%%%%%%%%%%%%%%%%%%%%%%%%%%%%%%%%%%%%%%%%%%%%%%%%%%%%%%%%%%%%%%%%
\textbf{AOTO} &
{\large \CheckedBox} &
{\large \CheckedBox} &
{\large \Square} &
{\large \Square} &
\begin{tabular}[l]{@{}l@{}}
Minimum spanning trees\\
Peer proximity heuristic for removing costly links
\end{tabular} &
\begin{tabular}[l]{@{}l@{}l@{}l@{}l@{}}
+ Spanning trees only to immediate neighbours so no flooding and at the same time no shrinked search scope.\\
+ Selective flooding effectiveness is detached from physical or overlay topologies.\\
+ The more logical neighbours the more effective selective flooding becomes\\
-- High recalculation costs.\\
-- No sophisticated selection policy for candidate non-flooding peers.
\end{tabular}
\\
\hline
%%%%%%%%%%%%%%%%%%%%%%%%%%%%%%%%%%%%%%%%%%%%%%%%%%%%%%%%%%%%%%%%%%%%%%%%%%%%%%%%
\textbf{LTM} &
{\large \CheckedBox} &
{\large \Square} &
{\large \Square} &
{\large \Square} &
\begin{tabular}[l]{@{}l@{}}
TTL detector (2-hop distance)\\
Delayed low productive connection cutting
\end{tabular} &
\begin{tabular}[l]{@{}l@{}l@{}l@{}}
+ Compared to AOTO, ACE and SBO achieves faster convergence speed.\\
-- Creates more overhead than AOTO, ACE and SBO.\\
-- Needs synchronisation of peer clocks.\\
-- Does not consider shortcuts created by powerful peers when choosing to disable connections (only uses delay metric).
\end{tabular}
\\
\hline
%%%%%%%%%%%%%%%%%%%%%%%%%%%%%%%%%%%%%%%%%%%%%%%%%%%%%%%%%%%%%%%%%%%%%%%%%%%%%%%%
\textbf{SBO} &
{\large \CheckedBox} &
{\large \CheckedBox} &
{\large \Square} &
{\large \Square} &
Red/white bipartite overlay &
\begin{tabular}[l]{@{}l@{}l@{}}
+ Efficient in both static and dynamic environments.\\
+ Compared to AOTO incurs half the overhead.\\
-- Needs almost double the steps of LTM in order to converge (static or dynamic environments).
\end{tabular}
\\
\hline
%%%%%%%%%%%%%%%%%%%%%%%%%%%%%%%%%%%%%%%%%%%%%%%%%%%%%%%%%%%%%%%%%%%%%%%%%%%%%%%%
\textbf{THANCS} &
{\large \CheckedBox} &
{\large \CheckedBox} &
{\large \Square} &
{\large \Square} &
\begin{tabular}[l]{@{}l@{}}
Local optimum heuristic\\
Piggybacking neighbour distance in queries
\end{tabular} &
\begin{tabular}[l]{@{}l@{}l@{}l@{}l@{}}
+ Completely distributed approach.\\
+ Presents trivial overhead compared to the query cost savings.\\
+ Convergent speed faster among AOTO, LTM, SBO.\\
+ Does not shrink the search scope.\\
-- Limited by design to not being easily extend to also support non-flooding-based systems.
\end{tabular}
\\
\hline
%%%%%%%%%%%%%%%%%%%%%%%%%%%%%%%%%%%%%%%%%%%%%%%%%%%%%%%%%%%%%%%%%%%%%%%%%%%%%%%%
\textbf{HAND} &
{\large \CheckedBox} &
{\large \Square} &
{\large \Square} &
{\large \Square} &
Triple-hop adjustment &
\begin{tabular}[l]{@{}l@{}l@{}l@{}l@{}l@{}}
+ No need for clock synch.\\
+ Fully distributed.\\
+ Low overhead for the triple hop adjustment.\\
+ Applicable to both static and dynamic environments.\\
+ Low query response time.\\
-- Compared to LTM has lower traffic reduction and query response rates.
\end{tabular}
\\
\hline
%%%%%%%%%%%%%%%%%%%%%%%%%%%%%%%%%%%%%%%%%%%%%%%%%%%%%%%%%%%%%%%%%%%%%%%%%%%%%%%%
\textbf{APS} &
{\large \CheckedBox} &
{\large \Square} &
{\large \Square} &
{\large \Square} &
machine learning adaptive mechanism &
\begin{tabular}[l]{@{}l@{}}
+ Fully dynamic switching decision policy.\\
- Low convergence due to the learning process.
\end{tabular}
\\
\hline
%%%%%%%%%%%%%%%%%%%%%%%%%%%%%%%%%%%%%%%%%%%%%%%%%%%%%%%%%%%%%%%%%%%%%%%%%%%%%%%%
\textbf{ITA} &
{\large \CheckedBox} &
{\large \CheckedBox} &
{\large \Square} &
{\large \Square} &
\begin{tabular}[l]{@{}l@{}}
Short/long connections\\
Local flooding
\end{tabular} &
\begin{tabular}[l]{@{}l@{}l@{}l@{}}
+ Low clustering.\\
+ Large peer coverage.\\
+ Reduced duplication.\\
+ Low or no impact to other mechanisms of unstructured p2p networks (e.g. 1-hop replication, dynamic quering).
\end{tabular}
\\
\hline
%%%%%%%%%%%%%%%%%%%%%%%%%%%%%%%%%%%%%%%%%%%%%%%%%%%%%%%%%%%%%%%%%%%%%%%%%%%%%%%%
\textbf{EGOIST} &
{\large \CheckedBox} &
{\large \Square} &
{\large \Square} &
{\large \Square} &
Selfish shortest path routing &
-- constructs a global view of the network
\\
\hline
%%%%%%%%%%%%%%%%%%%%%%%%%%%%%%%%%%%%%%%%%%%%%%%%%%%%%%%%%%%%%%%%%%%%%%%%%%%%%%%%
\textbf{BNS} &
{\large \CheckedBox} &
{\large \Square} &
{\large \CheckedBox} &
{\large \Square} &
\begin{tabular}[l]{@{}l@{}l@{}}
ISP clustering (tracker-side or ISP-side detection)\\
Bandwidth throttling\\
Caching
\end{tabular} &
\begin{tabular}[l]{@{}l@{}l@{}l@{}}
+ Localises traffic within an ISP.\\
+ Preserves the efficiency of bittorrent protocol.\\
-- Needs ISPs to either provide information or infrastructure changes.\\
-- Locality-based approaches do not treat fair all peers.
\end{tabular}
\\
\hline
%%%%%%%%%%%%%%%%%%%%%%%%%%%%%%%%%%%%%%%%%%%%%%%%%%%%%%%%%%%%%%%%%%%%%%%%%%%%%%%%
\textbf{Ono} &
{\large \CheckedBox} &
{\large \Square} &
{\large \Square} &
{\large \CheckedBox} &
\begin{tabular}[l]{@{}l@{}}
ISP clustering\\
Landmarking based on existing CDN infrastructure (CDN redirection measurements)
\end{tabular} &
\begin{tabular}[l]{@{}l@{}l@{}l@{}}
+ Needs no ISP cooperation.\\
+ Needs no extra infrastructure.\\
+ Needs no network topology information.\\
-- Locality based approaches do not treat fair all peers
\end{tabular}
\\
\hline
%%%%%%%%%%%%%%%%%%%%%%%%%%%%%%%%%%%%%%%%%%%%%%%%%%%%%%%%%%%%%%%%%%%%%%%%%%%%%%%%
\textbf{Locality-Awareness in BitTorrent-like P2P Applications} &
{\large \CheckedBox} &
{\large \Square} &
{\large \Square} &
{\large \Square} &
AS hop count minimisation on neighbour selection, on chocking/unchocking mechanisms and on next-chunk picking &
\begin{tabular}[l]{@{}l@{}}
+ Optimisation of the inter-AS traffic.\\
-- Locality based approaches do not treat fair all peers.
\end{tabular}
\\
\hline
%%%%%%%%%%%%%%%%%%%%%%%%%%%%%%%%%%%%%%%%%%%%%%%%%%%%%%%%%%%%%%%%%%%%%%%%%%%%%%%%
\textbf{TopBT} &
{\large \CheckedBox} &
{\large \Square} &
{\large \Square} &
{\large \Square} &
\begin{tabular}[l]{@{}l@{}}
Peer selection metric that tacks both downloading speed and network topology into account\\
Applied in multiple places of the bittorrent protocol (bootstrap, connection establishment/replacement, unchocking)
\end{tabular} &
\begin{tabular}[l]{@{}l@{}l@{}}
+ No need for additional infrastructure.\\
+ Enhances both traffic and downloading.\\
-- Needs offline processing of BGP dumps.
\end{tabular}
\\
\hline
%%%%%%%%%%%%%%%%%%%%%%%%%%%%%%%%%%%%%%%%%%%%%%%%%%%%%%%%%%%%%%%%%%%%%%%%%%%%%%%%
\textbf{UTAPS} &
{\large \CheckedBox} &
{\large \Square} &
{\large \Square} &
{\large \CheckedBox} &
Network tomography to construct a picture for the underlying network &
\begin{tabular}[l]{@{}l@{}l@{}}
+ Reduced ISP burden.\\
+ Better downloading speeds.\\
-- not thoroughly evaluated.
\end{tabular}
\\
\hline
%%%%%%%%%%%%%%%%%%%%%%%%%%%%%%%%%%%%%%%%%%%%%%%%%%%%%%%%%%%%%%%%%%%%%%%%%%%%%%%%
\textbf{An Effective Network-Aware Peer Selection Algorithm in Bittorrent} &
{\large \CheckedBox} &
{\large \Square} &
{\large \Square} &
{\large \CheckedBox} &
Cluster peers in a swarm into local, intra- and inter-ISP &
\begin{tabular}[l]{@{}l@{}l@{}}
+ Reduced ISP burden.\\
+ Better downloading speeds.\\
-- Not thoroughly evaluated.
\end{tabular}
\\
\hline
%%%%%%%%%%%%%%%%%%%%%%%%%%%%%%%%%%%%%%%%%%%%%%%%%%%%%%%%%%%%%%%%%%%%%%%%%%%%%%%%
\textbf{PROP} &
{\large \CheckedBox} &
{\large \Square} &
{\large \Square} &
{\large \Square} &
Neighbour exchange between peers &
\begin{tabular}[l]{@{}l@{}}
+ Cooperation between peers.\\
+ Guarrantees the connectivity of the network between exchanges.\\
\end{tabular}
\\
\hline
%%%%%%%%%%%%%%%%%%%%%%%%%%%%%%%%%%%%%%%%%%%%%%%%%%%%%%%%%%%%%%%%%%%%%%%%%%%%%%%%
\textbf{Resolving the Topology Mismatch Problem in Unstructured Peer-to-Peer Networks} &
? &
? &
? &
? &
? &
?
\\
\hline
%%%%%%%%%%%%%%%%%%%%%%%%%%%%%%%%%%%%%%%%%%%%%%%%%%%%%%%%%%%%%%%%%%%%%%%%%%%%%%%%
\textbf{DDNO} &
{\large \CheckedBox} &
{\large \Square} &
{\large \Square} &
{\large \Square} &
Domain name topology detection (Split-Hash and dnMatch) &
\begin{tabular}[l]{@{}l@{}l@{}}
+ Can be applied to both fully unstructured and super-peer based architectures.\\
+ Secures connectivity of the network.\\
+ Reduces cost of message exchange.
\end{tabular}
\\
\hline
%%%%%%%%%%%%%%%%%%%%%%%%%%%%%%%%%%%%%%%%%%%%%%%%%%%%%%%%%%%%%%%%%%%%%%%%%%%%%%%%
\textbf{CTAG} &
{\large \CheckedBox} &
{\large \Square} &
{\large \Square} &
{\large \Square} &
Clustering based on longest matching IP segment &
+ Focuses on both construction and adaptation.
\\
\hline
%%%%%%%%%%%%%%%%%%%%%%%%%%%%%%%%%%%%%%%%%%%%%%%%%%%%%%%%%%%%%%%%%%%%%%%%%%%%%%%%
\textbf{Landmark Binning} &
{\large \CheckedBox} &
{\large \Square} &
{\large \Square} &
{\large \CheckedBox} &
Landmark binning &
\begin{tabular}[l]{@{}l@{}l@{}@{}l@{}l@{}l@{}}
+ It is independent of the overlay model.\\
+ The technique can be considered scalable.\\
-- Uses not so reliable network latency metric (this can lead to load imbalance etc).\\
-- The use of landmark servers renders the technique not fully distributed.\\
-- Excessive traffic flow towards the landmark servers is possible.\\
-- Fixed points in a network are inherently more exposed to malicius attacks.\\
-- Coarse-grained scheme.
\end{tabular}
\\
\hline
%%%%%%%%%%%%%%%%%%%%%%%%%%%%%%%%%%%%%%%%%%%%%%%%%%%%%%%%%%%%%%%%%%%%%%%%%%%%%%%%
\textbf{mOverlay} &
{\large \CheckedBox} &
{\large \Square} &
{\large \Square} &
{\large \CheckedBox} &
Dynamic landmarks &
\begin{tabular}[l]{@{}l@{}l@{}}
+ fully distributed.\\
+ It is independent of the overlay model.\\
- Coarse-grained scheme.
\end{tabular}
\\
\hline
%%%%%%%%%%%%%%%%%%%%%%%%%%%%%%%%%%%%%%%%%%%%%%%%%%%%%%%%%%%%%%%%%%%%%%%%%%%%%%%%

















%\begin{tabular}[l]{@{}}
%+ ID reduces the messages especially in upper levels of the tree\\
%+ DBFS\\
%+ LI reduces aggregate bandwidth usage and improves query efficiency\\
%- ID needs evaluation time between iterations\\
%- DBFS uses heuristics so it depends on their efficient choise\\
%- LI add index update overhead which might be heavy and might not work at all in systems with high churn
%\end{tabular} &


%\textbf{Narada} & \textbf{Overlay optimization
%based}. Creates a mesh (richer connected graph) and builds minimum spanning
%trees on this mesh & & Small and sparse groups \\

%\hline
%\textbf{Gia} & \textbf{Broadcast based} Replaces
%Gnutella flooding with random walk, and introduces KaZaA style supernodes. Uses
%dynamic topology adaptation protocol &
% Gnutella &  Better than Gnutella  \\

%\hline
%\textbf{Adaptive Overlay Topology Optimization} & \textbf{Overlay optimization
%based}. Creates overlay multicast tree with Selective Flooding protocol&
%Gnutella &  Better than Gnutella \\

% \hline
% \textbf{Location-aware Topology Matching} &
% \textbf{Overlay Optimization Based}. Uses \textit{TTL2-detector flooding}, \textit{low productive
% connection cutting}, and \textit{source peer probing}. & Gnutella &  Better than Gnutella \\
% 
% \hline
% \textbf{Replication Strategies in Unstructured P2P Networks} &
% \textbf{Cache Based}. Uses uniform, proportional and square root allocation
% strategies to replicate data. & Gnutella &  Better than Gnutella \\
% 
% % \hline
% \textbf{Tracing a large-scale Peer to Peer System: an hour in the life of Gnutella.} &
% \textbf{Cache Based}. Proposes a caching algorithm based on the traces of the Gnutella traffic & Gnutella & Better than Gnutella \\
% 
% \hline
% \textbf{Improving search in P2P networks} &
% \textbf{Broadcast Based}. Uses \textit{iterative deepening}, \textit{directed
% BFS}, and \textit{local indices} to improve efficiency. & Gnutella &  Better than Gnutella \\
% 
% \hline
% \textbf{Distributed Cycle Minimization Protocol} &
% \textbf{Broadcast based} Uses a decentralized cycle elimination protocol  &  &  \\
% 
% \hline
% \textbf{Scalable Bipartite Overlay} &
% \textbf{Overlay optimization based} Uses bipartite partition graph and builds
% local minimum spanning trees  & Gnutella  & Better than Gnutella \\
% 
% \hline
% \textbf{Adaptive Connection Establishment} &
% \textbf{Overlay optimization based} Forms Neighbour Cost Tables, builds local
% minimum spanning trees and perform local optimizations & Adaptive Overlay
% Topology Optimization (AOTO), Gnutella & Better than Gnutella \\

% \hline
% \textbf{Hops Adaptive Neighbour Discovery} &  & &  \\
% 
% \hline
% \textbf{Two-Hop-Away Neighbour Comparison and Selection (THANCS)} &
% \textbf{Overlay optimization based} Uses piggybacking to discover neighbour
% distances and selects neighbours  & Gnutella  & \\

% \hline
% \textbf{mOverlay} &\textbf{Landmark based proximity} Uses dynamic landmarks to find node locality
% & & Due to dynamic landmarks and grouping, more scalable than tree-based or mesh-based protocols \\
% 
% \hline
% \textbf{Distributed Domain Name Order (DDNO)} &
% \textbf{Overlay optimization based} Connects half of the nodes connections to
% the nodes in the same domain and the other half to random nodes, therefore
% supports locality and topological connection  &  & Yes, by using super
% peers \\

% \hline
% \textbf{Peer-exchange Routing Optimization Protocols} & \textbf{Overlay optimization based} Optimizes overlay by the exchange of
% neighbors among peers  & Can work with both decentralized structured and
% unstructured architecture & Yes \\

% \hline
% \textbf{MAY OMIT - I CHANGED IT TO STRUCTURED SINCE THERE IS A REFERENCE FOR
% DHT (OF COURSE IT MIGHT POSIBLE TO BE APPLIED TO BOTH. MAYBE NEED TO CHECK) -
% T2MC} &
% \textbf{Overlay optimization based} Uses traceroute results for clustering the
% nodes  & & \\
% 
% \hline
% \textbf{Unnamed-unstructured} &
% \textbf{Overlay optimization based} Minimizes the communication delay and
% maximizes the broadcasting range & & Better than THANCS and mOverlay \\

% \hline
% \textbf{Landmark Binning} & \textbf{Landmark based proximity} Uses network latency to partition
% nodes into bins & Can work with both decentralized structured and unstructured architecture & \\
% 
% \hline
%\end{tabular}
\end{longtable}
\end{landscape}
\end{center}
\vspace{-2.5ex}
\vspace{-2.5ex}
%\end{figure}