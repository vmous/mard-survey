%\renewcommand\arraystretch{1.9}% (MyValue=1.0 is for standard)

\onecolumn

%\begin{landscape}
%\begin{figure}[h!]
\hspace{-3ex}
\begin{center}
\footnotesize
%\begin{tabular}{
\renewcommand*{\arraystretch}{1.6}
\begin{longtable}{
m{2cm}
m{0.35cm}
m{0.35cm}
m{0.35cm}
m{0.35cm}
m{3cm}
m{5cm}
}
% |>{\columncolor[gray]{.7}}m{0.1\columnwidth}
% |>{\columncolor[gray]{.9}}m{0.1\columnwidth}
% |>{\columncolor[gray]{.8}}m{0.1\columnwidth}
% |>{\columncolor[gray]{.9}}m{0.1\columnwidth}
% |>{\columncolor[gray]{.9}}m{0.1\columnwidth}
% |>{\columncolor[gray]{.8}}m{0.1\columnwidth}
% |>{\columncolor[gray]{.9}}m{0.1\columnwidth}
% |}
\caption[Summary table for unstructured algorithms]{Summary table for unstructured algorithms.} \label{unstructured:table} \\
%\hline
%%%%%%%%%%%%%%%%%%%%%%%%%%%%%%%%%%%%%%%%%%%%%%%%%%%%%%%%%%%%%%%%%%%%%%%%%%%%%%%%
% first head
%\rowcolor[gray]{.5}
\rot{\textbf{Algorithm / Paper}} &
\rot{\textbf{Topology Adaptation}} &
\rot{\textbf{Forwarding Optimization}} &
\rot{\textbf{Caching / Replication}} &
\rot{\textbf{Landmarking}} &
\rot{\textbf{Highlights}} &
\rot{\textbf{Pros / Cons}}\\
\hline
\endfirsthead
%%%%%%%%%%%%%%%%%%%%%%%%%%%%%%%%%%%%%%%%%%%%%%%%%%%%%%%%%%%%%%%%%%%%%%%%%%%%%%%%
% subsequent heads
\multicolumn{7}{c}%
{\tablename\ \thetable\ -- \textit{Continued from previous page}} \\
%\hline
%\rowcolor[gray]{.5}
\rot{\textbf{Algorithm / Paper}} &
\rot{\textbf{Topology Adaptation}} &
\rot{\textbf{Forwarding Optimization}} &
\rot{\textbf{Caching / Replication}} &
\rot{\textbf{Landmarking}} &
\rot{\textbf{Highlights}} &
\rot{\textbf{Pros / Cons}}\\
\hline
\endhead
%%%%%%%%%%%%%%%%%%%%%%%%%%%%%%%%%%%%%%%%%%%%%%%%%%%%%%%%%%%%%%%%%%%%%%%%%%%%%%%%
% foot
\hline \multicolumn{7}{r}{\textit{Continued on next page}} \\
\endfoot
%%%%%%%%%%%%%%%%%%%%%%%%%%%%%%%%%%%%%%%%%%%%%%%%%%%%%%%%%%%%%%%%%%%%%%%%%%%%%%%%
% last foot
\hline
\endlastfoot
%%%%%%%%%%%%%%%%%%%%%%%%%%%%%%%%%%%%%%%%%%%%%%%%%%%%%%%%%%%%%%%%%%%%%%%%%%%%%%%%
% data
\cite{YG-M2002} &
{\large \Square} &
{\large \CheckedBox} &
{\large \CheckedBox} &
{\large \Square} &
\begin{tabular}[l]{m{3cm}}
Iterative Deepening (ID).\\
Directed BFS (DBFS).\\
Local Indices (LI).
\end{tabular} &
\begin{tabular}[l]{m{5cm}}
+ ID reduces the messages especially in upper levels of the tree.\\
%+ DBFS ??????\\
+ LI reduces aggregate bandwidth usage and improves query efficiency.\\
-- ID needs evaluation time between iterations.\\
-- DBFS uses heuristics so it depends on their efficient choice.\\
-- LI add index update overhead which might be heavy process especially in high-churn systems.
\end{tabular}
\\
\hline
%%%%%%%%%%%%%%%%%%%%%%%%%%%%%%%%%%%%%%%%%%%%%%%%%%%%%%%%%%%%%%%%%%%%%%%%%%%%%%%%
DAPS - \cite{ZL2005} &
{\large \CheckedBox} &
{\large \CheckedBox} &
{\large \Square} &
{\large \Square} &
\begin{tabular}[l]{m{3cm}}
Clustered routing tables based on delay.\\
Pruning flood, an iterative deepening and multiple BFS approach with a pruning
boundary.
\end{tabular} &
+ It is a hybrid system with both structured and unstructured characteristics.
\\
\hline
%%%%%%%%%%%%%%%%%%%%%%%%%%%%%%%%%%%%%%%%%%%%%%%%%%%%%%%%%%%%%%%%%%%%%%%%%%%%%%%%
Gia - \cite{CRBLS2003} &
{\large \CheckedBox} &
{\large \CheckedBox} &
{\large \CheckedBox} &
{\large \Square} &
\begin{tabular}[l]{m{3cm}}
Random Walks (RW).
\end{tabular} &
\begin{tabular}[l]{m{5cm}}
+ RWs issue one copy of the query thus not flooding the whole network.\\
-- RWs can reduce search scope.\\
\end{tabular}
\\
\hline
%%%%%%%%%%%%%%%%%%%%%%%%%%%%%%%%%%%%%%%%%%%%%%%%%%%%%%%%%%%%%%%%%%%%%%%%%%%%%%%%
LTM - \cite{LLXNZ2004} &
{\large \CheckedBox} &
{\large \Square} &
{\large \Square} &
{\large \Square} &
\begin{tabular}[l]{m{3cm}}
TTL detector (2-hop distance).\\
Delayed low productive connection cutting.
\end{tabular} &
\begin{tabular}[l]{m{5cm}}
+ Compared to AOTO, ACE and SBO achieves faster convergence speed.\\
-- Creates more overhead than AOTO, ACE and SBO.\\
-- Needs synchronization of peer clocks.\\
-- Does not consider shortcuts created by powerful peers when choosing to disable connections (only uses delay metric).
\end{tabular}
\\
\hline
%%%%%%%%%%%%%%%%%%%%%%%%%%%%%%%%%%%%%%%%%%%%%%%%%%%%%%%%%%%%%%%%%%%%%%%%%%%%%%%%
DCMP - \cite{ZKB2008} &
{\large \CheckedBox} &
{\large \Square} &
{\large \Square} &
{\large \Square} &
\begin{tabular}[l]{m{3cm}}
Cycle detection.
\end{tabular} &
\begin{tabular}[l]{m{5cm}}
+ Drastically reduces duplicate messages.\\
-- Cannot detect cycles in distance bigger than the TTL value of the IC message.\\
\end{tabular}
\\
\hline
%%%%%%%%%%%%%%%%%%%%%%%%%%%%%%%%%%%%%%%%%%%%%%%%%%%%%%%%%%%%%%%%%%%%%%%%%%%%%%%%
\cite{CS2002} \& \cite{LCCLS2002} &
{\large \Square} &
{\large \CheckedBox} &
{\large \CheckedBox} &
{\large \Square} &
\begin{tabular}[l]{m{3cm}}
Uniform replication.\\
Proportional replication.\\
Square root replication allocation.\\
\emph{k}-walker query scheme.
\end{tabular} &
\begin{tabular}[l]{m{5cm}}
+ Uniform replication reduces time spent on unsuccessful searches.\\
+ Reduces search time for frequent queries.\\
\emph{k}-walker querying scheme can reduce network traffic up to two orders of magnitude.\\
-- Proportional replication struggles in locating rare objects.
\end{tabular}
\\
\hline
%%%%%%%%%%%%%%%%%%%%%%%%%%%%%%%%%%%%%%%%%%%%%%%%%%%%%%%%%%%%%%%%%%%%%%%%%%%%%%%%
% \textbf{Tracing a large-scale Peer to Peer System: an hour in the life of Gnutella} &
% ? &
% ? &
% ? &
% ? &
% ? &
% ?
% \\
% \hline
%%%%%%%%%%%%%%%%%%%%%%%%%%%%%%%%%%%%%%%%%%%%%%%%%%%%%%%%%%%%%%%%%%%%%%%%%%%%%%%%
Narada - \cite{CRZ2000} &
{\large \Square} &
{\large \CheckedBox} &
{\large \Square} &
{\large \Square} &
\begin{tabular}[l]{m{3cm}}
Mess creation.\\
Minimum spanning trees.
\end{tabular} &
\begin{tabular}[l]{m{5cm}}
+ Mess and trees are kept up-to-date in high churn environments.\\
-- Works well only for small groups of peers.
\end{tabular}
\\
\hline
%%%%%%%%%%%%%%%%%%%%%%%%%%%%%%%%%%%%%%%%%%%%%%%%%%%%%%%%%%%%%%%%%%%%%%%%%%%%%%%%
AOTO - \cite{LZXN2003} &
{\large \CheckedBox} &
{\large \CheckedBox} &
{\large \Square} &
{\large \Square} &
\begin{tabular}[l]{m{3cm}}
Minimum spanning trees.\\
Peer proximity heuristic for removing costly links.
\end{tabular} &
\begin{tabular}[l]{m{5cm}}
+ Spanning trees only to immediate neighbors so no flooding and at the same time
no shrinked search scope.\\
+ Selective flooding effectiveness is detached from physical or overlay topologies.\\
+ The more logical neighbors, the more effective selective flooding becomes\\
-- High recalculation costs.\\
-- No sophisticated selection policy for candidate non-flooding peers.
\end{tabular}
\\
\hline
%%%%%%%%%%%%%%%%%%%%%%%%%%%%%%%%%%%%%%%%%%%%%%%%%%%%%%%%%%%%%%%%%%%%%%%%%%%%%%%%
ACE - \cite{LZXN2004} &
{\large \CheckedBox} &
{\large \CheckedBox} &
{\large \Square} &
{\large \Square} &
\begin{tabular}[l]{m{3cm}}
Minimum spanning trees.\\
1-hop proximity heuristic.
\end{tabular} &
\begin{tabular}[l]{m{5cm}}
+ No flooding.\\
+ Less overhead compared to AOTO since computation is done within a certain diameter from the source peer.
-- Slow convergence speed.\\
-- Enhanced topology optimization comes to the expense of higher communication/computation overhead.
\end{tabular}
\\
\hline
%%%%%%%%%%%%%%%%%%%%%%%%%%%%%%%%%%%%%%%%%%%%%%%%%%%%%%%%%%%%%%%%%%%%%%%%%%%%%%%%
SBO - \cite{LXN2007} &
{\large \CheckedBox} &
{\large \CheckedBox} &
{\large \Square} &
{\large \Square} &
\begin{tabular}[l]{m{3cm}}
Red/white bipartite overlay.
\end{tabular} &
\begin{tabular}[l]{m{5cm}}
+ Efficient in both static and dynamic environments.\\
+ Compared to AOTO incurs half the overhead.\\
-- Needs almost double the steps of LTM to converge (static or dynamic environments).
\end{tabular}
\\
\hline
%%%%%%%%%%%%%%%%%%%%%%%%%%%%%%%%%%%%%%%%%%%%%%%%%%%%%%%%%%%%%%%%%%%%%%%%%%%%%%%%
THANCS - \cite{LNXE2005} &
{\large \CheckedBox} &
{\large \CheckedBox} &
{\large \Square} &
{\large \Square} &
\begin{tabular}[l]{m{3cm}}
Local optimum heuristic\\
Piggybacking neighbor distance in queries
\end{tabular} &
\begin{tabular}[l]{m{5cm}}
+ Completely distributed approach.\\
+ Presents trivial overhead compared to the query cost savings.\\
+ Convergent speed faster among AOTO, LTM, SBO.\\
+ Does not shrink the search scope.\\
-- Design cannot be extended to support non-flooding-based systems.
\end{tabular}
\\
\hline
%%%%%%%%%%%%%%%%%%%%%%%%%%%%%%%%%%%%%%%%%%%%%%%%%%%%%%%%%%%%%%%%%%%%%%%%%%%%%%%%
HAND - \cite{CLZHC2006} &
{\large \CheckedBox} &
{\large \Square} &
{\large \Square} &
{\large \Square} &
\begin{tabular}[l]{m{3cm}}
Triple-hop adjustment.
\end{tabular} &
\begin{tabular}[l]{m{5cm}}
+ No need for clock sync.\\
+ Fully distributed.\\
+ Low overhead for the triple hop adjustment.\\
+ Applicable to both static and dynamic environments.\\
+ Low query response time.\\
-- Compared to LTM has lower traffic reduction and query response rates.
\end{tabular}
\\
\hline
%%%%%%%%%%%%%%%%%%%%%%%%%%%%%%%%%%%%%%%%%%%%%%%%%%%%%%%%%%%%%%%%%%%%%%%%%%%%%%%%
APS - \cite{BFLZ2003} &
{\large \CheckedBox} &
{\large \Square} &
{\large \Square} &
{\large \Square} &
\begin{tabular}[l]{m{3cm}}
machine learning adaptive mechanism.
\end{tabular} &
\begin{tabular}[l]{m{5cm}}
+ Fully dynamic switching decision policy.\\
- Low convergence due to the learning process.
\end{tabular}
\\
\hline
%%%%%%%%%%%%%%%%%%%%%%%%%%%%%%%%%%%%%%%%%%%%%%%%%%%%%%%%%%%%%%%%%%%%%%%%%%%%%%%%
ITA - \cite{PRFM2013} &
{\large \CheckedBox} &
{\large \CheckedBox} &
{\large \Square} &
{\large \Square} &
\begin{tabular}[l]{m{3cm}}
Short/long connections.\\
Local flooding.
\end{tabular} &
\begin{tabular}[l]{m{5cm}}
+ Low clustering.\\
+ Large peer coverage.\\
+ Reduced duplication.\\
+ Low or no impact to other mechanisms of unstructured p2p networks (e.g. 1-hop
replication, dynamic querying).
\end{tabular}
\\
\hline
%%%%%%%%%%%%%%%%%%%%%%%%%%%%%%%%%%%%%%%%%%%%%%%%%%%%%%%%%%%%%%%%%%%%%%%%%%%%%%%%
EGOIST - \cite{SLLBBR2008} &
{\large \CheckedBox} &
{\large \Square} &
{\large \Square} &
{\large \Square} &
\begin{tabular}[l]{m{3cm}}
Selfish shortest path routing
\end{tabular} &
\begin{tabular}[l]{m{3cm}}
-- constructs a global view of the network
\end{tabular}
\\
\hline
%%%%%%%%%%%%%%%%%%%%%%%%%%%%%%%%%%%%%%%%%%%%%%%%%%%%%%%%%%%%%%%%%%%%%%%%%%%%%%%%
BNS - \cite{BCCMSBZ2006} &
{\large \CheckedBox} &
{\large \Square} &
{\large \CheckedBox} &
{\large \Square} &
\begin{tabular}[l]{m{3cm}}
ISP clustering (tracker-side or ISP-side detection).\\
Bandwidth throttling.\\
Caching.
\end{tabular} &
\begin{tabular}[l]{m{5cm}}
+ Localizes traffic within an ISP.\\
+ Preserves the efficiency of BitTorrent protocol.\\
-- Needs ISPs to either provide information or infrastructure changes.\\
-- Locality-based approaches do not treat all peers fairly.
\end{tabular}
\\
\hline
%%%%%%%%%%%%%%%%%%%%%%%%%%%%%%%%%%%%%%%%%%%%%%%%%%%%%%%%%%%%%%%%%%%%%%%%%%%%%%%%
Ono - \cite{CB2008} &
{\large \CheckedBox} &
{\large \Square} &
{\large \Square} &
{\large \CheckedBox} &
\begin{tabular}[l]{m{3cm}}
ISP clustering.\\
Landmarking based on existing CDN infrastructure (CDN redirection measurements).
\end{tabular} &
\begin{tabular}[l]{m{5cm}}
+ Needs no ISP cooperation.\\
+ Needs no network topology information.\\
-- Depends on feeds from major Internet infrastructures and large deployment
of its client.\\
-- ISP-friendly approaches do not seem to have a great impact on Internet-scale
basis.
\end{tabular}
\\
\hline
%%%%%%%%%%%%%%%%%%%%%%%%%%%%%%%%%%%%%%%%%%%%%%%%%%%%%%%%%%%%%%%%%%%%%%%%%%%%%%%%
\cite{LCLX2009} &
{\large \CheckedBox} &
{\large \Square} &
{\large \Square} &
{\large \Square} &
\begin{tabular}[l]{m{3cm}}
AS hop count minimization on neighbor selection, on chocking/unchocking
mechanisms and on next-chunk picking.
\end{tabular} &
\begin{tabular}[l]{m{5cm}}
+ Optimization of the inter-AS traffic.\\
-- Locality based approaches do not treat all peers fairly.
\end{tabular}
\\
\hline
%%%%%%%%%%%%%%%%%%%%%%%%%%%%%%%%%%%%%%%%%%%%%%%%%%%%%%%%%%%%%%%%%%%%%%%%%%%%%%%%
TopBT - \cite{RTLCGZ2010} &
{\large \CheckedBox} &
{\large \Square} &
{\large \Square} &
{\large \Square} &
\begin{tabular}[l]{m{3cm}}
Peer selection metric that takes both downloading speed and network topology
into account.\\
Applied in multiple places of the BitTorrent protocol (bootstrap, connection
establishment/replacement, unchocking).
\end{tabular} &
\begin{tabular}[l]{m{5cm}}
+ No need for additional infrastructure.\\
+ Enhances both traffic and downloading.\\
-- Needs off-line processing of BGP dumps.
\end{tabular}
\\
\hline
%%%%%%%%%%%%%%%%%%%%%%%%%%%%%%%%%%%%%%%%%%%%%%%%%%%%%%%%%%%%%%%%%%%%%%%%%%%%%%%%
UTAPS - \cite{LCY2008} &
{\large \CheckedBox} &
{\large \Square} &
{\large \Square} &
{\large \CheckedBox} &
\begin{tabular}[l]{m{3cm}}
Network tomography to construct a picture for the underlying network.
\end{tabular} &
\begin{tabular}[l]{m{5cm}}
+ Reduced ISP burden.\\
+ Better downloading speeds.\\
-- Small, laboratory-scale evaluation setup.
\end{tabular}
\\
\hline
%%%%%%%%%%%%%%%%%%%%%%%%%%%%%%%%%%%%%%%%%%%%%%%%%%%%%%%%%%%%%%%%%%%%%%%%%%%%%%%%
\cite{QLZG2009} &
{\large \CheckedBox} &
{\large \Square} &
{\large \Square} &
{\large \CheckedBox} &
\begin{tabular}[l]{m{3cm}}
Cluster peers in a swarm into local, intra- and inter-ISP.
\end{tabular} &
\begin{tabular}[l]{m{5cm}}
+ Reduced ISP burden.\\
+ Better downloading speeds.\\
-- Similarly to UTAPS, small scale evaluation setup.
\end{tabular}
\\
\hline
%%%%%%%%%%%%%%%%%%%%%%%%%%%%%%%%%%%%%%%%%%%%%%%%%%%%%%%%%%%%%%%%%%%%%%%%%%%%%%%%
PROP - \cite{QCYCZ2007} &
{\large \CheckedBox} &
{\large \Square} &
{\large \Square} &
{\large \Square} &
\begin{tabular}[l]{m{3cm}}
Neighbor exchange between peers.
\end{tabular} &
\begin{tabular}[l]{m{5cm}}
+ Cooperation between peers.\\
+ Guarantees the connectivity of the network between exchanges.\\
\end{tabular}
\\
\hline
%%%%%%%%%%%%%%%%%%%%%%%%%%%%%%%%%%%%%%%%%%%%%%%%%%%%%%%%%%%%%%%%%%%%%%%%%%%%%%%%
% \textbf{Resolving the Topology Mismatch Problem in Unstructured Peer-to-Peer Networks} &
% ? &
% ? &
% ? &
% ? &
% ? &
% ?
% \\
% \hline
%%%%%%%%%%%%%%%%%%%%%%%%%%%%%%%%%%%%%%%%%%%%%%%%%%%%%%%%%%%%%%%%%%%%%%%%%%%%%%%%
DDNO - \cite{Z-YK2005} &
{\large \CheckedBox} &
{\large \Square} &
{\large \Square} &
{\large \Square} &
\begin{tabular}[l]{m{3cm}}
Domain name topology detection (Split-Hash and dnMatch).
\end{tabular} &
\begin{tabular}[l]{m{5cm}}
+ Can be applied to both fully unstructured and super-peer based architectures.\\
+ Secures connectivity of the network.\\
+ Reduces cost of message exchange.
\end{tabular}
\\
\hline
%%%%%%%%%%%%%%%%%%%%%%%%%%%%%%%%%%%%%%%%%%%%%%%%%%%%%%%%%%%%%%%%%%%%%%%%%%%%%%%%
CTAG - \cite{ZL2006} &
{\large \CheckedBox} &
{\large \Square} &
{\large \Square} &
{\large \Square} &
\begin{tabular}[l]{m{3cm}}
Clustering based on longest matching IP segment.
\end{tabular} &
\begin{tabular}[l]{m{5cm}}
+ Focuses on both construction and adaptation.
\end{tabular}
\\
\hline
%%%%%%%%%%%%%%%%%%%%%%%%%%%%%%%%%%%%%%%%%%%%%%%%%%%%%%%%%%%%%%%%%%%%%%%%%%%%%%%%
Landmark Binning - \cite{RHKS2002} &
{\large \CheckedBox} &
{\large \Square} &
{\large \Square} &
{\large \CheckedBox} &
\begin{tabular}[l]{m{3cm}}
Landmark binning.
\end{tabular} &
\begin{tabular}[l]{m{5cm}}
+ It is independent of the overlay model.\\
+ The technique can be considered scalable.\\
-- Uses potentially unreliable network latency metric (this can lead to load imbalance etc).\\
-- The use of landmark servers renders the technique not fully distributed.\\
-- Excessive traffic flow towards the landmark servers is possible.\\
-- Fixed points in a network are inherently more exposed to malicious attacks.\\
-- Coarse-grained scheme.
\end{tabular}
\\
\hline
%%%%%%%%%%%%%%%%%%%%%%%%%%%%%%%%%%%%%%%%%%%%%%%%%%%%%%%%%%%%%%%%%%%%%%%%%%%%%%%%
mOverlay - \cite{ZZZSZ2004} &
{\large \CheckedBox} &
{\large \Square} &
{\large \Square} &
{\large \CheckedBox} &
\begin{tabular}[l]{m{3cm}}
Dynamic landmarks.
\end{tabular} &
\begin{tabular}[l]{m{5cm}}
+ Fully distributed.\\
+ It is independent of the overlay model.\\
+ Load balanced.\\
-- Coarse-grained scheme.
\end{tabular}
\\
\hline
%%%%%%%%%%%%%%%%%%%%%%%%%%%%%%%%%%%%%%%%%%%%%%%%%%%%%%%%%%%%%%%%%%%%%%%%%%%%%%%%


%\begin{tabular}[l]{@{}}
%+ ID reduces the messages especially in upper levels of the tree\\
%+ DBFS\\
%+ LI reduces aggregate bandwidth usage and improves query efficiency\\
%- ID needs evaluation time between iterations\\
%- DBFS uses heuristics so it depends on their efficient choice\\
%- LI add index update overhead which might be heavy and might not work at all in systems with high churn
%\end{tabular} &


%\textbf{Narada} & \textbf{Overlay optimization
%based}. Creates a mesh (richer connected graph) and builds minimum spanning
%trees on this mesh & & Small and sparse groups \\

%\hline
%\textbf{Gia} & \textbf{Broadcast based} Replaces
%Gnutella flooding with random walk, and introduces KaZaA style super-nodes.
%Uses
%dynamic topology adaptation protocol &
% Gnutella &  Better than Gnutella  \\

%\hline
%\textbf{Adaptive Overlay Topology Optimization} & \textbf{Overlay optimization
%based}. Creates overlay multi-cast tree with Selective Flooding protocol&
%Gnutella &  Better than Gnutella \\

% \hline
% \textbf{Location-aware Topology Matching} &
% \textbf{Overlay Optimization Based}. Uses \textit{TTL2-detector flooding}, \textit{low productive
% connection cutting}, and \textit{source peer probing}. & Gnutella &  Better than Gnutella \\
% 
% \hline
% \textbf{Replication Strategies in Unstructured P2P Networks} &
% \textbf{Cache Based}. Uses uniform, proportional and square root allocation
% strategies to replicate data. & Gnutella &  Better than Gnutella \\
% 
% % \hline
% \textbf{Tracing a large-scale Peer to Peer System: an hour in the life of Gnutella.} &
% \textbf{Cache Based}. Proposes a caching algorithm based on the traces of the Gnutella traffic & Gnutella & Better than Gnutella \\
% 
% \hline
% \textbf{Improving search in P2P networks} &
% \textbf{Broadcast Based}. Uses \textit{iterative deepening}, \textit{directed
% BFS}, and \textit{local indices} to improve efficiency. & Gnutella &  Better than Gnutella \\
% 
% \hline
% \textbf{Distributed Cycle Minimization Protocol} &
% \textbf{Broadcast based} Uses a decentralized cycle elimination protocol  &  &  \\
% 
% \hline
% \textbf{Scalable Bipartite Overlay} &
% \textbf{Overlay optimization based} Uses bipartite partition graph and builds
% local minimum spanning trees  & Gnutella  & Better than Gnutella \\
% 
% \hline
% \textbf{Adaptive Connection Establishment} &
% \textbf{Overlay optimization based} Forms Neighbor Cost Tables, builds local
% minimum spanning trees and perform local optimizations & Adaptive Overlay
% Topology Optimization (AOTO), Gnutella & Better than Gnutella \\

% \hline
% \textbf{Hops Adaptive Neighbor Discovery} &  & &  \\
% 
% \hline
% \textbf{Two-Hop-Away Neighbor Comparison and Selection (THANCS)} &
% \textbf{Overlay optimization based} Uses piggybacking to discover neighbor
% distances and selects neighbors  & Gnutella  & \\

% \hline
% \textbf{mOverlay} &\textbf{Landmark based proximity} Uses dynamic landmarks to find node locality
% & & Due to dynamic landmarks and grouping, more scalable than tree-based or mesh-based protocols \\
% 
% \hline
% \textbf{Distributed Domain Name Order (DDNO)} &
% \textbf{Overlay optimization based} Connects half of the nodes connections to
% the nodes in the same domain and the other half to random nodes, therefore
% supports locality and topological connection  &  & Yes, by using super
% peers \\

% \hline
% \textbf{Peer-exchange Routing Optimization Protocols} & \textbf{Overlay optimization based} Optimizes overlay by the exchange of
% neighbors among peers  & Can work with both decentralized structured and
% unstructured architecture & Yes \\

% \hline
% \textbf{MAY OMIT - I CHANGED IT TO STRUCTURED SINCE THERE IS A REFERENCE FOR
% DHT (OF COURSE IT MIGHT POSSIBLE TO BE APPLIED TO BOTH. MAYBE NEED TO CHECK) -
% T2MC} &
% \textbf{Overlay optimization based} Uses trace-route results for clustering
% the nodes  & & \\
% 
% \hline
% \textbf{Unnamed-unstructured} &
% \textbf{Overlay optimization based} Minimizes the communication delay and
% maximizes the broadcasting range & & Better than THANCS and mOverlay \\

% \hline
% \textbf{Landmark Binning} & \textbf{Landmark based proximity} Uses network latency to partition
% nodes into bins & Can work with both decentralized structured and unstructured architecture & \\
% 
% \hline
%\end{tabular}
\end{longtable}
\end{center}
\vspace{-2.5ex}
\vspace{-2.5ex}
%\end{figure}
%\end{landscape}

\twocolumn
