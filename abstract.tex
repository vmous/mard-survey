% As a general rule, do not put math, special symbols or citations
% in the abstract or keywords.

\begin{abstract}
\emph{Peer-to-peer} (\p) systems have enjoyed immense attention and have been
widely deployed on the Internet for well over a decade.
They are often implemented via an overlay network abstraction
atop the Internet's best-effort IP infrastructure. 
\p\ systems support a plethora of desirable features to
distributed applications including
anonymity, high availability, robustness, 
load balancing, quality of service and scalability to name just a few.
Unfortunately, inherent weaknesses 
of early deployments of \p\ systems, prevented 
applications from leveraging the full potential of the paradigm.
One major weakness, identified early on, 
is the \emph{topology mismatch problem}
between the overlay network and the underlying IP topology.
This mismatch can impose an extraordinary amount of unnecessary stress on
network resources and can adversely affect 
both the scalability and efficiency of the operating applications.
In this paper, we survey over a decade's worth of research efforts
aimed at alleviating the topology mismatch problem in both 
\emph{structured} and \emph{unstructured} \p\ systems.   
We provide a fine-grained categorization of the suggested solutions
by discussing their novelty, advantages and weaknesses.
Finally, we offer an analysis as well as pictorial comparisons 
of the reviewed approaches since we aim to offer a comprehensive
reference for developers, system architects and researchers in the field.
\end{abstract}

% Note that keywords are not normally used for peerreview papers.
\begin{IEEEkeywords}
Distributed systems, peer-to-peer, overlay network, topology mismatch,
topology awareness
\end{IEEEkeywords}