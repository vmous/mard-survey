% \renewcommand\arraystretch{1.1}

\begin{landscape}
%\begin{figure}[h!]
\hspace{-3ex}
\begin{center}
\footnotesize
%\begin{tabular}{
\begin{longtable}{
|m{2cm}
|m{1cm}
|m{1cm}
|m{1cm}
|m{1cm}
|m{1cm}
|m{1cm}
|m{3cm}
|m{5cm}
|
}
%|>{\columncolor[gray]{.7}}m{0.15\columnwidth}
%|>{\columncolor[gray]{.9}}m{0.05in}
%|>{\columncolor[gray]{.8}}m{0.05\columnwidth}
%|>{\columncolor[gray]{.9}}m{0.05\columnwidth}
%|>{\columncolor[gray]{.9}}m{0.05\columnwidth}
%|>{\columncolor[gray]{.8}}m{0.05\columnwidth}
%|>{\columncolor[gray]{.9}}m{0.05\columnwidth}
%|>{\columncolor[gray]{.8}}m{0.1\columnwidth}
%|>{\columncolor[gray]{.9}}m{0.1\columnwidth}
%|>{\columncolor[gray]{.8}}m{0.1\columnwidth+}
%|
%}
\caption[Summary table for structured algorithms]{Summary table for structured algorithms.} \label{structured:table} \\
\hline
%%%%%%%%%%%%%%%%%%%%%%%%%%%%%%%%%%%%%%%%%%%%%%%%%%%%%%%%%%%%%%%%%%%%%%%%%%%%%%%%
% first head
\rowcolor[gray]{.5}
\textbf{Algorithm / Paper} &
\textbf{Topology Adaptation} &
\textbf{Landmarking} &
\textbf{Proximity Routing} &
\textbf{Proximity Neighbor Selection} &
\textbf{Geographic Layout} &
\textbf{Caching / Replication} &
\textbf{Highlights} &
\textbf{Pros / Cons}\\
\hline
\endfirsthead
%%%%%%%%%%%%%%%%%%%%%%%%%%%%%%%%%%%%%%%%%%%%%%%%%%%%%%%%%%%%%%%%%%%%%%%%%%%%%%%%
% subsequent heads
\multicolumn{9}{c}%
{\tablename\ \thetable\ -- \textit{Continued from previous page}} \\
\hline
\rowcolor[gray]{.5}
\textbf{Algorithm / Paper} &
\textbf{Topology Adaptation} &
\textbf{Landmarking} &
\textbf{Proximity Routing} &
\textbf{Proximity Neighbor Selection} &
\textbf{Geographic Layout} &
\textbf{Caching / Replication} &
\textbf{Highlights} &
\textbf{Pros / Cons}\\
\hline
\endhead
%%%%%%%%%%%%%%%%%%%%%%%%%%%%%%%%%%%%%%%%%%%%%%%%%%%%%%%%%%%%%%%%%%%%%%%%%%%%%%%%
% foot
\hline \multicolumn{9}{r}{\textit{Continued on next page}} \\
\endfoot
%%%%%%%%%%%%%%%%%%%%%%%%%%%%%%%%%%%%%%%%%%%%%%%%%%%%%%%%%%%%%%%%%%%%%%%%%%%%%%%%
% last foot
\hline
\endlastfoot
%%%%%%%%%%%%%%%%%%%%%%%%%%%%%%%%%%%%%%%%%%%%%%%%%%%%%%%%%%%%%%%%%%%%%%%%%%%%%%%%
% data
\textbf{Global Soft-State} &
{\large \CheckedBox} &
{\large \CheckedBox} &
{\large \Square} &
{\large \Square} &
{\large \CheckedBox} &
{\large \Square} &
\begin{tabular}[l]{m{3cm}}
Hybrid landmark binning.\\
Probing scheme for proximity detection.\\
Strategic injection of proximity info around the network.\\
Subscription-notification system to dynamically adapt to network changes.
\end{tabular} &
\begin{tabular}[l]{m{5cm}}
+ Greatly reduces routing latency to far away nodes.\\
-- Host state maintenance.\\
-- Unable to identify nodes that are close to routers/gateways.\\
-- Its static nature sacrifices the self-organizing attribute of DHTs.
\end{tabular}
\\
\hline
%%%%%%%%%%%%%%%%%%%%%%%%%%%%%%%%%%%%%%%%%%%%%%%%%%%%%%%%%%%%%%%%%%%%%%%%%%%%%%%%
\textbf{Mithos} &
{\large \Square} &
{\large \Square} &
{\large \CheckedBox} &
{\large \Square} &
{\large \CheckedBox} &
{\large \Square} &
\begin{tabular}[l]{m{3cm}}
Directed incremental probing.\\
Synthetic coordinates.
\end{tabular} &
\begin{tabular}[l]{m{5cm}}
-- Does not effectively manage dynamic peer arrival and leave.
\end{tabular}
\\
\hline
%%%%%%%%%%%%%%%%%%%%%%%%%%%%%%%%%%%%%%%%%%%%%%%%%%%%%%%%%%%%%%%%%%%%%%%%%%%%%%%%
\textbf{LAPTOP} &
{\large \CheckedBox} &
{\large \Square} &
{\large \Square} &
{\large \Square} &
{\large \CheckedBox} &
{\large \CheckedBox} &
\begin{tabular}[l]{m{3cm}}
Tree-based hierarchy.
\end{tabular} &
\begin{tabular}[l]{m{5cm}}
+ Reduces hops during message routing.\\
+ Minimal maintenance overhead.\\
-- Heartbeat approach incurs overhead even when not needed.
\end{tabular}
\\
\hline
%%%%%%%%%%%%%%%%%%%%%%%%%%%%%%%%%%%%%%%%%%%%%%%%%%%%%%%%%%%%%%%%%%%%%%%%%%%%%%%%
\textbf{Landmark Binning} &
{\large \Square} &
{\large \CheckedBox} &
{\large \Square} &
{\large \Square} &
{\large \CheckedBox} &
{\large \Square} &
Binning &
\\
\hline
%%%%%%%%%%%%%%%%%%%%%%%%%%%%%%%%%%%%%%%%%%%%%%%%%%%%%%%%%%%%%%%%%%%%%%%%%%%%%%%%
\textbf{Proximity in Kademlia} &
{\large \CheckedBox} &
{\large \Square} &
{\large \CheckedBox} &
{\large \CheckedBox} &
{\large \Square} &
{\large \Square} &
\begin{tabular}[l]{m{3cm}}
% Replacing XOR metric with a function that minimizes the underlying cost.\\
% Clustering (MaxMind geolocation technology).
\end{tabular} &
\begin{tabular}[l]{m{5cm}}
+ Proximity routing works in Kademlia to improve connection locality.
\end{tabular}
\\
\hline
%%%%%%%%%%%%%%%%%%%%%%%%%%%%%%%%%%%%%%%%%%%%%%%%%%%%%%%%%%%%%%%%%%%%%%%%%%%%%%%%
\textbf{CHOP6} &
{\large \Square} &
{\large \Square} &
{\large \CheckedBox} &
{\large \Square} &
{\large \Square} &
{\large \Square} &
\begin{tabular}[l]{m{3cm}}
Ipv6 format exploitation.\\
Additional RTT information.
\end{tabular} &
\\
\hline
%%%%%%%%%%%%%%%%%%%%%%%%%%%%%%%%%%%%%%%%%%%%%%%%%%%%%%%%%%%%%%%%%%%%%%%%%%%%%%%%
\textbf{T2MC} &
{\large \Square} &
{\large \CheckedBox} &
{\large \CheckedBox} &
{\large \Square} &
{\large \Square} &
{\large \Square} &
\begin{tabular}[l]{m{3cm}}
Clustering using trace-route logs.
\end{tabular} &
\begin{tabular}[l]{m{5cm}}
+ Prioritize interaction of peers and edge gateways.\\
-- Extended overhead.\\
-- Trace-route is sometimes disabled by ISPs.
\end{tabular}
\\
\hline
%%%%%%%%%%%%%%%%%%%%%%%%%%%%%%%%%%%%%%%%%%%%%%%%%%%%%%%%%%%%%%%%%%%%%%%%%%%%%%%%
\textbf{PChord} &
{\large \Square} &
{\large \Square} &
{\large \CheckedBox} &
{\large \Square} &
{\large \Square} &
{\large \Square} &
\begin{tabular}[l]{m{3cm}}
Maintenance of a proximity list.
\end{tabular} &
\begin{tabular}[l]{m{5cm}}
+ Dynamic maintenance of the proximity list.\\
+ Reduces routing costs.\\
+ Can constrain costly jumps in and out of network partitions.\\
+ Maintenance cost kept to a minimal (node join/leave heartbeat for lists)
\end{tabular}
\\
\hline
%%%%%%%%%%%%%%%%%%%%%%%%%%%%%%%%%%%%%%%%%%%%%%%%%%%%%%%%%%%%%%%%%%%%%%%%%%%%%%%%
\textbf{AChord} &
{\large \Square} &
{\large \Square} &
{\large \CheckedBox} &
{\large \Square} &
{\large \Square} &
{\large \Square} &
\begin{tabular}[l]{m{3cm}}
Exploits IPv6's any-cast mechanism.
\end{tabular} &
\begin{tabular}[l]{m{5cm}}
+ Relieved from the bootstrapping process.\\
+ Easy to implement (just the additional neighborhood table).
\end{tabular}
\\
\hline
%%%%%%%%%%%%%%%%%%%%%%%%%%%%%%%%%%%%%%%%%%%%%%%%%%%%%%%%%%%%%%%%%%%%%%%%%%%%%%%%
\textbf{Chord6} &
{\large \Square} &
{\large \Square} &
{\large \CheckedBox} &
{\large \Square} &
{\large \Square} &
{\large \Square} &
\begin{tabular}[l]{m{3cm}}
Exploits IPv6's hierarchical features.
\end{tabular} &
\begin{tabular}[l]{m{5cm}}
+ Reduces inter-domain traffic between ISPs.\\
+ Easily portable to other DHTs.\\
-- Cannot distinguish between nearby domains to identify the best
next hop when an inter-domain step must be taken.
\end{tabular}
\\
\hline
%%%%%%%%%%%%%%%%%%%%%%%%%%%%%%%%%%%%%%%%%%%%%%%%%%%%%%%%%%%%%%%%%%%%%%%%%%%%%%%%
\textbf{DHT-PNS} &
{\large \Square} &
{\large \Square} &
{\large \Square} &
{\large \CheckedBox} &
{\large \Square} &
{\large \Square} &
\begin{tabular}[l]{m{3cm}}
Grouping through synthetic coordinates.\\
Partitioning using a concentric cycle clustering scheme.
\end{tabular} &
\begin{tabular}[l]{m{5cm}}
-- It assumes uniform node distribution which is not always the case in a synthetic coordinate system.\\
-- Clustering nodes might be a single point of failure.
\end{tabular}
\\
\hline
%%%%%%%%%%%%%%%%%%%%%%%%%%%%%%%%%%%%%%%%%%%%%%%%%%%%%%%%%%%%%%%%%%%%%%%%%%%%%%%%
\textbf{Quasi-Chord} &
{\large \Square} &
{\large \CheckedBox} &
{\large \Square} &
{\large \CheckedBox} &
{\large \Square} &
{\large \Square} &
\begin{tabular}[l]{m{3cm}}
Geometric space coordinates (GNP)\\
Transformation of the coordinate space into 1-d Cantor space for easier mapping to the Chord hierarchy\\
Two finger tables (clockwise, anti-clockwise)
\end{tabular} &
\begin{tabular}[l]{m{5cm}}
-- Not fully distributed (GNP is landmark based).\\
-- Makes an indirect assumption of a maximum number of allowable hosts.\\
-- Doubles the required routing information which needs to be created and maintained.
\end{tabular}
\\
\hline
%%%%%%%%%%%%%%%%%%%%%%%%%%%%%%%%%%%%%%%%%%%%%%%%%%%%%%%%%%%%%%%%%%%%%%%%%%%%%%%%
\textbf{IPBC} &
{\large \Square} &
{\large \Square} &
{\large \Square} &
{\large \CheckedBox} &
{\large \Square} &
{\large \Square} &
\begin{tabular}[l]{m{3cm}}
IP address prefixes (16 bit for IPv4) to detect proximity.
\end{tabular} &
\begin{tabular}[l]{m{5cm}}
+ Prefix is stored in the DHT so the proximity identification becomes as easy as to query the prefix.\\
+ DHT maintenance mechanisms for both voluntary or ungraceful departures of
peers.\\
-- Performance/accuracy trade-off in choosing the prefix length to be used.
\end{tabular}
\\
\hline
%%%%%%%%%%%%%%%%%%%%%%%%%%%%%%%%%%%%%%%%%%%%%%%%%%%%%%%%%%%%%%%%%%%%%%%%%%%%%%%%
\textbf{Cone} &
{\large \Square} &
{\large \CheckedBox} &
{\large \Square} &
{\large \CheckedBox} &
{\large \Square} &
{\large \Square} &
\begin{tabular}[l]{m{3cm}}
2-layer ID space.\\
Group concept for dividing nodes according to a common Chord id prefix.
\end{tabular} &
\begin{tabular}[l]{m{5cm}}
+ The first step of the routing scheme exploits proximity thus reducing message exchange costs.\\
-- Not fully distributed as it relies on landmarks.\\
-- Maintenance of three routing tables is needed.
\end{tabular}
\\
\hline
%%%%%%%%%%%%%%%%%%%%%%%%%%%%%%%%%%%%%%%%%%%%%%%%%%%%%%%%%%%%%%%%%%%%%%%%%%%%%%%%
\textbf{DynaMo} &
{\large \Square} &
{\large \CheckedBox} &
{\large \Square} &
{\large \CheckedBox} &
{\large \Square} &
{\large \Square} &
\begin{tabular}[l]{m{3cm}}
Random Landmarking (RLM).\\
Closest Neighbor Prefix Assignment (CPNA).\\
Making last hop as local as possible.
\end{tabular} &
\begin{tabular}[l]{m{5cm}}
+ Developed with mobile, ad-hoc networks in mind.\\
+ Evenly distributed IDs.\\
+ Dynamic landmarking is a fully distributed approach.\\
-- RLM imposes extra network stress especially to landmark nodes.\\
-- CPNA is a coarse-grained proximity approach.
\end{tabular}
\\
\hline
%%%%%%%%%%%%%%%%%%%%%%%%%%%%%%%%%%%%%%%%%%%%%%%%%%%%%%%%%%%%%%%%%%%%%%%%%%%%%%%%
\textbf{SAT-Match} &
{\large \Square} &
{\large \Square} &
{\large \Square} &
{\large \CheckedBox} &
{\large \Square} &
{\large \Square} &
\begin{tabular}[l]{m{3cm}}
Selective jumps to adjust peer positioning in the DHT.\\
Stretch reduction scheme.
\end{tabular} &
\begin{tabular}[l]{m{5cm}}
+ Continuously adaptive mechanism.\\
+ It can be considered lightweight.\\
+ Can coexist with other approaches (like landmarking).\\
+ Works sufficiently for large scopes as well as environments with high churn.\\
+ Compared to Mithos, scales much better.\\
-- Contention situation in selective jump phase.\\
-- Probing phase incurs unnecessary traffic.
\end{tabular}
\\
\hline

%%%%%%%%%%%%%%%%%%%%%%%%%%%%%%%%%%%%%%%%%%%%%%%%%%%%%%%%%%%%%%%%%%%%%%%%%%%%%%%%

% \hline
% \textbf{Global Softstate} &
% \textbf{Geographic layout} Uses first landmark clustering
% then measures RTTs to identify close nodes &  &  \\

% \hline
% \textbf{Mithos} &
% \textbf{Geographic layout} Uses nodes as topology landmarks and directed
% incremental probing to optimize topology & & Scales well as all
% operations are local ??? \\
% 
% \hline
% \textbf{Self-Adaptive Topology Matching} &
% \textbf{Proximity neighbor selection} Uses lightweight probing and
% selective jumps to optimize the topology & CAN &  Better than Mithos \\

% \hline
% \textbf{Delay Aware P2P System} & \textbf{} & & \\

% \hline
% \textbf{VERSION OF CHORD - DHT-PNS} &
% \textbf{Proximity neighbor selection} Uses Proximity Neighbor Selection and
the Vivaldi
% system & Chord  &  \\

% \hline
% \textbf{MAY OMIT- VERSION OF CHORD - Quasi-Chord} &
% \textbf{Proximity based}  & Chord  &  \\

% \hline
% \textbf{LAPTOP} &
% \textbf{Geographic layout} Hierarchical overlay structure  &  & routing path length $\log{_d N}$,
% join/leave overhead $d\log{_d N}$ \\

% \hline
% \textbf{IP-Based Clustering} &
% \textbf{Proximity neighbor selection} Proximity neighbor selection based on
longest common
% prefix of IP addresses &    &  \\

% \hline
% \textbf{CHOord considering Proximity on IPv6} &
% \textbf{Proximity routing} Uses IPv6 address format to provide proximity &  Chord
%  & Better than Chord \\

% \hline
% \textbf{Proximity in Kademlia} &
% \textbf{Proximity routing} Applies  proximity neighbor selection (PNS) and
proximity route selection (PRS)
% to Kademlia & Kademlia &   \\

% \hline
% \textbf{Cone} &
% \textbf{Proximity neighbor selection} Uses proximity neighbor selection (PNS)
& Chord  & Better than Chord \\

% \hline
% \textbf{DynaMO} &  &  &  \\
% 
% \hline
% \textbf{MAY OMIT-BADLY WRITTEN-PChord} &
% \textbf{Proximity based}  &  Chord  & \\
% 
% \hline
% \textbf{AChord} &  & & \\

% \hline
% \textbf{Chord6} &
% \textbf{Proximity routing} Uses IPv6 hierarchical address format to cluster
% topologically close nodes & Chord  &  \\

% \hline
\end{longtable}
\end{center}
\vspace{-2.5ex}
\vspace{-2.5ex}
%\label{fig:struct_compare_table}
%\end{figure}
 \end{landscape}
