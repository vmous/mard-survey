% TODO: Should be reviewed
\section{Conclusion}
\label{section:conclusion}

Research and development on application systems that follow the 
\p\ architectural paradigm, have flourished over the last decade.
By and large, such systems exploit the loose-coupling and self-organization 
of participating nodes to shape substrates upon which diverse applications 
can run; the latter offer services including, but not limited to, 
distributed naming, 
Internet telephony and video conferencing, 
storage and indexing, 
content and file sharing, 
web crawling and caching, 
event notification and subscribing/publishing.
These services are provided atop the 
best-effort infrastructure of today's Internet;
services have to comply with an array of key requirements and 
offer quality attributes that include node availability, robustness,
scalability, load-sharing, quality-of-service and user anonymity. 
%%AD rephrased below as above - too long sentence it was!
% and they have to effectively deal with an array of key quality attributes that
% they may need or have to offer node availability, robustness,
% scalability, load-sharing, quality-of-service and user anonymity.
%%
The difficulty to handle such characteristics effectively, 
emanates from the fact that
peers  operate at the edge of the Internet and no 
adequate attention is paid to the structure of the physical network 
during application-level network design. 
The deviation of the logical distributed overlay formation from the 
physical network yields suboptimal use of the underlying 
infrastructure and is known as the topology mismatch
problem.
%%AD rephrased what is below as above..
% The deviation of the structure of the overlay network from the
% optimal, according to the underlying physical structure, constitutes what is
% termed topology mismatch and the problem is known as the topology mismatch
% problem.

Development of \p--systems and applications
typically involve juggling a set of tradeoffs (e.g.,
choosing higher accuracy at the cost of increased system overhead).
% Computer systems design is typically about juggling a set of tradeoffs (e.g.,
% choosing higher accuracy at the cost of increased system overhead.)  
In surveying more than
a decade's worth of research efforts aimed at solving the topology mismatch
problem in both unstructured and structured \p\ networks, we find this
tussle-of-tradeoffs property to hold.  
With regards to the three criteria we used --efficiency,
overhead, and scalability-- we find that none of the proposed solutions 
is superior to the others on all three fronts.  This is not surprising.
Nonetheless, by
\begin{inparaenum}[\itshape i\upshape)]
  \item presenting an analysis of each of the solutions,
including what distinguishes it from the
others and its pros and cons and
  \item offering a pictorial
comparison of how each solution fares in regards to others as far as
efficiency, overhead, and scalability are concerned,
\end{inparaenum}
we hope to have provided \p\
developers and researchers with enough insight and perspective;
as they face specific problems, they will be readily able to 
draw the best possible design decisions for their own application
environments.
%%AD sentence was too long - this "giant" thing to much ass-kissing... no.
% when facing a specific problem, they would 
% when they face a specific problem, th 
% so that depending
% on the particular problem in hand, they are able to make a
% correct choice amongst these surveyed here or, by firmly standing on
% the shoulders of giants, propose a new solution of their own.
