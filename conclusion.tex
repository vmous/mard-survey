% TODO: Should be reviewed
\section{Conclusion}
\label{section:conclusion}

Research and development for computer systems
built using the \p\ paradigm have flourished  in the last decade.
By and large, such systems exploit the loose-coupling and self-organization 
of participating nodes to shape substrates upon which diverse applications 
can run; the latter offer services including but not limited to 
distributed naming, 
Internet telephony and video conferencing, 
storage and indexing, 
content and file sharing, 
web crawling and caching, 
event notification and subscribe/publish.
These services are provided atop the 
best-effort infrastructure of today's Internet 
and they have to effectively deal with an array of key issues
emanating from the fact that \p\ nodes 
operate at the edge of the Internet; such critical issues include
node availability, robustness, scalability, load-sharing,
quality-of-service and user anonymity.
The majority of such issues emerges as 
application-level \p--networks are formed without paying 
adequate attention to the design of the physical network beneath.
The latter constitutes what is known as the
topology mismatch problem between the \p-overlays and the physical networks.
In addition, the development of \p--systems and applications
typically involved juggling a set of tradeoffs (e.g.,
choosing higher accuracy at the cost of increased system overhead.)

% Computer systems design is typically about juggling a set of tradeoffs (e.g.,
% choosing higher accuracy at the cost of increased system overhead.)  
In surveying more than
a decade's worth of research efforts aimed at solving the topology mismatch
problem in both unstructured and structured \p\ networks, we find this
tussle-of-tradeoffs property to hold.  
With regards to the three criteria we used --efficiency,
overhead, and scalability-- we find that none of the proposed solutions 
is superior to the others on all three fronts.  This is not surprising.
Nonetheless, by
1)~presenting an analysis of each of the solutions, 
including what distinguishes it from the
others and its advantages and disadvantages and 
2)~offering a pictorial
comparison of how each solution fares compared with the others with regards
to efficiency, overhead, and scalability, we hope to have provided \p\
researchers and developers with enough perspective so that depending 
on the needs of their particular application, they are able to select
amongst these surveyed solutions or propose a new solution of their own.
