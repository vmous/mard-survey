% TODO: Should be reviewed
\section{Conclusion}
\label{section:conclusion}

Computer systems design is typically about juggling a set of tradeoffs (e.g.,
choosing higher accuracy at the cost of increased system overhead.)  
In surveying more than
a decade's worth of research efforts aimed at solving the topology mismatch
problem in both unstructured and structured P2P networks, we find this
tussle-of-tradeoffs property to hold.  With regards to the three criteria 
we used (efficiency,
overhead, and scalability), we find that none of the proposed solutions 
is superior to the others on all three fronts.  This is not surprising.
Nonetheless, by presenting 1) an
analysis of each of the solutions, including what distinguishes it from the
others and its advantages and disadvantages and 
2) a pictorial
comparison of how each solution fares compared with the others with regards
to efficiency, overhead, and scalability, we hope to have provided P2P
developers and researchers 
with enough perspective so that depending on the needs of their particular
application, they are able to select
amongst these surveyed solutions or propose a new solution of their own.


%Peer-to-peer architecture has been in the center of research attention in the
%advantages of loose coupling and self organization of computing nodes to form
%application-layer networks on top of the physical, best-effort infrastructure of
%the Internet that exhibit interesting properties. Scalability problems arouse
%quickly, though, because of the inefficient construction of this overlay that
%was built with no concern for the underlying physical network that causes a
%great deal of redundant traffic. The problem was identified by the research
%community as the topology mismatch problem between the overlay and the
%corresponding underlying physical network and a great deal of effort has been
%set towards alleviating it. Some fruits of this effort have been gathered and
%presented in this survey. Measurement of link cost through latency or RTT and
%deletion of inefficient established connections are the key concepts of almost
%all approaches. Others, manage to address the problem through hierarchical peer
%clustering (e.g. best IP matching). Additionally, some protocols focus on
%specific problems that furtherly arise, such as overlay partitioning, search
%scope reduction or convergence speed, to name just a few. Another desire is to
%form a protocol that could be applied to both decentralized unstructured and
%decentralized structured networks. Unfortunately no approach has equally
%addressed these problems in order to form a robust solution. So the field seems
%to be, still, fertile for any, new, clever idea.
