\section{Conclusion}
\label{section:conclusion}

%% Removing after Mema's suggestion
%Research and development in applications and systems that follow the 
%\p\ architectural paradigm have flourished for more than a decade now.
%By and large, such artifacts exploit the loose-coupling and self-organization 
%of participating nodes to shape substrates upon which diverse applications 
%can run; the latter offer services including, but not limited to, 
%distributed naming, 
%Internet telephony and video conferencing, 
%storage and indexing, 
%content and file sharing, 
%web crawling and caching, 
%event notification and subscribing/publishing.
%%%
%These services are provided atop the 
%best-effort infrastructure of today's Internet 
%and they have to effectively deal with an array of key quality attributes that
%they may have to offer, such as node availability, robustness,
%scalability, load-sharing, quality-of-service and user anonymity.
%%%
%The difficulty to handle such features effectively, emanates from the fact that
%peer nodes operate at the edge of the Internet and no adequate attention 
%is paid to the structure of the physical network during the 
%network formation at the application-level.
%The deviation of the distributed logically-formed overlay from the 
%physical network yields suboptimal use of the underlying 
%infrastructure and is known as the topology mismatch
%problem.
%
Development of \p--systems and applications
typically involve juggling a set of tradeoffs (e.g.,
choosing higher accuracy at the cost of increased system overhead).
% Computer systems design is typically about juggling a set of tradeoffs (e.g.,
% choosing higher accuracy at the cost of increased system overhead.)  
In surveying more than
a decade's worth of research efforts aimed at solving the topology mismatch
problem in both unstructured and structured \p\ networks, we find this
tussle-of-tradeoffs property to hold.  
With regards to the three criteria we used --efficiency,
overhead, and scalability-- we find that none of the proposed solutions 
is superior to the others on all three fronts.  This is not surprising.
Nonetheless, by
\begin{inparaenum}[\itshape i\upshape)]
  \item presenting an analysis of each of the solutions,
including what distinguishes it from 
others as well as its advantages and disadvantages and 
  \item offering a pictorial
comparison of how each solution fares with regards to others 
as far as
efficiency, overhead, and scalability are concerned,
\end{inparaenum}
we hope to provide \p\
developers and researchers with enough insight and perspective;
as they face specific problems, they will be readily able to 
draw the best possible design decisions for their applications.
