%\documentclass[a4paper,10pt]{article}
\documentclass[acmcsur,acmnow]{acmtrans2m}
%\documentclass[acmtocl,acmnow]{acmtrans2m}

%\acmVolume{V}
%\acmNumber{N}
%\acmYear{YY}
%\acmMonth{Month}

% packages
\usepackage[utf-8]{inputenc}
\usepackage{paralist}
\usepackage{subfigure}
\usepackage{graphicx}
\usepackage{colortbl}
\usepackage{color,soul}
\usepackage{longtable}
\usepackage{pdflscape}

% definitions
\newtheorem{theorem}{Theorem}[section]
\newtheorem{conjecture}[theorem]{Conjecture}
\newtheorem{corollary}[theorem]{Corollary}
\newtheorem{proposition}[theorem]{Proposition}
\newtheorem{lemma}[theorem]{Lemma}
\newdef{definition}[theorem]{Definition}
\newdef{remark}[theorem]{Remark}

%\newtheorem{theorem}{Theorem}[section]
%\newtheorem{definition}[theorem]{Definition}

%\textwidth      6.5in
%\textheight     9.2in
%\oddsidemargin  -0.5cm
%\evensidemargin -0.5cm
%\parindent      0pt              
%\topmargin      -0.3in
%\headsep        20pt
%
%\parskip        5pt
%\parindent      10pt

\markboth{}{}{}

\title{Alleviating the Topology Mismatch Problem in Unstructured and Structured
Overlay Networks: A Survey}

\author{
VASSILIS S. MOUSTAKAS\\\emph{Department of Informatics and Telecommunications}\\\emph{National and Kapodistrian University of Athens}\\
H\"USEYIN AKCAN\\\emph{Department of Software Engineering}\\\emph{Izmir University of Economics}\\
MEMA ROUSSOPOULOS\\\emph{Department of Informatics and Telecommunications}\\\emph{National and Kapodistrian University of Athens}\\
AND\\
ALEX DELIS\\\emph{Department of Informatics and Telecommunications}\\\emph{National and Kapodistrian University of Athens}\\
%  Vassilis Moustakas$^1$, H\"useyin Akcan$^2$, Mema Roussopoulou$^1$ and Alex
%Delis$^1$\\
%  $^1$Department of Informatics and Telecommunications,\\
%  National and Kapodistrian University of Athens\\
%  \mbox{\texttt{\{b.moystakas, mema, ad\}@di.uoa.gr}}\\
%  $^2$Department of Software Engineering,\\
%  Izmir University of Economics, Izmir, Turkey \\
%  \mbox{\texttt{huseyin.akcan@ieu.edu.tr}}
}

\begin{abstract}
Peer-to-peer (P2P) systems is a rapidly growing application architecture. The
overlay network abstraction on top of a best-effort infrastructure provides a
powerful tool to realize a plethora of ever-wanted exotic features like
anonymity, high availability and robustness, load balancing, quality-of-service
(QoS) at the application layer and many more. Unfortunately weaknesses of early
effectuations constrained applications from unleashing the full potential of the
paradigm. One of them is the topology mismatch problem between the overlay and
the physical underlying network, a problem that imposes a huge amount of
unnecessary stress to the network resources and, thus, impede applications from
scaling. In this survey we investigate on the recent research developments
towards the alleviation of this problem, in both unstructured and structured
overlay architectures. We visit the various algorithms available in the
lietrature, focusing on the presentation of how each of them tries to tackle the
problem. Additionally, we provide a finer grained categorisation of the
approaches based on the generic type of methodology they use in order to
accomplish this goal.
% TODO: Review this after the comparison sections are fixed.
Finally we discuss the advantages, disadvantages and novelties introduced by the
algotithms and offer pictorial comparisons for them in order to further
strengthen and organise understanding of the information and knowledge cited, in
an attempt to make this survey a one-stop-shop reference for anyone interested
in the topic, from researchers and students, to system architects and
developers.
\end{abstract}

\category{}{}{}

\terms{Topology mismatch, distributed, peer-to-peer, P2P, overlay}

\keywords{Topology mismatch problem, distributed algorithm, peer-to-peer
architecture, overlay network }

\begin{document}

\begin{bottomstuff}
SAY SOMETHING COOL HERE...
\end{bottomstuff}

\maketitle

\section{Introduction}
\label{section:intro}

Peer-to-peer (\p) networks are self-organizing, distributed systems where
participating nodes, called \emph{peers}, act as both resource providers and
resource consumers in contrast to the conventional \emph{client-server} model
where nodes undertake specific roles.
For over a decade, \p\ networks have been widely deployed and have
enjoyed immense popularity from Internet communities, primarily because
of the great number of features they offer to distributed applications 
built atop them. 
Such diverse features include:  high availability and robustness,
load-balancing, quality of service, scalability, decentralized administration,
and anonymity. 

The peer-to-peer paradigm gave impetus to two ``killer'' applications:
file-sharing and Internet telephony.
The {\sl Napster} file-sharing system was widely acknowledged as the 
``fastest growing Internet application ever'' in $2001$ when it topped
$26$ million users sharing over $80$ million songs.
Other file-sharing applications followed suit, 
including {\sl Gnutella} and {\sl Limewire} enjoying $3$ 
million concurrently-connected peers, as well as 
{\sl BitTorrent} connecting over $150$ million monthly users by January $2012$~\cite{bittorrentusers}.
\p\ telephony saw explosive growth with the advent of {\sl Skype}.
Since its introduction in $2003$,
{\sl Skype} has become extremely popular with more than $650$ million users 
in $2011$~\cite{skypetotalusers} and an astonishing 
$50$+ million concurrently-online users~\cite{skypesymusers}.
Moreover, it has consistently eroded  the traffic handled by 
traditional telephony carriers by slicing away a staggering 
$214$ billion minutes of pertinent traffic in $2013$ alone~\cite{skypetraffic}.

%%
Apart from the above, a startling number of diverse and successful
applications have been built based on \p\ architectures. Some of them include:
distributed search engines~\cite{yaci}, 
distributed data-storage systems~\cite{kbc_oceanstore_2000,bdet_fsdfs_2000,dkkms_cfs_2001,dr_pastutility_2001,abc_farsite_2002,mmfc_ivy_2002,arla,agebh_dks_2003},
Web caches, archives and publishing systems~\cite{ird_squirrel_2002,bags_youserv_2002,wrc_publius_2000,wm_tangler_2001},
messaging and dissemination applications~\cite{threedegrees,icpp08-pd}, 
event-notification infrastructures~\cite{rkcd_scribe_2001,cdkr_scribe_2002,agebh_dks_2003}, 
naming services~\cite{cmm_chorddns_2002}, 
censor-resistant stores~\cite{cswh_freenet_2001} and
lately, even cloud-based platforms~\cite{mgpj_cloudsnap_2011}.

\p\ systems are implemented using \emph{overlay networks}. 
An overlay network, is a virtual system of nodes featuring logical interconnects (or links)
created above an existing network; overlays provide an abstraction that
enables the implementation of efficient, fully distributed, application-layer
services such as routing messages to destinations that are not known in advance
or offering QoS guarantees (i.e., in content-distribution) over best-effort
infrastructures. Overlay nodes communicate through \emph{virtual connections}
each of which may correspond to a path of possibly many physical links 
in the underlying network.
Figure~\ref{figure:overlay} illustrates a simple four-node overlay constructed
over a wide-area network.
%%
\begin{figure}[ht]
\centering
  \includegraphics[scale=0.45]{img/pdf/under-over-lay.pdf}
\caption{An example overlay network.}
\label{figure:overlay}
\end{figure}

The single key issue that determines the efficiency of an overlay network,
is how well the overlay maps to the underlying network topology on which it
``rests''. 
Consider two nodes\footnote{In \p\ networks,
the participating nodes are typically user-PCs operating at the edge of the
Internet.} that are connected with each other via a path of overlay links.
If the application running on the nodes, generates heavy traffic along
the overlay path, it would be beneficial to
construct the overlay topology in a way that the number of 
underlying IP links between these two nodes is minimized.
Should the overlay network be constructed so that
it does not match the underlying topology well, 
the inherent \emph{topology mismatch} creates two
major problems. 
First, the performance of the application per se, can be
adversely affected since traffic must flow over a larger,
redundant, number of physical hops resulting in poor user experience 
entailing noticeable latencies or jitter. 
%%
Second, other applications running
on the underlying network infrastructure may be adversely affected as well.
Studies have shown that highly popular \p\ applications contribute 
the largest portion of the overall 
Internet traffic~\cite{seroiu_analysiscds_2002,sen_analyzep2ptraffic_2004,krp_ispfear_2005}, with some reporting that more than $60$\% of this traffic 
to be \p-related~\cite{cachelogic,ipoque2009};
it was also projected that this traffic would reach
$7$~\emph{Exabytes}-per-month by $2018$~\cite{CVNI2014}! 
This constitutes a major burden for 
Internet Service Providers (ISPs) who must route 
all of this traffic to destinations at the edge of the Internet. 
If the \p\ overlay topology is poorly designed, 
the demand on the Internet's backbone infrastructure may 
substantially increase as traffic might have to flow
``back and forth'' several times between two neighboring ISPs
while trying to travel from the source to its destination node
in the overlay.
Hence, it is critical that \p\ networks be laid out 
in ways that their topology matches the underlying IP topology as 
closely as possible.

For over a decade, researchers have extensively investigated  
various aspects of the topology mismatch problem.
This research area started to become active after  The mismatch problem is Interest around All interesting work around the topology We omitted centralized Centralized systems are not 

The main objective of this paper is to offer a comprehensive 
survey of the work carried out in the area
and provide a
taxonomy of the proposed solutions. We point out synergies, as well as
similarities and differences in the published approaches. 
Ultimately, our goal is to help readers sift through 
the voluminous literature, to help them
understand the advantages and disadvantages of each work, and 
to provide them with enough perspective so that 
when the need arises, they are able to
select, amongst the different approaches, the one that is most suitable for
their particular application.
%%
The rest of the paper is organized as follows: 
In Section~\ref{section:background}, we provide background on
overlay architectures including centralized, decentralized-unstructured and
decentralized-structured \p\ systems.
We also formally define the problem of topology mismatch 
and offer the rationale behind our 
evaluation of the techniques discussed in this paper. 
Work in the area was inspired mostly by the volatile and fully
distributed nature of the decentrilized overlays so we skip
centralized systems and in Sections~\ref{section:unstructured} and~\ref{section:structured}
we outline the surveyed research efforts
for unstructured and structured \p\ overlays, respectively.
We conclude in Section~\ref{section:conclusion}.


%%%%%%%%%%%%%%%%%%%%%%%%%%%%%%%%%%%%%%%%%%%%%%%%%%%%%%%%%%%%%%%%%%%%%%%%%%%%%%%
% start
%TODO: Maybe this section should be reviewed or deleted or go to another section
%Studying the behaviour of the various peer-to-peer schemes
%\cite{matei_mapgnutella_2002, lv_randomwalks_2002, merugu_str2unstr_2003}
%showed that the ad-hoc network topology of unstructured overlay networks that
%preserve \emph{Power Law} and \emph{Small World} characteristics\footnote{Power
%Law describes the node degree while the Small World describes characteristics
%of path length and clustering coefficient. The clustering coefficient for a
%node $\upsilon$ in a graph $G = \left( V, E \right)$ is defined as the ratio of
%the existing connections between $\upsilon$'s neighbouring nodes to $\gamma
%\times \left( \gamma - 1 \right)$, where $\gamma$ is the number of neighbouring
%nodes of $\upsilon$. High cluster coefficient means that neighbouring nodes of
%any node $\upsilon$ likely connect one another.} \cite{faloutsos_powerlaw_1999,
%saroiu_measurefileshare_2002} offer a more promising approach. Particularly:
%\begin{itemize}
%  \item Peer-to-peer clients are extremely \emph{transient}. Unstructured
%systems can have high maintenance traffic in delivering messages, updating the
%mapping, discovering failures and replicating lost data or pointers, making
%them insufficient on highly volatile networks.
%  \item \emph{Keyword searches} versus \emph{exact-match queries}. In DHTs
%there is a tight control between the data placement and the topology of the
%network. For this reason it is hard to efficiently support partially matched
%queries while Gnutella and other similar systems effortlessly support keyword
%searches and other complex queries since the mechanism is realized locally, on
%a node-by-node basis.
%  \item Popular content is located at multiple peers and thus it is more likely
%for a flooding-based search to return results. DHTs, on the other hand, fit
%better in the systems which require ability to reliably locate content, even in
%the extreme case that only a single-copy exists in the network.
%\end{itemize}
% end
%%%%%%%%%%%%%%%%%%%%%%%%%%%%%%%%%%%%%%%%%%%%%%%%%%%%%%%%%%%%%%%%%%%%%%%%%%%%%%%

% TODO: how 
%For this reasons, efforts have been placed for optimizing the efficiency of
%decentralized unstructured peer-to-peer networks. Research mainly focuses on
%\begin{inparaenum}[\itshape i\upshape)]
%  \item reducing unnecessary, redundant communication traffic, and
%  \item exploiting physical locality to reduce communication response.
%\end{inparaenum}
%The goal can be achieved at, both, the application-level network as well as the
%underlying physical one. In the first case by refining the message relay
%techniques, while in the second one, by adaptively reconstructing the
%application network to map as well as possible to the the physical network.

%%%%%%%%%%%%%%%%%%%%%%%%%%%%%%%%%%%%%%%%%%%%%%%%%%%%%%%%%%%%%%%%%%%%%%%%%%%%%%%
%\section{The Topology Mismatch Problem}

%%%%%%%
%%%%%%% TODO: NEEDS REWORDING
%%%%%%% Additional Material
%%%%%%%
%%%%%%%
%%%%%%%
% P4P [16],instead, aimed to provide a general framework for P2P
%applications by developing ISP-application interfaces so that
%application software can use the network layer information,
%such as congestion status and network topology, for better
%performance and resource utilization.

%P4P [37] project attempts to address
%the problem through custom trackers, both for ISPs and P2P
%systems, using an interface based on a primal-dual decomposition
%of an optimization problem. This interface design simplifies
%the realization of traffic-engineering objectives from each parties’
%perspective and ensures the extensibility of the approach. Through
%simulation-based studies and limited experimental deployments,
%these collaborative approaches have been shown to effectively
%reduce network costs while minimally impacting application performance.
%A clear advantage of these proposals is that they
%allow ISPs to incorporate aggregated traffic policies in their tracker
%recommendations (e.g., a particular traffic balance ratio between
%peering providers). However, all of them require deployments
%of oracles for each participating ISP and their effectiveness is
%ultimately predicated on their adoption by P2P applications and a
%trust relationship between P2P users and their ISPs.

% TODO: Taxomize the approaches that have been proposed in the literature
% for unstructured and structured overlay networks
% UNDER CONSTRUCTION
% VERY IMPORTANT SECTION
%\section{Taxonomy}
%
%% TODO: Unstructured
%% UNDER CONSTRUCTION
%Several overlay protocols for dealing with the problems derived from the
%topology mismatch problem have been proposed in the literature. Each
%implementation has different goals but all share some common ground on the
%aspect of the network functionality they focus on, in order to achieve them. In
%this section, the general taxonomy used throughout this paper to categorize the
%algorithms based on their functionality is presented.  The brief description of
%each category and sub-category is stated below.
%
%{\sethlcolor{yellow}\hl{CITE this paper {\cite{liu_ltm_2004}}}}
%
%\subsection{Forwarding based}
%
%Forwarding based methodology is generally adapted by distributed unstructured
%overlay systems. In this approach, a peer selects only a subset of its
%neighbours to re-broadcast query messages. The selection is made using one or a
%combination of various statistical metrics. Examples of such metrics are the
%number of responses received by a neighbour or the connection latency of the
%link between the nodes, etc. The forwarding based approach enhances search
%efficiency but has also several drawbacks. First, the search scope is reduced
%drastically.  Expanding the search scope, on the other hand, is no easy task
%because the overhead of forming multicast trees is proportional to the multicast
%group size.  Second, forwarding based schemes do not consider dynamic joining
%and leaving of peers so they do not scale well on dynamic environments.
%
%\subsection{cache based} 
%
%Caching based protocols are effectively used to reduce traffic costs and
%response times. The caching policy varies depending on the way protocol handles
%the index and the content. Centralized P2P systems use central index servers,
% while local caching systems, such as KazaA, use super peers
%to cache indices in a distributed way. Content caching is also possible in P2P
%systems, where nodes cache the forwarded content for further retrievals.
%Although caching has the above mentioned advantages,  duplication
%of messages still exist, which limits the scalability of these approaches.
%Therefore, cache based approaches are analyzed in the following categories:
%  \begin{itemize}
%    \item \emph{data index caching},
%    \item \emph{content index caching},
%    \item \emph{centralized}, and
%    \item \emph{local}.
%  \end{itemize}
%
%\subsection{overlay optimization based} 
%
%The overlay optimization based protocols modify the topology of the P2P network
%using various techniques. These approaches include creating spanning trees using 
%connection graphs, creating cluster of physically close nodes, or using latency
%information to detect proximity. The brief description of each category is
%presented below:
%
%  \begin{itemize}
%    \item \emph{Spanning tree based}. These approaches construct of a rich graph
%    based on the network connections and build minimum spanning tree on the
%    graphs, causing large traffic overhead to the system
%    \cite{chu_esm_2000,chu_esm_2002}.
%
%    \item \emph{Cluster based}. These approaches select to link physically
%    closer nodes with each other, therefore shrink the search scope
%    significantly while mapping accuracy is not always guaranteed.
%
%    \item \emph{Minimum latency first}. Use of latency as a metric to calculate
%    distance among peers. They require global latency information
%    ``landmarks''\footnote{Measuring latency between peers and stable Internet
%    servers.}.
%
%  \end{itemize}
%
%
%% TODO: Structured
%% UNDER CONSTRUCTION
%\subsection{proximity based}
%Xu \textit{et al.} \cite{xu_globstate_2003} state that there are three ways generating proximity information
%\begin{itemize}
% \item Expanding Ring Search\\
%Expanding ring search can be of two forms. First it can utilize the multicast
%infrastructure in the underlying network in order to emit its messages.
%Unfortunately such infrastructures are not widely deployed thus the
%implementations of this way of generating proximity information is limited to
%blindly flooding the neighbourhood to obtain reasonable results.
%
% \item Heuristics\\
%Heuristics are used in order to reduce the blindness of the expanding ring
%search and make realizations more efficient and effective. Unfortunately a
%common problem of all heuristic approaches is the local minimum pitfall in which
%the search might be caught into.
%
% \item Landmark Binning\\
%Landmark clustering is based on the view that nodes with similar distances to a
%set of predefined well-known landmark nodes are pretty likely of being close to
%each other. But this approach has its weaknesses as well, such as the fact that
%is a rather coarse grained approximation, therefore not particularly well suited
%for differentiating nodes within close distance to each other.
%
%\end{itemize}
%
%Castro \textit{et al.} \cite{castro_proximitydht_2002,castro_topawareroute_2002} on
%the other hand, comment on the three basic approaches in exploiting proximity
%in DHT protocols suggested by Ratnasamy \textit{et al.} \cite{ratnasamy_openq_2002}.
%These are:
%
%\begin{itemize}
% \item Geographic Layout\\
%The node IDs are assigned in such a way that nodes close by in the physical
%network topology, be close in the node ID space as well. Implementations that
%work relatively well with this approach have been incorporated into CAN. Nodes
%measure the RTT between themselves and a set of landmarks in order to match the
%CAN space as much as possible to the physical one. Unfortunately, the approach
%requires well known landmark servers and for that matter is  not fully
%self-organizing which can further lead to imbalanced node distribution. On
%other DHTs, such as Chord or Pastry, another problem emerges.  To gain fault
%tolerant properties, these protocols, replicate key-value pairs on neighbouring
%(in the ID space) nodes. When a proximity-based node ID assignment has been
%used, the needed failure resilience is undermined by the fact that close by
%nodes are more likely to suffer collective failures.
%
% \item Proximity Routing\\
%Proximity routing does not require routing tables be built using any knowledge
%about network proximity. On the other hand it exploits such knowledge in order
%to choose the best next hop during routing a message. This approach 
%balances between choosing the node that will further progress the routing
%towards the destination and choosing the closest entry in the routing table, in
%terms of network proximity. Thus, it is relatively less effective than
%geographical layout when applied to CAN(-like) implementations. Moreover, the
%technique has been incorporated into a version of Chord (TODO: PChord |
%additionally check whether this is true) causing an increase on the overhead of
%node joins and the size as well as maintenance cost of finger tables.
%\cite{dabek_cfs_2001} proposes a server selection scheme for the Chord DHT, on
%the domain of proximity routing selection. In \emph{CFS}, each node predicts
%the entire lookup latency as a function of the total number of nodes and the
%average overlay next routing peer. The problem is that it is very difficult to
%have a clear picture on the total number of nodes and the average hop latency
%from the local. This leads to rough estimations that consequently decreases
%overall performance.
%
% \item Proximity Neighbour Selection\\
%Finally, the third approach, constructs the routing tables using proximity
%knowledge. Tapestry and Pastry's mechanisms of routing table
%maintenance try to minimize the distance to nodes appearing in a peer's routing
%table. Since routing is based on longest node ID prefix match, messages 
%are gradually forwarded to nearby nodes at each routing step.
%\cite{castro_proximityp2p_2002} argues on how Pastry exploits proximity
%neighbour selection in order to create a scheme that is (more) location-aware
%compared to the other well-known DHTs (CAN, Chord).
%
%\end{itemize}
%
%{\sethlcolor{yellow}\hl{
%HA: Possible criteria: based on which protocol (eg Gnutella, CAN etc), peer
%selection (topology selection, cluster, cache etc), supports dynamic update,
%runtimes}}
%%\textbf{Algorithm} & \textbf{Overlay structure} & \textbf{Forwarding} &
%\textbf{Cache} & \textbf{Overlay optimization} & \textbf{Proximity information}
%& \textbf{Base protocol} & \textbf{Dynamic update} & \textbf{Runtime} \\
%
%
%{\sethlcolor{yellow}\hl{
%HA: Possible criteria2: Overlay optimization structure, base protocol (eg
%Gnutella, CAN etc), dynamic update, runtimes, scalability}}
%
%{\sethlcolor{yellow}\hl{
%HA: maybe add also the year of publication, and see if there is a pattern in
%terms of the method and the year??}}

\section{Unstructured \p\ Networks}
\label{section:unstructured}

In this section, we present algorithms that tackle the topology mismatch
problem in unstructured \p\ networks. We classify them based on their
use of the overlay structure, their message forwarding scheme 
for peer communication and the techniques they use for detecting
proximity, to be used to optimize their overlay topology.

\subsection{Algorithms for Unstructured Architectures}

% TODO: READ A Near-Optimal Algorithm Attacking the Topology Mismatch Problem in Unstructured Peer-to-Peer Networks)

% BROADCAST OPTIMIZATION

%%%%%%%%%%%%%%%%%%%%%%%%%%%%%%%%%%%%%%%%%%%%%%%%%%%%%%%%%%%%%%%%%%%%%%%%%%%%%%%%
% \subsubsection{Improving search in peer-to-peer networks}

% \begin{figure}[ht]
% \centering
% \subfigure[Iterative deepening with three levels.] {
%   \includegraphics[scale=0.4]{img/algorithms/iterative_deepening}
%   \label{figure:dbfs:iterdeep}
% }\qquad\qquad
% \subfigure[Directed BFS.] {
%   \includegraphics[scale=0.4]{img/algorithms/directed_bfs}
%   \label{figure:dbfs:dbfs}
% }\qquad\qquad
% \subfigure[Local indices with radius size equal to $2$.] {
%   \includegraphics[scale=0.4]{img/algorithms/local_indices}
%   \label{figure:dbfs:localindx}
% }
% \caption{Improving search in P2P networks}
% \label{figure:dbfs}
% \end{figure}

To improve over {\sl Gnutella}'s ``blind flooding'' approach,
\cite{YG-M2002} proposed a practical and easy to implement solution 
weaved around three different message forwarding methods:
\emph{iterative deepening},
\emph{directed BFS}, and \emph{local indices}.

In \emph{iterative deepening}, 
%(see Figure~\ref{figure:dbfs:iterdeep})
the search is
performed on a BFS tree with multiple preset depths. 
The depth limit is iteratively increased by the source node for each query, 
based on the quality of the results. 
The source node may issue a new request with increased depth limit
that will trigger the nodes at the last depth level to resume the search. 
The iterative approach avoids restarting the entire search process 
from scratch at each iteration and reduces the load on the nodes of 
the upper levels of the tree. 
Its major drawback is the delay between successive iterations, as the
source-node has to examine the results at each attempt before deciding to
either quit or ``unfreeze'' the query.

The \emph{directed BFS}
tries to avoid the aforementioned delay by forwarding 
query messages only to a selected subset of available neighbors. The selection
criteria varies, from the number of results received previously or the distance
in terms of hops traveled to locate these results, to the bandwidth, or the
query load of the neighbor. That way, fewer nodes are visited but quality of
query responses is maintained, to a large degree, on the premise that the
selected heuristics can direct the search to the right path.
%
In \emph{local indices},
each node indexes data within a radius of $r$ hops and uses 
this local index to answer queries % on behalf of them 
without generating additional traffic. Local indices
greatly reduce the aggregate bandwidth usage of the network and 
improve query efficiency. However, updating such indices in 
the presence of frequent node joins/departures 
may introduce significant overhead should the radius is kept broad.
%%
Below, we outline how the techniques in~\cite{YG-M2002}
fare against the $3$ major performance criteria 
of Section~\ref{section:background}.
% \begin{center}
% \begin{tabular}{ccc}
% \textbf{Efficiency} & \textbf{Overhead} & \textbf{Scalability} \\
% \hline
% Iterative Deepening & ? & ? & ?
% Directed BFS & ? & ? & ?
% Local Indices & ? & ? & ?
% \end{tabular}
% \end{center}
\begin{center}
{\footnotesize
\begin{tabular}{rccc}
\multicolumn{1}{r}{} &
\multicolumn{1}{c}{\emph{Efficiency}} &
\multicolumn{1}{c}{\emph{Overhead}} &
\multicolumn{1}{c}{\emph{Scalability}}
\\
\cline{2-4}
\emph{Iterative Deepening} &
% needs recalculation in every iteration step
% extra delay imposed in a high churn environment that prohibits the algorithm
% to start from the last level of nodes and results in a restart of the whole
% process
low &
% the process in not started from the beginning at each iteration step
low &
% 
medium \\
\emph{Directed BFS} &
% Better time to satisfaction compared to iterative deepening
medium &
% Better quality and quicker results mean more aggregate bandwidth and
% processing power needs to be consumed since more nodes are involved (death
% spiral).
medium &
% BFS is Gnutella's non efficient communication method
medium \\
\emph{Local Indices} &
% efficiency is enhanced greatly since data pointers are replicated across the
% network
high &
% updating indices is a time and resource consuming process.
medium &
% the scalability is constrained by the indices update in a highly dynamic
% environment (high churn)
medium \\
\end{tabular}
}
\end{center}

%%%%%%%%%%%%%%%%%%%%%%%%%%%%%%%%%%%%%%%%%%%%%%%%%%%%%%%%%%%%%%%%%%%%%%%%%%%%%%%%
% \paragraph*{ \bf Delay Aware P2P System}
A \emph{delay--aware} \p\ system termed {\sl DAPS} 
was introduced in~\cite{ZL2005}.
{\sl DAPS}  seeks to attain reduced look-up times by dividing 
peer routing tables into several sectors of increasing delay. 
The source node that issues the query designates the delay 
boundary it may tolerate, providing so a \emph{pruning factor}.
In this context, user requests are forwarded only to 
nodes whose expected delay is less than or equal to the indicated boundary. 
{\sl DAPS} primarily focuses on user ``experience'' and deploys an
end-to-end delay monitoring mechanism that  may enable
the clustering of routing tables. 
In an dynamic environment, the formation 
of very accurate such routing tables might be however an elusive goal.
%With the clustered
%routing tables and the loose organization of DAPS' overlay network it is
%considered by is be tween structured and unstructured.
%%
In terms of the $3$ performance criteria, {\sl DAPS} stands as follows:
\begin{center}
{\footnotesize
\begin{tabular}{ccc}
\emph{Efficiency} & \emph{Overhead} & \emph{Scalability} \\
\hline
% Does not check for close by nodes, it just sorts the contents of a node's
% routing table.
% Also in a high churn environment a lot of time could be spend probing for
% distances, resorting routing tables and creating sectors. Moreover it just
% prunes distant neighbout
low &
% Sorting routing table entries is done locally so it is not affecting.
low &
%
medium
\end{tabular}
}
\end{center}

%%%%%%%%%%%%%%%%%%%%%%%%%%%%%%%%%%%%%%%%%%%%%%%%%%%%%%%%%%%%%%%%%%%%%%%%%%%%%%%%
% \subsubsection{Gia}

The key objective of the {\sl Gia} system~\cite{CRBLS2003} is 
to help alleviate the scalability omnipresent
in unstructured \p\ file-sharing systems.
At first, {\sl Gia} replaced {\sl Gnutella}'s blind flooding 
with random walks.
Although this adoption was 
a step in the right direction~\cite{lv_randomwalks_2002},
issuing a single copy of the query within the network
effectively reduces the search scope and may negatively affect 
the success rate of the query in question.
To overcome this limitation, {\sl Gia} introduces
a token-based flow control mechanism 
that gradually redirects queries to nodes which are more
likely to answer. This flow control mechanism also helps prevent node
overloading as each peer ``announces'' the number of query requests it 
can handle, in terms of tokens, to its neighbors; to this end, 
peers only forward query requests to nodes that they previously
received tokens from. Further refinement to the search mechanism is the
support for \emph{one--hop replication} of pointers to content.
{\sl Gia} also acknowledges the heterogeneity in peer bandwidth,
processing power, disk speed, etc., of \p\ nodes and uses this
information when connecting nodes to each other.
By using a topology adaptation algorithm, 
{\sl Gia} places low capacity nodes within 
short proximity to peers with high performance features.
This topology adaptation algorithm is based on the metric 
each node maintains about its satisfaction --ranging between $0$ and $1$--
for the neighbors it finds itself associated with. 
Through message exchange, a peer can establish new connections 
or drop superseded ones in order to improve its satisfaction.
Despite the fact that {\sl Gia}'s topology adaptation algorithm 
that primarily focuses on the satisfaction of peers improves the network's 
scalability, it falls short in addressing the mismatch problem
as considerations for the underlying network are not handled in explicit
terms.
%
In terms of the $3$ stated criteria, {\sl Gia}'s approach is as follows:
\begin{center}
{\footnotesize
\begin{tabular}{ccc}
\emph{Efficiency} & \emph{Overhead} & \emph{Scalability} \\
\hline
% Reduces the search scope.
medium &
% Only one copy of the query is issued to the network.
% A token based flow control mechanism redirects queries to nodes that are most
% likely to fulfill them.
low &
% Even thought the low overhead scalability is reduced by the reducing of
% search scope.
medium
\end{tabular}
}
\end{center}

%%%%%%%%%%%%%%%%%%%%%%%%%%%%%%%%%%%%%%%%%%%%%%%%%%%%%%%%%%%%%%%%%%%%%%%%%%%%%%%%
% \subsubsection{Distributed Cycle Minimization Protocol (DCMP)}
Another problem with topology unaware
systems is the duplication of messages due to cycles in the network graph.
Interestingly, such cycles appear even along the correct forwarding path.
\cite{ZKB2008}~focuses on this exact deficiency of overlay networks and
introduces the \emph{Distributed Cycle Minimization Protocol
(DCMP)}, a dynamic, fully distributed method that removes cycles;
this is accomplished without sacrificing
overlay connectivity, resilience and other key properties of unstructured
\p\ architectures and by avoiding a hierarchical organization of peers. 
Once a cycle is detected in {\it DCMP}, the most powerful node in that cycle 
is elected as the \emph{Gate Peer} and the cycle is then broken in that place
that it will result in the minimization of the distance between the
\emph{Gate Peer} and all other nodes that are currently part of the cycle. 
This process is managed by using two specialized message types 
namely, \emph{Information Collection Message (ICM)} and \emph{Cut Message (CM)}. 
\emph{DCMP} bases its operation on messages whose travel is limited by an
imposed $TTL$ value (max set to $7$ for most cases). This inherently limits
the protocol in detecting cycles that span for more than $TTL$ nodes.
Even though cycle elimination does improve network performance, 
it cannot  directly contribute in the solution of the topology mismatch problem.
%
%\begin{figure}
%\centering
%  \includegraphics[scale=0.4]{img/dcmp.jpeg}
%\caption{Cycle elimination methods in DCMP}
%\label{figure:dcmp}
%\end{figure}
%
% TODO: SOME DISCUSSION
%Actually, the duplicate packet problem seems to hurt more the active nodes;
%those with higher capacities and bandwidth that contribute the most to the
%network.
The expected \emph{DCMP} behavior as far as our $3$ criteria is:
%
\begin{center}
{\footnotesize
\begin{tabular}{ccc}
\emph{Efficiency} & \emph{Overhead} & \emph{Scalability} \\
\hline
% It returns more result than LTM according to experiments in
% \cite{ZKB2008}
medium &
% Duplicates are an important fraction of the redundant overhead imposed by
% topology mismatch and thus DCMP minimizes this as possible.
% The control overhead is also one to two orders of magnitude less than LTM due
% to its ``lazy'' broadcasting of control messages (as opposed to LTM's periodic
% approach) - \cite{ZKB2008}
low &
% fully distributed method
high
\end{tabular}
}
\end{center}

%       CACHING

%%%%%%%%%%%%%%%%%%%%%%%%%%%%%%%%%%%%%%%%%%%%%%%%%%%%%%%%%%%%%%%%%%%%%%%%%%%%%%%%
% \subsubsection{Replication Strategies in Unstructured Peer-to-Peer Networks}

In~\cite{CS2002}, replication is used as a way to improve inefficient blind
search. As the number of replicas and/or cached copies increases in the network,
it would relatively easy to locate items even if search is random.
To analyze the feasibility of such an approach, \cite{CS2002} investigates
$3$ replication strategies: \emph{uniform}, \emph{proportional}, and
\emph{square root allocation}. In the uniform model, the copies of items are
uniformly spread over the network, while in the proportional, items are
replicated based on their query rate, so that frequently queried items can be
found more often.
The square root is a compromise between uniform and proportional allocation
models. The overall costs for both successful and unsuccessful
searches are compared to ascertain the effectiveness of the $3$ suggested 
approaches. Experimental outcomes indicate that the uniform allocation 
minimizes the maximum search length and so it can reduce the time spent on
unsuccessful searches. 
The proportional strategy effectively decreases the search time 
for popular items, but suffers when needing to locate the rare entities.
When it comes to expected successful search size, it is the same for 
for both uniform and proportional models and any approach in between 
to always behave much better. The square root allocation  aims at minimizing the 
expected search size for successful queries in \p\ networks.
The table below outlines how the $3$ strategies fare:
%%
\begin{center}
{\footnotesize
\begin{tabular}{rccc}
\multicolumn{1}{r}{} &
\multicolumn{1}{c}{\emph{Efficiency}} &
\multicolumn{1}{c}{\emph{Overhead}} &
\multicolumn{1}{c}{\emph{Scalability}}
\\
\cline{2-4}
\emph{Uniform Replication} &
% Proportional and Uniform are the worst “reasonable” strategies
low &
% When item is created, replicate its key in a fixed number of hosts.
low &
%
medium \\
\emph{Proportional Replication} &
% Proportional and Uniform are the worst “reasonable” strategies
low &
% for each query, replicate the key in a fixed number of hosts (need to know or
% estimate the query rate)
medium &
%
low \\
\emph{Square Root Replication Allocation} &
% All (strictly) in-between strategies are (strictly) better than Uniform and
% Proportional - replication theory
medium &
% Assuming that each query keeps track of the search size then each time a query
% is finished the object is copied to a number of sites proportional to the
% number of probes that the search required.
medium &
% e.g. path replication is easy to scale, just detect the path along which the
% query ultimately got fulfilled
medium \\
\end{tabular}
}
\end{center}

%%%%%%%%%%%%%%%%%%%%%%%%%%%%%%%%%%%%%%%%%%%%%%%%%%%%%%%%%%%%%%%%%%%%%%%%%%%%%%%%
% \subsubsection{Tracing a large-scale Peer to Peer System: an hour in the life
% of Gnutella}
% TODO: This looks better as an analysis for the unstructured networks in
% general and specifically for caching, so it doesn't seem to contribute any
% new approach.... At least we do not present anything here. Maybe we need to
% revisit the paper itself once more
% \cite{Markatos02} analyses Gnutella network traffic traces and by concluding
% there is locality among query requests, proposes a caching algorithm that tries
% to exploit these findings . The analysis of the trace data reveals other
% important facts about the structure of the Gnutella network and the query data.
% One significant observation is that the geographic locations of clients do not
% have a correlation with the number of query requests they receive. This is an
% obvious result of the topology mismatch problem caused by the overlay structure
% of the Gnutella network. Gnutella's traffic is observed to be bursty both for
% query requests and query responses, even in longer intervals. It is observed
% that a peer receives 50 query messages per second on average! Moreover nine out
% of ten queries do not generate any response due to the inefficient design of
% the Gnutella network. When developing a caching system to exploit locality,
% applying an approach similar to web caching does not fit well with P2P systems.
% Caches in P2P systems not only have to consider the query string, but also
% the TTL value, the source of the query, and the time of the query as well.  In
% general, even though optimum caching is hard to achieve, it is reported that
% it improves the overall performance of the Gnutella network.

% \begin{center}
% \begin{tabular}{ccc}
% \emph{Efficiency} & \emph{Overhead} & \emph{Scalability} \\
% \hline
%
% ? &
%
% ? &
%
% ?
% \end{tabular}
% \end{center}


%       OVERLAY OPTIMIZATION

%%%%%%%%%%%%%%%%%%%%%%%%%%%%%%%%%%%%%%%%%%%%%%%%%%%%%%%%%%%%%%%%%%%%%%%%%%%%%%%%
% \subsubsection{Narada}

% \begin{figure}[ht]
% \centering
% \subfigure[Underlying network with edge costs.] {
%   \includegraphics[scale=0.4]{img/algorithms/narada}
%   \label{figure:narada:underlying}
% }\qquad\qquad
% \subfigure[Peer A sends a message to the rest using regular broadcast.] {
%   \includegraphics[scale=0.4]{img/algorithms/narada2}
%   \label{figure:narada:regu}
% }\qquad\qquad
% \subfigure[Peer A sends a message to the rest using end system multicast.] {
%   \includegraphics[scale=0.4]{img/algorithms/narada3}
%   \label{figure:narada:multicast}
% }
% \caption{A visualization of the Narada protocol.}
% \label{figure:narada}
% \end{figure}

{\sl Narada}~\cite{CRZ2000,CRSZ2001,CRSZ2002} is a generic
protocol for creating self-adapting overlay networks capable of 
application-layer multicast communications without requiring IP multicast
infrastructure at the network layer.
Although IP--multicast would present a choice in implementing
{\sl Narada}, it is in general considered that it violates the 
stateless design of the current Internet.
Despite the fact the {\sl Narada} was not designed as \p\ system per se,
it was the first (along with \emph{Scattercast} \cite{C2000}) to consider the
feasibility of overlay-based, application-layer services over the Internet while
taking into account bandwidth and latency properties of the underlying physical
infrastructure. In the context of this work, the inefficiency caused by the
topology mismatch and attempted to be tackled by building an highly connected
graph termed \emph{mesh}
with each source featuring its own minimum spanning tree. A gossip-protocol was
deployed for the creation of these spanning trees~\cite{LYL2008}.
%as shown in Figure~\ref{figure:narada}.
Both graph and trees are dynamically updated as nodes keep joining 
or departing the network. The protocol aims to ease the physical link stress,
the overall resource usage as well as the relative delay among end-systems. 
Unfortunately, the main limitation for {\sl Narada} is that 
although it works reasonably well for small groups, 
it does not scale well for larger networks. 
Hence, it is not a suitable choice for potentially large \p\ networks:
%%
\begin{center}
{\footnotesize
\begin{tabular}{ccc}
\emph{Efficiency} & \emph{Overhead} & \emph{Scalability} \\
\hline
%
medium &
% The overhead of Narada is proportional to the multicast group size
high &
% Maintaining the mesh for large multicast groups is expensive.
low
\end{tabular}
}
\end{center}

%%%%%%%%%%%%%%%%%%%%%%%%%%%%%%%%%%%%%%%%%%%%%%%%%%%%%%%%%%%%%%%%%%%%%%%%%%%%%%%%
% \subsubsection{Adaptive Overlay Topology Optimization (AOTO)}

Along with {\sl Narada}, the \emph{Adaptive Overlay Topology 
Optimization (AOTO)}~\cite{LZXN2003} is one of the first attempts to address the
topology mismatch problem. \emph{AOTO} is a distributed algorithm that seeks to
optimize the usage of the underlying physical resources and operates in
$2$ phases:
\emph{Selective Flooding} and \emph{Active Topology}.
In \emph{Selective Flooding}, a minimum spanning tree is built for each peer and
its immediate neighbors so that queries do not flood the entire network and at
the same time preventing their search scope from shrinking. This way some
neighbors become non--flooding. During \emph{Active Topology}, each peer
replaces independently such non--flooding neighbors, with closer nodes as an attempt
to revise the overlay links so that they can reflect more closely the physical
network topology. Picking a replacement out of these non-flooding
neighbors follows a random policy called \emph{Randomized AT} algorithm.
To accomplish the above actions, a peer has to keep track of 
its communication costs with all its neighbors (e.g., network delays)
as well as the costs between any pair of neighbors. The \emph{randomized AT}
algorithm is applied by a source peer
every time its neighbor list is updated or an updated neighbor cost 
table is received.
%%
%\paragraph*{Selective Flooding (SF)}
%
% TODO: I DON'T REMEMBER WHY THE FOLLOWING IS MENTIONING LTM WHICH IS ANOTHER
%       ALGORITM. MAYBE THIS CAN BE USED AS A PART OF DISCUSSION AND/OR
%       COMPARISON OF AOTO AND LTM
% TODO: EDIT: LTM SEEMS A MISTAKE HERE BETTER NOT USE IT FOR DISCUSSION (THIS
%             IS AOTO)
%LTM's SF effectiveness has be proven to be detached from the different physical
% or overlay topologies. On the other hand, SF is more effective with large
%number of logical neighbors. It can reach an average optimization rate of 87.4
%percent on a logical topology with an average of 30 logical neighbors.
%
%\paragraph*{Active Topology (AT)}
%
%Different numbers of average logical neighbors has little to do with the
% effectiveness of AT. If the source has $n$ non-flooding peers, there are $n$
%potential neighbor replacements. The overhead to exhaust all $n$ possible
%replacements can be high, so in practice, after each replacement the source
%peer can decide whether it needs to find another candidate peer. This is done
%by computing the cost improvement ratio greater than some predefined
%termination threshold. The larger the threshold, the slower, in the number of
%optimization steps, the reduction of the normalized average distance. As a
%whole the average response time is significantly reduced when more optimization
%steps taken.
%%
\emph{AOTO}'s performance regarding the $3$ criteria is as follows:
%%
\begin{center}
{\footnotesize
\begin{tabular}{ccc}
\emph{Efficiency} & \emph{Overhead} & \emph{Scalability} \\
\hline
%
medium &
% tracking of communication costs to all peer's neighbors as well as between
% any pair of neighbors. Whenever a new neighbor cost table is received or
% there is a change of neighbors, the source peer has to re-calculate the
% multicast tree and apply the randomized AT algorithm.
high &
% needs global knowledge
medium
\end{tabular}
}
\end{center}

%%%%%%%%%%%%%%%%%%%%%%%%%%%%%%%%%%%%%%%%%%%%%%%%%%%%%%%%%%%%%%%%%%%%%%%%%%%%%%%%
% \subsubsection{Adaptive Connection Establishment (ACE)}
The \emph{Adaptive Connection Establishment (ACE)} 
approach~\cite{LZXN2004} uses the network delay as a metric 
to estimate the costs between nodes. Each peer computes the costs to its
logical one-hop neighbors and forms a \emph{neighbor cost table (NCT)} 
using a special routing message type. 
Any pair of neighboring peers exchange their \emph{NCT}s 
and so a minimal overlay topology can be formed. Based on the obtained
\emph{NCT}s a minimum spanning tree for each peer and its one-hop neighbors is
created. Finally, neighbors located physically far away 
are replaced by physically-closer counterparts. 
In particular, a peer $S$ iteratively probes the distance $d$ between
itself and every of its non-flooding neighbor nodes $G$ as well as
the distance between $S$ and $G$'s neighbors designated as $H$.
If $d_{SG} > d_{SH}$, then link $SG$ is replaced by link $SH$. If, on the other
hand, $d_{SG} < d_{SH}$ then also $d_{GH}$ and then we have the following
options. Either $d_{SH} < d_{GH}$ in which case link $SH$ is kept or
$d_{SH} > d_{GH}$ in which case $S$ will keep probing other $G$'s direct
neighbors.
%%%
The above optimization is conducted within $1$-neighbor
closure using as base, a peer and checking all its direct neighbors.
Evidently the scope of such an operation could be extended.
Should  a larger scope be used, a better topology matching can be obtained 
at a greater computational overhead.
%
% TODO: SOME DISCUSSION
%
%Simulations in \cite{liu_acesims_2004} show that the average scope of each
% query to cover the same scope of nodes is reduced by about 65 percent without
%losing any autonomy feature, while the average response time can be reduced by
%35 percent. Larger diameter topologies lead to better topology optimization
%rate but also to higher communication and computation overhead. It was also
%found that it is more effective in higher connectivity dense topologies.
%Compared to LTM, it comes short of convergence speed. In
%\cite{ni_mismatch_2004} shows reduction of both total traffic (90 percent) and
%response time (80 percent) to message queries without shrinking the search
%scope.Last but
%not least, it is concluded that work must be done on incorporating a more
%sophisticated selection policy for candidate non-flooding peers.
%

%In \emph{Adaptive Connection Establishment (ACE)} \cite{LZXN2004}, the
%authors extend the idea of AOTO by introducing optimizations based on the
%depths of the minimum spanning trees. But since the network delay is not
%always a reliable estimation method to detect the physical location of peers,
%the algorithm still suffers from the discrepancies caused by mis-located nodes.
%%
%%
\begin{center}
{\footnotesize
\begin{tabular}{ccc}
\emph{Efficiency} & \emph{Overhead} & \emph{Scalability} \\
\hline
% compared to LTM it has slow convergence speed.
medium &
% A little bit better that AOTO since the computation here is done within a
% certain diameter from the source peer
medium &
% Larger diameter topologies lead to better topology optimization
% rate but also to higher communication and computation overhead.
medium
\end{tabular}
}
\end{center}

%%%%%%%%%%%%%%%%%%%%%%%%%%%%%%%%%%%%%%%%%%%%%%%%%%%%%%%%%%%%%%%%%%%%%%%%%%%%%%%%
% \subsubsection{Location-aware Topology Matching (LTM)}

\emph{Location-aware Topology Matching (LTM)}~\cite{LLXNZ2004} 
seeks to optimize an overlay \p\ structure based on the physical topology.
To achieve this, peers issue special messages called
\textit{TTL-detector}s whose \emph{TTL} values are $0$ or $1$;
in this regard, peers 
discover $1$- and $2$-hop neighbor sets, designated as $N$ and $N^2$
respectively, and proceed to compute communication costs.
Time-stamps are used to derive network latency measurements that are then used 
to improve the overlay network without sacrificing the search scope.
Each node compares the latency information
received from its direct neighbors;
peers with longer latencies are placed on a
\emph{will-cut} list where they remain for a certain period of time 
after which they are finally eliminated 
and are placed on the peer's \emph{cut-list}. 
Thus, low-productivity connections are dropped and replaced by 
more efficient ones, reducing the
latency on the overall network. 
Although \emph{LTM} improves the overall efficiency of
the \p\ network, it is unable to offer any warranty for 
effectively addressing the mismatch problem.
%
% TODO: SOME DISCUSSION
%
%\paragraph*{Low productive connection cutting}
%There are three cases for any peer $P$ who receives $d(i, S, v)$ multiple
% times:
%\begin{inparaenum}[\itshape i\upshape)]
%  \item $P$ receives both $d(i, S, 1)$ and $d(i, S, 0)$
%  \item $P$ receives multiple $d(i, S, 0)$s from different paths, and P
% randomly chooses to process one
%  \item $P$ receives one $d(i, S, 1)$ and multiple $d(i, S, 0)$s, and $P$
% processes $d(i, S, 1)$ and one randomly selected $d(i, S, 0)$
%\end{inparaenum}
%If the link with the largest cost is found and is a direct neighbor then the
% connection is put in a will-cut list and stays there for a certain period of
%time. If it is not, then it is handled by other peers. After that period,
%connections are cut and recorded to $P$'s cut-list.
%
%\paragraph*{Source probing}
%For a peer $P\in(N^2(S) - N(S))$ who receives only one $d(i, S, 0)$, the cost
% of $PS$ is obtained (with list look-up or probing). Then $P$ compares it with
%the cost from each hop and if $PS$ has the largest cost, $P$ will not keep this
%connection, while otherwise the connection will be created.
%
%\paragraph{}
%Supposing $n$ is the number of peers, $c_n$ is the average number of neighbors
% and $c_e$ is the average cost of logical links, then in the flooding-based
%search the traffic incurred by one query from an arbitrary peer in a
%peer-to-peer network is $O(n)$. As observed in the Gnutella network
%\cite{sripanidkulchai_gnutella_2001}, each peer issues $0.3$ queries per minute
%in average, thus the per minute traffic incurred by the network with $n$ peers
%is $O(n^2)$. Because each $d(i, S, v)$ has a TTL of $2$ in each source peer,
%the traffic for one time LTM optimization in all peers is at most $2nc_n^2c_e$.
%If each peer uses LTM $k$ times per minute, the total traffic incurred is
%$2knc_n^2c_e$. Simulation shows the best value for $k$ is $2$ or $3$. So, the
%traffic overhead caused by LTM to the network is $O(n)$.
%
%TTLj-detectors, with $j > 2$, would detect and break cycles with more than 4
% links. LTM though, does not use such detectors because detector-flood traffic
%would increase significantly, and cut links between two end-peers, could cause
%queries initiated by them to traverse a path much more expensive than the cost
%on the the cut link.
%
%\paragraph{}
%LTM disadvantages are
%\begin{inparaenum}[\itshape i\upshape)]
%  \item disagreement of measured delay due to unsynchronized clocks causes
% problems when deciding the cut positions, which can influence the network
%connectivity, and
%  \item the network delay metric mainly focuses on disabling the connections
% between peers physically far away without considering the shortcuts created by
%powerful peers.
%\end{inparaenum}
%
\emph{LTM} fares as follows regarding our $3$  criteria:
\begin{center}
{\footnotesize
\begin{tabular}{ccc}
\emph{Efficiency} & \emph{Overhead} & \emph{Scalability} \\
\hline
% Authors claim 75% reduction on traffic cost and about 65% reduction on query
% response time.
medium &
%
medium &
%
medium
\end{tabular}
}
\end{center}

%%%%%%%%%%%%%%%%%%%%%%%%%%%%%%%%%%%%%%%%%%%%%%%%%%%%%%%%%%%%%%%%%%%%%%%%%%%%%%%%
% \subsubsection{Scalable Bipartite Overlay (SBO)}

The \emph{Scalable Bipartite Overlay (SBO)}~\cite{LXN2004,LXN2007} 
reduces the overhead of creating
and maintaining a minimum spanning tree 
by randomly dividing the nodes into
two groups --\emph{red}s and \emph{white}s-- 
and assigning different tasks to the two groups.
When a peer joins the network, it is randomly assigned with an initial
color (red or white).
Then the network bootstrap host node furnishes the joining peer with a list
of active peers along with their color information. 
The joining peer uses this list to 
establish connections to differently colored peers. 
In this regard, all peers form a a bipartite overlay. 
Using the network delay as a metric, white-peers 
peers measure distances from red counterparts
and report the encountered red neighbors.
Having information on all their $2$-hop neighbors 
($N^2$)
red-peers create minimum spanning trees for the neighbors in question and
assign efficient $2$-hop forwarding paths. This process can render a white
peer $E$ non--forwarding neighbor of the red peer $P$. Direct neighbor
replacement is a process conducted by $E$ in order to replace $P$ with a
$P' \in N^2(P)$ as its new neighbor. This adaptation tackles the
topology mismatch problem by reducing message duplication and shorten response
times caused by the problem identified as \emph{Revisit Not known (RN)}. RN is
the situation when a node is visited some times as a relay stop before it is
visited as an overlay peer, thus creating duplicate unnecessary messages to the
network and delaying query responses.
%
% TODO: SOME DISCUSSION
%
%In a static environment LTM may reduce traffic cost by around 80 to 85 percent
% while SBO reduces traffic cost between 85 and 90 percent. However, LTM  is
%proved to converge in around 2-3 steps while SBO needs 4-5 steps. Moreover LTM
%reduces response time by more than 60 percent in 3 steps while SBO needs 8. In
%a dynamic environment (10 minute average peer lifetime, 0.3 queries/sec by each
%peer) SBO and LTM reduce the average traffic cost per query (including the
%overhead due to the optimization steps) by 85 and 80 percent, respectively.
%Moreover LTM reduces the response time per query to 30 percent while SBO to 35
%percent.
%
%\cite{ni_mismatch_2004} shows SBO, achieves approximately 85 percent reduction
% on traffic cost and about 60 percent reduction on query response time.
In regards to the $3$ stated criteria, \emph{SBO} behaves as follows:
\begin{center}
{\footnotesize
\begin{tabular}{ccc}
\emph{Efficiency} & \emph{Overhead} & \emph{Scalability} \\
\hline
% SBO reduces traffic cost between 85 and 90 percent
high &
% compared to LTM it has slower convergence speed thus incurs more, total,
% overhead.
medium &
%
medium
\end{tabular}
}
\end{center}

%%%%%%%%%%%%%%%%%%%%%%%%%%%%%%%%%%%%%%%%%%%%%%%%%%%%%%%%%%%%%%%%%%%%%%%%%%%%%%%%
% \subsubsection{Two-Hop-Away Neighbor Comparison and Selection (THANCS)}

The distributed heuristic termed 
\emph{Two-Hop-Away Neighbor Comparison and Selection (THANCS)}~\cite{LNXE2005,L2008}
attempts to minimize overlay hop costs.
\emph{THANCS} is essentially a \emph{local search method} as it aims 
to find a locally optimum solution by exploiting knowledge within 
a $2$-hop radius. The algorithm consists of two main components:
\emph{piggybacking neighbor distance on queries} and
\emph{neighbor comparison and selection}.

The \emph{piggybacking component} requires peers 
to probe immediate neighbors using delay distance measurements 
and store information locally. The algorithm introduces a special query message
type, the \emph{Piggy Message (PM)} which includes information about 
the neighbor identification (IP) and measured distance ($d$). PMs are
piggybacked to normal
Gnutella messages. When node $P$ receives a query from $Q$ it constructs a
$PM_{PQ}$, piggybacks it to the query and forwards it to $P$'s neighbors. When
the neighbors recieve it they detach the $PM_{PQ}$, record the distance
$d_{PQ}$ and process the query normaly. Selection of which incoming queries
should be piggybacked with a \emph{PM} is proposed by either
\emph{pure probability-based (PPB)} or \emph{new neighbor triggered (NNT)}
policies.

The \emph{neighbor comparison and selection} component defines that a peer $S$
probe all his $2$-hops away neighbors (a set denoted as $ N^2(S)$) not probed
thus far. Let $P$ be a $1$-hop distance neighbor of $S$, and $Q$ be a probed
peer by $S$. When $S$ receives a message piggybacked with a $PM_{PQ}$ we
identify the following cases.
\begin{itemize}
  \item If $Q$ is already a direct neighbor of $S$ then we check the distances
  $d_{SQ}$, $d_{SP}$ and $d_{PQ}$. If either $d_{SQ}$ or $d_{SP}$ is the longest, then the longest
  link will be put in a \emph{will-cut} list\footnote{A peer will not send or
  forward queries to connections in its will-cut list but will preserve them
  for some time in order not to effect ongoing response messages travelling the
  inverse path.}. If $d_{PQ}$ is the longest then nothing is done by $S$; being
  fully distributed, neighbor comparison and selection process, will have the
  oportunity to handle $PQ$ link when initiated by peers $P$ or $Q$.

  \item If $Q$ is strictly a $2$-hop neighbor of $S$ and have never probed $Q$
  in the past, stores distance $d_{SQ}$ and checks distances $d_{SQ}$, $d_{SP}$ and $d_{PQ}$.
  If $d_{SQ}$ is the longest, $S$ will not create link $SQ$. If $d_{SP}$ is the
  longest, $SQ$ will be created and $SP$ will be put into the will-cut list.
  If $PQ$ is the longest, links $SP$ and $SQ$ will be preserved expecting that
  $P$ or $Q$ will disconnect $PQ$ later.
\end{itemize}

%
% TODO: SOME DISCUSSION
%
% \begin{inparaenum}[\itshape i\upshape)]
%   \item is completely distributed and needs no global knowledge,
%   \item presents trivial overhead compared to the query cost savings
%   \item its convergent speed of the algorithm is fast enough (faster than
% minimum spanning tree approaches) so that is effective to dynamic
% environments, and
%   \item does not shrink the search scope.
% \end{inparaenum}
%
%In a static environment THANCS has been proven to be effective; optimizing 45
% percent out of the 60 percent of mismatched paths, constructing a nearly
%optimal overlay. This leads to a 60 percent reduction in traffic cost as well
%as a 40 percent decrease in query response time. In dynamic environments
%(Gnutella 0.6/Limewire super-peer-like and Ion flat-like), THANCS saves up to
%70 percent of the traffic cost in the super-peer topology and 55 percent for
%the flat one. Average response time is also decreased by 60 and 45 percent,
%respectively. Generally, THANCS has similar performance to LTM, without needing
%synchronization. SBO, incurring half the  overhead of AOTO, reduces the traffic
%cost the most, while THANCS has lower response time and converges faster than
%SBO. THANCS is, thus, more suitable for a more dynamic environment. In
%addition, THANCS is easy to implement and its operation overhead is trivial,
%compared with the other three approaches. This design, however, has the
%limitation of not being easily extend to also support non-flooding-based
%systems.

As seen above, the method is fully distributed and needs only minimum knowledge
of, at most, a $2$-hop distant peers. This makes it a good candidate for big
overlay networks. Also topology adaptation helps construct a well fit overlay
network with restect to the underlying one (at least with respect to network
distances). Morever will-cut list and distance cache (which stores already
probed peers) minimize the unnecessary messages for the network maintenance.
These characteristics are depicted to the $3$ evaluation criteria we use in this
survey as follows:

\begin{center}
{\footnotesize
\begin{tabular}{ccc}
\emph{Efficiency} & \emph{Overhead} & \emph{Scalability} \\
\hline
% 
medium &
% trivial overhead compared to the query cost savings and its convergent speed
% is faster than minimum spanning tree approaches so less overall overhead cost.
low &
% completely distributed and needs no global knowledge,
high
\end{tabular}
}
\end{center}

%%%%%%%%%%%%%%%%%%%%%%%%%%%%%%%%%%%%%%%%%%%%%%%%%%%%%%%%%%%%%%%%%%%%%%%%%%%%%%%%
The \emph{Hops Adaptive Neighbor Discovery (HAND)} algorithm~\cite{CLZHC2006}
uses a fully-distributed triple--hop adjustment strategy, applied to a network
graph $G$ in order to create an optimal graph $G^{*}$, free of the implications
of topology mismatch. This optimality is attained if all peer hop 
sequences $(v_1, v_2, \ldots, v_k)$ of $G$ exist in $G^{*}$ and are in the same
order. The latter indicates that in practice triple sequences $(v_1, v_2, v_3)$
are used. A mismatch is detected as follows: supposing that we want to verify
a sequence, say $v_2-v_1-v_3$, two probing messages are dispatched from $v_1$ to
$v_2$ and $v_3$ and yield delays of $x$ and $z$ respectively.
%
When the probing message arrives at $v_2$, it gets forwarded
directly to $v_3$. 
Similarly, the message reaching $v_3$, it is directly forwarded to $v_2$.
The above two forwarding actions seek to quantify 
the corresponding delays of $(v_2,v_3)$ and $(v_3,v_2)$ physical paths
denoted by $y$. 
%%
If $y=z-x\pm\varepsilon$, the sequence $v_2-v_1-v_3$ is mismatched and should
be adjusted to $v_1-v_2-v_3$ by deleting edge $(v_1,v_3)$ and 
then adding a new $(v_2,v_3)$. 
If $y=x-z\pm\varepsilon$, sequence $v_2-v_1-v_3$ is mismatched and
should be adjusted to $v_1-v_3-v_2$ by deleting edge $(v_1,v_2)$ and adding a
new $(v_3,v_2)$. 
The  $\varepsilon$ is a small positive real number denoting
additional delays caused by possible forwarding and jitter delays. 
%%
The advantages of \emph{HAND} are that it
\begin{inparaenum}[\itshape i\upshape)]
  \item does not need any clock synchronization,
  \item is a fully distributed algorithm.
  \item involves low traffic overhead,
  \item can be used in dynamic \p\ environments, and 
  \item furnishes low query response times.
\end{inparaenum}
%
% TODO: SOME DISCUSSION
%
%Measurements conducted for evaluation purposes showed that in a static
% environment the algorithm can effectively decrease traffic cost by about 77
%percent and shorten the query response time by about 49 percent in less than
%two minutes. In a dynamic environment it shows similar behavior and with the
%size of the overlay network having little impact on the effectiveness of the
%algorithm. Compared to LTM both algorithms have almost the same traffic
%reduction rate, however on the response time reduction rate HAND has a higher
%one by about 4 percent. The traffic overhead of HAND is much less than that of
%LTM by an average of 55 percent.
In terms of the $3$ stated criteria, \emph{HAND} fares as follows:
\begin{center}
{\footnotesize
\begin{tabular}{ccc}
\emph{Efficiency} & \emph{Overhead} & \emph{Scalability} \\
\hline
% similar traffic reduction rate savings to LTM
medium &
% The traffic overhead of HAND is much less than that of LTM by an average of
% 55 percent.
low &
% no clock syncing needed, fully distributed and with low control overhead the
% algorithm can be considered scalable.
high
\end{tabular}
}
\end{center}

%%%%%%%%%%%%%%%%%%%%%%%%%%%%%%%%%%%%%%%%%%%%%%%%%%%%%%%%%%%%%%%%%%%%%%%%%%%%%%%%
% \subsubsection{Adaptive Peer Selection (APS)}
The \emph{Adaptive Peer Selection (APS)} approach uses 
machine learning techniques to form peer selection strategies based on past
experience~\cite{BFLZ2003}. A decision tree is used to rate peers based on
information collected for the characteristics of established connections. Such
features include connection link load, bandwidth, and past uploading experience.
Subsequently, a Markov decision process is introduced as a mechanism to shape
the policy for switching among the peers. The success of approach depends on
how fast the introduced peer selection functions. Admittedly, \emph{APS} is slow 
due to the learning process and its inherent complexity. As far as the $3$
criteria is concerned, \emph{APS} has as follows:
\begin{center}
{\footnotesize
\begin{tabular}{ccc}
\emph{Efficiency} & \emph{Overhead} & \emph{Scalability} \\
\hline
% machine learning the best way to adapt the overlay 
high &
% high computation overhead for learning and very slow convergence speed for
% the learning process.
high &
% Complexity of computation, and slow convergence speed prevents the approach
% to scale, especially in high churn environments.
low
\end{tabular}
}
\end{center}

%%%%%%%%%%%%%%%%%%%%%%%%%%%%%%%%%%%%%%%%%%%%%%%%%%%%%%%%%%%%%%%%%%%%%%%%%%%%%%%%
The \emph{Innocuous Topology Aware Overlay Construction (ITA)}
approach~\cite{PRFM2009} seeks to offer both an overlay formation approach and
an effective way to carry out searching. When constructing the overlay,
\emph{ITA} exploits the notion of \emph{short} and \emph{long} connections.
Should $N$ be the number of network peers,  $\alpha \leq 1 $ a system-wide magic
number and $x$ an $\alpha$-related latency threshold, the bootstrapping peer
randomly selects $\alpha \ast N$ ``close'' (latency below $x$) and 
$\left( 1 - \alpha \right) \ast N$ distant (latency above $x$) nodes
as its neighbors.
%%AD how do they select all these magic constants???
%%VM This α constant seems to be the percentage of close neighbours selected by
% a peer out of its total degree. Can't find the criterion with which it is
% chosen though. Maybe we can ask... Mema :)
Searching occurs in two phases: in the initial phase, the querying node
floods its distant neighborhood  with $TTL = 1$.
Subsequently, peers receiving a query over a ``long link'' commence a 
local flood with $TTL=ttl$ where \emph{ttl} is system-defined.
%%AD how are you supposed to define these params?? in a fixed manner??
%%
%% VM Don't understand... the paper says that for increased awareness we need a
%% small α (page 7). But in the begining they give that short links (close-by)
%% are given by α*Ν. I guess this gets smaller for smaller α.... Also on
%% page 7 they say they choose α to be 0.1, 0.05 and 0.033 and that
%% this equals to ``10, 20, 30 long and and the same number of short links''.
%% I don't understand how they deduce these numbers. Since short links is given
%% by α*Ν and long by (1-α)*Ν at least the numbers should be different from
%% short and long links...
The main objective of the overlay construction phase is to yield a network that
exhibits low clustering. In turn, this is a beneficial characteristic for
random graphs as it can lead to a larger coverage of peers and at the same time
helps reduce duplication of messages travelling around the network. This
combination offers efficient lookups that pose minimal or no negative impact to
other mechanisms possibly employed by the \p\ systems (e.g., 1-hop replication
or dynamic querying mechanisms).
A 50\% reduction in communication latency among peers by cutting off 
some 20\% of the \emph{IP}-level traffic generated 
in reported in~\cite{PRFM2009}.
%%
\emph{ITA} places as follows in the context of the $3$ criteria:
%
\begin{center}
{\footnotesize
\begin{tabular}{ccc}
\emph{Efficiency} & \emph{Overhead} & \emph{Scalability} \\
\hline
% low clustering with larger coverage of peers with the same number of messages
% and reduced duplication.
medium &
% construction at peer join, while searching using local ``floods''
low &
% the selection of neighbors is done at join time. No overlay adaptation means
% low scalability in dynamic environments.
medium
\end{tabular}
}
\end{center}

%%%%%%%%%%%%%%%%%%%%%%%%%%%%%%%%%%%%%%%%%%%%%%%%%%%%%%%%%%%%%%%%%%%%%%%%%%%%%%%%
% \subsubsection{EGOIST}
\emph{EGOIST}~\cite{SLLBBR2008} is a set of algorithms 
to construct and manage overlay networks. 
\emph{EGOIST} utilizes a selfish approach in the sense that every participating
peer continuously updates its neighbors so as to minimize the sum of distances
to all destinations under shortest-path routing. 
A newly arriving peer, randomly connects to an already participating 
node through a bootstrap server. Once connected, the peer starts receiving
information throughout the link-state mechanism and thus, after some time, 
it has a complete picture of the overlay graph. 
Then the peer estimates the delay to all other nodes to
determine its potential neighbors and ultimately connects 
to the overlay using some policy; such a policy might be 
for example the minimization of the average delay to all its
neighbors. \emph{EGOIST} requires extensive resource usage to continuously
update the overlay connections of all the nodes in the system. This needs
$O(n^2)$ measurements. Fortunatelly, each node does not need to do these
measurements for maintenance (monitoring and updating) with all other
participating nodes, but just with a number of $k<n$ nodes that it chooses to
establish links with. This gives a reduced $O(kn)$ time complexity for
maintenance while $O(n^2)$ is needed less frequently, when nodes reevaluate
their connections.

Regarding the $3$ criteria, \emph{EGOIST}'s behavior is as follows:
\begin{center}
{\footnotesize
\begin{tabular}{ccc}
\emph{Efficiency} & \emph{Overhead} & \emph{Scalability} \\
\hline
% peer neighborhood selected to minimize, for example, average delay
high &
% link state info (heartbeat) and delay estimation to all other nodes
% resulting to a complete picture of the overlay graph.
high &
% need global knowledge
low
\end{tabular}
}
\end{center}

%%%%%%%%%%%%%%%%%%%%%%%%%%%%%%%%%%%%%%%%%%%%%%%%%%%%%%%%%%%%%%%%%%%%%%%%%%%%%%%%
The main objective of the \emph{Biased Neighbor Selection (BNS)} 
approach~\cite{BCCMSBZ2006} is to 
strengthen {\sl BitTorrent}'s~\cite{c_bittorrent_2003} locality 
by carefully choosing most of the neighbors of a peer to come from 
the same \emph{ISP}, while at the same time adhering to the near-optimal
download performance of the protocol.
%%
By and large, {\sl BitTorrent} deploy mechanisms that have proved
aggressive for \emph{ISP}s' networking and accounting as they function 
in a ``without-borders'' approach. The default {\sl BitTorrent} specification
designates $35$ connections for each peer regardless if such connections are
placed within or outside the borders of the \emph{ISP} that hosts the node.
%%
\emph{BNS} proposes that only $k$ out of these $35$ connections are 
established with nodes beyond the {\it ISP}-borders;
the $k$ connections offer an extended and ``global'' view of 
the network at large and seek to strike a balance between 
the load imposed inside and outside the \emph{ISP}.
The latter refrains peers from exchanging traffic through
expensive network links should there exist alternative local connections
offering the sought services. 
\emph{BNS} is realized either through 
the modification on the tracker so that \emph{ISP}-local resources are rapidly 
identified,
or through \p\ traffic shaping devices installed on \emph{ISP} edge routers. 
%%
In~\cite{BCCMSBZ2006}, a combination of 
biased neighbor selection along with bandwidth throttling
and caching techniques for near-optimal lookup results is proposed.
Unfortunately, \emph{BNS} deployment requires knowledge of 
\emph{ISP} network mappings and awareness of possible changes 
that occur in the infrastructure limiting its widespread adoption;
the following table qualitatively depicts the expected behavior of \emph{BNS}:
%%
\begin{center}
{\footnotesize
\begin{tabular}{ccc}
\emph{Efficiency} & \emph{Overhead} & \emph{Scalability} \\
\hline
% inter-ISP cost minimized while retaining the near optimal performance of
% BitTorrent download.
high &
% this info is taken at swarm join
low &
% trackers should maintain and update proximity info/proximity info is
% retrieved from special hardware.
medium
\end{tabular}
}
\end{center}

%%%%%%%%%%%%%%%%%%%%%%%%%%%%%%%%%%%%%%%%%%%%%%%%%%%%%%%%%%%%%%%%%%%%%%%%%%%%%%%%
\emph{Ono}~\cite{CB2008} is a protocol that helps contain
{\sl BitTorrent} traffic within individual \emph{ISP}s
and enhances the downloading rates by favoring connections
within the borders of a single autonomous system. Contrary to
\emph{BSN}~\cite{BCCMSBZ2006},
\emph{Oro} leads to improved performance 
without requiring any explicit cooperation between 
\emph{ISP} subscribers or any additional
infrastructure and network topology information.
\emph{Oro}'s selection approach is 
landmark-based and leverages existing {\sl CDN}-infrastructure 
for peer distance estimation. {\sl CDN}s already use both static (i.e.,
geographical, IP-clustering) and dynamic (i.e., network delay measurement)
information for their replica selection. Thus, the algorithm leverages the
observation that peers that are redirected to a specific CDN replica does not
only mean that these are close by this replica but to each other as well.
%%
The redirection behavior is modeled in terms of \emph{ratio map}, 
a vector of \emph{$<$replica-server,ratio$>$} tuples, where \emph{ratio} is
the percentage of times the {\sl CDN} redirects the peer 
to the specific \emph{replica-server}.
%%
The protocol's bootstrapping phase consists of the newcomming peer performing
{\sl DNS} lookup to {\sl CDN}-names to build its redirection information.
%%
Initially, the {\sl DNS} polling interval is set to $30$ seconds 
and increases by $1$ minute every time redirection information 
to the {\sl CDN} remains unchanged after the lookup.
The {\sl DNS} polling interval is halved if pertinent 
redirection information has changed.
To avoid the bootstrapping phase when the \emph{ratio map} 
is sufficiently fresh, 
\emph{Oro} persists the map after the end of a BitTorrent session.
%%
\begin{center}
{\footnotesize
\begin{tabular}{ccc}
\emph{Efficiency} & \emph{Overhead} & \emph{Scalability} \\
\hline
% Authors claim minimization of inter-ISP traffic cost while enhancing the,
% already, near optimal performance of BitTorrent download.
high &
% periodic DNS lookups and CDN redirection imposes medium control overhead to
% the method.
medium &
% the use of CDNs make the approach not fully distributed thus less scalable
low
\end{tabular}
}
\end{center}

%%%%%%%%%%%%%%%%%%%%%%%%%%%%%%%%%%%%%%%%%%%%%%%%%%%%%%%%%%%%%%%%%%%%%%%%%%%%%%%%
% \subsubsection{Locality-Awareness in BitTorrent-like P2P Applications}
Three approaches that ``inject'' locality awareness 
in {\sl BitTorrent}-like applications are discussed in~\cite{LCLX2009}. The
first \emph{macroscopic-level} approach
focuses on improving the neighbor selection process. 
When a peer asks a tracker to join, 
the later sorts its swarm peers according to
their hop-count distance from the requesting peer 
send out the top-$k$ list of nodes (e.g., $k$=$50$).
%%
The second approach, applied in an \emph{intermediate-level}, manipulates the
{\sl BitTorrent} chocking/unchocking mechanism. A peer unchockes its $4$
closest in terms of 
% VM I put ``autonomous system'' back in again because in the context of this
% paper we are dealing on the number of AS crossings (one-hop = one AS crossing)
% untli we reach a neighbor.
Autonomous System
hop-count interested neighbors. 
The same applies also when the peer turns to 
the seeding state;  this  \emph{intermediate-level} approach favors
least distance and is in contrast to the original {\sl BitTorrent} 
implementation which favors uploading speed.
%%
Finally, the third approach that operates at 
a \emph{microscopic-level}, substitutes the rare-first chunk picking of the
original {\sl BitTorrent} protocol that picks the rarest of the available
chunks to download next, with a locality-first policy which picks first the
chunck that is closer. The distance value of a chunk is computed as the mean
value of the distances of the peers that posses it.
Through experimentation on a real-world Internet topology simulated on
PlanetLab, the locality-based approach achieves traffic optimization.
Nevertheless, it does not do that well when compared with the
random approach of the standard {\sl BitTorrent} protocol.

All methods target in exposing the locality of the basic functions of the
BitTorrent protocol. Even though they do not restructure the network to map
better to the underlying infrastructure they favour the intra-AS communication
over inter-AS communication thus reducing the general cost of communication.

\begin{center}
{\footnotesize
\begin{tabular}{ccc}
\emph{Efficiency} & \emph{Overhead} & \emph{Scalability} \\
\hline
% minimization of inter-AS traffic but unfair among peers.
medium &
% info (AS hop distances) acquired at swarm join
low &
% does not result in a better map to the underlying topology, it just eases
% the implications of complete locality unawareness.
medium
\end{tabular}
}
\end{center}

%%%%%%%%%%%%%%%%%%%%%%%%%%%%%%%%%%%%%%%%%%%%%%%%%%%%%%%%%%%%%%%%%%%%%%%%%%%%%%%%
% \subsubsection{TopBT}
\emph{TopBT}~\cite{RTLCGZ2010} seeks to enhance 
proximity awareness in the {\sl BitTorrent}
without the need of any additional infrastructure. The key conjecture upon
which \emph{TopBT} is built is that
a good peer selection metric should take into account both the
downloading speed and the network topology. 
In this context, the approach suggests a metric that designates
peers demonstrating high download rate, low reciprocal upload demand 
and low routing hops.
The metric is defined as the ratio of download to upload rate ${d}/{u}$
divided by the either the link-level $l$ or the number of routing hops required $a$.
This metric should guarantee selection of peers with high download rate, low
reciprocal upload demand and decreased routing hops.
The above \emph{TopBT}-metric is applied in various
aspects of the original {\sl BitTorrent} protocol to ``inject'' topology
awareness for improved peer selection. 
%%
In an effort to essentially improve on the inherent topology mismatch problem,
\emph{TopBT} modifies the peer list the tracker returns when someone first tries
to connect to the swarm, the initial connection establishment, and the
unchoking mechanism in the {\sl BitTorrent}. The important feature of TopBT is
that it does not need wide deployment of its same types in order to see results.
It also supports both link-hop and AS-hop metrics to identify proximity in
different levels.
%
%A peer that runs the TopBT protocol evaluates its neighbors by periodically
%emitting pings or trace-routes in order to unchock those peer that exhibit less
%hops to reach and higher download rates. \emph{Link-hops} are measured by using
%the TTL value that the originator receives as a response from the remote
%peer\footnote{Initial TTL values are known for the different operating systems
%(e.g. 64 for Linux or 128 for Windows NT/2000/XP) so the originator can
%calculate the hops using the value of the TTL on the response message it
%receives.}. Also the peer, when off-line, builds a table that maps IPs to
%\emph{AS-hops}, using BGP dumps.
%%
Through experimentation on hundreds of {\it PlanetLab} and residential hosts,
\emph{TopBT} offers more than 25\% traffic reduction and 15\% faster downloads
if compared with popular {\sl BitTorrent} implementations.
In reference to the stated $3$ criteria, \emph{TopBT} fares as follows:
\begin{center}
{\footnotesize
\begin{tabular}{ccc}
\emph{Efficiency} & \emph{Overhead} & \emph{Scalability} \\
\hline
% it takes into account both dl speed and network topology. The later though is
% rather coarse grained since it is taken into account on the AS level.
medium &
% the algorithm does not involve extensive calculations.
low &
% Does not need infrastructure and the calculations are distributed so it can
% scale
high
\end{tabular}
}
\end{center}

%%%%%%%%%%%%%%%%%%%%%%%%%%%%%%%%%%%%%%%%%%%%%%%%%%%%%%%%%%%%%%%%%%%%%%%%%%%%%%%%
% \subsubsection{Underlying Topology-Aware Peer Selection (UTAPS)}
The \emph{Underlying Topology-Aware Peer Selection (UTAPS)}~\cite{LCY2008}
is an effort to furnish an enhanced peer selection strategy. \emph{UTAPS}
essentially works in two stages: in the first,
it collects information to develop a picture of the underlying topology 
and in the second, it proceeds to make the peer selection based on 
this knowledge. In the first stage, \emph{UTAPS} employs
network tomography, a technique that probes from a large network's 
endpoints to infer its internal characteristics~\cite{chny_tomography_2002}. 
%%
Upon arrival of a new peer, the tracker trace-routes 
it to obtain some basic knowledge 
including IP address, routers involved, \emph{RTT} and number of 
hops traversed.
The more the peers in the swarm the better 
picture a tracker has for the underlying infrastructure.  
The tracker provides this information to the newcomer in the form
of the bootstrapping peer list. These newly joined peers traceroute the tracker
peer list themselves and return back to the tracker in order to further enhance
the tracker's view of the infrastructure.
%%
In the second stage, \emph{UTAPS} employs a set of heuristics
which target those peers that expose low \emph{RTT} and are found within
certain hop-counts away from
the routers observed during the initial traceroute.

It is reported in~\cite{LCY2008} that peers running \emph{UTAPS}
instead of the random peer selection, achieve better downloading rates
and at the same time offer reduced burden on the underlying \emph{ISP}s. 
\emph{UTAPS} stands as follows in regard to the $3$ criteria:
%%
\begin{center}
{\footnotesize
\begin{tabular}{ccc}
\emph{Efficiency} & \emph{Overhead} & \emph{Scalability} \\
\hline
% coarse grained approach of the network tomography doesn't provide a detailed
% picture of the underlying network that the peers can then exploit
medium &
% trace-route and hop counts and calculation from both the tracker and the peers
% (the later for all peers in the bootstrapping list) increases the control
% overhead of the approach.
medium &
%
medium
\end{tabular}
}
\end{center}

%%%%%%%%%%%%%%%%%%%%%%%%%%%%%%%%%%%%%%%%%%%%%%%%%%%%%%%%%%%%%%%%%%%%%%%%%%%%%%%%
% \subsubsection{An Effective Network-Aware Peer Selection Algorithm in
% BitTorrent}
A clustering approach that differentiates peers encountered in a swarm
in those that are local, intra-\emph{ISP} and inter-\emph{ISP} is 
presented in~\cite{QLZG2009}.
The classification is carried with the help of the core routers used 
while peers communicate with the tracker. It is assumed that even though the
network may be highly volatile, these core routers are are usually more stable.
For this, a newly arriving
{\sl BitTorrent} peer, trace-routes the tracker and stores vectors containing
info like the IP address and the hop number of a traced router, as well as the
link delay of the traced router and the previous hop along the trace-route path.
These vectors are reported to the tracker and the latter uses this information
to classify the router through a $k$-means classification algorithm, with $k = 3$. 
For peer selection, \cite{QLZG2009} proposes a biased approach 
where the tracker returns $c$ close peers 
and $d\;=\;N-c$ distant peers, where $N$ is the length of the returned list. 
In choosing $c$, the tracker employs an iterative search process
first in the peer's local neighborhood. If not sufficient number of 
peers are available there, the tracker goes to intra-\emph{ISP} 
and subsequently, if needed, to the inter-\emph{ISP} cluster. 
%%
Experimentation in a controlled environment indicates up to
5\% faster downloading times as well as up to 22\% reduction of 
the total cross-\emph{ISP} traffic.
This clustering approach fares as follows in regard 
to the $3$ criteria:
%%
\begin{center}
{\footnotesize
\begin{tabular}{ccc}
\emph{Efficiency} & \emph{Overhead} & \emph{Scalability} \\
\hline
% It incorporates clustering of peers to inter- and intra ISP which is a
% rather coarse grained method.
medium &
% only on peer arrival proximities to routers are estimated and sent to the
% tracker. Even though the tracker is backing the identification of a peer
% being close or distant the actual computation is done in a distributed manner
% thus not incurring too much additional controlling overhead.
medium &
% TODO: how intensive with regards to computing power is an on-line k-means
% (with k=3) classification???
% VM: do we have an idea?
medium
\end{tabular}
}
\end{center}

%%%%%%%%%%%%%%%%%%%%%%%%%%%%%%%%%%%%%%%%%%%%%%%%%%%%%%%%%%%%%%%%%%%%%%%%%%%%%%%%
% \subsubsection{Peer-exchange Routing Optimization Protocols (PROP)}
In~\cite{QCYCZ2007}, two protocols termed 
\emph{Peer-exchange Routing Optimization Protocols (PROP)} 
are introduced to help adjust the neighborhood graph of an overlay network
while aiming at reducing the network's overall link latency.
\emph{PROP} algorithms are weaved around the concept of
``exchange'' of neighbors among peers; 
this is carried out so that participant node can collectively 
benefit from the attained reduction in network delays. 
%%
This ``collaboration'', is what differentiates this approach 
from others by allowing two peers to optimize
their neighborhood environment, than simply letting each node to ``selfishly''
choose its own strategy. 

%%
%% VM ADDING SOME OPERATIONAL DETAILS BELOW.
The PROP algorithms operate, basically, into two phases. During the \emph{warm-up}
phase a node $u$ probes its neighbors to collect distance information. Then it
contacts, at a fixed time rate $timer$, a random node $v$, $nhops$ away from it. New distance
information is calculated independently by nodes $u$ and $v$, now for the
hypothetical state when the potential exchange would occur. If the benefit
gained, in terms of reducing the average distance among the peers involved, is
above some predefined threshold then the exchange will actually occur,
otherwise no further action will be taken. The \emph{maintenance} phase, on the
other hand, differs from the warm-up one, in that the probability
of peers to be probed and the $timer$ interval that this happens, now depends on
past peer exchange results.

In the generic form of the \emph{PROP-G} protocol, a peer can exchange all of
its neighbors with those of another peer. This essentially results in the two
nodes exchanging positions so the topology and connectivity of the overlay is
not affected by the operation of the algorithm. The optimized version
\emph{PROP-O} permits peers to exchange selected sets of their neighbors. It is
not allowed to exchange neighbors allong the path of the nodes that perform the
exchange in order to ensure that in the end of the process the peers will remain
connected. Additionally, the algorithm needs to preserve the characteristics of
the network (i.e., the graph remains connected, preserve the degree of the
nodes), so the exchange always involves equal number of neighbors.

The following table depicts the behavior of the \emph{PROP} 
algorithms regarding the stated criteria:
%%
\begin{center}
{\footnotesize
\begin{tabular}{rccc}
\multicolumn{1}{r}{} &
\multicolumn{1}{c}{\emph{Efficiency}} &
\multicolumn{1}{c}{\emph{Overhead}} &
\multicolumn{1}{c}{\emph{Scalability}}
\\
\cline{2-4}
\emph{PROP-G} &
% The authors show that an Internet-like underlying physical topology has much
% better performance
medium &
% Exchanging neighbors within a radius of 2 (TTL value) reduces stretch
% significantly, while keeping the additional overhead at reasonable levels
low &
% The effectiveness of the scheme is reduced while the system size becomes
% larger, but as the number grows, this reduction becomes smaller
medium \\
\emph{PROP-O} &
% The algorithm can better reclaim the heterogeneity of peers (those with faster
% response) to reduce the system's aggregate delay. 
medium &
% Same as above
medium &
% The queries are directed through fast nodes so it is important such nodes
% be part of the network
medium \\
\end{tabular}
}
\end{center}

% TODO: SOME DISCUSSION
% PROP-G can be applied to both structured and unstructured overlays.

%%%%%%%%%%%%%%%%%%%%%%%%%%%%%%%%%%%%%%%%%%%%%%%%%%%%%%%%%%%%%%%%%%%%%%%%%%%%%%%%
% \subsubsection{Resolving the Topology Mismatch Problem in Unstructured
% Peer-to-Peer Networks}
% TODO: to be reviewed

% In \cite{hsiao_redblue_2009}, Hsiao et al, claim to construct topology-aware
% unstructured overlays that \emph{guarantee} performance qualities in terms of
% \begin{inparaenum}[\itshape i\upshape)]
%   \item the expected communication latency among any two overlay peers
% regardless of the network size, and
%   \item the broadcasting scope of each participating peer
% \end{inparaenum}
% . The algorithm constructs an undirected graph $G = \left( V, E \right)$
% comprised by two subgraphs. The first, namely $G^{\left( red \right)} = \left(
% V^{\left( red \right)}, E^{\left( red \right)} \right)$ in the paper's context,
% includes all vertices of $G$ and ensures the connectivity of the graph by
% securing at least one path between any two nodes. In contrast, $G^{\left( blue
% \right)} = \left( V^{\left( blue \right)}, E^{\left( blue \right)} \right)$,
% contains those vertices of $G$ that have free edges to link to other nodes and
% because these are fully utilized, the following also stands $E = E^{\left( red
% \right)} \cup E^{\left( blue \right)}$. A joining peer $u$, partitions its
% neighbors into two subsets, the $B_u^{\left( red \right)}$ and $B_u^{\left(
% blue \right)}$. In order to populate the $B_u^{\left( red \right)}$ subset, peer
% $u$ samples peers uniformly and at random. Then, each of these selected peers
% discovers a routing path starting from itself towards the node with the smallest
% (or the largest) ID in the system.

% TODO: CHECK ONCE AGAIN
% \begin{center}
% \begin{tabular}{ccc}
% \emph{Efficiency} & \emph{Overhead} & \emph{Scalability} \\
% \hline
%
% ? &
%
% ? &
%
% ?
% \end{tabular}
% \end{center}

%%%%%%%%%%%%%%%%%%%%%%%%%%%%%%%%%%%%%%%%%%%%%%%%%%%%%%%%%%%%%%%%%%%%%%%%%%%%%%%%% 
% \subsubsection{Distributed Domain Name Order (DDNO)}
The \emph{Distributed Domain Name Order (DDNO)} approach~\cite{Z-YK2005} 
uses domain names to detect topologically-close nodes.
The fundamental assumption of the approach is the nodes found in 
the same domain are also topologically close.
%%
\emph{DDNO}'s outcome is a flat overlay topology which, with some adjustments, 
can be utilized in super-peer architectures as well. According to the algorithm,
half of the possible connections of a node are used to connect to local peers
(called \emph{sibling} connections) and rest are used to randomly
connect to the peers anywhere on the network (called \emph{random} connections).
The former ensure the reduction of long distance traveling for messages, while
the later secure the connectivity of the structure and prevent network
partitioning.
%% VM NEW ATTEMPT TO EXPLAIN THE ALGORITHM
Upon arrival the newcomer peer $n$ asks a \emph{host-cache} to establish its
random connections. For establishing its sibling connections, the newcomer
multicasts a specialized \emph{message} $l$, to locate the right candidate(s).
Initially, the behaviour of this message, is modeled as a random walker,
until $l$ reaches some peer $d$ capable to guide it through its next step.
This capability is enabled through a \emph{ZoneCache}, a data structure that
contains information for nodes accessible in an $r$-hop radius from $d$.
Ultimatelly $l$ will reach a node $m$, candidate to become $n$'s sibling. $m$
will issue a broadcast message to its own siblings and then all togeather will
inform the initiating peer $n$ of who is willing to accept new connections.

The population and maintenance of ZoneCaches across the network is an important
task for the efficiency of the constructed overlay. While seeking to address the
mismatch problem, \emph{DDNO} is a heuristic approach that uses
\emph{Split-Hash} and \emph{dnMatch} algorithms to effectively encode domain
names and determine domain locality respectively. In terms of the $3$ criteria,
\emph{DDNO} fares are follows:
\begin{center}
{\footnotesize
\begin{tabular}{ccc}
\emph{Efficiency} & \emph{Overhead} & \emph{Scalability} \\
\hline
% Clustering approaches reduce search scope (or at least increase search
% satisfaction time). The algorithm tries to ease this problem by allowing some
% connections to be random
medium &
% overhead of split-hash and dnMatch is of low cost (TODO: double check this).
% Walker message is multicasted only when sibling connections must be
% established.
low &
% it can be utilized in hierarchical architectures as well meaning increased
% scalability. Tests show that an increase from 5K- to 10K-node system slightly
% increases the average hops needed to satisfy a lookupDN message.
medium
\end{tabular}
}
\end{center}

%%%%%%%%%%%%%%%%%%%%%%%%%%%%%%%%%%%%%%%%%%%%%%%%%%%%%%%%%%%%%%%%%%%%%%%%%%%%%%%%
% \subsubsection{Critical Topology-Aware Grouping (CTAG)}
The \emph{Critical Topology-Aware Grouping (CTAG)} approach~\cite{ZL2006}
advocates a grouping strategy of peers based on the 
\emph{Internet Assigned Numbers Authority (IANA)} and 
the respective \emph{Regional Internet Registry (RIR)}'s IP assignments.
%% VM CHANGED DESCRIPTION
Based on these assignments, nodes that belong in the same 
organization are always addressed from the same block of IP addresses.
\emph{CTAG}, exploits this observation to construct topology-aware unstructured
overlays. On top of this, \emph{Adjacency Measurement (AM)} is proposed, as
a technique which uses the longest matching \emph{IP} segment criterion to
compute node proximity. \emph{CTAG} focuses on both the construction of the
overlay as well as its constant and dynamic adaptation.
In the first phase,
called \emph{bootstrapping grouping},
the {\sl Gnutella} web caching mechanism
is modified so that a newcomer may choose the closest, in terms of AM,
cache to retrieve the bootstrapping list. Similarly, during the second phase,
called \emph{dynamic revision}, the standard connection establishement mechanism
is also altered so that nodes with the highest AM metric are chosen. When a node
reaches its maximum number of neighbor
connections, it disconnects those neighbors with the lowest AM.
\emph{CTAG} behaves as follows regarding the $3$ criteria:
\begin{center}
{\footnotesize
\begin{tabular}{ccc}
\emph{Efficiency} & \emph{Overhead} & \emph{Scalability} \\
\hline
% The technique partitions the system reducing the search efficiency
low &
% IP-based clustering involves low overhead
low &
% Has average scalability mainly to its low overhead. In larger applications the
% clustering of the system prevents the system from scaling smoothly.
medium
\end{tabular}
}
\end{center}

%###############################################################################
%###############################################################################
%###############################################################################
%       LANDMARKING
%###############################################################################
%###############################################################################
%###############################################################################


%%%%%%%%%%%%%%%%%%%%%%%%%%%%%%%%%%%%%%%%%%%%%%%%%%%%%%%%%%%%%%%%%%%%%%%%%%%%%%%%
% \subsubsection{Landmark Binning (LM)}\label{sec:landmark_binning}
% \begin{figure*}
% \centering
%   \includegraphics[scale=0.4]{img/algorithms/landmark_binning}
% \caption{Example 2D coordinate overlay and a sample routing path from node D to
% (x,y).}
% \label{fig:landmark_binning}
% \end{figure*}

\emph{Landmark Binning}~\cite{RHMKS2002} partitions close-by nodes 
into bins based on their distance from well known anchor nodes across the
Internet.%, as shown in Figure~\ref{fig:landmark_binning}.
To detect locality, peers mainly use network latency (i.e. \emph{RTT}).
%  as a measurement technique. 
Despite the fact that such delays are not always accurate,
they are used in~\cite{RHMKS2002} for they are 
non-intrusive, transparent and easy to apply.
%%
For the binning to work, a few anchor servers with known
physical locations need to be installed in strategic positions 
across the the Internet. 
It is conjectured~\cite{RHMKS2002} that $12$ such servers can prove
sufficient for the task.
An arriving node measures its distance from these landmarks
and unilaterally decides to join a specific bin based on its measurements.
Specifically, the node measures its round-trip time to each of the landmarks and
orders the resulting \emph{RTT}-values in a decreasing order. 
%%AD I do not understand what the following sentence says..
%%	for each peer there is a different bin right?
%
%The ordering represents a \emph{bin},
%in the sense of close-by nodes having the same landmark ordering and hence
%belong to the same bin. 
%
%% VM I CHANGED THE ABOVE AS BELOW TO REPHRASE IT.
Each permutation of the set of landmarks represents a specific bin. Should there
$m$ landmarks be adopted, potentially $m!$  different bins exist. Peers that
end up with the same such ordering, belong to the same bin, in the sense that
if they experience similar overall latencies from fixed network points then it
is very likelly that they are close to each other.
%%
The operation of the method is independent of the model 
incorporated by the overlay network and thus it can be
applied with no significant changes to 
both structured and unstructured \p\ systems.
%%
The main disadvantage of the approach is that 
landmark servers must be installed and maintained throughout the world.
Provided that a \p-system often may have a few million nodes 
connected at any given point in time, landmark servers do 
inherently play a pivotal role in the operation of the network.
Although latency estimation does not drain network resources,
the scale of the approach does call for an alternative design.
In this respect, \cite{RHMKS2002} advocates the replacement of 
each landmark server by a cluster of servers.
However, this approach may not entirely avoid the creation of 
excessive network traffic flow through its landmark server-clusters.

%%%%%%%%%%%%%%%%%%%%%%%%%%%%%%%%%%%%%%%%%%%%
%
%%
%To answer the question as to whether the algorithm actually contributes
% positively to the construction of an enhanced overlay, the paper defines the
%\emph{gain ratio} as the factor by which the latency reduces when someone
%communicates with a random node from the same bin than with one not in the bin.
%This is implemented with an inter-bin to an intra-bin latency ratio.
%
% TODO: LANDMARK BINNING FOR UNSTRUCTURED OVERLAYS
%
%For unstructured overlays the paper assumes \emph{a set of $n$ nodes where each
% node picks any $k$ neighbor nodes so that the average routing latency on the
%resultant overlay is low (assuming shortest path routing)}. According to the
%proposed heuristic algorithm called \emph{BinShort-Long}, a node picks its
%neighbors by choosing its $\frac{k}{2}$ closest\footnote{If the node's bin is
%not large enough for it to pick these $\frac{k}{2}$ neighbors, it picks the
%required nodes from the bin that matches the most in terms of landmark
%ordering.} ones (named \emph{short links}), using the \emph{binning} scheme and
%the rest $\frac{k}{2}$ randomly (\emph{long links}). The former set produces
%well-connected \emph{pockets} of nearby nodes while the later preserves the
%connectivity of the graph, both yielding a proximity factor of $\alpha = 0.5$
%in an attempt to preserve the beneficial properties of unstructured
%topologies\cite{merugu_str2unstr_2003}.
%
% TODO: SOME DISCUSSION
%
%A potential bottleneck could be the extra load that this
%``ping''-like scheme imposes to the landmarks, especially when we need instant
%reaction from our topology when dealing with the dynamic nature of the p2p
%networks.
%
%One disadvantage of this landmark scheme is related to the additional burden
% imposed to the landmark sites. The authors claim though that the algorithm
%requires so little work by the landmarks (maybe just echo to ping messages)
%that could in effect, act as ``unsuspecting participants''. Even if this is the
%case, the fact that it is not fully distributed, renders the protocol's
%scalability directly vulnerable to any system size increase as well as
%suitable for highly dynamic networks such as ad-hoc networks. Moreover, fixed
%points in a network are inherently more exposed to malicious attacks. The most
%significant downside of the algorithm though is that it can lead to an
%extremely uneven overlay ID distribution causing load unbalances and hot spots.
%Lastly, the scheme is coarse grained when it comes to distinguishing relatively
%close nodes\footnote{In the worst case, all nodes could ve clustered into a
%single bin.}.
%
%%
The landmark binning behaves as follows:
\begin{center}
{\footnotesize
\begin{tabular}{ccc}
\emph{Efficiency} & \emph{Overhead} & \emph{Scalability} \\
\hline
% The technique is coarse grained thus doesn't achieve optimal results
% (especially in small networks)
medium &
% The algorithm needs only nodes to compute distances to a small number of
% predefined nodes without exchanging any additional information.
low &
% The introduction of landmark servers renders the approach not fully
% decentralized, thus preventing it from scaling smoothly. Communicating
% and overloading landmark servers in high-churn systems is another scalability
% concern.
low
\end{tabular}
}
\end{center}

%%%%%%%%%%%%%%%%%%%%%%%%%%%%%%%%%%%%%%%%%%%%%%%%%%%%%%%%%%%%%%%%%%%%%%%%%%%%%%%%
% \subsubsection{mOverlay}
The \emph{mOverlay} \cite{ZZZSZ2004} approach addresses scalability issues
that might arise when static landmark servers are in use.
To this end, the use of dynamic landmarks is proposed.
\emph{mOverlay}'s founding notion is that of a \emph{group} that 
designates a set of peers found in close proximity.
This proximity is user-defined in the protocol and 
may involve metrics including \emph{RTT}s and network latencies.
By and large a clustering approach, \emph{mOverlay} seeks to 
recreate small-world-like properties by producing a 
two-level hierarchical structure:
at the top level, there are only connections among groups 
while at the bottom, only \emph{intra-group} connections occur among peers.

%%AD how do you identify the initial groups????
%%
%% VM REPHRASING IN ORDER TO ANSWER THE QUESTION ABOVE
Clearly, identifying groups and accurately finding the closest group 
to a peer is a fundamental concern in the creation of the overlay.
Nodes are grouped based on their distance to 
the groups already in the network, rendering
the latter be the \emph{dynamic landmarks} in the process. For a newcomming
peer $Q$ the \emph{grouping critirion} says that when the distance of $Q$ and
some group $A$’s neighbor groups is the same as the distance
between group $A$ and group $A$’s neighbor groups, then host $Q$
should belong to group $A$. During network initialization or when grouping
criterion is not met, new groups are created.
It is shown that in the proposed scheme, a new peer can 
reach its group by expending at most $O(logN)$ messages
within the network. 
Finally, \emph{mOverlay} maintains stability and constant 
overheads when a host either fails or departs the network;
this is achieved through periodic cache updates and group 
leader selections, should one either leaves or dies.
%
%\paragraph{Locating process} A new coming host, $Q$, first connects to a
%globally known host cache called the \emph{rendezvous point (RP)} in order to
%retrieve the starting point in the overlay, say $A$ in group $1$. Host $Q$
%then, measures its distance to host $A$. At the same time, the later, sends
%information about the neighbor groups of group $1$ back to host $Q$. This list
%is called \emph{candidate group list}, and the new coming host sequentially
%measures its distance to each of them in seek for the closest one. If the
%\emph{grouping criterion} is met, host $Q$ belongs to group $1$. If not, a boot
%host from the closest group is found and the algorithm is re-run until the
%criterion is met or after a predefined number of repetitions. In the later
%case, $Q$ creates a new group comprising itself only. The above protocol does
%not favor hot-spots as it spreads the probability of visiting a group across
%the whole overlay and limits the overhead in the level of $O \left ( log N
%\right )$.
%
%\paragraph{General overlay operations} A set of additional protocols, are also
% introduced, similar to those found in traditional unstructured networks, but
%modified focusing on scalability and robustness. For example a protocol for
%\emph{group formation} is introduced that exploits the inherent characteristic
%of proximity, in the overlay, in order to efficiently detect the neighboring
%groups of a newly formed group from the set of adjacent groups of its closest
%neighbor. Additionally, during \emph{group joining} the corresponding protocol
%denotes the exchange of important information for group maintenance. This can
%be further improved by \emph{information sharing} between nodes of the same
%group, functionality handled by a dedicated flood-like protocol\footnote{Since
%nodes that belong to the same group are physically close this can be achieved
%at a minimum price.}. Moreover, another set of distributed protocols handle the
%\emph{information update}. The information that needs update, in the proposed
%architecture, is
%\begin{inparaenum}[\itshape i\upshape)]
%  \item the host cache, when a new node joins, and
%  \item the neighbors of groups, when a close-by group is generated.
%\end{inparaenum}
%Finally, in case of \emph{host failure} or \emph{host departure} the system is
% able to maintain its stability since there are defined operations for
%periodical host cache update and group leader selection if one leaves or dies.
%
%%
In terms of the stated criteria, \emph{mOverlay} fares as follows:
\begin{center}
{\footnotesize
\begin{tabular}{ccc}
\emph{Efficiency} & \emph{Overhead} & \emph{Scalability} \\
\hline
% The technique is coarse grained thus doesn't achieve optimal results
% (especially in small networks)
medium &
% The algorithm is iterative through the available groups. At each group
% probing of a candidate list must be performed. This process is done at
% bootstrapping time so the overhead increases in high-churn systems
medium &
% The introduction of dynamic landmark servers renders this approach much more
% scalable than the traditional static landmarking techniques 
medium
\end{tabular}
}
\end{center}

%%%%%%%%%%%%%%%%%%%%%%%%%%%%%%%%%%%%%%%%%%%%%%%%%%%%%%%%%%%%%%%%%%%%%%%%%%%%%%%%
%%%%%%%%%%%%%%%%%%%%%%%%%%%%%%%%%%%%%%%%%%%%%%%%%%%%%%%%%%%%%%%%%%%%%%%%%%%%%%%%
\subsection{Discussion on the Algorithms for Unstructured Architectures}
%%%%%%%%%%%%%%%%%%%%%%%%%%%%%%%%%%%%%%%%%%%%%%%%%%%%%%%%%%%%%%%%%%%%%%%%%%%%%%%%
%%%%%%%%%%%%%%%%%%%%%%%%%%%%%%%%%%%%%%%%%%%%%%%%%%%%%%%%%%%%%%%%%%%%%%%%%%%%%%%%




%%%%%%%%%%%%%%%%%%%%%%%%%%%%%%%%%%%%%%%%%%%%%%%%%%%%%%%%%%%%%%%%%%%%%%%%%%%%%%%%
%
% TODO: HOW CAN UNSTRUCTURED SCHEMES BE REFINED
%
%For this reasons, efforts have been placed for optimizing the efficiency of
%decentralized unstructured peer-to-peer networks. Research mainly focuses on
%\begin{inparaenum}[\itshape i\upshape)]
%  \item reducing unnecessary, redundant communication traffic, and
%  \item exploiting physical locality to reduce communication response.
%\end{inparaenum}
%The goal can be achieved at, both, the application-level network as well as the
%underlying physical one. In the first case by refining the message relay
%techniques, while in the second one, by adaptively reconstructing the
%application network to map as well as possible to the the physical network.
%
%%%%%%%%%%%%%%%%%%%%%%%%%%%%%%%%%%%%%%%%%%%%%%%%%%%%%%%%%%%%%%%%%%%%%%%%%%%%%%%%

While surveying efforts to overcome the mismatch problem
in the area of unstructured \p\ systems,
we came to identify four key methodologies 
utilized by the discussed approaches; these methodologies are:
\begin{enumerate}[\itshape i\upshape)]
  \item topology adaptation
  \item forwarding optimization
  \item caching and replication, and
  \item landmarking
\end{enumerate}
%%VM Added the following to mention the hybrid/combo methodologies
An important aspect of many implementations, is that since the above
methodologies use heuristics to solve topology mismatch, an
intrinsically difficult problem, no single approach can offer a robust solution.
Thus, many of the algorithms we surveyed are actually combinations of multiple
such methodologies, in an attempt to refine the results of using just a single one.
For example, \emph{mOverlay} combines topology adaptation with landmark binning or \emph{Gia}
combines all three topology adaptation, forwarding optimization and caching
methodologies. Some times combination results in win/loose trade-offs. For
instance, the effectiveness of a caching/replication component can be undermined
by a continuously adapting overlay that removes important links between peers.
For this reason, the reader is advised to criticaly approach either the overall
algorithms or the separate methodologies when evaluating them for different
application domains.

Below, we discuss how the proposed protocols use elements of the four
methodologies and we offer a summary qualitative comparison for all surveyed
approaches applicable for \emph{unstructured} \p \emph{systems}.
%%VM I brought the following paragraph here from the end of this subsection.
Moreover, Table~\ref{unstructured:table} offers a summary overview for each effort,
showing which of the aforementioned methodologies it incorporates, along with its
highlights and a rough estimation of the positive and negative aspects of its
implementation.

%%%%%%%%%%%%%%%%%%%%%%%%%%%%%%%%%%%%%%%%%%%%%%%%%%%%%%%%%%%%%%%%%%%%%%%%%%%%%%%%
\subsubsection{Topology Adaptation Methodology}

Protocols based on topology adaptation 
modify the topology of the \p\ network
using various techniques. The two most commonly used approaches to
topology adaptation are creating
\emph{spanning tree}s using connection graphs
and creating \emph{clusters} of physically close nodes.

\emph{Narada} and subsequent algorithms including 
\emph{AOTO, LTM, SBO}, try to solve the problem
by building a richer connected graph and forming 
minimum spanning trees over
this graph that can efficiently route messages among peers. The \emph{AOTO, LTM}
and \emph{SBO} try to
overcome this limitation with ingeneous schemes
like forming minimum spanning trees
for the $2$-hop away neighbors for each node, seperating participant nodes into
groups, sometimes with different responsiblities and tasks at hand (e.g.,
\emph{SBO}). The advantage of building minimum spanning trees is that
they maintain the connectivity on the network in
an efficient manner while still preserving the overall search scope.
However, their construction and the update costs,
especially in dynamic, high-churn environments,
causes large additional traffic overhead on the underlying
network~\cite{CRZ2000,CRSZ2001,CRSZ2002}.

The cluster-based approaches, on the other hand, 
link physically-close nodes to each other. 
\emph{T2MC}, for example, uses trace-route logs and 
\emph{DDNO} exploits domain names to cluster nodes in proximity at peer join.
Further enhancements may include dynamic local restructuring of the overlay
graph through neighbour exchange like in \emph{PROP} or cycle-cut like in
\emph{DCMP} to achieve continoous adaptation thoughout a peer's lifecycle.
Unfortunately, commonly used methods for proximity detection across Internet do
not always return
reliable results and therefore, mapping accuracy is not guaranteed. Specifically
for traceroute, its overhead is not negligible and routers or firewalls in a
network may have already been configured to disable traceroute response from
the start. The most problematic aspect of clustering, though, is
its nature per se. 
Limited connectivity among the various local domains can
significantly shrink the search scope, negatively affecting the query response
time that the \p\ user experiences. Among others, \emph{DDNO} tries to 
balance the efficiency of the clustering approach 
with enhanced node connectivity by forcing half of each node's connections to be
with other, randomly selected, nodes.

As a methodology, topology adaptation, ultimately aims to reduce the 
average path traversing cost from one node to another.
In doing so, topology adaptation re-arranges the
overlay network so that it becomes a better fit 
for the underlying \emph{IP} network.
% as Figure~\ref{figure:topology-adaptation} depicts.
Despite the fact that the above is a plausible proposition,
it is not always advantageous though. For example, a topology adaptive
algorithm can exchange a slow edge in the overlay with a faster one
thus reducing the average latency of the network communication.
The pitfall here is that this new virtual link
can traverse a fast \emph{AS}--to--\emph{AS} link meaning that even though
message round-trip-time is reduced, it has actually additional cost in terms of
inter-ISP communication accounting and management.

\begin{figure}[ht]
\centering
\subfigure[Inefficient overlay topology averaging $\simeq 16$ delay units.] {
  \includegraphics[scale=0.4]{img/pdf/topology-adaptation-before.pdf}
  \label{figure:topology-adaptation:before}
}\qquad\qquad
\subfigure[Efficient overlay topology after adaptation averaging $\simeq 11$ delay units.] {
  \includegraphics[scale=0.4]{img/pdf/topology-adaptation-after.pdf}
  \label{figure:topology-adaptation:after}
}
\caption{A simplified example on how overlay topology adaptation may improve matters.}
\label{figure:topology-adaptation}
\end{figure}
%%

%%%%%%%%%%%%%%%%%%%%%%%%%%%%%%%%%%%%%%%%%%%%%%%%%%%%%%%%%%%%%%%%%%%%%%%%%%%%%%%%
\subsubsection{Forwarding Optimization Methodology}
Approaches based on forwarding-optimization 
propose intelligent forwarding link selection for a message's next
hop across a routing path. 
The selection criteria vary depending on the
algorithm, but they mainly use one or more statistical heuristics. 
These can range from simple connection speed, to exploiting specific peer
attributes such as high processing capacity e.t.c. An example is \emph{Gia} that
exploits the so called heterogeneity of peers as well as employs a token-based
flow control mechanism to guide the messages towards peers that are more likelly
to answer
a query. More sophisticated alternatives also exist,
taking into account application level requirements or exploiting machine
learning techniques to adjust to overlay network routing, like the \emph{APS}
algorithm.

Figure~\ref{figure:forwarding-optimisation} shows a simple example
of how forwarding optimization may work.
%%
\begin{figure}[ht]
\centering
\subfigure[Typical blind flooding.] {
\includegraphics[scale=0.20]{img/pdf/forwarding-optimization-before.pdf}
  \label{figure:forwarding-optimisation:before}
}\qquad\qquad
\subfigure[A node can decide where to forward the messages.] {
  \includegraphics[scale=0.20]{img/pdf/forwarding-optimization-after.pdf}
  \label{figure:forwarding-optimisation:after}
}
\caption{Forwarding optimization in constraining message flooding.}
\label{figure:forwarding-optimisation}
\end{figure}
%%
Fig.~\ref{figure:forwarding-optimisation:before} 
shows a protocol that simply floods the entire network 
in search of object found in the nodes colored in black.
Alternatively, a node can decide to forward its
messages not to every output link but to a specific 
or subset of its outgoing links.
In Fig.~\ref{figure:forwarding-optimisation:after}, 
the node in the middle, does not flood its neighbors as instead 
in only picks one of them.
The dashed line on the right, shows that, 
for this routing process, the protocol has
rendered two potential forwarding paths as inefficient (e.g., the target nodes
show overloading signs). 
On one hand, forward
optimisation approaches have 
the advantage of enhancing the
search responsiveness and reducing the aggregate resource 
usage of the physical network. On the other, 
they suffer from drastic reduction of the search
scope (in Fig.~\ref{figure:forwarding-optimisation:after} the object on
the far right is not reached) thus limiting the scalability of the whole
network. 
Moreover, they adress the problem of the mismatch in a limited manner
since they do not provide any guarantees 
that overlay and underlying topologies are
aligned with each other let alone quantify the mismatch 
and try to alleviate.
Forwarding based optimisations are commonly applied in
conjunction with other methodologies to improve 
the quality of the \p\ systems.


%%%%%%%%%%%%%%%%%%%%%%%%%%%%%%%%%%%%%%%%%%%%%%%%%%%%%%%%%%%%%%%%%%%%%%%%%%%%%%%%
\subsubsection{Caching and Replication Methodology}

Caching is widely used to exploit locality and minimize redundant
transfer of data. Caching has with much success been successfully 
adopted by web and file server application environments. 
Since peers in a \p\ system also operate as servers,
it is intuitively expected that \p\ file sharing systems can also benefit from
caching in improving performance and reducing overall resource usage. 
However, the design of caches in this context 
is non-trivial compared to the web-based caching. 
Due to the fact nodes play the double role of server and client,
two important issues have to be considered at design time. 
First, the lifetime of a query is short, as the nodes join 
and leave frequently. Second, the result
of a single query string is not always the same, as this depends on the
source of the query, the \emph{TTL} value set for the messages, 
the current interconnection of peers and the 
high volatility of the environment. 
Thus, to develop a successful caching system for a \p\ architecture, these
parameters also have to be carefully considered. 
\p\ caching/replication can be applied
at two different levels, namely caching indices or pointers to data 
(Fig.~\ref{figure:replication:index}) or caching the data itself 
(Fig.~\ref{figure:replication:data}). 
%%%
\begin{figure}[ht]
\centering
\subfigure[Indexing can reduce the cost of the last hop.] {
\includegraphics[scale=0.3]{img/pdf/replication-index.pdf}
  \label{figure:replication:index}
}\qquad\qquad
\subfigure[Data replication along the path of a successful query.] {
  \includegraphics[scale=0.3]{img/pdf/replication-data.pdf}
  \label{figure:replication:data}
}
\caption{Index and data replication strategies.}
\label{figure:replication}
\end{figure}
%%
Successful implementations have
already been developed in some commercial \p\ systems, like 
\emph{KaZaA} or less well known like \emph{Gia} and \emph{BNS} as we saw during
the survey.
%%%
Even though the state of the art in \p\ protocols using caching methods
helps reduce the burden of network resources, 
their contribution in addressing the actual mismatch between the overlay and
and the underlying networks remains limited.

% TODO SOME DISCUSSION
%The caching policy varies depending on the way protocol handles the
%index and the
%content. Centralized P2P systems
%use central index servers, while local caching systems, such as KazaA, use
%super peers
%to cache indices in a distributed way. Content caching is also possible in P2P
%systems, where nodes cache the forwarded content for further retrievals.
%Although caching has the above mentioned advantages,  duplication
%of messages still exist, which limits the scalability of these approaches.
%Therefore, cache based approaches are analyzed in the following categories:
%  \begin{itemize}
%    \item \emph{data index caching},
%    \item \emph{content index caching},
%    \item \emph{centralized}, and
%    \item \emph{local}.
%  \end{itemize}

%%%%%%%%%%%%%%%%%%%%%%%%%%%%%%%%%%%%%%%%%%%%%%%%%%%%%%%%%%%%%%%%%%%%%%%%%%%%%%%%
\subsubsection{Landmarking}\label{sec:landmark}

In landmark-based algorithms, nodes use network delay (e.g., \emph{RTT}) as a
distance measurement method to position themselves with respect to ``a priori''
known servers on the Internet, like \emph{Ono} which uses the CDN-infrastructure
for this purpose.
These landmark servers are used
by nodes to estimate their positions based on the intuitive assumption that
nodes
with similar distances to a set of landmarks, are physically close to each
other, as well, over the network. 
In Figure~\ref{figure:landmarking} the newly
arriving peer, measures its distance to an array of landmark point denoted as
$L_n$ and and makes a sorted list of the peers, say in increasing order. Then in
order to choose the already participating peers with which to connect to,
compares its list to the list of the potential neighbor. Peers with similar
measured distances to the these landmark points are likely to be close to
each other as well.
%%
\begin{figure}[ht]
\centering
  \includegraphics[scale=0.4]{img/pdf/landmarking.pdf}
\caption{Landmark binning during node bootstrap.}
\label{figure:landmarking}
\end{figure}

Landmark based protocols have four important drawbacks: First, the network
delay is not a reliable distance estimation method. For example, based on the
load on the network, the delay to certain nodes or networks can change from time
to time, which will eventually affect the distance measurements and wrong
measurements will lead to wrong estimated positions for the nodes, or incorrect
and non optimal clusterings of the nodes. Second, relying on predefined nodes
make the whole paradigm not fully distributed and the landmark system prone to
become a single point of failure. Third, using
landmark servers requires costly installation and maintenance of landmark
infrastructure across the whole Internet and for all the different autonomous
system domains. 
As popular \p\ file sharing applications usually have millions
of peers connected at any time, the network costs
of maintaining these landmark servers will likely be quite high. 
A possible solution to
the scalability problem of the static landmark servers is to use ordinary nodes
as dynamic landmarks, like in \emph{mOverlay} once they estimate their own positions. Even though this
approach scales much better than static landmark servers, the measurement
accuracy problem affects the overall performance of the system. Moreover,it can
be characterized as a rather coarse-grained approximation, therefore not
particularly well suited for detecting the correct positions of nodes within
close distance to each other.

%%%%%%%%%%%%%%%%%%%%%%%%%%%%%%%%%%%%%%%%%%%%%%%%%%%%%%%%%%%%%%%%%%%%%%%%%%%%%%%%
%\renewcommand\arraystretch{1.9}% (MyValue=1.0 is for standard)

\onecolumn

%\begin{landscape}
%\begin{figure}[h!]
\hspace{-3ex}
\begin{center}
\footnotesize
%\begin{tabular}{
\renewcommand*{\arraystretch}{1.6}
\begin{longtable}{
m{2cm}
m{0.35cm}
m{0.35cm}
m{0.35cm}
m{0.35cm}
m{3cm}
m{5cm}
}
% |>{\columncolor[gray]{.7}}m{0.1\columnwidth}
% |>{\columncolor[gray]{.9}}m{0.1\columnwidth}
% |>{\columncolor[gray]{.8}}m{0.1\columnwidth}
% |>{\columncolor[gray]{.9}}m{0.1\columnwidth}
% |>{\columncolor[gray]{.9}}m{0.1\columnwidth}
% |>{\columncolor[gray]{.8}}m{0.1\columnwidth}
% |>{\columncolor[gray]{.9}}m{0.1\columnwidth}
% |}
\caption[Summary table for unstructured algorithms]{Summary table for unstructured algorithms.} \label{unstructured:table} \\
%\hline
%%%%%%%%%%%%%%%%%%%%%%%%%%%%%%%%%%%%%%%%%%%%%%%%%%%%%%%%%%%%%%%%%%%%%%%%%%%%%%%%
% first head
%\rowcolor[gray]{.5}
\rot{\textbf{Algorithm / Paper}} &
\rot{\textbf{Topology Adaptation}} &
\rot{\textbf{Forwarding Optimization}} &
\rot{\textbf{Caching / Replication}} &
\rot{\textbf{Landmarking}} &
\rot{\textbf{Highlights}} &
\rot{\textbf{Pros / Cons}}\\
\hline
\endfirsthead
%%%%%%%%%%%%%%%%%%%%%%%%%%%%%%%%%%%%%%%%%%%%%%%%%%%%%%%%%%%%%%%%%%%%%%%%%%%%%%%%
% subsequent heads
\multicolumn{7}{c}%
{\tablename\ \thetable\ -- \textit{Continued from previous page}} \\
%\hline
%\rowcolor[gray]{.5}
\rot{\textbf{Algorithm / Paper}} &
\rot{\textbf{Topology Adaptation}} &
\rot{\textbf{Forwarding Optimization}} &
\rot{\textbf{Caching / Replication}} &
\rot{\textbf{Landmarking}} &
\rot{\textbf{Highlights}} &
\rot{\textbf{Pros / Cons}}\\
\hline
\endhead
%%%%%%%%%%%%%%%%%%%%%%%%%%%%%%%%%%%%%%%%%%%%%%%%%%%%%%%%%%%%%%%%%%%%%%%%%%%%%%%%
% foot
\hline \multicolumn{7}{r}{\textit{Continued on next page}} \\
\endfoot
%%%%%%%%%%%%%%%%%%%%%%%%%%%%%%%%%%%%%%%%%%%%%%%%%%%%%%%%%%%%%%%%%%%%%%%%%%%%%%%%
% last foot
\hline
\endlastfoot
%%%%%%%%%%%%%%%%%%%%%%%%%%%%%%%%%%%%%%%%%%%%%%%%%%%%%%%%%%%%%%%%%%%%%%%%%%%%%%%%
% data
\textbf{\cite{YG-M2002}} &
{\large \Square} &
{\large \CheckedBox} &
{\large \CheckedBox} &
{\large \Square} &
\begin{tabular}[l]{m{3cm}}
Iterative Deepening (ID).\\
Directed BFS (DBFS).\\
Local Indices (LI).
\end{tabular} &
\begin{tabular}[l]{m{5cm}}
+ ID reduces the messages especially in upper levels of the tree.\\
%+ DBFS ??????\\
+ LI reduces aggregate bandwidth usage and improves query efficiency.\\
-- ID needs evaluation time between iterations.\\
-- DBFS uses heuristics so it depends on their efficient choice.\\
-- LI add index update overhead which might be heavy process especially in high-churn systems.
\end{tabular}
\\
\hline
%%%%%%%%%%%%%%%%%%%%%%%%%%%%%%%%%%%%%%%%%%%%%%%%%%%%%%%%%%%%%%%%%%%%%%%%%%%%%%%%
\textbf{DAPS \cite{ZL2005}} &
{\large \Square} &
{\large \Square} &
{\large \Square} &
{\large \Square} &
\begin{tabular}[l]{m{3cm}}
Clustered routing tables based on delay.\\
Pruning flood, an iterative deepening and multiple BFS approach with a pruning
boundary.
\end{tabular} &
+ It is a system between structured and unstructured.
\\
\hline
%%%%%%%%%%%%%%%%%%%%%%%%%%%%%%%%%%%%%%%%%%%%%%%%%%%%%%%%%%%%%%%%%%%%%%%%%%%%%%%%
\textbf{Gia \cite{CRBLS2003}} &
{\large \CheckedBox} &
{\large \CheckedBox} &
{\large \CheckedBox} &
{\large \Square} &
\begin{tabular}[l]{m{3cm}}
Random Walks (RW).
\end{tabular} &
\begin{tabular}[l]{m{5cm}}
+ RWs issue one copy of the query thus not flooding the whole network.\\
-- RWs can reduce search scope.\\
\end{tabular}
\\
\hline
%%%%%%%%%%%%%%%%%%%%%%%%%%%%%%%%%%%%%%%%%%%%%%%%%%%%%%%%%%%%%%%%%%%%%%%%%%%%%%%%
\textbf{LTM \cite{LLXNZ2004}} &
{\large \CheckedBox} &
{\large \Square} &
{\large \Square} &
{\large \Square} &
\begin{tabular}[l]{m{3cm}}
TTL detector (2-hop distance).\\
Delayed low productive connection cutting.
\end{tabular} &
\begin{tabular}[l]{m{5cm}}
+ Compared to AOTO, ACE and SBO achieves faster convergence speed.\\
-- Creates more overhead than AOTO, ACE and SBO.\\
-- Needs synchronization of peer clocks.\\
-- Does not consider shortcuts created by powerful peers when choosing to disable connections (only uses delay metric).
\end{tabular}
\\
\hline
%%%%%%%%%%%%%%%%%%%%%%%%%%%%%%%%%%%%%%%%%%%%%%%%%%%%%%%%%%%%%%%%%%%%%%%%%%%%%%%%
\textbf{DCMP \cite{ZKB2008}} &
{\large \CheckedBox} &
{\large \Square} &
{\large \Square} &
{\large \Square} &
\begin{tabular}[l]{m{3cm}}
Cycle detection.
\end{tabular} &
\begin{tabular}[l]{m{5cm}}
+ Drastically reduces duplicate messages.\\
-- Cannot detect cycles in distance bigger than the TTL value of the IC message.\\
\end{tabular}
\\
\hline
%%%%%%%%%%%%%%%%%%%%%%%%%%%%%%%%%%%%%%%%%%%%%%%%%%%%%%%%%%%%%%%%%%%%%%%%%%%%%%%%
\textbf{\cite{CS2002} \& \cite{LCCLS2002}} &
{\large \Square} &
{\large \CheckedBox} &
{\large \CheckedBox} &
{\large \Square} &
\begin{tabular}[l]{m{3cm}}
Uniform replication.\\
Proportional replication.\\
Square root replication allocation.\\
\emph{k}-walker query scheme.
\end{tabular} &
\begin{tabular}[l]{m{5cm}}
+ Uniform replication reduces time spend on unsuccessful searches.\\
+ Reduces search time for frequent queries.\\
\emph{k}-walker quering scheme can reduce network traffic up to two orders of magnitude.\\
-- Proportional replication struggles in locating rare objects.
\end{tabular}
\\
\hline
%%%%%%%%%%%%%%%%%%%%%%%%%%%%%%%%%%%%%%%%%%%%%%%%%%%%%%%%%%%%%%%%%%%%%%%%%%%%%%%%
% \textbf{Tracing a large-scale Peer to Peer System: an hour in the life of Gnutella} &
% ? &
% ? &
% ? &
% ? &
% ? &
% ?
% \\
% \hline
%%%%%%%%%%%%%%%%%%%%%%%%%%%%%%%%%%%%%%%%%%%%%%%%%%%%%%%%%%%%%%%%%%%%%%%%%%%%%%%%
\textbf{Narada \cite{CRZ2000}} &
{\large \Square} &
{\large \CheckedBox} &
{\large \Square} &
{\large \Square} &
\begin{tabular}[l]{m{3cm}}
Mess creation.\\
Minimum spanning trees.
\end{tabular} &
\begin{tabular}[l]{m{5cm}}
+ Mess and trees are kept up to date in high churn environments.\\
-- Works well only for small groups of peers.
\end{tabular}
\\
\hline
%%%%%%%%%%%%%%%%%%%%%%%%%%%%%%%%%%%%%%%%%%%%%%%%%%%%%%%%%%%%%%%%%%%%%%%%%%%%%%%%
\textbf{AOTO \cite{LZXN2003}} &
{\large \CheckedBox} &
{\large \CheckedBox} &
{\large \Square} &
{\large \Square} &
\begin{tabular}[l]{m{3cm}}
Minimum spanning trees.\\
Peer proximity heuristic for removing costly links.
\end{tabular} &
\begin{tabular}[l]{m{5cm}}
+ Spanning trees only to immediate neighbors so no flooding and at the same time
no shrinked search scope.\\
+ Selective flooding effectiveness is detached from physical or overlay topologies.\\
+ The more logical neighbors, the more effective selective flooding becomes\\
-- High recalculation costs.\\
-- No sophisticated selection policy for candidate non-flooding peers.
\end{tabular}
\\
\hline
%%%%%%%%%%%%%%%%%%%%%%%%%%%%%%%%%%%%%%%%%%%%%%%%%%%%%%%%%%%%%%%%%%%%%%%%%%%%%%%%
\textbf{ACE \cite{LZXN2004}} &
{\large \CheckedBox} &
{\large \CheckedBox} &
{\large \Square} &
{\large \Square} &
\begin{tabular}[l]{m{3cm}}
Minimum spanning trees.\\
1-hop proximity heuristic.
\end{tabular} &
\begin{tabular}[l]{m{5cm}}
+ No flooding.\\
+ Less overhead compared to AOTO since computation is done within a certain diameter from the source peer.
-- Slow convergence speed.\\
-- Enhanced topology optimization comes to the expence of higher communication/computation overhead.
\end{tabular}
\\
\hline
%%%%%%%%%%%%%%%%%%%%%%%%%%%%%%%%%%%%%%%%%%%%%%%%%%%%%%%%%%%%%%%%%%%%%%%%%%%%%%%%
\textbf{SBO \cite{LXN2007}} &
{\large \CheckedBox} &
{\large \CheckedBox} &
{\large \Square} &
{\large \Square} &
\begin{tabular}[l]{m{3cm}}
Red/white bipartite overlay.
\end{tabular} &
\begin{tabular}[l]{m{5cm}}
+ Efficient in both static and dynamic environments.\\
+ Compared to AOTO incurs half the overhead.\\
-- Needs almost double the steps of LTM to converge (static or dynamic environments).
\end{tabular}
\\
\hline
%%%%%%%%%%%%%%%%%%%%%%%%%%%%%%%%%%%%%%%%%%%%%%%%%%%%%%%%%%%%%%%%%%%%%%%%%%%%%%%%
\textbf{THANCS \cite{LNXE2005}} &
{\large \CheckedBox} &
{\large \CheckedBox} &
{\large \Square} &
{\large \Square} &
\begin{tabular}[l]{m{3cm}}
Local optimum heuristic\\
Piggybacking neighbor distance in queries
\end{tabular} &
\begin{tabular}[l]{m{5cm}}
+ Completely distributed approach.\\
+ Presents trivial overhead compared to the query cost savings.\\
+ Convergent speed faster among AOTO, LTM, SBO.\\
+ Does not shrink the search scope.\\
-- Design cannot be extended to support non-flooding-based systems.
\end{tabular}
\\
\hline
%%%%%%%%%%%%%%%%%%%%%%%%%%%%%%%%%%%%%%%%%%%%%%%%%%%%%%%%%%%%%%%%%%%%%%%%%%%%%%%%
\textbf{HAND \cite{CLZHC2006}} &
{\large \CheckedBox} &
{\large \Square} &
{\large \Square} &
{\large \Square} &
\begin{tabular}[l]{m{3cm}}
Triple-hop adjustment.
\end{tabular} &
\begin{tabular}[l]{m{5cm}}
+ No need for clock sync.\\
+ Fully distributed.\\
+ Low overhead for the triple hop adjustment.\\
+ Applicable to both static and dynamic environments.\\
+ Low query response time.\\
-- Compared to LTM has lower traffic reduction and query response rates.
\end{tabular}
\\
\hline
%%%%%%%%%%%%%%%%%%%%%%%%%%%%%%%%%%%%%%%%%%%%%%%%%%%%%%%%%%%%%%%%%%%%%%%%%%%%%%%%
\textbf{APS \cite{BFLZ2003}} &
{\large \CheckedBox} &
{\large \Square} &
{\large \Square} &
{\large \Square} &
\begin{tabular}[l]{m{3cm}}
machine learning adaptive mechanism.
\end{tabular} &
\begin{tabular}[l]{m{5cm}}
+ Fully dynamic switching decision policy.\\
- Low convergence due to the learning process.
\end{tabular}
\\
\hline
%%%%%%%%%%%%%%%%%%%%%%%%%%%%%%%%%%%%%%%%%%%%%%%%%%%%%%%%%%%%%%%%%%%%%%%%%%%%%%%%
\textbf{ITA \cite{PRFM2013}} &
{\large \CheckedBox} &
{\large \CheckedBox} &
{\large \Square} &
{\large \Square} &
\begin{tabular}[l]{m{3cm}}
Short/long connections.\\
Local flooding.
\end{tabular} &
\begin{tabular}[l]{m{5cm}}
+ Low clustering.\\
+ Large peer coverage.\\
+ Reduced duplication.\\
+ Low or no impact to other mechanisms of unstructured p2p networks (e.g. 1-hop
replication, dynamic querying).
\end{tabular}
\\
\hline
%%%%%%%%%%%%%%%%%%%%%%%%%%%%%%%%%%%%%%%%%%%%%%%%%%%%%%%%%%%%%%%%%%%%%%%%%%%%%%%%
\textbf{EGOIST \cite{SLLBBR2008}} &
{\large \CheckedBox} &
{\large \Square} &
{\large \Square} &
{\large \Square} &
\begin{tabular}[l]{m{3cm}}
Selfish shortest path routing
\end{tabular} &
\begin{tabular}[l]{m{3cm}}
-- constructs a global view of the network
\end{tabular}
\\
\hline
%%%%%%%%%%%%%%%%%%%%%%%%%%%%%%%%%%%%%%%%%%%%%%%%%%%%%%%%%%%%%%%%%%%%%%%%%%%%%%%%
\textbf{BNS \cite{BCCMSBZ2006}} &
{\large \CheckedBox} &
{\large \Square} &
{\large \CheckedBox} &
{\large \Square} &
\begin{tabular}[l]{m{3cm}}
ISP clustering (tracker-side or ISP-side detection).\\
Bandwidth throttling.\\
Caching.
\end{tabular} &
\begin{tabular}[l]{m{5cm}}
+ Localizes traffic within an ISP.\\
+ Preserves the efficiency of BitTorrent protocol.\\
-- Needs ISPs to either provide information or infrastructure changes.\\
-- Locality-based approaches do not treat fair all peers.
\end{tabular}
\\
\hline
%%%%%%%%%%%%%%%%%%%%%%%%%%%%%%%%%%%%%%%%%%%%%%%%%%%%%%%%%%%%%%%%%%%%%%%%%%%%%%%%
\textbf{Ono \cite{CB2008}} &
{\large \CheckedBox} &
{\large \Square} &
{\large \Square} &
{\large \CheckedBox} &
\begin{tabular}[l]{m{3cm}}
ISP clustering.\\
Landmarking based on existing CDN infrastructure (CDN redirection measurements).
\end{tabular} &
\begin{tabular}[l]{m{5cm}}
+ Needs no ISP cooperation.\\
+ Needs no network topology information.\\
-- Depends on feeds from major internet infrastructures and large deeployment
of its client.\\
-- ISP-friendly approaches do not seem to have a great impact on Internet-scale
basis.
\end{tabular}
\\
\hline
%%%%%%%%%%%%%%%%%%%%%%%%%%%%%%%%%%%%%%%%%%%%%%%%%%%%%%%%%%%%%%%%%%%%%%%%%%%%%%%%
\textbf{\cite{LCLX2009}} &
{\large \CheckedBox} &
{\large \Square} &
{\large \Square} &
{\large \Square} &
\begin{tabular}[l]{m{3cm}}
AS hop count minimization on neighbor selection, on chocking/unchocking
mechanisms and on next-chunk picking.
\end{tabular} &
\begin{tabular}[l]{m{5cm}}
+ Optimization of the inter-AS traffic.\\
-- Locality based approaches do not treat fair all peers.
\end{tabular}
\\
\hline
%%%%%%%%%%%%%%%%%%%%%%%%%%%%%%%%%%%%%%%%%%%%%%%%%%%%%%%%%%%%%%%%%%%%%%%%%%%%%%%%
\textbf{TopBT \cite{RTLCGZ2010}} &
{\large \CheckedBox} &
{\large \Square} &
{\large \Square} &
{\large \Square} &
\begin{tabular}[l]{m{3cm}}
Peer selection metric that takes both downloading speed and network topology
into account.\\
Applied in multiple places of the BitTorrent protocol (bootstrap, connection
establishment/replacement, unchocking).
\end{tabular} &
\begin{tabular}[l]{m{5cm}}
+ No need for additional infrastructure.\\
+ Enhances both traffic and downloading.\\
-- Needs off-line processing of BGP dumps.
\end{tabular}
\\
\hline
%%%%%%%%%%%%%%%%%%%%%%%%%%%%%%%%%%%%%%%%%%%%%%%%%%%%%%%%%%%%%%%%%%%%%%%%%%%%%%%%
\textbf{UTAPS \cite{LCY2008}} &
{\large \CheckedBox} &
{\large \Square} &
{\large \Square} &
{\large \CheckedBox} &
\begin{tabular}[l]{m{3cm}}
Network tomography to construct a picture for the underlying network.
\end{tabular} &
\begin{tabular}[l]{m{5cm}}
+ Reduced ISP burden.\\
+ Better downloading speeds.\\
-- Small, laboratory-scale evaluation setup.
\end{tabular}
\\
\hline
%%%%%%%%%%%%%%%%%%%%%%%%%%%%%%%%%%%%%%%%%%%%%%%%%%%%%%%%%%%%%%%%%%%%%%%%%%%%%%%%
\textbf{\cite{QLZG2009}} &
{\large \CheckedBox} &
{\large \Square} &
{\large \Square} &
{\large \CheckedBox} &
\begin{tabular}[l]{m{3cm}}
Cluster peers in a swarm into local, intra- and inter-ISP.
\end{tabular} &
\begin{tabular}[l]{m{5cm}}
+ Reduced ISP burden.\\
+ Better downloading speeds.\\
-- Similarly to UTAPS, small scale evaluation setup.
\end{tabular}
\\
\hline
%%%%%%%%%%%%%%%%%%%%%%%%%%%%%%%%%%%%%%%%%%%%%%%%%%%%%%%%%%%%%%%%%%%%%%%%%%%%%%%%
\textbf{PROP \cite{QCYCZ2007}} &
{\large \CheckedBox} &
{\large \Square} &
{\large \Square} &
{\large \Square} &
\begin{tabular}[l]{m{3cm}}
Neighbor exchange between peers.
\end{tabular} &
\begin{tabular}[l]{m{5cm}}
+ Cooperation between peers.\\
+ Guarantees the connectivity of the network between exchanges.\\
\end{tabular}
\\
\hline
%%%%%%%%%%%%%%%%%%%%%%%%%%%%%%%%%%%%%%%%%%%%%%%%%%%%%%%%%%%%%%%%%%%%%%%%%%%%%%%%
% \textbf{Resolving the Topology Mismatch Problem in Unstructured Peer-to-Peer Networks} &
% ? &
% ? &
% ? &
% ? &
% ? &
% ?
% \\
% \hline
%%%%%%%%%%%%%%%%%%%%%%%%%%%%%%%%%%%%%%%%%%%%%%%%%%%%%%%%%%%%%%%%%%%%%%%%%%%%%%%%
\textbf{DDNO \cite{Z-YK2005}} &
{\large \CheckedBox} &
{\large \Square} &
{\large \Square} &
{\large \Square} &
\begin{tabular}[l]{m{3cm}}
Domain name topology detection (Split-Hash and dnMatch).
\end{tabular} &
\begin{tabular}[l]{m{5cm}}
+ Can be applied to both fully unstructured and super-peer based architectures.\\
+ Secures connectivity of the network.\\
+ Reduces cost of message exchange.
\end{tabular}
\\
\hline
%%%%%%%%%%%%%%%%%%%%%%%%%%%%%%%%%%%%%%%%%%%%%%%%%%%%%%%%%%%%%%%%%%%%%%%%%%%%%%%%
\textbf{CTAG \cite{ZL2006}} &
{\large \CheckedBox} &
{\large \Square} &
{\large \Square} &
{\large \Square} &
\begin{tabular}[l]{m{3cm}}
Clustering based on longest matching IP segment.
\end{tabular} &
\begin{tabular}[l]{m{5cm}}
+ Focuses on both construction and adaptation.
\end{tabular}
\\
\hline
%%%%%%%%%%%%%%%%%%%%%%%%%%%%%%%%%%%%%%%%%%%%%%%%%%%%%%%%%%%%%%%%%%%%%%%%%%%%%%%%
\textbf{Landmark Binning \cite{RHKS2002}} &
{\large \CheckedBox} &
{\large \Square} &
{\large \Square} &
{\large \CheckedBox} &
\begin{tabular}[l]{m{3cm}}
Landmark binning.
\end{tabular} &
\begin{tabular}[l]{m{5cm}}
+ It is independent of the overlay model.\\
+ The technique can be considered scalable.\\
-- Uses not so reliable network latency metric (this can lead to load imbalance etc).\\
-- The use of landmark servers renders the technique not fully distributed.\\
-- Excessive traffic flow towards the landmark servers is possible.\\
-- Fixed points in a network are inherently more exposed to malicious attacks.\\
-- Coarse-grained scheme.
\end{tabular}
\\
\hline
%%%%%%%%%%%%%%%%%%%%%%%%%%%%%%%%%%%%%%%%%%%%%%%%%%%%%%%%%%%%%%%%%%%%%%%%%%%%%%%%
\textbf{mOverlay \cite{ZZZSZ2004}} &
{\large \CheckedBox} &
{\large \Square} &
{\large \Square} &
{\large \CheckedBox} &
\begin{tabular}[l]{m{3cm}}
Dynamic landmarks.
\end{tabular} &
\begin{tabular}[l]{m{5cm}}
+ Fully distributed.\\
+ It is independent of the overlay model.\\
+ Load balanced.\\
-- Coarse-grained scheme.
\end{tabular}
\\
\hline
%%%%%%%%%%%%%%%%%%%%%%%%%%%%%%%%%%%%%%%%%%%%%%%%%%%%%%%%%%%%%%%%%%%%%%%%%%%%%%%%


%\begin{tabular}[l]{@{}}
%+ ID reduces the messages especially in upper levels of the tree\\
%+ DBFS\\
%+ LI reduces aggregate bandwidth usage and improves query efficiency\\
%- ID needs evaluation time between iterations\\
%- DBFS uses heuristics so it depends on their efficient choice\\
%- LI add index update overhead which might be heavy and might not work at all in systems with high churn
%\end{tabular} &


%\textbf{Narada} & \textbf{Overlay optimization
%based}. Creates a mesh (richer connected graph) and builds minimum spanning
%trees on this mesh & & Small and sparse groups \\

%\hline
%\textbf{Gia} & \textbf{Broadcast based} Replaces
%Gnutella flooding with random walk, and introduces KaZaA style super-nodes.
%Uses
%dynamic topology adaptation protocol &
% Gnutella &  Better than Gnutella  \\

%\hline
%\textbf{Adaptive Overlay Topology Optimization} & \textbf{Overlay optimization
%based}. Creates overlay multi-cast tree with Selective Flooding protocol&
%Gnutella &  Better than Gnutella \\

% \hline
% \textbf{Location-aware Topology Matching} &
% \textbf{Overlay Optimization Based}. Uses \textit{TTL2-detector flooding}, \textit{low productive
% connection cutting}, and \textit{source peer probing}. & Gnutella &  Better than Gnutella \\
% 
% \hline
% \textbf{Replication Strategies in Unstructured P2P Networks} &
% \textbf{Cache Based}. Uses uniform, proportional and square root allocation
% strategies to replicate data. & Gnutella &  Better than Gnutella \\
% 
% % \hline
% \textbf{Tracing a large-scale Peer to Peer System: an hour in the life of Gnutella.} &
% \textbf{Cache Based}. Proposes a caching algorithm based on the traces of the Gnutella traffic & Gnutella & Better than Gnutella \\
% 
% \hline
% \textbf{Improving search in P2P networks} &
% \textbf{Broadcast Based}. Uses \textit{iterative deepening}, \textit{directed
% BFS}, and \textit{local indices} to improve efficiency. & Gnutella &  Better than Gnutella \\
% 
% \hline
% \textbf{Distributed Cycle Minimization Protocol} &
% \textbf{Broadcast based} Uses a decentralized cycle elimination protocol  &  &  \\
% 
% \hline
% \textbf{Scalable Bipartite Overlay} &
% \textbf{Overlay optimization based} Uses bipartite partition graph and builds
% local minimum spanning trees  & Gnutella  & Better than Gnutella \\
% 
% \hline
% \textbf{Adaptive Connection Establishment} &
% \textbf{Overlay optimization based} Forms Neighbor Cost Tables, builds local
% minimum spanning trees and perform local optimizations & Adaptive Overlay
% Topology Optimization (AOTO), Gnutella & Better than Gnutella \\

% \hline
% \textbf{Hops Adaptive Neighbor Discovery} &  & &  \\
% 
% \hline
% \textbf{Two-Hop-Away Neighbor Comparison and Selection (THANCS)} &
% \textbf{Overlay optimization based} Uses piggybacking to discover neighbor
% distances and selects neighbors  & Gnutella  & \\

% \hline
% \textbf{mOverlay} &\textbf{Landmark based proximity} Uses dynamic landmarks to find node locality
% & & Due to dynamic landmarks and grouping, more scalable than tree-based or mesh-based protocols \\
% 
% \hline
% \textbf{Distributed Domain Name Order (DDNO)} &
% \textbf{Overlay optimization based} Connects half of the nodes connections to
% the nodes in the same domain and the other half to random nodes, therefore
% supports locality and topological connection  &  & Yes, by using super
% peers \\

% \hline
% \textbf{Peer-exchange Routing Optimization Protocols} & \textbf{Overlay optimization based} Optimizes overlay by the exchange of
% neighbors among peers  & Can work with both decentralized structured and
% unstructured architecture & Yes \\

% \hline
% \textbf{MAY OMIT - I CHANGED IT TO STRUCTURED SINCE THERE IS A REFERENCE FOR
% DHT (OF COURSE IT MIGHT POSSIBLE TO BE APPLIED TO BOTH. MAYBE NEED TO CHECK) -
% T2MC} &
% \textbf{Overlay optimization based} Uses trace-route results for clustering
% the nodes  & & \\
% 
% \hline
% \textbf{Unnamed-unstructured} &
% \textbf{Overlay optimization based} Minimizes the communication delay and
% maximizes the broadcasting range & & Better than THANCS and mOverlay \\

% \hline
% \textbf{Landmark Binning} & \textbf{Landmark based proximity} Uses network latency to partition
% nodes into bins & Can work with both decentralized structured and unstructured architecture & \\
% 
% \hline
%\end{tabular}
\end{longtable}
\end{center}
\vspace{-2.5ex}
\vspace{-2.5ex}
%\end{figure}
%\end{landscape}

\twocolumn


%%%%%%%%%%%%%%%%%%%%%%%%%%%%%%%%%%%%%%%%%%%%%%%%%%%%%%%%%%%%%%%%%%%%%%%%%%%%%%%%
%%%%%%%%%%%%%%%%%%%%%%%%%%%%%%%%%%%%%%%%%%%%%%%%%%%%%%%%%%%%%%%%%%%%%%%%%%%%%%%%
%                             UNSTRUCTURED
%%%%%%%%%%%%%%%%%%%%%%%%%%%%%%%%%%%%%%%%%%%%%%%%%%%%%%%%%%%%%%%%%%%%%%%%%%%%%%%%
%%%%%%%%%%%%%%%%%%%%%%%%%%%%%%%%%%%%%%%%%%%%%%%%%%%%%%%%%%%%%%%%%%%%%%%%%%%%%%%%

%\pgfplotsset{width=7cm,compat=newest}

%%%%%%%%%%%%%%%%%%% EFFICIENCY %%%%%%%%%%%%%%%%%%%
\begin{landscape}
\begin{center}
\begin{tikzpicture}
\begin{axis}[
  xbar,
  bar width=7pt,
  xlabel=\emph{Efficiency},
  ylabel=\emph{Algorithm},
  symbolic x coords={low, medium, high},
  symbolic y coords={
    IterativeDeepening,
    DirectedBFS,
    LocalIndices,
    Gia,
    DCMP,
    UniformReplication,
    ProportionalReplication,
    SqrtReplication,
    %Markatos02,
    Narada,
    AOTO,
    ACE,
    LTM,
    SBO,
    THANCS,
    HAND,
    APS,
    ITA,
    EGOIST,
    BNS,
    Ono,
    LiuEtAl,
    TopBT,
    UTAPS,
    QinEtAl,
    PROP-G,
    PROP-O,
    %hsiao_redblue_2009,
    DDNO,
    CTAG,
    LB,
    mOverlay
  },
  every axis y label/.style=
    {at={(ticklabel cs:0.5)},rotate=90,anchor=near ticklabel},
%   x tick label style={rotate=45,anchor=east},
  xtick=data, ytick=data,
%   ymin=low,ymax=high,ytickmin=low,
  height=\textheight - 0.3cm,
  width=\textwidth,
  enlargelimits=0.05,
  title=\emph{Efficiency} Pictorial Comparison of Unstructured Approaches.
]

\addplot[black,fill=green] table[x=EFFICIENCY,y=ALGORITHM]
{unstructured-plot.dat};

\end{axis}
\end{tikzpicture}
\end{center}
\end{landscape}

%%%%%%%%%%%%%%%%%%% OVERHEAD %%%%%%%%%%%%%%%%%%%
\begin{landscape}
\begin{center}
\begin{tikzpicture}
\begin{axis}[
  xbar,
  bar width=7pt,
  xlabel=\emph{Overhead},
  ylabel=\emph{Algorithm},
  symbolic x coords={low, medium, high},
  symbolic y coords={
    IterativeDeepening,
    DirectedBFS,
    LocalIndices,
    Gia,
    DCMP,
    UniformReplication,
    ProportionalReplication,
    SqrtReplication,
    %Markatos02,
    Narada,
    AOTO,
    ACE,
    LTM,
    SBO,
    THANCS,
    HAND,
    APS,
    ITA,
    EGOIST,
    BNS,
    Ono,
    LiuEtAl,
    TopBT,
    UTAPS,
    QinEtAl,
    PROP-G,
    PROP-O,
    %hsiao_redblue_2009,
    DDNO,
    CTAG,
    LB,
    mOverlay
  },
  every axis y label/.style=
    {at={(ticklabel cs:0.5)},rotate=90,anchor=near ticklabel},
%   x tick label style={rotate=45,anchor=east},
  xtick=data, ytick=data,
%   ymin=low,ymax=high,ytickmin=low,
  height=\textheight - 0.3cm,
  width=\textwidth,
  enlargelimits=0.05,
  title=\emph{Overhead} Pictorial Comparison of Unstructured Approaches.
]

\addplot[black,fill=red] table[x=OVERHEAD,y=ALGORITHM]
{unstructured-plot.dat};

\end{axis}
\end{tikzpicture}
\end{center}
\end{landscape}

%%%%%%%%%%%%%%%%%%% SCALABILITY %%%%%%%%%%%%%%%%%%%
\begin{landscape}
\begin{center}
\begin{tikzpicture}
\begin{axis}[
  xbar,
  bar width=7pt,
  xlabel=\emph{Scalability},
  ylabel=\emph{Algorithm},
  symbolic x coords={low, medium, high},
  symbolic y coords={
    IterativeDeepening,
    DirectedBFS,
    LocalIndices,
    Gia,
    DCMP,
    UniformReplication,
    ProportionalReplication,
    SqrtReplication,
    %Markatos02,
    Narada,
    AOTO,
    ACE,
    LTM,
    SBO,
    THANCS,
    HAND,
    APS,
    ITA,
    EGOIST,
    BNS,
    Ono,
    LiuEtAl,
    TopBT,
    UTAPS,
    QinEtAl,
    PROP-G,
    PROP-O,
    %hsiao_redblue_2009,
    DDNO,
    CTAG,
    LB,
    mOverlay
  },
  every axis y label/.style=
    {at={(ticklabel cs:0.5)},rotate=90,anchor=near ticklabel},
%   x tick label style={rotate=45,anchor=east},
  xtick=data, ytick=data,
%   ymin=low,ymax=high,ytickmin=low,
  height=\textheight - 0.3cm,
  width=\textwidth,
  enlargelimits=0.05,
  title=\emph{Scalability} Pictorial Comparison of Unstructured Approaches.
]

\addplot[black,fill=blue] table[x=SCALABILITY,y=ALGORITHM]
{unstructured-plot.dat};

\end{axis}
\end{tikzpicture}
\end{center}
\end{landscape}




\section{Structured \p\ Networks}
\label{section:structured}

This section offers description and analysis of algorithms proposed
to address the topology mismatch problem in structured \p\ networks. 
The categorization of the approaches is based on how different
levels of proximity information is gathered as well as  
the way peer routing tables are maintained and used in an effort make hops in
overlays as efficient as possible. 
%%AD what does it really mean this below???
The categorization is based on previous work found in the
literature~\cite{CDHR2002,CDCR2002,RSS2002}.

%%COMM Rephrased as above to keep it analogous to the opening paragraph of
%% the unstructured part.
%%
% This section presents a description and analysis of algorithms proposed
% to address the topology mismatch problem in structured \p\ networks.
% In structured \p\ networks, the construction of the routing tables
% held by participating nodes determines the efficiency of message forwarding.
% Routing tables that closely match the underlying IP topology
% achieve increased performance. As a result, the proposed solutions for
% the topology mismatch problem for structured \p\ networks
% use different levels of proximity
% information to optimize the routing tables used by peers.
% The categorization of the algorithms we present here is based on the
% categorization suggested by previous
% work~\cite{CDHR2002,CDCR2002,RSS2002}.

\subsection{Algorithms for Structured Architectures}

%###############################################################################
%###############################################################################
%###############################################################################
%       GEOGRAPHIC LAYOUT
%###############################################################################
%###############################################################################
%###############################################################################

%%%%%%%%%%%%%%%%%%%%%%%%%%%%%%%%%%%%%%%%%%%%%%%%%%%%%%%%%%%%%%%%%%%%%%%%%%%%%%%%
% \subsubsection{Global Soft-State}

% \begin{figure*}
% \centering
%   \includegraphics[scale=0.8]{img/algorithms/global_softstate}
% \caption{(left) Position of $p$ in the space of $3$ landmarks. (right) Position
% of $p$ on the map.}
% \label{fig:global_softstate}
% \end{figure*}

\emph{Global Soft-State} \cite{XTZ2003} builds a global map to help
choose shorter routing paths. This is accomplished by 
combining the landmark binning method with small
scale distance probes to reveal the proximity properties 
of the underlying network. 
%Figure \ref{fig:global_softstate} presents the landmark
%space and the mapping of the position of the nodes.
This global view of the state is made available to 
\emph{all} nodes to help them find the best way to route their messages. 
%%AD when you say "all" above means what?? all in the structured network?? 
%%	how is the structured network organized?  (a line or 2 for this?)
\emph{Global Soft-State} operates in $2$ phases:
\emph{generation} and \emph{use} of the proximity information. 
For \emph{generating} the proximity information, 
a hybrid approach is proposed: it uses landmark clustering as a
preprocessing step to select a number of potential nearest neighbor
candidates and then, refines the selection by incorporating a
round-trip-time-based scheme to ultimately choose the closest node. 
%% 
The algorithm follows a different path from that of the
classic gossiping approaches for the construction and 
maintenance of the overlay when 
it comes to \emph{use} the proximity information.
%%AD isn't the following sentence somewhat of repetition??? (of what 
%%	is stated above?? 
The approach is based on landmark clustering 
for generation and strategic
placement of proximity information on the overlay. 
This information is called a ``map'' and concerns a 
region of the overlay; 
the latter may be part of the Cartesian space in
structures like \emph{eCAN}~\cite{xu_ecan_2002} or 
a set of nodes sharing a particular
prefix in overlays such as \emph{Pastry}\cite{antony_pastry_2001}. 
%%
Thus, each node not only can access such information from 
these landmarks but can also subscribe to
relevant \emph{soft states} and get notified on dynamic network changes. 
A node may appear in a maximum of $log(N)$ such maps.
%%
Although this approach does reduce the routing latency to far nodes, it
can become expensive as it is 
possible to end up with a very large number of regions
each of which has to maintain a map. 
The fact that a peer may appear in multiple
such maps points into additional traffic for probing and notifications
and so renders the approach not scalable \cite{RGJZ2004}. 
Moreover, to refine measurements
additional inner-bin measurements are needed~\cite{WZS2004}.
Also, improvements within a region can be minimal as there exists
limited knowledge about neighboring zones.
%%AD rephrased what is below -2 lines- as above..
% Also improvement
% is done within a region because of little knowledge about neighboring zones.
%%%%%%%%%%%%%%%%%%%%%%%%%%%%%%%%%%%%%%%%%%%%
%%
%% COMM Removed most of the paragraph below which 
%%		contains info that we cannot back
%%
%% Rescanned papers and wrote the paragraph below.
%% DO NOT PERMANENTLY DELETE!
%%
% Unfortunately, maintained of several host states at different layers,
% renders any content migration a costly process. Additionally, the method does
% not make any continuing effort to remap
% the overlay structure after nodes successfully join resulting into poor
% adaptation to changing conditions. Although this approach greatly reduces the
% routing latency to far nodes, it is unable to dynamically identify nodes that
% are close to routers and gateways in order to construct the secondary overlay.
% Nevertheless, static recognition of such nodes is currently done based on BGP
% reports and pre-chosen landmarks, sacrificing the self-organizing attribute of
% traditional DHTs.
%
%
% TODO: SOME DISCUSSION
%
% Xu \textit{et al.} \cite{XTZ2003} state that there are three ways
% generating proximity information
%\begin{itemize}
% \item Expanding Ring Search\\
%Expanding ring search can be of two forms. First it can utilize the multi-cast
%infrastructure in the underlying network in order to emit its messages.
%Unfortunately such infrastructures are not widely deployed thus the
%implementations of this way of generating proximity information is limited to
%blindly flooding the neighborhood to obtain reasonable results.
%
% \item Heuristics\\
%Heuristics are used in order to reduce the blindness of the expanding ring
%search and make realizations more efficient and effective. Unfortunately a
%common problem of all heuristic approaches is the local minimum pitfall in
%which
%the search might be caught into.
%
% \item Landmark Binning\\
%Landmark clustering is based on the view that nodes with similar distances to a
%set of predefined well-known landmark nodes are pretty likely of being close to
%each other. But this approach has its weaknesses as well, such as the fact that
%is a rather coarse grained approximation, therefore not particularly well
%suited
%for differentiating nodes within close distance to each other.
%
%\end{itemize}
%
%
%
% TODO: IN PAPER INTRODUCTION A DISCUSSION ABOUT
%TOPOLOGY-AWARE CAN
%
%Techniques to exploit topology information in overlay
%routing include geographic layout, proximity routing and
%proximity neighbor selection [3]. With geographic layout
%such as topology-aware CAN [12], the overlay structure is
%constrained by underlying network topology. This tech-
%nique, unfortunately, can create uneven distribution of
%nodes in the overlay, increasing the chances of overloading
%nodes and rendering the maintenance cost formidable. Our
%study shows that, for a typical 10,000-node topology-aware
%CAN, 5% nodes occupy 85-98% of the entire Cartesian
%space, and some nodes have to maintain 450-1500 neigh-
%bors. In Proximity routing, physical topology is not consid-
%ered when constructing the overlay.
%
In terms of our $3$ criteria, the approach fares as follows:
\begin{center}
{\footnotesize
\begin{tabular}{ccc}
\emph{Efficiency} & \emph{Overhead} & \emph{Scalability} \\
\hline
% Unable to identify nodes that are close to routers but it can reduce routing
% latency to far away node by approximately 50-75%.
medium &
% maintenance of node states, subscriptions and notifications need global
% overview of the overlay resulting in high control overhead
high &
% Statically implemented sacrificing the decentralized nature of DHTs and
% thus poorly scaling. Also we can end up with a very large number of regions.
% Because of the fact that a peer may appear in multiple
% such maps we can conclude that additional traffic for probing and notifications
% renders the approach does not scalable.
low
\end{tabular}
}
\end{center}

%%%%%%%%%%%%%%%%%%%%%%%%%%%%%%%%%%%%%%%%%%%%%%%%%%%%%%%%%%%%%%%%%%%%%%%%%%%%%%%%
% \subsubsection{Mithos}
\emph{Mithos} \cite{WR2003} is a \p\ protocol which incorporates directed
incremental probing to find near optimal node placement and integrates
geographic layout and proximity routing overlay optimization methods.
%%AD in general some mention of the mismatch problem must be made right???
%%		so that a reasonable connection can be established... 

The bootstrap phase of \emph{Mithos} starts with a subset 
of existing members as the initial set of candidates and 
while the closest nodes are detected by iterative
probing of the neighborhood, the candidate neighbor--list gets updated. 
To avoid local minima, \emph{Mithos} probes all the neighbors 
within a two-hop distance from the current minimum 
before concluding the process.

After finding its first neighbor, an 
% new-coming   %%AD overall the term new-coming is somewhat odd.. incoming? 
incoming peer is assigned 
an \emph{ID} using information gathered during 
the iterative neighbor selection phase. 
%%AD use SIMPLE sentences that are well structured to say what u want
Based on the two closest nodes found, along with these nodes neighbors and the corresponding
distances a system of virtual springs is set so that the tension of each spring
is inversely proportional to the distance. 
%%AD the above is impossible to read.. "based on,... along with... " NO!!!
%%AD I am not sure what you are talking about here - I mean below... rework
%%	and simplify.. 
The system is then left to settle,
and the state in which it settles is used for its \emph{ID}. 
The benefit of these synthetic coordinates are that
they are explicitly used as the node's \emph{ID}, 
and distance computations among nodes
can be done in \emph{ID} space without requiring physical measurements.
%%%%%%%%%%%%%%%%%%%%%%%%%%%%%%%%%%%% upto here %%%%%%%%%%%%%%%%

The last step of the algorithm is the 
interconnection among neighbors. 
\emph{Mithos} uses a \emph{quadrant-based} mechanism in which 
each node establishes a link to the closest neighbor in each quadrant. 
%%AD what is the quadrant??? Hugh??? 
During forwarding, the next hop is performed towards a neighbor 
in the same quadrant of the final destination. 
An incoming node may not know of other neighbors in all
quadrants, therefore, the node first identifies neighbors in all quadrants
using a mechanism based on ideas similar to a perimeter walk~\footnote{Used in
the Greedy Perimeter Stateless Routing (\emph{GPSR}) protocol.}  %%AD ref??
and then improves the results using parallel path processing 
by taking into account further geometric properties of node relationships.
%%AD the above 5-6 lines sounds impressive but there is little help 
%%	for the average ignorant reader.. what is perimeter walk in 
%%	general? how quadrants are formed? What is their relationship? 

One major limitation of \emph{Mithos} is that the protocol cannot effectively
handle dynamic arrivals and departures from the network. 
\cite{CCRK2004}~points out the extensive probing to determine 
distances and evaluates \emph{Mithos} results as less accurate
than those derived with \emph{GNP}~\cite{ng_gnp_2001}. 
%%AD what is GNP??? How did it "land" here???
%%
The latter has been also attributed by~\cite{WSS2005} to 
the iterative neighbor selection process which that is 
reportedly prone to premature termination at local minima.
%%
\cite{cox_vivaldi_2004} argues that elements of the protocol's
spring relaxation technique require a centralized implementation 
affecting \emph{Mithos} scalability.
%%AD rephrased what is below as above...
% Another point that has negative effect to the
% scalability of the algorithm is that parts of its spring relaxation technique
% require a centralized implementation~\cite{cox_vivaldi_2004}.
%%%%%%%%%%%%%%%%%%%%%%%%%%%%%%%%%%%%%%%%%%%%%%%%%%%%%%%%%%%%%%%%
In regard to the $3$ criteria, \emph{Mithos} stands as follows:
\begin{center}
{\footnotesize
\begin{tabular}{ccc}
\emph{Efficiency} & \emph{Overhead} & \emph{Scalability} \\
\hline
% It has very good results in reducing communication latency among peers but its
% efficiency is being constrained by the fact that the proximity is calculated
% at bootstrap, assigning virtual coordinates that are used throughout of a
% nodes life cycle and thus the overlay is not adapting to a high-churn
% environment.
medium &
% Incremental probing is applied in quadrants and in two-hop distance at each
% step that does not generate a lot o control overhead.
medium &
% It is constrained by the fact that the protocol does not respond to dynamic
% peer arrivals and departures
low
\end{tabular}
}
\end{center}

%%%%%%%%%%%%%%%%%%%%%%%%%%%%%%%%%%%%%%%%%%%%%%%%%%%%%%%%%%%%%%%%%%%%%%%%%%%%%%%%
% \subsubsection{LAPTOP}
\emph{LAPTOP} \cite{WLH2007} organizes the overlay 
%%AD what is the nature of the "structure" overlay here???
into a tree-based hierarchy so that 
hops required during message routing be reduced and 
maintenance overhead be minimized.
A caching scheme is also used so as
to further lower routing table update costs. 
It is theoretically shown that 
\emph{LAPTOP} routing path length is bounded by $O(log_d(N))$ and node
joining and leaving in the overlay network is bounded by
$O(d*log_d(N))$ hops in a balanced overlay tree with $N$ representing the
number of nodes and $d$ the maximum degree of each node. 
\emph{LAPTOP} 
%% implements 
adopts the geographical layout approach  
and constructs its layout in a self-organizing and 
efficient fashion. 
The physical distances are estimated
with round trip times to a few existing nodes in the overlay network.
%%
Each node is assigned a dot-formatted address (e.g., 1.3.4). 
Each octet ranges from
$1$ to $d$.
%  where $d$ is the maximum degree of the nodes. 
%%AD this definition was provided just above.. 
The assignment process is done by appending a unique octet 
to the address of the parent of each node, while
the root node is assigned address $1$.

%The protocol is based on four definitions.
%\begin{enumerate}
% \item The amplitude of all possible measured RTTs is divided into intervals.
%  Each node measures its distance to its parent and is assigned a label $L_i$
% where $i$ denotes the configurable RTT interval in which the measured distance
% falls into. A special kind of node, the root, is initially assigned the $L_1$
% label and maintains (as all $L_1$ nodes do) a list of other $L_1$ nodes in the
% overlay.
% \item Any node can have children with level lower than theirs, except an
%  $L_{max}$ node which can only have $L_{max}$ level children and only in the
% case when its parent has reached its maximum degree.
% \item Each node is assigned a dot-formated address (e.g 1.3.4). Each octet
% ranges from
% $1$ to $d$, where $d$ is the maximum degree of the nodes. The assignment
% process
% is done by appending a unique octet to the address of each nodes parent, while
% root node is assigned address $1$.
% \item For any descendant node $Y$ of a node $X$, the measured distance among
%  each other, must always be less than the lower bound of the RTT interval
% denoted by $X$'s label.
%\end{enumerate}
%

\emph{LAPTOP}'s routing scheme is similar to the 
longest-prefix \emph{IP} matching scheme. 
At each forwarding hop, a message travels up the tree 
until the first common ancestor of both source and destination nodes 
is reached and then, the message starts descending towards its target. 
During tree traversals, special entries in the routing tables, 
called \emph{routing cache}, 
%%AD clarify - routing caches are the special entries OR the routing tables???
are maintained to increase routing efficiency and achieve finer--load balance. 
Caching enables a node to forward a message to a better 
longest-prefix match than that of its direct ancestor yielding
a larger, quicker and more cost--effective step through the overlay towards
the destination. 
To improve scalability, the number of child nodes and the
size of the routing cache are limited. 
To contain maintenance costs, \emph{LAPTOP} employs 
a simple \emph{heartbeat}-based regime in which 
a parent is responsible for the monitor of its own children.
%%AD rephrased what is below as above.. 
% In terms of overlay maintenance, \emph{LAPTOP}
% incorporates a simple \emph{heartbeat}-based technique 
% where each parent node is
% responsible for monitoring its children.
%%%%%%%%%%%%%%%%%%%%%%%%%%%%%%%%%%%%%%%%
%
%The new comer $N$ locates the root node and the later responds with a list of
%  $L_1$ nodes. $N$ then probes each of the $L_1$ nodes in search for the
% closest one. If the measured RTT to the closest $L_1$ falls into the first
% interval then the new comer becomes a $L_1$ node as well. Otherwise node $N$
% sets the closest $L_1$ node as its potential parent node. This potential
% parent becomes the actual parent if it does not have any other children. If it
% has, $N$ gets a list of these $L_i$ nodes and by measuring the RTT to each of
% them tries to spot a new potential parent in order to repeat the above
% process.
%
%
% At join process the new coming node is assigned its level label as well as its
% address by its parent node. Additionally, it initializes its routing table
%(with normal and caching entries) as it traverses the overlay in search for
%its parent node.  During a graceful departure, the referred node checks for
%children in the overlay. If it does not have any, it simply notifies its
%parent and leaves. If it has, it selects the child node with the lowest
%round-trip time to it in order to take its place so that the locality
%property is preserved. On the other hand, in the case of an arbitrary
%failure, after the children of the failed node detect their parent's absence
%(through the aforementioned heartbeat mechanism), they start emitting special
%messages to their grandparent node (the address of which is stored during
%join process}. In case of no response from the grandparent node, the children
%invoke the joining process to detect their new parent node.The parent of the
%failed node, aggregates the notifications from the above children nodes for a
%period of time, and then chooses the one with the lowest level label as the
%takeover node. Potential ties are broken by favoring the lower RTT. The
%parent of the failed node, finally informs accordingly all the children about
%the change in the hierarchy.
%%%
%
%%%%%%%%%%%%%%%%%%%%%%%%%%%%%%%%%%%%%%%%%%%%%%%%%%%%%%%%%%%%%%%%%%%%%%%%%%%%%%%%
%
% TODO: HERE STRUCTURED LANDMARK BINNING APPROACH LIES
%
%Assuming a structured approach based on CAN\cite{ratnasamy_can_2001} and $m$
% landmark nodes. The coordinate space is then partitioned into $m!$ equally
%sized portions, each corresponding to a single ordering of the landmarks. To do
%this, the first dimension is divided into $m$ areas each of which is further
%divided (second dimension) into $m - 1$ sections and so on. Having set this $m$
%dimensional space, at joining time, a node measures the delay to the set of
%landmarks in order to determine its associated bin and thus position itself in
%that portion of the coordinate space associated with its landmark ordering.
%Even though this scheme can
%
\emph{LAPTOP}'s behavior for the $3$ criteria stands as follows:
\begin{center}
{\footnotesize
\begin{tabular}{ccc}
\emph{Efficiency} & \emph{Overhead} & \emph{Scalability} \\
\hline
% It minimizes routing length, but may suffer load imbalance since nodes higher
% in the tree hierarchy are more likely to route most of the messages in the
% overlay.
medium &
% Heartbeat approach incurs extra control overhead to the network. Fortunately
% involves only the parent and the children of the departing/failing node.
% nodes.
low &
% Seems to scale nicely from 10K- to 1M-node testbeds.
high
\end{tabular}
}
\end{center}


%###############################################################################
%###############################################################################
%###############################################################################
%       PROXIMITY ROUTING
%###############################################################################
%###############################################################################
%###############################################################################


%%%%%%%%%%%%%%%%%%%%%%%%%%%%%%%%%%%%%%%%%%%%%%%%%%%%%%%%%%%%%%%%%%%%%%%%%%%%%%%%
% \subsubsection{Proximity in Kademlia}
% Before discussing {\em Proximity in Kademlia}, it would be helpful to briefly
% summarize the highlights about the {\em Kademlia} protocol. {\em Kademlia}
% \cite{maymounkov_kademlia_2002} is a distributed hash table (DHT) based
% protocol for \p\ networks, which uses an iteration based lookup algorithm. The
% protocol uses the standard $160$ bit ID system for nodes and locates the nodes
% in a prefix binary tree, where IDs are used as prefixes. The iterative lookup
% operations are done over this prefix binary tree, which converges to
%logarithmic
% lookup times. The ID's are assigned randomly, therefore in Kademlia, there is
%no
% proximity control, which results in inefficient use of the underlying network
% during lookup and retrieve operations.


In order to optimize the underlying network use,
\cite{KLKP2008} introduces proximity
controls over the \emph{vanilla Kademlia} 
protocol's random \emph{ID} assignment.
% The work in this paper
% can also be considered as a more generic study of incorporating proximity into
% iterative lookup algorithms.
In its quest to improve routing performance, the overlay uses 
a cost ``yardstick'' (i.e., metric) provided by the underlying network.
% The overlay tier improves its routing performance using a cost--metric that is
% provided by the underlying tier. 
Such cost information can be acquired by
\begin{inparaenum}[\itshape i\upshape)]
  \item measuring previous lookups,
  \item exchanging or jointly-computing measurements with 
	others, or 
  \item using a local database to look--up information.
\end{inparaenum}

\cite{KLKP2008} exploits $2$ routing techniques: 
\emph{proximity routing selection (PRS)} and 
\emph{proximity neighbor selection (PNS)}.
%%
In \emph{PRS}, the goal is to choose the best next hop while routing a message.
\emph{Plain old Kademlia} (\emph{POK}) always selects the closest--node 
with respect to the \emph{XOR} metric.
With \emph{PRS} in place however, \emph{Kademlia} may also choose 
the node that exhibits the smallest underlay metric cost; in this way, 
it balances a trade-off between the logical overlay and 
actual underlying distances.
%%AD rephrased as above.. the "underlay" I am not sure what it is...
%  but with proximity routing, Kademlia should also choose the node
% that exhibits the smallest underlay metric cost, balancing a trade-off between
% the overlay and underlay progress distances.
%%
In \emph{PNS}, on the other hand, the goal is to keep peers 
with the least contact cost in the routing table. 
%%AD WHAT is this "least contact cost"?? Sounds funny! CHECK
As \emph{Kademlia} continually learns about new
peers via incoming requests and an iterative lookup process, 
%%AD what is this iterative lookup process???
no specialized algorithm is required for searching for 
more cost-effective peers 
and the system simply goes for the best peers ``seen'' thus far.

\emph{Kademlia} proximity injection improves 
locality of connections among peers.
To enhance traffic locality, the use of 
{\sl MaxMind GeoIP} database~\footnote{\url{http://www.maxmind.com}} 
is suggested to form clusters of nearby peers.
To this end, Vivaldi coordinates~\cite{cox_vivaldi_2004} could be 
also used as an alternative for morphing locality. 
\cite{KLKP2008} however suggests that the clustering approach 
when combined with \emph{PRS}/\emph{PNS} helps contain 
$40\%$ of messages within an \emph{ISP} and allows only a $6\%$ 
to go through the backbone.
The behavior of the proposal with respect to the stated criteria is deemed as follows:
%%
\begin{center}
{\footnotesize
\begin{tabular}{ccc}
\emph{Efficiency} & \emph{Overhead} & \emph{Scalability} \\
\hline
% Uses static information to cluster nodes which is not the most efficient
% approach in dynamic systems
% PNS reduces lookup latencies by 66\% compared to POK
medium &
% 
medium &
% MaxMind is commercial and cannot be used by FOSS P2P systems. Trying to
% expose information from P2P client can result in security issues.
medium
\end{tabular}
}
\end{center}

%%%%%%%%%%%%%%%%%%%%%%%%%%%%%%%%%%%%%%%%%%%%%%%%%%%%%%%%%%%%%%%%%%%%%%%%%%%%%%%%
% \subsubsection{CHOord considering Proximity on IPv6 (CHOP6)}
\emph{CHOP6} \cite{MT2007} roughly estimates the proximity among
nodes by exploiting the {\sl IPv6} address format and if available, 
\emph{RTT} information.
The proximity estimation is achieved by introducing a $64$-bit 
\emph{ID} scheme in which
the least significant bits are the {\sl IPv6} global routing prefix.
Thus, a longest--prefix match scheme is established.
% enabling a longest prefix match scheme. 
%%AD check is the word "longest" is correct..
The protocol is designed based on the
observation that it is feasible to estimate a node's geographical location by
examining the upper $32$-bits of its {\sl IPv6} address. 
Moreover, similar to \emph{Chord}~\cite{stoica_chord_2001}, 
\emph{CHOP6} uses a finger table, whose entries hold
more than one candidate node.
%
%\paragraph{}
% There are three cases in which a node chooses the next hop according to
% information it possesses.
%\begin{itemize}
% \item When no information about RTT is available candidate next hops, the
%  sender just selects a node in the finger table entry which shares the longest
% prefix with the destination.
% \item There is another possibility that after some communication with other
%  nodes, the source node should know of the RTTs to some of the nodes in finger
% table entry. In this case, the source node chooses the one with the smallest
% RTT with probability $p$. If there is a node with no measured RTT then the
% sender can select such a node with probability $1 - p$
% \item In this last case the source node has already communicated with all
%  candidate node in the finger table entry. Thus, the node selects the node
% whose RTT is the smallest with probability $p$, the one with the second
% smallest with probability $q$ and so on, where $0 \leq \ldots < r < q < p < 1$
% and $p+q+r+\ldots \leq 1$
%\end{itemize}
%
With respect to the $3$ qualitative criteria, \emph{CHOP6} stands as follows:
\begin{center}
{\footnotesize
\begin{tabular}{ccc}
\emph{Efficiency} & \emph{Overhead} & \emph{Scalability} \\
\hline
% Simple approach that exploits proximity but it is coarse grained and thus not
% efficient.
low &
% Its static nature imposes low overhead to the network
low &
% IPv6 is still not widely applied and supported.
medium
\end{tabular}
}
\end{center}
%%AD how does CHOP6 helps with the mismatch problem?? 


%%%%%%%%%%%%%%%%%%%%%%%%%%%%%%%%%%%%%%%%%%%%%%%%%%%%%%%%%%%%%%%%%%%%%%%%%%%%%%%%
% \subsubsection{T2MC}
\emph{T2MC} \cite{SLCGZ2008} uses traceroute logs to detect 
the boundaries of an autonomous system (\emph{AS}) and
cluster nearby peers; the objective here is to 
diminish the redundant multiple messages that may 
unnecessarily cross \emph{AS}-boundaries.
\emph{T2MC} uses a customized $k$-mean classification
algorithm with $k = 2$ to perform the classification 
and exploits the stable structure of the Internet routers 
to help guide the clustering in question. 
Even though {\sl traceroute}
provides detailed information about the network structure, 
it extensive use may create overhead on the overall network structure. 
In this respect, it is often the case that administrators 
disable {\sl traceroute} support on their routers.
The latter may affect \emph{T2MC}'s performance.
%%
Although overall experimental results have been encouraging, 
\emph{T2MC} is considered coarse-grained
as it classifies peers into only ``remote'' and ``close''
groups \cite{QLZG2009}.
%%AD a better connection with the mismatch problem has to be made... 
%%		.. in the form of 1-2 lines (or short statements). 
%%
%
%T2MC \cite{SLCGZ2008} uses trace-route logs to detect different autonomous
%systems and cluster close by peers, thus reducing redundant traffic that crosses
%autonomous system boundaries. T2MC uses a customized k-mean classification
%algorithm with $k=2$ to perform the classification, and exploits the stable
%structure of the Internet routers to guide clustering. The peer chooses the
%minimum and maximum Latency results from the trace-route to initialize the
%centroids of two sets. Then the peer calculates, for all hops along its
%trace-routed path, the peer calculates the absolute distance to these centroids
%of both sets and assigns the routers to that centroid with which it has the
%smaller absolute distance. After that, the mean and variance values are
%calculated and if the variance is larger than a predefined threshold then the
%algorithm sets the two latency mean values as new centroids for the two sets and
%loops again. Ultimately peer will end up with two sets having the minimum
%intra-set variance. The peer, then, chooses the router from the second set with
%the minimum hops attribute and sets it as a threshold. The selected router and
%all others whose hops attribute is larger than the threshold are classified as
%\emph{remote} router cluster. The remaining are classified as \emph{near}. From
%the near class, the peer chooses the one with maximum hops attribute as its edge
%router, and registers it along with all near clusters into the DHT of the
%\p\ overlay. As new peers join the network, those that share the same edge
%router or any of the members of the near router clusters, would form a close
%peer cluster. As Edge routers can provide more valuable information than other
%members of the near set, T2MC was designed to prioritize interaction of peers
%and edge gateways.
%
%
% TODO: SOME DISCUSSION
%
%The use of Trace-route as a tool for implementing the distance measuring
%infrastructure raise concerns about its efficiency and scalability. Being,
%primarily, a network diagnostic utility,
% Even though trace-route provides detailed information about the network
% structure, its excessive use creates additional overhead that stresses further
% the overall network infrastructure. This forces network administrators to
% disable its support from the routers they manage, thus jeopardizing the
% effectiveness of the T2MC algorithm.
%
\emph{T2MC} fares as follows:
%%
\begin{center}
\begin{tabular}{ccc}
\emph{Efficiency} & \emph{Overhead} & \emph{Scalability} \\
\hline
% The efficiency of choosing close by peers is very good and reported almost
% 65\% better than that of GNP (and thus generally of static landmarking as a
% whole)
high &
% trace-route is a very heavy tool and thus incurs excessive burden to the
% network resources.
high &
% Trace-route is categorized as too heavy-weighted and intrusive for use in a
% larger scale system \cite{RKHS2002}. Additionally, disabling
% ICMP is a common administrative policy for edge sites to enforce security,
% while dumping BGP routing tables\cite{krishnamurthy_bgpclust_2000} is not
% directly available to the application layer.
low
\end{tabular}
\end{center}

%%%%%%%%%%%%%%%%%%%%%%%%%%%%%%%%%%%%%%%%%%%%%%%%%%%%%%%%%%%%%%%%%%%%%%%%%%%%%%%%
% \subsubsection{PChord}
\emph{PChord} \cite{HLYW2005} is based on the \emph{Chord DHT} and adds
proximity into its routing mechanism. 
The main modification over its predecessor is the inclusion 
of a \emph{proximity list} into its routing table;
in this way, the next hop can be decided based on both the best progress
towards the key and the physical proximity of the candidate nodes.
%%AD I am not sure about the "best progress" term - what does it really mean???
%%
At join time, this list in empty. 
As the \emph{PChord} node starts to interact with
others, it applies a heuristic mechanism to fill the list. 
Entries are dynamically added or removed as the network state changes. 
The overlay maintenance cost is similar to that of \emph{Chord}. 
%%
Apart from the costs incurred when a  node joins in or departs and 
by heart-beat messages used for connection verification, \emph{PChord} adds
heart-beat among nodes it the proximity list. 
Fortunately the latter is kept to a minimum provided that 
the list entails nodes presenting very low \emph{RTT}s.
%%AD removed "lowest" to use lowest u must have a set or something...

As there is always progress towards the target in the \emph{ID}-space,
\emph{PChord} yields a reduced number of hopes that \emph{Chord}.
%%AD rephrased as above - the 2 last lines of the text below are 
%%		are difficult to decipher... 
% Due to the fact that, with each step, there is always 
% progress towards the target node in the ID space, 
% \emph{PChord} will result in a reduced hop number than
% Chord, since each hop in \emph{PChord} is larger or 
% at least equal in the key space.
%%%%%%%%%%%%%%%%%%%%%%%%%%%%%%%%%%%%%
%%
Passing through proximity links, in the underlying network, 
means reduced routing cost. 
Moreover, if the number of entries in the proximity list is the same
as the number of network partitions, 
\emph{PChord} prevents hops from jumping in and
out of network partitions the current node belongs to.
%%
On one hand, 
\emph{PChord}’s advantageous  points are that 
its routing attains fewer overlay hops, 
it crosses network partition(s) only once by following proximity links
and it maintains the light-weight maintenance cost of \emph{Chord}.
On the other, \emph{PChord} exhibits slow convergence speed 
and inefficiency in high-churn environments~\cite{DK2006}.
%%AD rephrased what is below as above... 
% \emph{PChord}’s advantages include the fact 
% that the main routing optimizations are of
% less overlay hops, following proximity links and crossing the same network
% partition only once along a routing path while in the same time keeping Chord's
% light-weight maintenance cost. 
% On the other hand, it exhibits slow convergence
	% speed and inefficiency in a high churn environment \cite{DK2006}.
%%%%%%%%%%%%%%%%%%%%%%%%%%%%%%%%%%%%%%%%%%%%%%%%%
\emph{PChord} qualitatively behaves as follows:
\begin{center}
{\footnotesize
\begin{tabular}{ccc}
\emph{Efficiency} & \emph{Overhead} & \emph{Scalability} \\
\hline
% Small decrease in needed overlay hops per object publishing but rigid in
% high churn environments.
medium &
% The are heartbeat mechanism for both finger table and proximity list adds a
% burden to the network by default. But the overall overhead is decreased as
% the proximity list becomes longer.
medium &
% Improvement of routing is achieved with more object pointers helping to
% shorten the routing path of object querying. When the distance of query source
% to target is far away both Chord and PChord achieve the average hop number to
% not increase.
medium
\end{tabular}
}
\end{center}

%%%%%%%%%%%%%%%%%%%%%%%%%%%%%%%%%%%%%%%%%%%%%%%%%%%%%%%%%%%%%%%%%%%%%%%%%%%%%%%%
% \subsubsection{AChord}
\emph{AChord} \cite{DK2006} uses the {\sl IPv6} any-cast mechanism to
relieve the protocol from complex joining procedures,
harness network proximity and achieve high routing efficiency. 
Any-cast delivers a message coming from the outside of 
an any-cast group to the physically closest node in
that group~\cite{M2002}. 
\emph{AChord} organizes all nodes participating in the
overlay network into an any-cast group. 
Any joining node comes outside of that group and, thus, is
automatically forwarded to the physically nearest node for bootstrapping;
this avoids the need to explicitly maintain such nodes, %%AD such nodes???
and the effort of locating the physically nearest. 
%%
The \emph{ID} of an incoming 
% the new-coming %%AD changed...
node is computed based on the \emph{ID} of the bootstrap node 
and the bootstrap's predecessor in a way that the constructed 
\emph{ID} is positioned between the aforementioned two. 
After joining, a \emph{finger} table is created ala \emph{Chord}. 
Moreover, an \emph{adjacency table} maintains 
information about the closest--known peers. 
The routing decision is made by using both adjacency and finger tables.
%%AD how are these 2 combined???
\emph{AChord} behavior has as follows:
\begin{center}
{\footnotesize
\begin{tabular}{ccc}
\emph{Efficiency} & \emph{Overhead} & \emph{Scalability} \\
\hline
% good routing efficiency but no adaptation to high churn environments.
medium &
% Relieves the system from joining overhead and topology aware ID construction
% but doubles the information needed to be stored and maintained by the
% introduction of adjacency table
medium &
% assumes the operation of ideal underlying any-cast mechanism (meaning that
% outside messages are always routed to the nearest node which in the current
% Internet, this is not always feasible. Moreover, is designed on the basis of
% IPv6 protocol which is not widely operational yet.
medium
\end{tabular}
}
\end{center}

%%%%%%%%%%%%%%%%%%%%%%%%%%%%%%%%%%%%%%%%%%%%%%%%%%%%%%%%%%%%%%%%%%%%%%%%%%%%%%%%
% \subsubsection{Chord6}
\emph{Chord6}~\cite{XZHL2005} is a \emph{Chord}-variant that
exploits {\sl IPv6}'s  hierarchical features in order to create 
a substrate that reduces inter-domain traffic among service providers. 
The key difference between
\emph{Chord6} and \emph{Chord} lies on the identifier definition. 
Therefore, the approach can be easily portable to other 
\emph{DHT}s including \emph{CAN}, \emph{Pastry} and \emph{Tapestry}. 
\emph{Chord6}'s identifier contains two parts: 
the higher bits are obtained by hashing the
node's {\sl IPv6} address prefix of specific length, 
while the remaining lower bits are the hash-value 
of the rest of the {\sl IPv6} address. 
As a result of this assignment, nodes in a domain are mapped 
onto a continuous key space on the overlay network;
the latter avoids unnecessary message--forwarding across different
service providers and helps minimize overall routing cost.

\emph{Chord6} does reduce the average system path length to be
logarithmically proportional to the number of available domains 
instead of the number of participating peers. 
However, due to the large number of Internet
domains and the possibility of occasionally having only 
small number of peers within each domain,
\cite{DK2006} challenges the approach's scalability and efficiency.
%%
Overall, the approach fares as follows:
\begin{center}
{\footnotesize
\begin{tabular}{ccc}
\emph{Efficiency} & \emph{Overhead} & \emph{Scalability} \\
\hline
% Can reduce inter-domain average path length but even though nodes in the same
% domain would have close identifiers, nodes in two close domains may have very
% different ones reducing the overall efficiency of the approach in general when
% considering inter-domain average path length. No active adaptation mechanism
% for high churn environments.
low &
% The construction of the identifier space is done at node join and in an
% independent fashion involving the application of two hash functions meaning
% resulting in low overhead stress to the network and its nodes.
low &
% Is designed on the basis of IPv6 protocol which is not widely operational yet.
low
\end{tabular}
}
\end{center}

%###############################################################################
%###############################################################################
%###############################################################################
%       PROXIMITY NEIGHBOR SELECTION
%###############################################################################
%###############################################################################
%###############################################################################

%%%%%%%%%%%%%%%%%%%%%%%%%%%%%%%%%%%%%%%%%%%%%%%%%%%%%%%%%%%%%%%%%%%%%%%%%%%%%%%%
% \subsubsection{DHT-PNS}
A proximity neighbor selection scheme  functioning atop 
the \emph{Chord DHT} is introduced in~\cite{DLTZZ2006}.
Its key objective here has been to group physically-close nodes as 
neighbors in the \emph{DHT} table while exploiting proximity information.
To detect proximity, virtual network coordinates for 
peers are used with the help of the Vivaldi protocol~\cite{cox_vivaldi_2004}. 
The virtual coordinates are then used to help map nodes to 
the identifier--space of the \emph{DHT}.
%%
The space in question is partitioned using a 
\emph{concentric circle clustering scheme} where successive cycles of
radiuses $\rho$, $2\rho$, $3\rho$ and so on, are constructed. 
The formed annuluses are then divided into 
$2\chi-1$ \emph{sectors}, where $\chi$ denotes the
sequence number of the annulus starting from $\chi = 1$ 
for the center cycle. 
In this way, 
\cite{DLTZZ2006} prove that each sector
occupies the same area as does the center cycle. 
Assuming uniform node distribution, this characteristic,
favors a more load--balanced clustering operation. 
Every sector in this $2d$ space is mapped to 
a unique \emph{region} in the \emph{DHT} space forming a
multi-layer node identifier space.  
Consequently, nodes belonging to the same
sector, are mapped to the same region as well, preserving their proximity
relationship.
%% unveiled by the use of the Vivaldi protocol. 
%%AD I am not sure u need the above line at the end of the previous sentence.
Individual elements
are uniquely mapped to the identifier--space, 
allowing logarithmic lookup operations with high 
probability on \emph{Chord}. 
%%

%%AD what is the section u say here??? -- also the first sentence 
%%	here has to do a better connection with what was discussed before!
The system described in the previous section 
allows any node to obtain its \emph{DHT} key and its region key, in a fully
distributed manner, just by applying a consistent hash function. 
%%AD where these RPCs come from? kind of disconnected???
Using an {\sl RPC} call in \emph{Chord}, 
the node can obtain the region's master node, 
called \emph{Cluster Node (CN)} which is responsible for clustering 
the nodes belonging in the same sector or region with that of the node at hand. 
An additional {\sl RPC} call, registers it to its corresponding region 
and publishes its information, through the special peer \emph{CN}. 
The node can, additionally, ask an \emph{CN} for other
nodes that have previously joined the region to add them into its
neighbor set for future routing table optimization. Q
%%AD the one below is not a parse-able sentence!!! What does it say???
%%%		restrict the length of your sentences to say at most 2.5 lines!
Even in the case when no
neighbor is detected in the current region, the search is expanded to adjacent
regions and towards an upper layer identifier space until one is found or the
first layer reached. Simulations suggest that the scheme has good convergence
property of the delay distribution and good optimizing performance.
%%AD up to here the text in this second paragraph for Chord DHT remains 
%%%%%		... rather cryptic - does need work and better connection 
%%%%%		with the 1st paragraph...
%%
Unfortunately, even though scaling appeared promising, only limited 
simulations carried out with configuration involving up to 2,048 nodes 
were carried out.
% where made on rather small test beds (of at most 2048 nodes), so no safe
% conclusions can be deduced. 
Moreover the approach assumes uniform node distribution which
is not always the case, especially in a synthetic coordinate system.
%%
\emph{Chord DHT} fares as follows in reference to the $3$ criteria:
\begin{center}
{\footnotesize
\begin{tabular}{ccc}
\emph{Efficiency} & \emph{Overhead} & \emph{Scalability} \\
\hline
%
medium &
%
medium &
%
medium
\end{tabular}
}
\end{center}



%%%%%%%%%%%%%%%%%%%%%%%%%%%%%%%%%%%%%%%%%%%%%%%%%%%%%%%%%%%%%%%%%%%%%%%%%%%%%%%%
% \subsubsection{Quasi-Chord}
\cite{SZ2008} proposes \emph{Quasi-Chord}, a \emph{Chord}-based network that
constructs its overlay network while attempting to match the underlying 
network topology.
The overlay creation follows $3$ stages:
first, each host acquires its coordinates in
a geometric space using the \emph{Global Network Position (GNP)} protocol
\cite{ng_gnp_2001} and a set of landmarks.
Second, using the \emph{Cantor} space--filling curve, 
the $2$-dimensional is converted to $1$-dimensional space. 
The latter is employed in the third and final stage to build 
the \emph{Quasi-Chord} circle according to the host's \emph{Cantor} value. 
Physical proximity is thus denoted by \emph{Cantor} value proximity. 
The \emph{Chord} cycle is constructed by sorting nodes in 
ascending order. In \emph{Quasi-Chord},
each node stores $2$ finger tables; one clockwise and one counter-clockwise.
%%AD what is the purpose of this? efficiency to go back and forth???
%%
%
%In the following paragraphs, these steps are further discussed.
%
%\paragraph{GNP coordinates}
%First of all the host must be positioned in the geometric space. The algorithm
%  models the \p\ network with a well defined distance function in such a way,
% it can predict, with high accuracy, the distance between any two points in the
% space by just evaluating the output of the distance function on the
% coordinates of these points. This is accomplished with the \emph{global
% network position (GNP)} protocol which\begin{inparaenum}[\itshape i\upshape)]
%  \item creates a reference set of $N$ landmark nodes so as to minimize the
%  error of ICMP measured distances and coordinate computed ones between them,
% and then
%  \item each host is able to measure its round-trip times to the $N$ Landmarks
% in order to compute its own coordinates.
%\end{inparaenum}
%
%\paragraph{Cantor SPF}
%After the $2$-d coordinates are set, in the next stage of the algorithm, each
%  participating peer is assigned a Cantor value, according to the application
% of a \emph{space filling curve} on the coordinate space. (TODO: add figure).
% This results in the conversion of the $2$-dimensional space to a
% $1$-dimensional Cantor space that can more easily be more mapped to the Chord
% hierarchy.
%
%\paragraph{Quasi-Chord construction}
%As can intuitively be inferred by observing the Cantor chart, close-by nodes in
%  the physical layer, are more likely to have similar Cantor values. This
% attribute is exploited in order to construct a topology-aware identifier space
% for the Chord DHT. The cycle is constructed by sorting nodes in ascending
% order. Each host maintains 2 finger tables, one for clockwise and one for
% counter-clockwise stepping. This helps with the connectivity of the network
% because its not allowed to connect the first node with the last one since this
% will incur heavy traffic to the later\footnote{After all their Cantor values
% denote that they are actually the furthest of each other.}.
%
%
%
% TODO: SOME DISCUSSION
%       NEEDS DOUBLE-CHECK
%
%The disadvantage of the algorithm is that the coordinate assignment in the
%  first stage is backed by a not fully distributed landmark-based algorithm.
% Moreover the Quasi-Chord model build-up is making an indirect assumption of a
% maximum number of allowable hosts since it is constructed. Last but not least
% the doubling of the required routing information which needs to be created and
% maintained is an additional negative point to the efficiency and the
% scalability of the algorithm.
%
%%
\emph{Quasi-Chord} qualitative behavior has as follows:
\begin{center}
{\footnotesize
\begin{tabular}{ccc}
\emph{Efficiency} & \emph{Overhead} & \emph{Scalability} \\
\hline
% The algorithm shows to reduce end-to-end delay as well as not produce
% duplicates of messages.
medium &
% The overhead is more compared to vanilla chord since mapping of 2-dimensional
% space to 1-dimensional with Cantor SPF adds some extra burden in time
% complexity. Also the addition of the counter-clockwise finger table doubles
% the data, per se, needed to be stored and maintained.
high &
% The GNP system is a static landmark approach making things not fully
% distributed and thus constrain the model's scalability.
medium
\end{tabular}
}
\end{center}

%%%%%%%%%%%%%%%%%%%%%%%%%%%%%%%%%%%%%%%%%%%%%%%%%%%%%%%%%%%%%%%%%%%%%%%%%%%%%%%%
% \subsubsection{IP-based Clustering (IPBC)}
Proximity neighbor selection algorithms often use probing 
% and other techniques
%%AD say what are these "other" than simply say so... 
to detect proximity. 
However, such methods suffer by imprecisions often due to network overload.
{\sl IP}-based clustering~\cite{KM2007} 
% is a proximity neighbor selection algorithm which  %%AD u do not need this!
uses {\sl IP} address prefixes ($16$ bit for {\sl IPv4}) 
to detect proximity. 
%%AD where this IPBC came from?
\emph{IPBC} clustering generates a key and its prefix %%AD this ... "and" is not 						      %%    clear!!!
and stores the prefix in the \emph{DHT} itself;
in this way, newly joining nodes with the same {\sl IP} prefix 
can query the \emph{DHT} and readily
identify all neighbors with the same prefix.  
Nodes periodically update their entries in the \emph{DHT} 
by removing those who leave or have timed-out.
%%

The overhead of the algorithm is very low as proximity detection requires only
a comparison of prefixes. On the other hand, the efficiency of the proximity
is caught into a vicious cycle with the storage space needed for \emph{DHT} 
object publishing (how many bits will be used for either). 
%%AD did not get the parenthesis above - what is "either"???
%%	I am not sure what the sentence below says and 
%%	how is connect to the above
Moreover, its static nature (no online measurements used) 
makes it difficult to handle~\cite{EWF2011}.
%%
% Moreover, observing the
% \emph{IANA IPv4 Address Space Registry} we can infer that, since blocks of
% consecutive octet prefixes are assigned to the same Regional Internet
% Registries (RIRs), nodes inside will be physically close to each other.
%
%
% In order to be published in the overlay, each node is first assigned a
% unique identifier and subsequently generates a key by hashing a fixed-length
% prefix of its IP. Authors argue on the prefix length to be used, as there is a
% thin balance between reducing the probability of finding closer nodes by
% adopting a long prefix and overload nodes that are responsible for information
% published by many peers when choosing to hash a short one. Their verdict is
% for
% the use of a 16-bit wide prefix for real world systems deployed in an Internet
% scale. In either case, ID and IP is then stored in the DHT using this
% generated key. This way, any node, at any time (either at join time or during
% peer lifetime for topology adaptation) can acquire information about close-by
% nodes just by querying the DHT for a specific key. Moreover, the algorithm
% takes care of the freshness of the proximity information in two ways. First,
% the advertising nodes themselves periodically update their advertisements or
% when they voluntarily leave the overlay, they explicitly remove their data. On
% the other hand, in case of a failure each publication is assigned an
% expiration time, and thus ultimately removed by the DHT maintenance
%mechanisms.
%
The suggested approach fares as follows:
\begin{center}
{\footnotesize
\begin{tabular}{ccc}
\emph{Efficiency} & \emph{Overhead} & \emph{Scalability} \\
\hline
% The efficiency of proximity detection through the prefix matching technique
% is caught in a vicious cycle with the storage space needed for DHT object
% publishing (how many bits will be used for either).
low &
% The overhead for detecting proximity is only comparison of prefixes
low &
%
medium
\end{tabular}
}
\end{center}

%%AD how the 2-level thing helps the mismatch  problem?? 
%%	ALSO, why the symbolism of the layers is required?? Ld_Cord Ld_Cone
%%%%%%%%%%%%%%%%%%%%%%%%%%%%%%%%%%%%%%%%%%%%%%%%%%%%%%%%%%%%%%%%%%%%%%%%%%%%%%%%
% \subsubsection{Cone}
\emph{Cone} \cite{HY2007} extends \emph{Chord} using a proximity neighbor
selection topology optimization algorithm. The proximity information is
generated using landmarks and round-trip-time-based distances to 
those landmarks. %%AD added those.. 
%%AD how is this better/worse than the one just above based on IP clustering???
\emph{Cone} uses a two-layered identifier space. The first layer is 
identical to that of \emph{Chord} and is denoted as $Id_{Chord}$.
% \emph{Chord}-layer identifier and denoted as $Id_{Chord}$, 
% is the same as in Chord. 
The second layer identifier known as \emph{Cone}-layer and denoted as 
$Id_{Cone}$ is generated using two IDs. 
The first, known as \emph{group identifier (gid)} denotes a
relevant group to which the node belongs to while the second, 
namely \emph{local identifier (lid)} indicates the % local 
identifier within the group. 
The group is used here as means to divide nodes according to  a
% concept is a way of dividing nodes according to a
common $Id_{Chord}$ prefix.  
The \emph{Cone} overlay retains the \emph{Chord}'s circular topology. 
The difference lies in the fact that, now, $2$ rings are created. 
A big one in which nodes with the same \emph{gid} are arranged 
% at each position.  %%AD I do not get thus at EACH position!!!
in the same position.
Each of these positions constitute a smaller ring for the particular
group's \emph{lid}s. 
%%%
Routing is achieved in both clockwise and
counter-clockwise directions in the big-ring, for which two routing tables are
maintained: \emph{front} and \emph{back} finger tables. 
Entries in these tables, display physical network proximity 
with the current node. 
Moreover, a third table called \emph{group} maintains 
information about other on-line peers
within the current node's group in a way that 
entries are now close in the \emph{id} space. %%AD check rephrasing...
%
%Routing in Cone, comprises of the inter-group algorithm and the intra-group
%  algorithm. First the group of nodes to which the desired key lies is
% detected, exploiting physical proximity information (front and back finger
% tables). Second during the next and last hop, the message is forwarded to the
% exact node that contains the desired key.
%
\emph{Cone} fares as follows: 
\begin{center}
{\footnotesize
\begin{tabular}{ccc}
\emph{Efficiency} & \emph{Overhead} & \emph{Scalability} \\
\hline
% The result of experiments show that the Cone, compared with Chord, has obvious
% improved on the delay of route and the hops of overlay networks…
medium &
% The approach incorporates a static landmark + RTT to these landmarks in order
% to exploit proximity information thus incurring low overall additional
% operational overhead but imposing extra stress to the landmarks themselves.
% Additional information must be maintained, three finger tables (front, back and
% group)
medium &
% landmark based approaches are not full distributed thus not smoothly scalable.
medium
\end{tabular}
}
\end{center}

%%%%%%%%%%%%%%%%%%%%%%%%%%%%%%%%%%%%%%%%%%%%%%%%%%%%%%%%%%%%%%%%%%%%%%%%%%%%%%%%
% \subsubsection{DynaMO}
\emph{DynaMO}~\cite{WZS2004} attempts to create an overlay that 
does not only consider physical proximity amongst peers but also
takes into account mobility attributes of nodes.
To adapt to mobile ad-hoc networks, \emph{DynaMO} strives
to maintain an evenly--distributed overlay \emph{ID}-space 
so that hot-spots are avoided. 
%%
Although based on the \emph{Pastry} overlay and 
it well understood built-in locality heuristics ~\cite{CDCR2002a},
\emph{DynaMO}'s own mechanisms are \emph{DHT}-independent and 
so, they can be applied to other \emph{DHT}s.
%%AD rephrased/shortened above - check
%%
\emph{DynaMO}'s approach may be viewed as bottom-up. 
An incoming node computes physical information about its neighborhood
and uses the outcome to unilaterally assign itself an \emph{ID}.
The proposal is based the fact that with every new overlay  routing-hop
the physical distance between nodes is likely to increase
resulting in a dominating 
last routing step~\cite{antony_pastry_2001,CDCR2002a}.
To identify this step, two techniques are used: 
\emph{Random LandMarking (RLM)} and 
\emph{Closest Neighbor Prefix Assignment (CNPA)}.
%%AD rephrased as above - check!
% A new-coming node is required
% to compute physical information about its neighborhood 
% and use it to unilaterally assign itself the an ID. 
% \emph{DynaMO} is building on the
% observation that with every new overlay routing hop the physical distance
% between nodes is likely to increase, resulting in a dominating last routing
% step \cite{antony_pastry_2001,CDCR2002a} and for this two methods are discussed,
% \emph{random landmarking (RLM)} and \emph{closest neighbor prefix assignment
% (CNPA)}.
%%%%%%%%%%%%%%%%%%%%%%%%%%

In \emph{RLM}, a set of peers are chosen to be responsible 
for a set of landmark keys. 
These keys have to be chosen in a way that they equally divide the overlay
space. 
%%AD how do you accomplish the above?? 
An incoming peer then computes distances from those landmarks,
sorts them, and assigns itself an \emph{ID} on this 
landmark ordering as follows:
% The new-coming peer then computes distances to those
% landmarks, sorts them and assigns itself an ID based on the landmark ordering.
the peer adopts as \emph{ID} having a prefix from its closest
landmark; the length is 
$l_{prefix}=|log_b k|$, where $b$ is the \emph{ID} base 
and $k$ the number of landmarks. 
The remaining of bits of the peer-\emph{ID} are assigned either 
randomly or computed with additional proximity optimizations.
In this way, two desirable features are attained:
% What is achieved is that
\begin{inparaenum}[\itshape i\upshape)]
  \item landmark assignment is dynamic, and
  \item the set of a node leaf may reference peers that are 
	physically close by.
	%%AD I am not sure why is that?? But it is ok...
\end{inparaenum}
The latter in particular helps reduce average routing cost(s)
for the leaf-set of a node is used to efficiently determine the 
last routing step.
% (since it contains the nodes closest in the overlay space).
%%AD I do not think you need the above.. %%AD Also rewriting..
%%
%%
\emph{CNPA} is proposed for network setups for which a lightweight
bootstrap phase is required.
It delegates to \emph{Pastry} the task to identify the physically
close--node with which an incoming peer gets bootstrapped. 
The new-comer assumes the bootstrapping nodes's \emph{ID} prefix 
and fills the remaining of its key using the techniques of \emph{RLM}.

\emph{RLM} gets evaluated as presenting good proximity in a load-balanced 
environments. Moreover, it is adaptable to high-churn settings.
Unfortunately, \emph{RLM} bootstrap overhead is substantial and can 
effect  the set of temporary landmarks.
\emph{CNPA} on the other hand, incurs less overhead but this come 
at the expense of being more coarse-grained.
%%AD  Rephrased as above.  -- In the above I am not clear 
%%	a) what are the "temporary" landmarks? b) what is coarse-grained 
%%	and why??? 
%%%%%%%%%%%%%%%%%%%%%%%%%%%%%%%%%%
% \emph{RLM} achieves good proximity evaluation in a load balanced and adaptable to
% high churn environments fashion. Unfortunately it comes with more bootstrap
% overhead that can especially effect the set of temporary landmarks. On the other
% hand, \emph{CNPA}'s less incurred overhead comes at the expense of being more
% coarse-grained.
%%%
%%
% \emph{DynaMO} tries to capitalize on the fact that in Pastry each node's table consists
% of rows equal to the number of digits in the overlay ID and columns equal to
% the ID's base. As we go down the table rows, the matching prefix between the
% current node's ID and the row's entries increases by one. Thus, the leaf set
% contains the closest nodes in the ID space. Additionally, as the prefix match
% increases by one, the result is exponentially fewer candidates that can fill the
% tables entries as the row increases. This leads to the observation
%  In this context, two approaches are considered namely
% \emph{Random Landmarking (RLM)} and \emph{Closest Neighbor Prefix Assignment
% (CNPA)}.
%
% TODO: SOME DISCUSSION
%
%Both algorithms, are protected against the formation of physical landmark
%  clusters or imbalanced ID distribution\footnote{More common during the
% initial formation of an overlay network} , by introducing the \emph{landmark
% gravitation range} as a threshold over which landmark keys are reassigned (for
% the RLM approach) or unutilized ID prefix ranges are detected and used (for
% the CPNA scheme) in order to balance the distribution of regions in the
% overlay.
%
\emph{DynoMO}'s elements fare as follows regarding our $3$ criteria:
\begin{center}
{\footnotesize
\begin{tabular}{rccc}
\multicolumn{1}{r}{} &
\multicolumn{1}{c}{\emph{Efficiency}} &
\multicolumn{1}{c}{\emph{Overhead}} &
\multicolumn{1}{c}{\emph{Scalability}}
\\
\cline{2-4}
\emph{RLM} &
%%
%% TAKEN FROM
%% Efficient service discovery mechanism for wireless sensor networks
%%
%%
% common prefix of the overlay ID with the closest landmark. The
% RLM utilizes the overlay lookup capabilities by reducing traffic
% overhead within an overlay cluster. However, it generates much
% control traffic for periodically measuring distances to all landmarks.
% Moreover, the hop stretch overhead still increases when
% performing overlay routing among overlay clusters.
medium &
%
medium &
% Dynamic landmarking approach helps the scalability of the algorithm. The
% authors claim that the temporary nature of RLM's landmark nodes does not
% prevent them to maintain a good ratio in the presence of high landmark
% digression rates.
high \\
\emph{CNPA} &
%
medium &
% % The paper suggests that this approach generates significantly lower overhead
% than the vanilla Pastry algorithm
low &
% The queries are directed through fast nodes so it is important such nodes
% be part of the network
medium \\
\end{tabular}
}
\end{center}

%%%%%%%%%%%%%%%%%%%%%%%%%%%%%%%%%%%%%%%%%%%%%%%%%%%%%%%%%%%%%%%%%%%%%%%%%%%%%%%%
% \subsubsection{SAT-Match: Self-Adaptive Topology Matching}
% \begin{figure}[ht]
% \centering
% \subfigure[Example physical topology of nodes.] {
%   \includegraphics[scale=0.6]{img/algorithms/SAT_match}
%   \label{figure:sat_match:before}
% }\qquad\qquad
% \subfigure[(left) A jumps to B in order to match the logical topology to the
% physical. (right) State of the logical topology after the jump.] {
%   \includegraphics[scale=0.6]{img/algorithms/SAT_match2}
%   \label{figure:sat_match:after}
% }
% \caption{The SAT-Match algorithm.}
% \label{figure:sat_match}
% \end{figure}

\emph{SAT-Match}~\cite{RGJZ2004} defines an iterative process that 
adaptively changes the overlay network to ease the topology mismatch problem. 
It consists of a \emph{probing phase} in which a peer queries nearby nodes 
for distances and a \emph{jumping phase} in which the peer decides 
to change its neighborhood in order to reduce average neighbor distances. 
This process is performed unilaterally
until no additional optimization is needed but it achieves optimization within a
sufficiently large scope.
%% VM This is used in other papers. Maybe we can put somewhere in the
% definition of Topology Mismatch Problem in the intro/background.
The notion of \emph{stretch} is used as th means for quantifying 
the \emph{topology match degree} of the constructed overlay;
stretch is defined as: 
$S = \bar{L_l}/\bar{L_p}$ where $\bar{L_l}$ is the average logical link
latency and $\bar{L_p}$ is the average physical link latency.
%%AD what are the indexes: l and p -- p for peer and l for new peer???
%%AD rephrased what is below as above... 
% 
% The paper defines \emph{stretch}
% $S = \bar{L_l}/\bar{L_p}$, where $\bar{L_l}$ is the average logical link
% latency while the $\bar{L_p}$ is the average physical link latency, and uses it
% a way to quantify the \emph{topology match degree} of the constructed overlay.

The \emph{probing phase} starts when the new-comer joins the \emph{DHT} 
using the standard mechanism of the algorithm at hand. 
The peer starts probing the neighborhood with 
messages of small \emph{TTL=k}~\cite{jiang_lightflood_2008}. 
Recipients return information to the source and forward the message 
if not expired. 
After discovering its \emph{TTL}-$k$ neighborhood, 
the source measures \emph{RTT}-times, sorts them
and picks the two nodes with the smallest \emph{RTT}s.

During the \emph{jumping phase}, the source node calculates the 
stretch change of its \emph{TTL--1} neighbors %%AD each of its neighbors???
and that of the \emph{TTL-1} neighbors of the first of
the peers picked during the previous phase. 
These calculations are made as if the jump had been made. 
Should the stretch reduction is over a predefined threshold
the jump is actually performed. Otherwise,
the second selected candidate is selected and the
same computation is performed. 
%%AD is this done only 2 times?? Or is it done recursively??
If again, the threshold is not met, then no jump is ultimately carried out. 
In case of a jump, this is performed as a combination
of \emph{leave} and \emph{join} operations, in the underlying context
(i.e., \emph{CAN}).
%%AD I am not sure of the last sentence + CAN... What does it say?? context???

%%AD The following sentence says what exactly????? 
% Even though no theoretical proof is given, valid results are
% obtained and demonstrated through experimentations. 
%%%%%%%
% It is argued that since it
% is observed to achieve a $27\%$ stretch reduction when compared to a landmark
% enhanced CAN-system, it can be deduced that it performs better than landmark
% binning in terms of the matching degree.
%%AD rephrased above as below.. .
Experimentation shows that \emph{SAT-Match} 
attains a $27\%$ stretch reduction over 
a landmark-enhanced \emph{CAN}-alternative.
Based on this result, it is conjured that \emph{SAT-Match} 
may perform better than landmark binning in terms of the matching degree. 
\emph{SAT-Match} behaves as follows with respect to the stated $3$ criteria:
%%AD matching degree is unqualified here... 
%%
% Figure~\ref{figure:sat_match} demonstrates the selective jump on an
% example topology.
%
%\paragraph{}
%Moreover the algorithm takes several issues into consideration in order to
%  further improve the resulted overlay. For example when multiple nodes try
% to jump simultaneously into a region, then the logical link brakes from one
% attempt may result in inaccurate computation of the gain factor, for an other.
% This situation is identified as \emph{contention} and the nodes use an
% exponential back-off algorithm to avoid it. An other problem is the
%unnecessary
% traffic incurred by the probing phase in a region that after several jumps
% has settled to a stable state. In these cases it is more likely for jump
% attempts to be proven worthless. The algorithm doubles the probing period of
% such nodes, every time a jump is not taken.
%
%\paragraph{}
%The authors claim that this continuously adaptive mechanism achieves global
%  topology matching optimization in a sufficiently large scope. This also
% secures the fast adaptation to frequent network changes. It also considered
% lightweight and can easily be embedded into current \p\ systems, as well
% as effectively combined with other techniques, such as landmark binning.
%%
%
% TODO: SOME DISCUSSION
%
% It is reported that,
% due to the selective jumps, SAT-Match achieves $40\%$ reduction in
% link stretch, and when used with the landmark binning (see
% Sec. \ref{sec:landmark_binning}), the reduction rate increases up to $60\%$.
% For dynamic environments, with frequent node arrival and leave, SAT-Match
% scales much better than Mithos, due to its self adaptation mechanism and
% selective jumps.
%%
\begin{center}
{\footnotesize
\begin{tabular}{ccc}
\emph{Efficiency} & \emph{Overhead} & \emph{Scalability} \\
\hline
% SAT-Match seems to outperform landmark binning in the sense of providing a
% higher grained differentiation of proximity, adding to the fact that
% SAT-match is fully distributed. Also to further enhance the efficiency it can
% work cooperatively with landmark binning itself.
% Security is compromised with arbitrary jumps!
high &
% distance probing and iterative computations add to the approach's overhead.
% Also the fact that the selective jumps are not always performed the added
% overhead to efficiency gained is not guaranteed in any configuration and
% circumstance.
medium &
% The self-adaptation nature of the algorithm and the selective jump mechanisms
% scale much better than, for example, Mithos. The test are on a relatively
% small amount of peers, thus, no safe conclusion can be made with respect to the
% algorithm's scalling performance.
medium
\end{tabular}
}
\end{center}

%%%%%%%%%%%%%%%%%%%%%%%%%%%%%%%%%%%%%%%%%%%%%%%%%%%%%%%%%%%%%%%%%%%%%%%%%%%%%%%%
%%%%%%%%%%%%%%%%%%%%%%%%%%%%%%%%%%%%%%%%%%%%%%%%%%%%%%%%%%%%%%%%%%%%%%%%%%%%%%%%
\subsection{Discussion on the Algorithms for Structured Architectures}
%%%%%%%%%%%%%%%%%%%%%%%%%%%%%%%%%%%%%%%%%%%%%%%%%%%%%%%%%%%%%%%%%%%%%%%%%%%%%%%%
%%%%%%%%%%%%%%%%%%%%%%%%%%%%%%%%%%%%%%%%%%%%%%%%%%%%%%%%%%%%%%%%%%%%%%%%%%%%%%%%

%% VM I commented out the following paragraph since it feels out of place
% (maybe better in the intro or background) or duplicated (from the intro or
% background).
%% 
% Structured \p\ network algorithms use a global distributed hash table or a
% prefix tree structure to uniquely lookup peers or their data in the overlay
% network. As all the data is kept within the overlay, each node behaves as a
% client and a server, therefore nodes join and leave according to rules
% determined by the integrity of the global data structure. The main advantage of
% the structured \p\ topology is that with the help of the global data structure,
% peers or their data can be found within the network even if there is only a
% single copy of that item present. However, each node join and leave creates
% maintenance overhead for the network due to updates required by the global data
% structure, and for networks with frequent node arrivals and departures, the
% overlay uses valuable network resources just to update the global structure.
% Nodes join the network by using a key value, which determines the location and
% the neighborhood of the new node within the network. However, assigning
% random key values to the newly inserted nodes creates non-optimal matching with
% the underlying physical network topology, therefore, increasing the overhead of
% the network even more. One solution for handling the topology mismatch problem
% is to consider the proximity of the peers when generating the key and joining
% the node to the network, so that nodes within the same network domains are
% selected as peers, or neighbors, during the overlay topology construction.

This section outlined
% gathered together  some of the most distinctive 
representative research that attempts 
to address the issue of inefficient network resource utilization 
as this is created by the network topology mismatch
in structured \p-systems.
Depending on the salient features the surveyed techniques seek to explore,
\cite{CDHR2002,CDCR2002,RSS2002} categorize the solutions based on:
%%AD the categorization is due to 3 above citations??? Us?? ;-) 
% A categorization of the methods used for achieving their goals is the
% following based on previous work from
% \cite{CDHR2002,CDCR2002,RSS2002}:
\begin{enumerate}[\itshape i\upshape)]
  \item geographic-layout,
  \item proximity routing, and 
  \item proximity neighbor selection,
\end{enumerate}

%%COM-VM Added the following to mention the hybrid/combo methodologies
Similarly to unstructured approaches, most of the approaches 
reviewed in this section entail features from multiple of the above categories. 
For example, \cite{KLKP2008}~uses both proximity routing 
and proximity neighbor selection to achieve topology adaptation 
or \emph{LAPTOP} combines geographic--layout with
replication strategies to refine its results. 
%%%
The heuristic nature of the suggested solutions opens them up 
to trade-offs which in most cases are 
highly affected by the application domain and the implementation itself.
%%AD this is like instructions for manual - I think the point is made and 
%%	so we can avoid this.. 
% Thus, the reader is advised to filter the results of each approach and try to
% think about how the points presented will work in her own particular problem.

Table~\ref{structured:table} offers a quick overview of how the surveyed 
approaches behave.  
The topics of topology adaptation, landmarking and caching/replication
are not further elaborated as they have been discussed 
in Section~\ref{section:unstructured}.
In a similar fashion to Table~\ref{unstructured:table}, we offer 
key properties of every approach 
that differentiate it from competitors and
highlight each algorithm's pros and cons.
%%AD rephrased what is below as above..
% A quick overview can be obtained in Table~\ref{structured:table}. 
% Topology adaptation, landmarking and caching/replication have been visited during the
% discussion on unstructured systems in Section~\ref{section:unstructured} so
% we will not elaborate more here. 
% As in the previous section, we provide the
% key properties of each algorithm that differentiate it from the others and
% highlight each algorithm's advantages and disadvantages.

%%%%%%%%%%%%%%%%%%%%%%%%%%%%%%%%%%%%%%%%%%%%%%%%%%%%%%%%%%%%%%%%%%%%%%%%%%%%%%%%
\subsubsection{Geographic-layout} \label{section:geographic_layout}

% In the \emph{geographic layout (GL)} category fall methods which practically
This category consists of methods that 
position physically nearby nodes together in the application space as 
Figure~\ref{figure:geographic-layout} depicts. 
This is accomplished by adjusting and maintaining the routing tables 
of all involved peers by exploiting proximity information 
that describe the overall geographic positioning of peers. 
%%AD I am not sure I follow the 1-line above... 
%%
\begin{figure}[ht]
\centering
  \includegraphics[scale=0.4]{img/pdf/geographic-layout.pdf}
\caption{Successive nodes in the overlay circle can be chosen to be from the
same or near autonomous systems (\emph{AS}).}
\label{figure:geographic-layout}
\end{figure}
%%
A number of the reviewed algorithms including 
\emph{Global Soft-State},
\emph{Mithos} and \emph{LAPTOP} resort to clustering close-by nodes in
order to reduce the network's average delay.
Landmark servers and \emph{RTT} measurements are two popular methods, which
can also be used in conjunction, to discover peers in physical proximity.
However, as already discussed, these
methods do not always yield reliable estimates for node positions over the
Internet. 
Landmark servers are not self-organizing and present maintenance
overheads. 
To serve a \p--network with millions of peers, multiple landmark
servers distributed over the world is required, a difficult proposition.
\emph{RTT}s can measure the delay between peers, but their use constitutes a
greedy method that can result in non-optimal overlay topologies;
the latter is especially true if nearby 
nodes happen to be connected through low bandwidth connections.

On one hand, managing the overlay structure based on \emph{geographic layout} 
of the nodes improves the query efficiency of the system in general. 
On the other, it tends to undermine the uniform distribution 
of peer identifiers which in turn terns to create 
hot-spots and limited node resilience to failures.
Such resilience appears to suffer as
nearby nodes are more likely to suffer collective failures~\cite{HY2007}.
%%AD changed the wording above - check...

% must
% cope with the inherent trade os associated with selecting
% identiers based on location: randomly selected identiers
% are essential for the scalability and security of the system
% and help to balance load while location-aware identiers pre-
% vent messages from taking extensive scenic routes (on the
% physical level). 


%%%%%%%%%%%%%%%%%%%%%%%%%%%%%%%%%%%%%%%%%%%%%%%%%%%%%%%%%%%%%%%%%%%%%%%%%%%%%%%%
%
% TODO: Geographic Layout
%
%\begin{itemize}
% \item Geographic Layout\\
%The node IDs are assigned in such a way that nodes close by in the physical
%network topology, be close in the node ID space as well. Implementations that
%work relatively well with this approach have been incorporated into CAN. Nodes
%measure the RTT between themselves and a set of landmarks in order to match the
%CAN space as much as possible to the physical one. Unfortunately, the approach
%requires well known landmark servers and for that matter is  not fully
%self-organizing which can further lead to imbalanced node distribution. On
%other DHTs, such as Chord or Pastry, another problem emerges.  To gain fault
%tolerant properties, these protocols, replicate key-value pairs on neighboring
%(in the ID space) nodes. When a proximity-based node ID assignment has been
%used, the needed failure resilience is undermined by the fact that close by
%nodes are more likely to suffer collective failures.
%

%%%%%%%%%%%%%%%%%%%%%%%%%%%%%%%%%%%%%%%%%%%%%%%%%%%%%%%%%%%%%%%%%%%%%%%%%%%%%%%%
\subsubsection{Proximity Routing (PR)}

This category of approaches % ({\emph{PR}) 
does not require routing tables to be built using any
knowledge about network proximity. 
%%AD so if no knowledge is required what happens? How you go about 
%%%
%%AD I am not sure what the following sentence says... ;-) 
On the other hand it exploits such knowledge
to choose the best next hop when routing a message as can be seen in
Figure~\ref{figure:proximity-routing}. 
%%AD The above opening 2 sentences require re-calibration.. 
%%
\begin{figure}[ht]
\centering
  \includegraphics[scale=0.4]{img/pdf/proximity-routing.pdf}
\caption{Proximity routing takes proximity information into account during the
routing process.}
\label{figure:proximity-routing}
\end{figure}
%%
We reviewed an array of methodologies that strictly adhere to 
the tenets of this category such as \emph{Mithos} and
\emph{PChord} as well as others that function along with 
proximity neighbor selection such as \emph{DynaMO}.
%%
\emph{PR} tries to balance between choosing the node that will
further progress the routing towards the destination 
and picking the closest entry in the routing table
in terms of network proximity. 
%%AD just above in this section you have not clarified the way 
%%	routing tables get populated - u have to clarify this point 
%%	in order for the contrast here to be valid.. 
As this is very difficult to achieve,  %%AD Why so? Why is it difficult??? 
the system might end up
using a greedy approach that selects low latency paths on the overlay
which ultimately maps to (suboptimal) longer paths on the physical topology. 
%%AD in the following you make a point that should be made in the 
%%	follow up segment - no???
Thus, in general, proximity neighbor selection is considered 
as a superior method to proximity routing, however, joint uses of these two protocols are also possible. 
Moreover, they are  %%AD who is they? Confused!
less effective than geographical layout when applied to CAN(-like)
implementations.
%%AD re-work carefully this segment... 


% Moreover, the technique has been
% incorporated into a version of Chord causing an increase in the overhead of node
% joins and the size as well as maintenance cost of finger tables.

%
%\cite{dabek_cfs_2001} proposes a server selection scheme for the Chord DHT, on
%the domain of proximity routing selection. In \emph{CFS}, each node predicts
%the entire lookup latency as a function of the total number of nodes and the
%average overlay next routing peer. The problem is that it is very difficult to
%have a clear picture on the total number of nodes and the average hop latency
%from the local. This leads to rough estimations that consequently decreases
%overall performance.
%

%%%%%%%%%%%%%%%%%%%%%%%%%%%%%%%%%%%%%%%%%%%%%%%%%%%%%%%%%%%%%%%%%%%%%%%%%%%%%%%%
\subsubsection{Proximity Neighbor Selection (PNS)}

% \emph{Proximity Neighbor Selection (PNS)} 
Algorithms in this category populate their routing tables using
proximity knowledge as Figure~\ref{figure:proximity-neighbour-selection} 
depicts. 
%%
\begin{figure}[ht]
\centering
  \includegraphics[scale=0.4]{img/pdf/proximity-neighbor-selection.pdf}
\caption{Proximity neighbor selection dictates that proximity is taken into
account during routing table maintenance.}
\label{figure:proximity-neighbour-selection}
\end{figure}
%%
The proximity information used here is different from that of 
landmark-based systems described in Section~\ref{section:geographic_layout};
here, \emph{TTL} values between nodes (i.e., \emph{SAT-Match}) or 
\emph{IP} address prefixes (i.e., \emph{DynaMO} or \emph{Cone}) 
are predominantly used to detect proximity. 
\cite{freedman_iploc_2005} states that $97\%$ of prefixes larger
than $24$ bits belong to a single geographical location. 
However, using a smaller number of bits creates less precise results 
and a larger number of bits may increase the burden in the network
and simultaneously reduce the possible number of neighbors. 
%%
Moreover, \cite{HLYW2005} argue that such approaches can
destroy the uniform distribution in the key space and do not work in
one-dimensional key space where the mapping is restricted to 
operate at the overlay.
Therefore, a careful selection is required in terms of
performance/accuracy trade-off. 
\emph{Tapestry}, \emph{Pastry}, and \emph{CAN} successfully incorporated
proximity neighbor selection into their algorithms. 
The routing protocol in \emph{Pastry} is based on longest node 
\emph{ID} prefix matching, while \emph{CAN} uses \emph{RTT} values to
detect nearby nodes. \cite{CDCR2002a} reports that proximity neighbor
selection is an effective proximity based method.
%%AD perhaps the ending of the previous segment should be used here...


% \item Proximity Neighbor Selection\\
%Finally, the third approach, constructs the routing tables using proximity
%knowledge. Tapestry and Pastry's mechanisms of routing table
%maintenance try to minimize the distance to nodes appearing in a peer's routing
%table. Since routing is based on longest node ID prefix match, messages
%are gradually forwarded to nearby nodes at each routing step.
%\cite{castro_proximityp2p_2002} argues on how Pastry exploits proximity
%neighbor selection in order to create a scheme that is (more) location-aware
%compared to the other well-known DHTs (CAN, Chord).
%
%\end{itemize}
%
%{\sethlcolor{yellow}\hl{
%HA: Possible criteria: based on which protocol (eg Gnutella, CAN etc), peer
%selection (topology selection, cluster, cache etc), supports dynamic update,
%runtimes}}
%%\textbf{Algorithm} & \textbf{Overlay structure} & \textbf{Forwarding} &
%\textbf{Cache} & \textbf{Overlay optimization} & \textbf{Proximity information}
%& \textbf{Base protocol} & \textbf{Dynamic update} & \textbf{Runtime} \\
%
%
%{\sethlcolor{yellow}\hl{
%HA: Possible criteria2: Overlay optimization structure, base protocol (eg
%Gnutella, CAN etc), dynamic update, runtimes, scalability}}
%
%{\sethlcolor{yellow}\hl{
%HA: maybe add also the year of publication, and see if there is a pattern in
%terms of the method and the year??}}


%%%%%%%%%%%%%%%%%%%%%%%%%%%%%%%%%%%%%%%%%%%%%%%%%%%%%%%%%%%%%%%%%%%%%%%%%%%%%%%%
% \renewcommand\arraystretch{1.1}

\begin{landscape}
%\begin{figure}[h!]
\hspace{-3ex}
\begin{center}
\footnotesize
%\begin{tabular}{
\begin{longtable}{
|m{2cm}
|m{1cm}
|m{1cm}
|m{1cm}
|m{1cm}
|m{1cm}
|m{1cm}
|m{3cm}
|m{5cm}
|
}
%|>{\columncolor[gray]{.7}}m{0.15\columnwidth}
%|>{\columncolor[gray]{.9}}m{0.05in}
%|>{\columncolor[gray]{.8}}m{0.05\columnwidth}
%|>{\columncolor[gray]{.9}}m{0.05\columnwidth}
%|>{\columncolor[gray]{.9}}m{0.05\columnwidth}
%|>{\columncolor[gray]{.8}}m{0.05\columnwidth}
%|>{\columncolor[gray]{.9}}m{0.05\columnwidth}
%|>{\columncolor[gray]{.8}}m{0.1\columnwidth}
%|>{\columncolor[gray]{.9}}m{0.1\columnwidth}
%|>{\columncolor[gray]{.8}}m{0.1\columnwidth+}
%|
%}
\caption[Summary table for structured algorithms]{Summary table for structured algorithms.} \label{structured:table} \\
\hline
%%%%%%%%%%%%%%%%%%%%%%%%%%%%%%%%%%%%%%%%%%%%%%%%%%%%%%%%%%%%%%%%%%%%%%%%%%%%%%%%
% first head
\rowcolor[gray]{.5}
\textbf{Algorithm / Paper} &
\textbf{Topology Adaptation} &
\textbf{Landmarking} &
\textbf{Proximity Routing} &
\textbf{Proximity Neighbor Selection} &
\textbf{Geographic Layout} &
\textbf{Caching / Replication} &
\textbf{Highlights} &
\textbf{Pros / Cons}\\
\hline
\endfirsthead
%%%%%%%%%%%%%%%%%%%%%%%%%%%%%%%%%%%%%%%%%%%%%%%%%%%%%%%%%%%%%%%%%%%%%%%%%%%%%%%%
% subsequent heads
\multicolumn{9}{c}%
{\tablename\ \thetable\ -- \textit{Continued from previous page}} \\
\hline
\rowcolor[gray]{.5}
\textbf{Algorithm / Paper} &
\textbf{Topology Adaptation} &
\textbf{Landmarking} &
\textbf{Proximity Routing} &
\textbf{Proximity Neighbor Selection} &
\textbf{Geographic Layout} &
\textbf{Caching / Replication} &
\textbf{Highlights} &
\textbf{Pros / Cons}\\
\hline
\endhead
%%%%%%%%%%%%%%%%%%%%%%%%%%%%%%%%%%%%%%%%%%%%%%%%%%%%%%%%%%%%%%%%%%%%%%%%%%%%%%%%
% foot
\hline \multicolumn{9}{r}{\textit{Continued on next page}} \\
\endfoot
%%%%%%%%%%%%%%%%%%%%%%%%%%%%%%%%%%%%%%%%%%%%%%%%%%%%%%%%%%%%%%%%%%%%%%%%%%%%%%%%
% last foot
\hline
\endlastfoot
%%%%%%%%%%%%%%%%%%%%%%%%%%%%%%%%%%%%%%%%%%%%%%%%%%%%%%%%%%%%%%%%%%%%%%%%%%%%%%%%
% data
\textbf{Global Soft-State} &
{\large \CheckedBox} &
{\large \CheckedBox} &
{\large \Square} &
{\large \Square} &
{\large \CheckedBox} &
{\large \Square} &
\begin{tabular}[l]{m{3cm}}
Hybrid landmark binning.\\
Probing scheme for proximity detection.\\
Strategic injection of proximity info around the network.\\
Subscription-notification system to dynamically adapt to network changes.
\end{tabular} &
\begin{tabular}[l]{m{5cm}}
+ Greatly reduces routing latency to far away nodes.\\
-- Host state maintenance.\\
-- Unable to identify nodes that are close to routers/gateways.\\
-- Its static nature sacrifices the self-organizing attribute of DHTs.
\end{tabular}
\\
\hline
%%%%%%%%%%%%%%%%%%%%%%%%%%%%%%%%%%%%%%%%%%%%%%%%%%%%%%%%%%%%%%%%%%%%%%%%%%%%%%%%
\textbf{Mithos} &
{\large \Square} &
{\large \Square} &
{\large \CheckedBox} &
{\large \Square} &
{\large \CheckedBox} &
{\large \Square} &
\begin{tabular}[l]{m{3cm}}
Directed incremental probing.\\
Synthetic coordinates.
\end{tabular} &
\begin{tabular}[l]{m{5cm}}
-- Does not effectively manage dynamic peer arrival and leave.
\end{tabular}
\\
\hline
%%%%%%%%%%%%%%%%%%%%%%%%%%%%%%%%%%%%%%%%%%%%%%%%%%%%%%%%%%%%%%%%%%%%%%%%%%%%%%%%
\textbf{LAPTOP} &
{\large \CheckedBox} &
{\large \Square} &
{\large \Square} &
{\large \Square} &
{\large \CheckedBox} &
{\large \CheckedBox} &
\begin{tabular}[l]{m{3cm}}
Tree-based hierarchy.
\end{tabular} &
\begin{tabular}[l]{m{5cm}}
+ Reduces hops during message routing.\\
+ Minimal maintenance overhead.\\
-- Heartbeat approach incurs overhead even when not needed.
\end{tabular}
\\
\hline
%%%%%%%%%%%%%%%%%%%%%%%%%%%%%%%%%%%%%%%%%%%%%%%%%%%%%%%%%%%%%%%%%%%%%%%%%%%%%%%%
\textbf{Landmark Binning} &
{\large \Square} &
{\large \CheckedBox} &
{\large \Square} &
{\large \Square} &
{\large \CheckedBox} &
{\large \Square} &
Binning &
\\
\hline
%%%%%%%%%%%%%%%%%%%%%%%%%%%%%%%%%%%%%%%%%%%%%%%%%%%%%%%%%%%%%%%%%%%%%%%%%%%%%%%%
\textbf{Proximity in Kademlia} &
{\large \CheckedBox} &
{\large \Square} &
{\large \CheckedBox} &
{\large \CheckedBox} &
{\large \Square} &
{\large \Square} &
\begin{tabular}[l]{m{3cm}}
% Replacing XOR metric with a function that minimizes the underlying cost.\\
% Clustering (MaxMind geolocation technology).
\end{tabular} &
\begin{tabular}[l]{m{5cm}}
+ Proximity routing works in Kademlia to improve connection locality.
\end{tabular}
\\
\hline
%%%%%%%%%%%%%%%%%%%%%%%%%%%%%%%%%%%%%%%%%%%%%%%%%%%%%%%%%%%%%%%%%%%%%%%%%%%%%%%%
\textbf{CHOP6} &
{\large \Square} &
{\large \Square} &
{\large \CheckedBox} &
{\large \Square} &
{\large \Square} &
{\large \Square} &
\begin{tabular}[l]{m{3cm}}
Ipv6 format exploitation.\\
Additional RTT information.
\end{tabular} &
\\
\hline
%%%%%%%%%%%%%%%%%%%%%%%%%%%%%%%%%%%%%%%%%%%%%%%%%%%%%%%%%%%%%%%%%%%%%%%%%%%%%%%%
\textbf{T2MC} &
{\large \Square} &
{\large \CheckedBox} &
{\large \CheckedBox} &
{\large \Square} &
{\large \Square} &
{\large \Square} &
\begin{tabular}[l]{m{3cm}}
Clustering using trace-route logs.
\end{tabular} &
\begin{tabular}[l]{m{5cm}}
+ Prioritize interaction of peers and edge gateways.\\
-- Extended overhead.\\
-- Trace-route is sometimes disabled by ISPs.
\end{tabular}
\\
\hline
%%%%%%%%%%%%%%%%%%%%%%%%%%%%%%%%%%%%%%%%%%%%%%%%%%%%%%%%%%%%%%%%%%%%%%%%%%%%%%%%
\textbf{PChord} &
{\large \Square} &
{\large \Square} &
{\large \CheckedBox} &
{\large \Square} &
{\large \Square} &
{\large \Square} &
\begin{tabular}[l]{m{3cm}}
Maintenance of a proximity list.
\end{tabular} &
\begin{tabular}[l]{m{5cm}}
+ Dynamic maintenance of the proximity list.\\
+ Reduces routing costs.\\
+ Can constrain costly jumps in and out of network partitions.\\
+ Maintenance cost kept to a minimal (node join/leave heartbeat for lists)
\end{tabular}
\\
\hline
%%%%%%%%%%%%%%%%%%%%%%%%%%%%%%%%%%%%%%%%%%%%%%%%%%%%%%%%%%%%%%%%%%%%%%%%%%%%%%%%
\textbf{AChord} &
{\large \Square} &
{\large \Square} &
{\large \CheckedBox} &
{\large \Square} &
{\large \Square} &
{\large \Square} &
\begin{tabular}[l]{m{3cm}}
Exploits IPv6's any-cast mechanism.
\end{tabular} &
\begin{tabular}[l]{m{5cm}}
+ Relieved from the bootstrapping process.\\
+ Easy to implement (just the additional neighborhood table).
\end{tabular}
\\
\hline
%%%%%%%%%%%%%%%%%%%%%%%%%%%%%%%%%%%%%%%%%%%%%%%%%%%%%%%%%%%%%%%%%%%%%%%%%%%%%%%%
\textbf{Chord6} &
{\large \Square} &
{\large \Square} &
{\large \CheckedBox} &
{\large \Square} &
{\large \Square} &
{\large \Square} &
\begin{tabular}[l]{m{3cm}}
Exploits IPv6's hierarchical features.
\end{tabular} &
\begin{tabular}[l]{m{5cm}}
+ Reduces inter-domain traffic between ISPs.\\
+ Easily portable to other DHTs.\\
-- Cannot distinguish between nearby domains to identify the best
next hop when an inter-domain step must be taken.
\end{tabular}
\\
\hline
%%%%%%%%%%%%%%%%%%%%%%%%%%%%%%%%%%%%%%%%%%%%%%%%%%%%%%%%%%%%%%%%%%%%%%%%%%%%%%%%
\textbf{DHT-PNS} &
{\large \Square} &
{\large \Square} &
{\large \Square} &
{\large \CheckedBox} &
{\large \Square} &
{\large \Square} &
\begin{tabular}[l]{m{3cm}}
Grouping through synthetic coordinates.\\
Partitioning using a concentric cycle clustering scheme.
\end{tabular} &
\begin{tabular}[l]{m{5cm}}
-- It assumes uniform node distribution which is not always the case in a synthetic coordinate system.\\
-- Clustering nodes might be a single point of failure.
\end{tabular}
\\
\hline
%%%%%%%%%%%%%%%%%%%%%%%%%%%%%%%%%%%%%%%%%%%%%%%%%%%%%%%%%%%%%%%%%%%%%%%%%%%%%%%%
\textbf{Quasi-Chord} &
{\large \Square} &
{\large \CheckedBox} &
{\large \Square} &
{\large \CheckedBox} &
{\large \Square} &
{\large \Square} &
\begin{tabular}[l]{m{3cm}}
Geometric space coordinates (GNP)\\
Transformation of the coordinate space into 1-d Cantor space for easier mapping to the Chord hierarchy\\
Two finger tables (clockwise, anti-clockwise)
\end{tabular} &
\begin{tabular}[l]{m{5cm}}
-- Not fully distributed (GNP is landmark based).\\
-- Makes an indirect assumption of a maximum number of allowable hosts.\\
-- Doubles the required routing information which needs to be created and maintained.
\end{tabular}
\\
\hline
%%%%%%%%%%%%%%%%%%%%%%%%%%%%%%%%%%%%%%%%%%%%%%%%%%%%%%%%%%%%%%%%%%%%%%%%%%%%%%%%
\textbf{IPBC} &
{\large \Square} &
{\large \Square} &
{\large \Square} &
{\large \CheckedBox} &
{\large \Square} &
{\large \Square} &
\begin{tabular}[l]{m{3cm}}
IP address prefixes (16 bit for IPv4) to detect proximity.
\end{tabular} &
\begin{tabular}[l]{m{5cm}}
+ Prefix is stored in the DHT so the proximity identification becomes as easy as to query the prefix.\\
+ DHT maintenance mechanisms for both voluntary or ungraceful departures of
peers.\\
-- Performance/accuracy trade-off in choosing the prefix length to be used.
\end{tabular}
\\
\hline
%%%%%%%%%%%%%%%%%%%%%%%%%%%%%%%%%%%%%%%%%%%%%%%%%%%%%%%%%%%%%%%%%%%%%%%%%%%%%%%%
\textbf{Cone} &
{\large \Square} &
{\large \CheckedBox} &
{\large \Square} &
{\large \CheckedBox} &
{\large \Square} &
{\large \Square} &
\begin{tabular}[l]{m{3cm}}
2-layer ID space.\\
Group concept for dividing nodes according to a common Chord id prefix.
\end{tabular} &
\begin{tabular}[l]{m{5cm}}
+ The first step of the routing scheme exploits proximity thus reducing message exchange costs.\\
-- Not fully distributed as it relies on landmarks.\\
-- Maintenance of three routing tables is needed.
\end{tabular}
\\
\hline
%%%%%%%%%%%%%%%%%%%%%%%%%%%%%%%%%%%%%%%%%%%%%%%%%%%%%%%%%%%%%%%%%%%%%%%%%%%%%%%%
\textbf{DynaMo} &
{\large \Square} &
{\large \CheckedBox} &
{\large \Square} &
{\large \CheckedBox} &
{\large \Square} &
{\large \Square} &
\begin{tabular}[l]{m{3cm}}
Random Landmarking (RLM).\\
Closest Neighbor Prefix Assignment (CPNA).\\
Making last hop as local as possible.
\end{tabular} &
\begin{tabular}[l]{m{5cm}}
+ Developed with mobile, ad-hoc networks in mind.\\
+ Evenly distributed IDs.\\
+ Dynamic landmarking is a fully distributed approach.\\
-- RLM imposes extra network stress especially to landmark nodes.\\
-- CPNA is a coarse-grained proximity approach.
\end{tabular}
\\
\hline
%%%%%%%%%%%%%%%%%%%%%%%%%%%%%%%%%%%%%%%%%%%%%%%%%%%%%%%%%%%%%%%%%%%%%%%%%%%%%%%%
\textbf{SAT-Match} &
{\large \Square} &
{\large \Square} &
{\large \Square} &
{\large \CheckedBox} &
{\large \Square} &
{\large \Square} &
\begin{tabular}[l]{m{3cm}}
Selective jumps to adjust peer positioning in the DHT.\\
Stretch reduction scheme.
\end{tabular} &
\begin{tabular}[l]{m{5cm}}
+ Continuously adaptive mechanism.\\
+ It can be considered lightweight.\\
+ Can coexist with other approaches (like landmarking).\\
+ Works sufficiently for large scopes as well as environments with high churn.\\
+ Compared to Mithos, scales much better.\\
-- Contention situation in selective jump phase.\\
-- Probing phase incurs unnecessary traffic.
\end{tabular}
\\
\hline

%%%%%%%%%%%%%%%%%%%%%%%%%%%%%%%%%%%%%%%%%%%%%%%%%%%%%%%%%%%%%%%%%%%%%%%%%%%%%%%%

% \hline
% \textbf{Global Softstate} &
% \textbf{Geographic layout} Uses first landmark clustering
% then measures RTTs to identify close nodes &  &  \\

% \hline
% \textbf{Mithos} &
% \textbf{Geographic layout} Uses nodes as topology landmarks and directed
% incremental probing to optimize topology & & Scales well as all
% operations are local ??? \\
% 
% \hline
% \textbf{Self-Adaptive Topology Matching} &
% \textbf{Proximity neighbor selection} Uses lightweight probing and
% selective jumps to optimize the topology & CAN &  Better than Mithos \\

% \hline
% \textbf{Delay Aware P2P System} & \textbf{} & & \\

% \hline
% \textbf{VERSION OF CHORD - DHT-PNS} &
% \textbf{Proximity neighbor selection} Uses Proximity Neighbor Selection and
the Vivaldi
% system & Chord  &  \\

% \hline
% \textbf{MAY OMIT- VERSION OF CHORD - Quasi-Chord} &
% \textbf{Proximity based}  & Chord  &  \\

% \hline
% \textbf{LAPTOP} &
% \textbf{Geographic layout} Hierarchical overlay structure  &  & routing path length $\log{_d N}$,
% join/leave overhead $d\log{_d N}$ \\

% \hline
% \textbf{IP-Based Clustering} &
% \textbf{Proximity neighbor selection} Proximity neighbor selection based on
longest common
% prefix of IP addresses &    &  \\

% \hline
% \textbf{CHOord considering Proximity on IPv6} &
% \textbf{Proximity routing} Uses IPv6 address format to provide proximity &  Chord
%  & Better than Chord \\

% \hline
% \textbf{Proximity in Kademlia} &
% \textbf{Proximity routing} Applies  proximity neighbor selection (PNS) and
proximity route selection (PRS)
% to Kademlia & Kademlia &   \\

% \hline
% \textbf{Cone} &
% \textbf{Proximity neighbor selection} Uses proximity neighbor selection (PNS)
& Chord  & Better than Chord \\

% \hline
% \textbf{DynaMO} &  &  &  \\
% 
% \hline
% \textbf{MAY OMIT-BADLY WRITTEN-PChord} &
% \textbf{Proximity based}  &  Chord  & \\
% 
% \hline
% \textbf{AChord} &  & & \\

% \hline
% \textbf{Chord6} &
% \textbf{Proximity routing} Uses IPv6 hierarchical address format to cluster
% topologically close nodes & Chord  &  \\

% \hline
\end{longtable}
\end{center}
\vspace{-2.5ex}
\vspace{-2.5ex}
%\label{fig:struct_compare_table}
%\end{figure}
 \end{landscape}


%%%%%%%%%%%%%%%%%%%%%%%%%%%%%%%%%%%%%%%%%%%%%%%%%%%%%%%%%%%%%%%%%%%%%%%%%%%%%%%%
%%%%%%%%%%%%%%%%%%%%%%%%%%%%%%%%%%%%%%%%%%%%%%%%%%%%%%%%%%%%%%%%%%%%%%%%%%%%%%%%
%                               STRUCTURED
%%%%%%%%%%%%%%%%%%%%%%%%%%%%%%%%%%%%%%%%%%%%%%%%%%%%%%%%%%%%%%%%%%%%%%%%%%%%%%%%
%%%%%%%%%%%%%%%%%%%%%%%%%%%%%%%%%%%%%%%%%%%%%%%%%%%%%%%%%%%%%%%%%%%%%%%%%%%%%%%%

%\pgfplotsset{width=7cm,compat=newest}

%%%%%%%%%%%%%%%%%%% EFFICIENCY %%%%%%%%%%%%%%%%%%%
\begin{landscape}
\begin{center}
\begin{tikzpicture}
\begin{axis}[
  ybar,
%  bar width=7pt,
  xlabel=Algorithm,
  ylabel= Efficiency,
  symbolic x coords={
    GlobalSoftState,
    Mithos,
    LAPTOP,
    %Landmark Binning,
    ProximityInKademlia,
    CHOP6,
    T2MC,
    PChord,
    AChord,
    Chord6,
    DHTPNS,
    QuasiChord,
    IPBC,
    Cone,
    DynaMo,
    SATMatch,
    DelayAwareP2PSystem
  },
  symbolic y coords={lo, med, hi},
  x tick label style={rotate=45,anchor=east},
  xtick=data, ytick=data,
%   ymin=low,ymax=high,ytickmin=low,
  height=\textheight - 2cm,
  width=\textwidth - 2cm,
  enlargelimits=0.05
]

\addplot[black,fill=green] table[x=ALGORITHM,y=EFFICIENCY]
{structured-plot.dat};

\end{axis}
\end{tikzpicture}
\end{center}
\end{landscape}

%%%%%%%%%%%%%%%%%%% OVERHEAD %%%%%%%%%%%%%%%%%%%
\begin{landscape}
\begin{center}
\begin{tikzpicture}
\begin{axis}[
  ybar,
%  bar width=7pt,
  xlabel=Algorithm,
  ylabel= Overhead,
  symbolic x coords={
    GlobalSoftState,
    Mithos,
    LAPTOP,
    %Landmark Binning,
    ProximityInKademlia,
    CHOP6,
    T2MC,
    PChord,
    AChord,
    Chord6,
    DHTPNS,
    QuasiChord,
    IPBC,
    Cone,
    DynaMo,
    SATMatch,
    DelayAwareP2PSystem
  },
  symbolic y coords={lo, med, hi},
  x tick label style={rotate=45,anchor=east},
  xtick=data, ytick=data,
%   ymin=low,ymax=high,ytickmin=low,
  height=\textheight - 2cm,
  width=\textwidth - 2cm,
  enlargelimits=0.05
]

\addplot[black,fill=red] table[x=ALGORITHM,y=OVERHEAD]
{structured-plot.dat};

\end{axis}
\end{tikzpicture}
\end{center}
\end{landscape}

%%%%%%%%%%%%%%%%%%% SCALABILITY %%%%%%%%%%%%%%%%%%%
\begin{landscape}
\begin{center}
\begin{tikzpicture}
\begin{axis}[
  ybar,
%  bar width=7pt,
  xlabel=Algorithm,
  ylabel= Scalability,
  symbolic x coords={
    GlobalSoftState,
    Mithos,
    LAPTOP,
    %Landmark Binning,
    ProximityInKademlia,
    CHOP6,
    T2MC,
    PChord,
    AChord,
    Chord6,
    DHTPNS,
    QuasiChord,
    IPBC,
    Cone,
    DynaMo,
    SATMatch,
    DelayAwareP2PSystem
  },
  symbolic y coords={lo, med, hi},
  x tick label style={rotate=45,anchor=east},
  xtick=data, ytick=data,
%   ymin=low,ymax=high,ytickmin=low,
  height=\textheight - 2cm,
  width=\textwidth - 2cm,
  enlargelimits=0.05
]

\addplot[black,fill=blue] table[x=ALGORITHM,y=SCALABILITY]
{structured-plot.dat};

\end{axis}
\end{tikzpicture}
\end{center}
\end{landscape}



%%%%%%%%%%%%%%%%%%%%%%%%%%%%%%%%%%%%%%%%%%%%%%%%%%%%%%%%%%%%%%%%%%%%%%%%%%%%%%%%
%\section{The Algorithms}
%Having understood the nature of the problem that peer-to-peer architecture face, this section starts the discussion of the academic work that has been conducted in the field the last few years. What follows this introduction, does not claim to be a thorough citation of all known protocols that are available out there, but instead, a carefull selection of those that left a distinctive fingerprint contribution in the efforts of the research community to alleviate the topology mismatch problem.
%
%%%%%%%%%%%%%%%%%%%%%%%%%%%%%%%%%%%%%%%%%%%%%%%%%%%%%%%%%%%%%%%%%%%%%%%%%%%%%%%%
%% UNSTRUCTURED
%%%%%%%%%%%%%%%%%%%%%%%%%%%%%%%%%%%%%%%%%%%%%%%%%%%%%%%%%%%%%%%%%%%%%%%%%%%%%%%%
%
%\subsection{Narada}
%
%\paragraph*{}
%\emph{End System Multicast} \cite{chu_esm_2000,chu_esm_2001,chu_esm_2002} as well as other  proposes an architecture, where end-systems implement all multicast related functionality (including membership management and packet replication). In \emph{Narada}, the protocol developed in the paper's context, end-systems
%\begin{inparaenum}[\itshape i\upshape)]
%  \item self-organize, in a fully distributed manner,
%  \item into an efficient overlay structure, both from the network and the application perspective,
%  \item with self improving properties,
%  \item adapting to network dynamics\footnote{Long-term variations in Internet path characteristics (i.e. bandwidth, latency e.t.c.)}.
%\end{inparaenum}
%
%\paragraph*{}
%What Narada really does is to construct overlay spanning trees in a two-step process:
%\begin{enumerate}
%  \item It builds a richer connected graph \footnote{ENarada \cite{li_enarada_2008} used Gossip protocol for the construction}, termed a \emph{mesh}, while trying to ensure a couple of desired performance properties:
%    \begin{itemize}
%      \item path quality between any pair of members should be comparable to the quality of the unicast path between them, and
%      \item each member should have a limited number of neighbours on the mesh.
%    \end{itemize}
%
%  \item It computes minimum spanning trees out of the mesh, each rooted at the corresponding source member.
%\end{enumerate}
%
%\paragraph*{}
%Narada incrementally improves the quality of the mesh by adding and dropping overlay links. Members probe each other at random and new links are added depending on what is the gain in doing so. On the other hand, participants, continuously  monitor the utility of existing links, and drop them if they are found to be not usefull.
%
%\paragraph*{}
%Narada protocol achieves relatively good performance for small and medium sized groups involving tens to hundreds of members. This is not, by no means, what a contemporary peer-to-peer system needs to be characterized as scalable, but this work is considered pioneering\footnote{Another similar approach is \emph{Scattercast}\cite{chawathe_scattercast_2000}}, in that it was the first to have conducted a detailed Internet evaluation to analyse and demonstrate the feasibility of an overlay-based, application layer, multicast architecture that can dynamically adapt to bandwidth and latency properties of the physical underlying infrastructure.
%
%\subsection{Gia}
%
%\paragraph*{}
%\emph{Gia} \cite{chawathe_gia_2003} is a decentralized unstructured peer-to-peer file sharing system with the following characteristics:
%\begin{inparaenum}[\itshape i\upshape)]
%  \item replaces Gnutella's flooding with random walks \cite{lv_randomwalks_2002},
%  \item recognizes the importance of overlay network's topology, while using the random walks, and exploits it by incorporating a topology adaptation algorithm,
%  \item introduces a token-based flow control mechanism, and
%  \item detects the significant heterogeneity in peer bandwidth, processing power, disk speed e.t.c. and takes advantage of it.
%\end{inparaenum}
%
%\paragraph*{}
%More specifically, there are four key components in the design of \emph{Gia} which are summarized bellow:
%\begin{enumerate}
%  \item A \emph{dynamic topology adaptation} protocol that puts participating nodes within short reach of high capacity nodes so that these \emph{high-degree} nodes, which due to their high connectivity receive most of the queries, actually have the capacity to handle them.
%  \item An \emph{active flow control} scheme is used to avoid overloaded hot-spots. Heterogeneity is detected and flow control tokens are given to nodes based on the available capacity.
%  \item Every node maintains pointers to to the content that is offered by their immediate neighbours, creating a \emph{one-hop replication} of pointers, scheme.
%  \item A \emph{search protocol} based on random walks, that is biased towards directing queries to high-capacity nodes that are typically best able to answer these queries.
%\end{enumerate}
%
%\subsection{Adaptive Overlay Topology Optimization}
%
%\paragraph*{}
%\emph{Adaptive Overlay Topology Optimization (AOTO)} \cite{liu_auto_2003} is an algorithm for building an overlay multicast tree among each source node and its direct logical neighbours in order to alleviate the mismatching problem while providing a larger query coverage range. It includes the steps of
%\begin{inparaenum}[\itshape i\upshape)]
%  \item \emph{Selective Flooding (SF)}, and
%  \item \emph{Active Topology (AT)}
%\end{inparaenum}
%which are furtherlly discussed in the following paragraphs.
%
%\paragraph*{Selective Flooding (SF)}
%Instead of flooding to all neighbours, SF builds an overlay multicast tree among each peer and its immediate logical neighbours. This reduces overall message burden to the network by making some neighbours non-flooding. The tree is formed using a minimum spanning tree algorithm. For this reason a peer needs to know the costs to all its logical neighbours (e.g. the network delay) as well as between any pair of neighbours. Additional, cost probing, messages need to be added to the original messages defined by the Gnutella protocol.
%
%LTM's SF effectiveness has be proven to be detached from the different physical or overlay topologies. On the other hand, SF is more effective with large number of logical neighbours. It can reach an average optimization rate of 87.4 percent on a logical topology with an average of 30 logical neighbours. 
%
%\paragraph*{Active Topology (AT)}
%At this step the overlay topology is reorganized. Each peer, independently, makes optimizations in the overlay network to alleviate topology mismatching by replacing non-flooding neighbours, with closer nodes. The approach incorporated by the paper is called \emph{Randomized AT} algorithm which picks up a candidate peer at random among the non-flooding neighbour's neighbours. Whenever a new neighbour cost table is received or there is a change of neighbours, the source peer has to re-calculate the multicast tree and apply the randomized AT algorithm.
%
%Different numbers of average logical neighbours has little to do with the effectiveness of AT. If the source has $n$ non-flooding peers, there are $n$ potential neighbour replacements. The overhead to exhaust all $n$ possible replacements can be high, so in practice, after each replacement the source peer can decide whether it needs to find another candidate peer. This is done by computing the cost improvement ratio greater than some predefined termination threshold. The larger the threshold, the slower, in the number of optimization steps, the reduction of the normalized average distance. As a whole the average response time is significantly reduced when more optimization steps taken.
%
%\subsection{Location-aware Topology Matching}
%
%\paragraph*{}
%\emph{Location-aware topology matching} or \emph{LTM} \cite{liu_ltm_2004}, is an algorithm of building an efficient overlay by disconnecting low productive connections and choosing physically closer nodes as logical neighbours while still retaining the search scope and reducing response time of queries. This is done by issuing detector messages by peers, to a small region so that receivers of the detector can record relative delay information and cut inefficient or redundant logical links as well as add closer nodes as direct neighbours. Three operations are defined in LTMband are furtherly discussed in the following paragraphs.
%
%\paragraph*{TTL2-detector flooding}
%\emph{TTL2-detector} messages are designed based on the specification of the Gnutella protocol. Suppose that $d(i, S, v)$ denotes the TTL2-detector which has the message ID of $i$ with TTL value of $v$ and is initiated by $S$. In addition to Gnutella's unified 23-byte header, such a message has a body of two types:
%\begin{itemize}
%  \item short format $d(i, S, 1)$ if $S$ is a direct neighbour of the receiver
%  \item long format $d(i, S, 0)$ if $S$ is within a two-hop distance from the receiver
%\end{itemize}
%This way a peer can calculate link costs from within a one- or two-hop area around it and thus reach peers in $N(S)$ and $N^2(S)$ sets, respectively, in the paper's context. Each peer floods a TTL-detector periodically while it calculates link cost to another peer by receiving other peer's detector messages and checking:
%\begin{itemize}
%  \item the Source Timestamp field and its own clock upon time of receipt for one-hop messages
%  \item the Source Timestamp, TTL1 Timestamp and its own clock upon receipt for two-hop messages
%\end{itemize}
%Of course peer clocks must be synchronized in order to do such checking.
%
%\paragraph*{Low productive connection cutting}
%There are three cases for any peer $P$ who receives $d(i, S, v)$ multiple times:
%\begin{inparaenum}[\itshape i\upshape)]
%  \item $P$ receives both $d(i, S, 1)$ and $d(i, S, 0)$
%  \item $P$ receives multiple $d(i, S, 0)$s from different paths, and P randomly chooses to process one
%  \item $P$ receives one $d(i, S, 1)$ and multiple $d(i, S, 0)$s, and $P$ processes $d(i, S, 1)$ and one randomly selected $d(i, S, 0)$
%\end{inparaenum}
%If the link with the largest cost is found and is a direct neighbour then the connection is put in a will-cut list and stays there for a certain period of time. If it is not, then it is handled by other peers. After that period, connections are cut and recorded to $P$'s cut-list.
%
%\paragraph*{Source probing}
%For a peer $P\in(N^2(S) - N(S))$ who receives only one $d(i, S, 0)$, the cost of $PS$ is obtained (with list look-up or probing). Then $P$ compares it with the cost from each hop and if $PS$ has the largest cost, $P$ will not keep this connection, while otherwise the connection will be created.
%
%\paragraph{}
%Supposing $n$ is the number of peers, $c_n$ is the average number of neighbours and $c_e$ is the average cost of logical links, then in the flooding-based search the traffic incurred by one query from an arbitrary peer in a peer-to-peer network is $O(n)$. As observed in the Gnutella network \cite{sripanidkulchai_gnutella_2001}, each peer issues $0.3$ queries per minute in average, thus the per minute traffic incurred by the network with $n$ peers is $O(n^2)$. Because each $d(i, S, v)$ has a TTL of $2$ in each source peer, the traffic for one time LTM optimization in all peers is at most $2nc_n^2c_e$. If each peer uses LTM $k$ times per minute, the total traffic incurred is $2knc_n^2c_e$. Simulation shows the best value for $k$ is $2$ or $3$. So, the traffic overhead caused by LTM to the network is $O(n)$.
%
%TTLj-detectors, with $j > 2$, would detect and break cycles with more than 4 links. LTM though, does not use such detectors because detector-flood traffic would increase significantly, and cut links between two end-peers, could cause queries initiated by them to traverse a path much more expensive than the cost on the the cut link.
%
%\paragraph{}
%LTM disadvantages are
%\begin{inparaenum}[\itshape i\upshape)]
%  \item disagreement of measured delay due to unsynchronized clocks causes problems when deciding the cut positions, which can influence the network connectivity, and
%  \item the network delay metric mainly focuses on disabling the connections between peers physically far away without considering the shortcuts created by powerful peers.
%\end{inparaenum}
%
%\begin{figure}
%\centering
%  \includegraphics[scale=0.4]{img/ltm_multid.jpeg}
%\caption{Peer $P$ receives $d(i, S, v)$ multiple times}
%\label{figure:ltm_multid}
%\end{figure}
%
%\begin{figure}
%\centering
%  \includegraphics[scale=0.4]{img/ltm_example.jpeg}
%\caption{A full example of LTM}
%\label{figure:ltm_example}
%\end{figure}
%
%\subsection{Scalable Bipartite Overlay}
%\emph{Scalable Bipartite Overlay (SBO)} \cite{liu_bipartite_2007} employs an efficient strategy to select query forwarding paths and logical neighbours. The topology construction and optimization of SBO consist of four phases:
%\begin{enumerate}[\itshape i\upshape)]
%  \item When a new peer is joining the network, it will randomly take an initial colour; say red or white. This colour is not to be changed for the time the peer stays connected. If it leaves and then rejoins it will, again, have to pick a colour, randomly. Thus all peers are separated into two groups, red or white. Then the bootstrap host will provide, the joining peer, with a list of active peers, including their colour information, in order for the later to establish connections to different colour peers. This way, all peers form a bipartite overlay.
%  \item White peers probe distances with their immediate (red) neighbours, form a cost table and send this table to their corresponding red neighbours.
%  \item Each red peer builds a minimum spanning tree using the obtained neighbour cost tables. These are two-hop diameter trees so white peers do not need to build one. The links that are part of a minimum spanning tree are called \emph{forwarding connections (FC)} while the rest \emph{non-forwarding (NFC)}.
%  \item Having a minimum spanning tree with two hops a red peer is able to send its queries within that range. Some white peers, though, have become non-forwarding neighbours. In this phase, such a (white) neighbour will try to find another red peer being two hops away from its current red neighbour to replace the later as its new neighbour.
%\end{enumerate}
%
%\begin{figure}
%\centering
%  \includegraphics[scale=0.4]{img/sbo_efficient_forward.jpeg}
%\caption{A red peer $P$ has a small overlay topology of $N(P)$ and $N^2(P)$ and computes the efficient forwarding paths}
%\label{figure:sbo_efficient_forward}
%\end{figure}
%
%\begin{figure}
%\centering
%  \includegraphics[scale=0.4]{img/sbo_neighbour_replace.jpeg}
%\caption{Neighbor replacement}
%\label{figure:sbo_neighbour_replace}
%\end{figure}
%
%In a static enviromnent LTM may reduce traffic cost by around 80 to 85 percent while SBO reduces traffic cost between 85 and 90 percent. However, LTM  is proved to converge in around 2-3 steps while SBO needs 4-5 steps. Moreover LTM reduces response time by more than 60 percent in 3 steps while SBO needs 8. In a dynamic environment (10 minute average peer lifetime, 0.3 queries/sec by each peer) SBO and LTM reduce the average traffic cost per query (including the overhead due to the optimization steps) by 85 and 80 percent, respectively. Moreover LTM reduces the response time per query to 30 percent while SBO to 35 percent.
%
%\subsection{Adaptive Connection Establishment}
%
%\paragraph{}
%\emph{Adaptive Connection Establishment (ACE)} \cite{liu_ace_2004} builds an overlay multicast tree among each source node and the peers within a certain diameter from the source peer and optimizes the neighbour connections that are not in that tree.
%
%\paragraph{}
%ACE indicates three phases for its algorithm:
%\begin{enumerate}[\itshape i\upshape)]
%  \item Calculate cost between nodes using network delay as a metric. Each peer probes the costs with its immediate logical neighbours and forms a \emph{neighbour cost table (NCT)} using a special routing message type. Two neighbouring peers exchange their NCTs in order for every peer to obtain the cost between any pair of its local neighbours forming a small overlay topology.
%  \item Based on obtained NCTs a minimum spanning tree among each peer and its immediate neighbours is built (Figure~\ref{figure:ace_phase2}).
%  \item Physically far away neighbours are replaced by physically close neighbours. In ACE a peer, say $S$, probes the distance between one of its non-flooding neighbour's neighbour, say $G$ and $H$ respectively. If the link to neighbour's neighbour is smaller than that to the neighbour, the later is cut. If this is not the case but $S$ finds that cost of $GH$ is even larger than that of the $SH$, $S$ will keep $H$ as a new neighbour. Obviously, if $SH$ is larger than $SG$ and $GH$, the connection will not be established and $S$ will continue probing another neighbour's neighbour. The algorithm described is conducted within $1$-neighbour closure (among its source peer and all its direct neighbours) but the optimization scope can be enlarged. The larger the scope, the better topology matching improvement but also the greater the computational overhead (Figure~\ref{figure:ace_phase3}).
%\end{enumerate}
%
%\paragraph{}
%Simulations in \cite{liu_acesims_2004} show that the average scope of each query to cover the same scope of nodes is reduced by about 65 percent without losing any autonomy feature, while the average response time can be reduced by 35 percent. Larger diameter topologies lead to better topology optimization rate but also to higher communication and computation overhead. It was also found that it is more effective in higher connectivity dense topologies. Compared to LTM, it comes short of convergence speed. In \cite{ni_mismatch_2004} shows reduction of both total traffic (90 percent) and response time (80 percent) to message queries without shrinking the search scope. SBO, on the other hand, achieves approximately 85 percent reduction on traffic cost and about 60 percent reduction on query response time. Last but not least, it is concluded that work must be done on incorporating a more sophisticated selection policy for candidate non-flooding peers.
%
%\begin{figure}
%\centering
%  \includegraphics[scale=0.4]{img/ace_phase2.jpeg}
%\caption{Second phase in ACE}
%\label{figure:ace_phase2}
%\end{figure}
%
%\begin{figure}
%\centering
%  \includegraphics[scale=0.4]{img/ace_phase3.jpeg}
%\caption{Third phase in ACE}
%\label{figure:ace_phase3}
%\end{figure}
%
%\subsection{Hops Adaptive Neighbour Discovery}
%
%\paragraph{}
%Paper \cite{chen_hand_2006} proposes a new algorithm named \emph{Hops Adaptive Neighbour Discovery (HAND)} which uses a fully distributed triple hop adjustment strategy to address the topology mismatch problem. The advantages of the algorithm compared to other approaches are that
%\begin{inparaenum}[\itshape i\upshape)]
%  \item it does not need any clock synchronization,
%  \item it is a fully distributed algorithm making it robust and reliable in decentralized systems,
%  \item the traffic overhead of the triple hop adjustment is very low,
%  \item it is applicable to dynamic peer-to-peer environments, and
%  \item maintains lower query response time.
%\end{inparaenum}
%
%\paragraph{}
%The algorithm's ultimate goal is the optimal overlay, the \emph{Logical Communication Network (LCN)} in the paper's context. The key concept of the algorithm is that a graph $G^{*}$ that describes an LCN and a graph $G$ that describes the current overlay are matched only if all peer hop sequences $(v_1, v_2, \ldots, v_k)$ in $G$ exist in $G^{*}$ and in the same order. In practice triple sequences $(v_1, v_2, v_3)$ are used.
%
%\paragraph{}
%The mismatching detection is done in the following way. Suppose we want to verify peer sequence $v_2-v_1-v_3$ (see Figure~\ref{figure:hand_com_overlay}). A pair of probing messages are sent from $v_1$ to $v_2$ and $v_3$. Suppose delays of $(v_1,v_2)$ and $(v_1,v_3)$ are are denoted as $x$ and $z$, respectively. When the probing message arrives to $v_2$ it forwards it directly to $v_3$. Similarly, when the probing message arrives to $v_3$, it forwards it directly to $v_2$. These last steps are performed in order to obtain delays of $(v_2,v_3)$ and $(v_3,v_2)$ physical paths, respectively, denoted by $y$.
%\begin{itemize}
%  \item If $y=z-x\pm\varepsilon$, sequence $v_2-v_1-v_3$ is mismatched and should be adjusted to $v_1-v_2-v_3$ by deleting edge $(v_1,v_3)$ and adding a new $(v_2,v_3)$ (see Figure~\ref{figure:hand_matchtriple_lemma3}).
%  \item if $y=x-z\pm\varepsilon$, sequence $v_2-v_1-v_3$ is mismatched and should be adjusted to $v_1-v_3-v_2$ by deleting edge $(v_1,v_2)$ and adding a new $(v_3,v_2)$ (see Figure~\ref{figure:hand_matchtriple_lemma4}).
%\end{itemize}
%$\varepsilon$ is a small positive real number denoting additional delays caused by possible forwarding and jitter delays.
%
%\begin{figure}
%\centering
%  \includegraphics[scale=0.4]{img/hand_com_overlay.jpeg}
%\caption{The communication in overlay}
%\label{figure:hand_com_overlay}
%\end{figure}
%
%\begin{figure}
%\centering
%  \includegraphics[scale=0.4]{img/hand_matchtriple_lemma3.jpeg}
%\caption{The matching triple in paper's Lemma 3}
%\label{figure:hand_matchtriple_lemma3}
%\end{figure}
%
%\begin{figure}
%\centering
%  \includegraphics[scale=0.4]{img/hand_matchtriple_lemma4.jpeg}
%\caption{The matching triple in paper's Lemma 4}
%\label{figure:hand_matchtriple_lemma4}
%\end{figure}
%
%\paragraph{}
%Measurements conducted for evaluation perposes showed that in a static environment the algorithm can effectively decrease traffic cost by about 77 percent and shorten the query response time by about 49 percent in less than two minutes. In a dynamic environment it shows similar behaviour and with the size of the overlay network having little impact on the effectiveness of the algorithm. Compared to LTM both algorithms have almost the same traffic reduction rate, however on the response time reduction rate HAND has a higher one by about 4 percent. The traffic overhead of HAND is much less than that of LTM by an average of 55 percent.
%
%\subsection{Distributed Cycle Minimization Protocol}
%
%\paragraph{}
%\emph{Distributed Cycle Minimization Protocol (DCMP)} which is introduced in \cite{zhu_dcmp_2008} is a dynamic, fully decentralized protocol that promises significant reduction of duplicate messages. To achieve that, it uses the slightly different approach of eliminating unnecessary cycles, while retaining the connectivity of the network and preserving fault resilience and load balancing properties of unstructured peer-to-peer schemas by avoiding the creation of a hierarchical organization.
%
%DCMP aims at cutting the cycle paths at strategic locations. Any peer that detects a duplicate message can initiate the cutting process which consists of the following steps:
%\begin{enumerate}
%  \item Peers in the cycle elect a leader called \emph{GatePeer}\footnote{GatePeers are important for maintaining the connectivity and optimal structure of the network while peers join and leave randomly.}.
%  \item The cycle is cut at a well-defind point with respect to the GatePeer
%\end{enumerate}
%
%\paragraph{}
%The first step after detecting a duplicate message by some peer is to gather information from all peers in the cycle using a new type of control message called \emph{Information Collecting Message} or \emph{ICM}. ICM contains:
%\begin{inparaenum}[\itshape i\upshape)]
%  \item a \emph{GUID}\footnote{Globally Unique IDentifier assigned to every query message generated by any node.} field same as the one of the duplicate message,
%  \item \emph{DetectionID} field which represents the direction of the connection where the duplicate was identified\footnote{This ensures the uniqueness of the ICM messages because as it travels through many cyclic paths, multiple peers will detect the duplicates and initiate an ICM message.}, and
%  \item \emph{Node Information Vector (NIV)} which contains information (bandwidth, CPU power, etc) about peers that propagated the ICM.
%\end{inparaenum}
%
%\paragraph{}
%Suppose $A$ detected the duplicate as depicted in Figure~\ref{figure:dcmp}. It then emits an ICM to $B$ and $F$, that initially contains information only about itself. Each peer that receives the ICM, appends its information and propagates it along the reverse path of the original message. Since two copies of ICM are sent, at some point, a peer, say peer $D$, will receive a duplicate ICM. Using the information in the NIVs of the ICMs, $D$, decides to cut (for example) the EF connection. To inform the other peers about its decision, it emits a \emph{Cut Message (CM)} which contains the GUID and DetectionID of the corresponding ICM and an additional field that identifies the connection to be cut. $D$, then, forwards the CM in the reverse directions from where the ICM arrived. Similarly CMs received by any peer are propagated toward the reverse path of the corresponding ICM. Eventually, either peer $E$ or peer $F$ will receive the CM and cut the connection, thus eliminating the cycle.
%
%Receiving a duplicate ICM denotes the existence of a cycle. The opposite is not true though. For example, if the cycle contains $2 \times TTL$ edges, it will not be detected, because the ICM messages will be discarded before they locate it. There is a tradeoff between preserving the connectivity of the network and minimizing the duplicates that makes such a possibility to be safely ignored.
%
%\begin{figure}
%\centering
%  \includegraphics[scale=0.4]{img/dcmp.jpeg}
%\caption{Cycle elimination methods in DCMP}
%\label{figure:dcmp}
%\end{figure}
%
%\paragraph{}
%Experiments in \cite{zhu_dcmp_2008} showed that DCMP incurs a lower delay, returns more results and decreases the number of duplicate messages by 22\%, compared to LTM. Additionally DCMP has one to two orders of magnitude less overhead, because it adopts a more ``LAZY'' approach than broadcasting control messages periodically like LTM does.
%
%\subsection{Two-Hop-Away Neighbour Comparison and Selection}
%
%\paragraph{}
%Work in \cite{liu_thancs_2005,liu_thancs_2008} proposes a distributed heuristic called \emph{Two-Hop-Away Neighbour Comparison and Selection (THANCS)} that
%\begin{inparaenum}[\itshape i\upshape)]
%  \item is completely distributed and needs no global knowledge,
%  \item presents trivial overhead compared to the query cost savings
%  \item its convergent speed of the algorithm is fast enough (faster than minimum spanning tree approaches) so that is effective to dynamic environments, and
%  \item does not shrink the search scope.
%\end{inparaenum}
%
%\paragraph{}
%THANCS is considered a \emph{local search method}, in the sense that it targets in finding a locally optimum solution, by exploiting knowledge within a 2-hop radius. The algorithm consists of two main components: \emph{piggybacking neighbour distance on queries} and \emph{neighbour comparison and selection} which are furtherly discussed bellow.
%
%\paragraph{Piggybacking neighbour distance on queries}
%Using network delay as a metric for measuring the distance, each peer probes distances with its immediate naighbours and stores information locally. For this reason a special query message type, \emph{Piggy Message (PM)}, is introduced. It is 6 bytes long and includes two fields: Neighbour IP Address and Neighbour Distance. A peer $P$ constructs a PM for its neighbour $Q$, which contains $Q$'s IP address and $Q$'s distance from $P$. When $P$ receives a query from $Q$, this PM will be piggybacked by the query that goes to all other neighbours of peer $P$. Upon receiving such a query message, each of the other neighbours will detach the PM, record the $PQ$ distance and process the query. This PM will not be further forwarded. In selecting which incoming queries  should piggyback a PM the paper proposes the \emph{pure propability-based (PPB)} and the \emph{new neighbour triggered (NNT)} policies.
%
%\paragraph{Neighbour comparison and selection}
%Figure~\ref{figure:thancs} illustrates this component of the THANCS algorithm. A peer $S$ probes the distance to all known unprobed $N^2(S)$\footnote{$N^2(S)$ denotes the set of peers being two hops away from $S$, while $N(S)$ denotes the set of direct logical neighbours of $S$.}. The distance of $SP$ is known to $S$. Upon receiving a PM from node $P$ with the distance of $PQ$, $S$ follows one of the following:
%\begin{itemize}
%  \item $Q \in \left( N(S) \cap N^2(S) \right)$, i.e. $Q$ is direct neighbour of $S$. In this case $S$ will compare cost of $SQ$, $SP$ and $PQ$. If the most costly connection is one of $SQ$ or $SP$ the corresponding link will be put into a \emph{will-cut list}\footnote{Links are not immediately disconnected when put in the will-cut list. There are useless for forwarding but are kept active, for some time, in order to serve query responses that are traveling to the source peer along the inverse search path.}. If the most costly connection is $PQ$ then $S$ will do nothing, as the fully distributed nature of the algorithm will give a chance to either $P$ or $Q$ to cut the connection.
%  \item $Q \in \left( N^2(S) - N(S) \right)$, i.e. $Q$ is a two-hop-away neighbour of $S$. If $S$ hasn't probed $Q$ before\footnote{A distance cache is maintained and looked up in such a case.}, it probes and stores the result in the \emph{distance cache}. Having the distance of $SQ$, $S$ compares costs of $SQ$, $SP$ and $PQ$. If $SQ$ is the most costly, $S$ will not establish the connection. If $SP$ is the most costly, $S$ will establish connection $SQ$ and put $SP$ in the will-cut list. If $PQ$ is the longest, $S$ will keep  the connection with both $P$ and $Q$, expecting that $P$ or $Q$ will eventually disconnect link $PQ$, later.
%\end{itemize}
%
%\begin{figure}
%\centering
%  \includegraphics[scale=0.4]{img/thancs.jpeg}
%\caption{Probing two-hop-away neighbours}
%\label{figure:thancs}
%\end{figure}
%
%\paragraph{}
%In a static environment THANCS has been proven to be effective; optimizing 45 percent out of the 60 percent of mismatched paths, constructing a nearly optimal overlay. This leads to a 60 percent reduction in traffic cost as well as a 40 percent decrease in query response time. In dynamic environments (Gnutella 0.6/Limewire super-peer-like and Ion flat-like), THANCS saves up to 70 percent of the traffic cost in the super-peer topology and 55 percent for the flat one. Average response time is also decreased by 60 and 45 percent, respectively. Generally, THANCS has similar performance to LTM, without needing synchronization. SBO, incurring half the  overhead of AOTO, reduces the traffic cost the most, while THANCS has lower response time and converges faster than SBO. THANCS is, thus, more suitable for a more dynamic environment. In addition, THANCS is easy to implement and its operation overhead is trivial, compared with the other three approaches. This design, however, has the limitation of not being easily extend to also support non-flooding-based systems.
%
%\subsection{mOverlay}
%Zhang et al. in \cite{zhang_moverlay_2004} focus on two aspects of the overlay construction. First, to achieve \emph{efficiency} of communication, a protocol should minimize unecessary long-distance hops and redundant traffic. Second, to be \emph{scalable} the overlay should be constructed in a distributed fashion and maintenance cost (data and locality management) should be minimized. Both of the above observations, are positively affected by the exploitation of node proximity at the underlying network. To accomplish it, the paper introduces the \emph{group} concept, according to which non-static (dynamic) landmarks are used to compute proximity, resulting in a highly robust and scalable overlay construction with reduced maintenance cost as well. The dynamic nature of landmarks help towards load-balance, meaning that hot-spots are avoided.
%
%\paragraph{}
%The proposed architecture, called \emph{mOverlay}, is a \emph{clustering approach} which creates a two-level hierarchical network, where on the top level we have connections between groups while on the bottom we have connections between hosts inside groups, thus attempting to recreate \emph{Small-World}-like properties for the constructed overlay network. The notion of a \emph{group}, in the paper's context, is a set of hosts that are close to each other with respect to any position $P$ in the underlying network. The distance between hosts can be 
%\begin{inparaenum}[\itshape i\upshape)]
%  \item network latency,
%  \item round-trip time,
%  \item minimum bandwidth on the links along the path connecting the two nodes, or
%  \item some other user-defined cost metric between the two nodes.
%\end{inparaenum}
%These nodes are refered to as neighbouring. Similarly groups can exchange messages with their neighbouring ones\footnote{Groups nearby, in the underlying physical network.}.
%
%According to the \emph{grouping criterion} set by the authors, a new host $Q$ belongs to some group $A$ if the distance between $Q$ and group $A$'s neighboring groups is the same as the distance between group $A$ and it's neighbouring groups. The neighbour groups thus, play the role of the dynamic landmarks and for accuracy and performance reasons the nearest of them are chosen to play that role.
%
%\paragraph{Locating process} A new coming host, $Q$, first connects to a globally known host cache called the \emph{rendezvous point (RP)} in order to retrieve the starting point in the overlay, say $A$ in group $1$. Host $Q$ then, measures its distance to host $A$. At the same time, the later, sends information about the neighbour groups of group $1$ back to host $Q$. This list is called \emph{candidate group list}, and the newcoming host sequnentially measures its distance to each of them in seek for the closest one. If the \emph{grouping criterion} is met, host $Q$ belongs to group $1$. If not, a boot host from the closest group is found and the algorithm is re-run until the criterion is met or after a predefined number of repetitions. In the later case, $Q$ creates a new group comprising itself only. The above protocol does not favour hotspots as it spreads the probability of visiting a group across the whole overlay and limits the overhead in the level of $O \left ( log N \right )$.
%
%\paragraph{General overlay operations} A set of additional protocols, are also introduced, similar to those found in traditional unstructred networks, but modified focusing on scalability and robustness. For example a protocol for \emph{group formation} is introduced that exploits the inherent characteristic of proximity, in the overlay, in order to efficiently detect the neighbouring groups of a newly formated group from the set of adjasent groups of its closest neighbour. Additionaly, during \emph{group joining} the coresponding protocol denotes the exchange of important information for group maintenance. This can be furtherly improved by \emph{information sharing} between nodes of the same group, functionality handled by a dedicated flood-like protocol\footnote{Since nodes that belong to the same group are physically close this can be achieved at a minimum price.}. Moreover, another set of distributed protocols handle the \emph{information update}. The information that needs update, in the proposed architecture, is
%\begin{inparaenum}[\itshape i\upshape)]
%  \item the host cache, when a new node joins, and
%  \item the neighbours of groups, when a close-by group is generated.
%\end{inparaenum}
%Finally, in case of \emph{host failure} or \emph{host departure} the system is able to maintain its stability since there are defined operations for periodical host cache update and group leader selection if one leaves or dies.
%
%\subsection{Distributed Domain Name Order}
%{\sethlcolor{yellow}\hl{HA:  In paper/introduction nice advantages of unstructured networks:
%
%Unstructured P2P networks o?er a number of
%important advantages: (i) An unstructured network
%imposes very small demands on individual nodes,
%and more speci?cally it allows nodes to join or leave
%the network without signi?cantly a?ecting the sys-
%tem performance. (ii) Unstructured networks are
%appropriate for content-based retrieval (e.g., key-
%word searches) as opposed to object identi?er loca-
%tion of structured overlays. (iii) Finally unstructured
%networks can easily accommodate nodes of vary-
%ing power. Consequently, they scale to very large
%sizes and they o?er more robust performance in
%the presence of node failures and connection
%unreliability.
%}}
%
%\paragraph{}
%\cite{z-yk_ddno_2005} proposes the \emph{Distributed Domain Name Order (DDNO)} technique which makes unstructured overlay networks, topologically aware. It results in a flat overlay topology but with some changes it can be utilized in \emph{superpeer} environments. In DDNO, a node of degree $d$, tries to connect to $\frac{d}{2}$ nodes that belong to the same domain (\emph{sibling} connections) and another $\frac{d}{2}$ of random nodes (\emph{random} connections). Connecting to \emph{sibling} nodes ensures the reduction of long distance message travelling and thus resulting to better performance, while \emph{random} nodes keep the structure connected.
%
%\paragraph{Discovering \emph{random} neighbours}
%Initially, when a new coming node wants to join a network $N$, uses a \emph{hostcache} mechanism to provide it with a list of nodes with which to establish its $\frac{d}{2}$ \emph{random} connections.
%
%\paragraph{Discovering \emph{sibling} neighbours}
%In order to discover the rest $\frac{d}{2}$, \emph{sibling}, nodes, the \emph{lookupDN} procedure is initiated. In this phase, a special $l$ walker message is multicasted, by the newcomer, in search of a node that knows\footnote{A special structure is used called ZoneCache that contains information about which nodes are reachable in an $r$-hop radius.} how to guide $n$ to a \emph{sibling} node.
%
%\paragraph{}
%Figure~\ref{figure:ddno_lookupdn} shows an $l$ message emitted by node $n$ which travels along the path [$a$, $b$, $c$, $e$, $b$, $d$]. At node $d$, $l$ finds information to make a decision on which neighbour to follow next\footnote{Contrary to the random choices, made at the previous hops.}. Ultimately $l$ arrives at $m$, which is a \emph{sibling} node to $n$. $m$ then issues a broadcast message to all its own \emph{siblings}. Each of the receiving nodes, $m$ included, will respond to $n$ with a designated $l^{'}$ message if it is willing to accept new connections and out of these answers, node $n$, will attempt to establish its, remaining, $\frac{d}{2}$ \emph{sibling} connections.
%
%\paragraph{}
%Unfortunately, use of DDNO results in networks which have a uniform distribution of node degrees, losing the valuable properties of Power Law and Small World networks, as mentioned previously in this survey.
%
%\begin{figure}
%\centering
%  \includegraphics[scale=0.375]{img/ddno_lookupdn.jpeg}
%\caption{Domain name look-up in a \emph{DDNO} topology}
%\label{figure:ddno_lookupdn}
%\end{figure}
%
%%\subsection{Critical Topology-Aware Grouping}
%%
%%\paragraph{}
%%\emph{Critical Topology-Aware Grouping (CTAG)} presented in \cite{zhao_ctag_2006}, is a grouping algorithm that tries to exploit low-cost and low-delay communication of physically close-by peers. The grouping strategy is based on the \emph{IANA}\footnote{Internet Assigned Numbers Authority} and the respective \emph{Regionanal Internet Registry (RIR)}'s IP assignment strategies, according to which nodes within the same organization are always addressed from the same block. The paper proposes the \emph{Adjacency Measurement (AM)} technique which uses the longest matching IP segment criterion to calculate node proximity.
%%
%%Observations made in various studies (e.g. \cite{matei_mapgnutella_2002}) concerning the distribution of nodes among \emph{Internet Service Providers (ISPs)} and \emph{Autonomous Systems (ASs)} have shown that only $2$ to $5$ percent of Gnutella connections link peers within a single AS while more than $40$ percent of all Gnutella peers are located within the top 10 ASs. Similarly, measurements in \cite{zeinalipour-yazti_gnudc_2002} used a $244,000$ IPs test-bed and results have shown that $45$ percent of the nodes belonged to only $10$ large ISPs and $58$ percent belong to only $20$. Such results mean that most overlay generated traffic crosses AS borders increasing topology mismatch cost.
%%
%%\emph{CTAG} focuses on both the construction of the overlay as well as dynamically revising it during node interaction, phases called \emph{bootstrapping grouping} and \emph{dynamic revision}, respectively in the paper's context.
%%
%%\paragraph{Bootstrapping grouping}
%%The \emph{Gnutella Web Caching} mechanism has been modified in order for a new coming node to choose the closest \emph{GWC} in order to retrieve the node list for bootstrapping.
%%
%%\paragraph{Dynamic revision}
%%Similarly to the bootstrapping phase, \emph{AM} metric is used to store hosts' addresses. read from \emph{X-Try} headers during handshake or from \emph{QueryHit} messages. Additionally, when a node reaches the max neighbour connections, node disconnects the neighbours with the lowest \emph{AM}.
%
%\subsection{Peer-exchange Routing Optimization Protocols}
%
%\paragraph{}
%Work presented in \cite{qiu_prop_2007} introduces \emph{peer-exchange}\footnote{A series of exchanges of neighbours between two peers. One exchange can be viewed as a pair of cut-add operations.} as a basic operation to adaptively adjust the connections of the overlay network, and efficiently reduce the average logical link latency of the whole system. There are two points that differentiate this scheme from others:
%\begin{inparaenum}[\itshape i\upshape)]
%  \item it utilizes the collaboration of two peers to optimize their neighbourhood environment, than simply letting each node to ``selfishly'' choose its own strategy, and
%  \item can be deployed effortlessly on both unstructured and structured peer-to-peer systems.
%\end{inparaenum}
%
%\emph{PROP-G}\footnote{\emph{G} stands for \emph{generic}.} is the simple and direct variant of \emph{peer-exchange} in which two neighbours swap all their neighbours. As depicted in figure~\ref{figure:prop-g} the mechanism can be viewed as, nodes, exchanging overlay ``positions''. Thus, the overall overlay topology is not affected by the this operation.
%
%\begin{figure}
%\centering
%  \includegraphics[scale=0.4]{img/prop-g.jpeg}
%\caption{PROP-G in which a neighbours are exchanged.}
%\label{figure:prop-g}
%\end{figure}
%
%The alternative protocol proposed, called \emph{PROP-O}\footnote{\emph{O} stands for optimized}, ensures that the degree of each participating node remains the same by selectively choose the same number of neighbours for the exchange. With this mechanism, 2 nodes exchange the same number of neighbours in order not to change the degree
%
%\begin{figure}
%\centering
%  \includegraphics[scale=0.4]{img/prop-o.jpeg}
%\caption{PROP-O exchanging \emph{m} neighbours.}
%\label{figure:prop-o}
%\end{figure}
%
%\subsection{T2MC}
%
%\paragraph{}
%\emph{T2MC}\cite{shi_t2mc_2008} exploits some properties of the Internet paradigm and clusters nodes belonging to the same ISP without any centralised control or predefined system parameterization. The algorithm considers the dynamic nature of peers and exploits the stable nature of routers in order to build a topology location relationship among end peers.
%
%\paragraph{}
%Some special routers split the physical network into autonomous system domains. Using a Traceroute mechanism T2MC searches for latency leaps among the path to a host in order to form ``near'' and ``remote'' router clusters. This is achieved by a 2-Means Classification, defined by the following steps:
%\begin{enumerate}
% \item the peer chooses the minimum and maximum Latency results from the Traceroute for initializing the cntroinds of two sets ``first'' and ``second''.
% \item The peer calculates, for all hops allong its tracerouted path, the absolute distance to the centroids of both sets and assigns the routers to that centroid with which it has the smaller absolute distance.
% \item The peer calculates the latncy mean and variance value of two sets
% \item If the variance is larger than a predefined threshold then the algorithm takes a loop from step 2 picking the two latency mean values as new centroids of sets ``first'' and ``second''.
%\end{enumerate}
%Ultimately peer will end up with two sets having the minimum intra-set variance. Finally the peer chooses the router from ``second'' set with the minimum hops attribute and sets it as a threshold. The selected router and all others whose hops attribute is larger than the threshold are classified as ``remote'' router cluster. The remaining are classified as ``near''. From the ``near'' class, the peer chooses the one with maximum hops attribute as its edge router, and registers it along with the all the ``near'' cluster into the DHT of the p2p overlay. As new peers join the network, those that share the same edge router or any of the members of the ``near'' router clustersm ther would gather to form a ``close'' peer cluster. As Edge routers can provide more valuable information than other members of the ``near'' set, T2MC was designed to prioritize interaction of peers and edge gateways.
%
%% TODO: figure t2mc.jpeg (a) stars represent edge routers (b) hop leaps
%
%\paragraph{}
%The use of Traceroute as a tool for implementing the distance measuring infrastructure raise concearns about its efficiency and scalability. Being, primarilly, a network diagnostic utility, it is concearned too heavy weighted and intrucive for use in a larger scale\cite{ratnasamy_binning_2002}. Additionally, disabling ICMP is a common administrative policy for edge sites to enforce security, while dumping BGP routing tables\cite{krishnamurthy_bgpclust_2000} is not directly available to the application layer.
%
%\subsection{Unnamed Unstructured!!}
%% TODO: to be reviewed
%
%\paragraph{}
%In \cite{hsiao_redblue_2009}, Hsiao et al, claim to construct topology-aware unstructured overlays that \emph{guarantee} performance qualities in terms of
%\begin{inparaenum}[\itshape i\upshape)]
%  \item the expected communication latency among any two overlay peers regardless of the network size, and
%  \item the broadcasting scope of each participating peer.
%\end{inparaenum}
%
%The algorithm constructs an undirected graph $G = \left( V, E \right)$ comprised by two subgraphs. The first, namely $G^{\left( red \right)} = \left( V^{\left( red \right)}, E^{\left( red \right)} \right)$ in the paper's context, includes all vertices of $G$ and ensures the connectivity of the graph by securing at least one path between any two nodes. In contrast, $G^{\left( blue \right)} = \left( V^{\left( blue \right)}, E^{\left( blue \right)} \right)$, contains those vertices of $G$ that have free edges to link to other nodes and because these are fully utilized, the following also stands $E = E^{\left( red \right)} \cup E^{\left( blue \right)}$.
%
%A joining peer $u$, partitions its neighbours into two subsets, the $B_u^{\left( red \right)}$ and $B_u^{\left( blue \right)}$. In order to populate the $B_u^{\left( red \right)}$ subset, peer $u$ samples peers uniformly and at random. Then, each of these selected peers discovers a routing path starting from itself towards the node with the smallest (or the largest) ID in the system.
%
%%%%%%%%%%%%%%%%%%%%%%%%%%%%%%%%%%%%%%%%%%%%%%%%%%%%%%%%%%%%%%%%%%%%%%%%%%%%%%%%
%% STRUCTURED
%%%%%%%%%%%%%%%%%%%%%%%%%%%%%%%%%%%%%%%%%%%%%%%%%%%%%%%%%%%%%%%%%%%%%%%%%%%%%%%%
%
%\subsection{Landmark Binning}
%The scheme proposed in \cite{ratnasamy_binning_2002} is based on the process of \emph{binning} close by nodes (in terms of network latency) into the same cluster. The authors set the following objectives for designing their proposed algorithm:
%\begin{enumerate}
% \item simplicity (minimal support from any measuring infrastructure),
% \item scalablility (no global knowledge of the network),
% \item complete distribution (nodes need no communication-cooperation).
%\end{enumerate}
%
%\paragraph{Binning scheme}
%The implementation requires a set of well-known \emph{landmark} machines spread accross the Internet. Every newly arriving node measures its distance from these landmarks and unilaterally decides to join a specific bin based on these results. In more detail the node measures its round-trip time to each of the landmarks and orders these measurements in a decreasing order. The ordering represents a ``bin'', in the sence of close-by nodes having the same landmark ordering and hance belong to the same ``bin''. This means that a landmark system consisting of $m$ such nodes results in $m!$ potentional different bins.
%
%The algorithm can be considered scalable, as nodes need only compute distances to a small number of predefined nodes and thus without exchanging any information. A potential bottleneck could be the extra load that this ``ping''-like scheme imposes to the landmarks, especially when we need instant reaction from our topology when dealing with the dynamic nature of the p2p networks.
%
%To answer the question as to whether the algorithm actually contributes possitivelly to the construction of an enhaned overlay, the paper defines the \emph{gain ratio} as the factor by which the latency reduces when someone communicates with a random node from the same bin than with one not in the bin. This is implemented with an inter-bin to an intra-bin latency ratio.
%
%\paragraph{Structured and Unstructured Binning}
%The \emph{binning} scheme can be incorporated within either a structured or an unstructured overlay construction algorithm. The authors provide an example for both these cases.
%
%\paragraph{}
%Assuming a structured approach based on CAN\cite{ratnasamy_can_2001} and $m$ landmark nodes. The coordinate space is then partitioned into $m!$ equally sized portions, each corresonding to a single ordering of the landmarks. To do this, the first dimension is divided into $m$ areas each of which is furtherly divided (second dimension) into $m - 1$ sections and so on. Having set this $m$ dimensional space, at joining time, a node measures the delay to the set of landmarks in order to determine its associated bin and thus position itself in that portion of the coordinate space associated with its landmark ordering. Even though this scheme can 
%
%For unstructured overlays the paper assumes \emph{a set of $n$ nodes where each node picks any $k$ neighbour nodes so that the average routing latency on the resultant overlay is low (assuming shortest path routing)}. According to the proposed heuristic algorithm called \emph{BinShort-Long}, a node picks its neighbours by choosing its $\frac{k}{2}$ closest\footnote{If the node's bin is not large enough for it to pick these $\frac{k}{2}$ neighbours, it picks the required nodes from the bin that matches the most in terms of landmark ordering.} ones (named \emph{short links}), using the \emph{binning} scheme and the rest $\frac{k}{2}$ randomly (\emph{long links}). The former set produces well-connected \emph{pockets} of nearby nodes while the later preserves the connectivity of the graph, both yielding a proximity factor of $\alpha = 0.5$ in an attempt to preserve the benefitial properties of unstructured topologies\cite{merugu_str2unstr_2003}.
%
%\paragraph{}
%% TODO double check validity
%One disadvantage of this landmark scheme is related to the additional burden imposed to the landmark sites. The authors claim though that the algorithm requires so little work by the landmarks (maybe just echo to ping messages) that could in effect, act as ``unsuspecting participants''. Even if this is the case, the fact that it is not fully distributed, renders the protocol's scalability directly volnerable to any system size increase as well as usuitable for highly dynamic networks such as ad-hoc networks. Moreover, fixed points in a network are inherently more exposed to malicius attacks. The most significant downside of the algorithm though is that it can lead to an extremely uneven overlay ID distribution causing load inbalances and hot spots. Lastly, the scheme is coarse grained when it comes to distinguishing relatively close nodes\footnote{In the worst case, all nodes could ve clustered into a single bin.}.
%
%\subsection{Global Soft-State}
%{\sethlcolor{yellow}\hl{HA: in paper introduction, discussion about
%disadvantages of top-aware CAN:
%
%Techniques to exploit topology information in overlay
%routing include geographic layout, proximity routing and
%proximity neighbor selection [3]. With geographic layout
%such as topology-aware CAN [12], the overlay structure is
%constrained by underlying network topology. This tech-
%nique, unfortunately, can create uneven distribution of
%nodes in the overlay, increasing the chances of overloading
%nodes and rendering the maintenance cost formidable. Our
%study shows that, for a typical 10,000-node topology-aware
%CAN, 5% nodes occupy 85-98% of the entire Cartesian
%space, and some nodes have to maintain 450-1500 neigh-
%bors. In Proximity routing, physical topology is not consid-
%ered when constructing the overlay.
%
%[snip]
%Studies [14] have shown that triangle inequal-
%ity may not hold in Internet topology. In fact, study from
%Pastry has shown that the proximity approximation is much
%worse when using the Mercator topology that is based on
%the real measurements of the Internet [3].
%
%RELATED WORK:
%
%Miguel Castro et al [3] divide techniques used to
%exploit network proximity into three categories: geographic
%layout, proximity routing and proximity neighbor selection.
%Proximity neighbor selection is superior in terms of load
%balancing and proximity approximation. The existing algo-
%
%}}
%
%\paragraph{}
%The approach argued by Xu et.al in \cite{xu_globstate_2003} is to build a global map to help choose shorter routing paths, combining the landmark binning method and small scale distance probes to reveal the proximity properties of the underlying network to the overlay. This global view of the state is made available to all nodes in order to help them find the best way to route their messages. The authors focus on two aspects in order to accomplish their goal.
%
%\paragraph{}
%Initially, is the \emph{generation} of proximity information and then its \emph{effective exploitation}. For the first, a hybrid approach is proposed, which uses landmark clustering as a preprossesing step in order to select a number of potential nearest neighbour candidates and then refine the selection by incorporating an RTT scheme to ultimately choose the closest node. For the second, the algorithm chooses a different path from the classic gossiping approaches for constructing and maintaining the overlay. It is based on landmark-clustering-based strategic placement of proximity information on the overlay enabling any node to access such information using a landmark number that reflects its physical position in the network. For various logical regions\footnote{This might be a high-order zone in the eCAN\cite{xu_ecan_2002} context or a set of nodes sharing a particular prefix in overlays such as Pastry.} maps of physical information are built and published where each node may appear in a maximum of $log\left( N \right)$ such maps.. To dynamicaly adapt to changing network conditions, a node subscribes to relevant \emph{soft states} that utilize a notification system in order to initialize any necessary neighbour re-selection.
%
%\paragraph{}
%Maintaining several host states at different layers, makes any content migration costly. Additionaly, the method does not make any continuing effort to remap the overlay structure after a node successfuly joins, in order to adapt its state to any occurance of condition change. Although this approach greatly reduces the routin latency to far nodes, it is unable to dynamicaly identify nodes that are close to routers and gatways in order to construct the secondary overlay. Nevertheless, static recognition of such nodes is currently done based on BGP reports and prechosen landmarks, sucrificing the self-organising attribute of traditional DHTs.
%
%\subsection{Self-Adaptive Topology Matching}
%
%\paragraph{}
%Ren et.all's goal while designing the \emph{SAT-Match} algorithm \cite{ren_satmatch_2004} was to create a protocol that would be decentralised and scalable, fine-grained in detecting changes and adaptively reform in response to the dynamic nature of a classic p2p environment and last, but certainly not least, of low incuring cost. The method is in a nutshell, a two-phased iterative process that focuses on local optimizations in order to collectively achieve a global full overlay optimization.  The iteration is finised when the node detects that it is physicaly close enough to its neigbours so that no additional optimization is needed. The paper defines \emph{stretch}, $S = \frac{\bar{L_l}}{\bar{L_p}}$, as a way quantify the \emph{topology match degree} of the constructed overlay, where $\bar{L_l}$ is the average logical link latency while the $\bar{L_p}$ is the average physical link latency.
%
%\paragraph{Probing phase}
%\emph{SAT-Match} uses a small TTL value for the probing messages in order to reduce redunduncy\cite{jiang_lightflood_2008}. This process begins as soon as the node joins the network using a DHT mechanism. Each probing message contains information about the source and a small TTL value. The recipient of such a message returns information about itself to the source and forwards the probing message to its neighbours if the TTL is non-zero. The nodes been covered, are refered to as $TTL-k$ neighbourhood of the source node\footnote{Especialy, $TTL-1$ neighbourhood referes to the source's direct neighbours}. The list of responses is used to measure RTT to these nodes and sort the list ascending RTT order.
%
%\paragraph{Jump phase}
%Blindly selecting the peer with the smallest RTT as neighbour is, generally, not the right desision to make in order to achieve global \emph{stretch} reduction, This is because, in a structured scheme, when a node jumps to connect to a physically close node, it may need to connect to other distant nodes to maintain the structure's integrity, thus creating an overall increase in the overlay's \emph{stretch}. The two nodes with the smallest RTT is then used in order to select one zone to jump in this phase. The algorithm is as follows: The source node $S$ calculates the stretch change of its $TTL-1$ neighbourhood and that of the $TTL-1$ neighbourhood of the first of the previously selected peers. These calculations are made as if the jump has been made. If the stretch reduction is over a predefined threshold the jump is performed, otherwise the second selected candidate is picked and the same computations are performed. If again, the threshold is not met, then no jump is ultimately done. In case of a jump, this is performed as a combination of \emph{leave} and \emph{join} operations, in the CAN context.
%
%\paragraph{}
%Moreover the algorithm takes several issues into consiteration in order to furtherly improve the resulted overlay. For example when multiple nodes try to jump simultaneously into a regeon, then the logical link brakes from one attempt may result in inacurate computation of the gain factor, for an other. This situation is identified as \emph{contention} and the nodes use an exponatial back-off algorithm to avoid it. An other problem is the uneseccary traffic incuerred by the probing phase in a region that after several jumps has settled to a stable state. In these cases it is more likely for jump attempts to be proven worthless. The algorithm doubles the probing period of such nodes, every time a jump is not taken.
%
%\paragraph{}
%The authors claim that this continuously adaptive mechanism achieves global topology matching optimization in a sufficiently large scope. This also secures the fast adaptation to frequent network changes. It also considered lightweight and can easily be embedded into current p2p systems, as well as effectively combined with other techniques, such as landmark binning.
%
%\subsection{Delay Aware P2P System}
%A new \emph{Delay Aware P2P System (DAPS)} is introduced in \cite{zhang_daps_2005}. Its main goal is to reduce the time $L$ of a look-up request by dividing the routing tables of peers into several sectors in increasing delay. The source node emitting the query message defines a delay boundary or the pruning factor $L_t$ in the paper's context. Request messages will be forwarded only to nodes whose delay less than or equal to $L_t$. With the clustered routing tables and the loose organisation the overlay network of \emph{DAPS} is between structured and unstructured.
%
%\subsection{Mithos}
%
%\paragraph{}
%Waldvogel and Rinaldi proposed a protocol\cite{waldvogel_mythos_2003} which incorporates a directed incremental probing to find near optimal node placement.
%
%\paragraph{Neighbour detection}
%During bootstraping, the new comming node needs to know how to contact at least one of the existing members. A subset of these nodes will be used as the first set of candidate neighbours. Then, iteratively, each of the peers in the cadidate set is asked for their neighbours in order for each of the later to be probed for their network distance from the new coming one. The closest node is then used as the new candidate neighbour and the process is repeated until no further improvement is detected. Due to the fact that the algorithm may ultimately reach a local minima instead of a global one, \emph{Mithos}' approach is to first probe all neighbours within a two hop distance from the current minimum before concluding the process.
%
%\paragraph{ID assignment}
%After finding its first neighbour, the newcoming peer must be assigned its ID, a very crusial selection in order not to create many local minima that will prevent efficient future neighbourhood location. \emph{Mithos} uses information gathered during the previous step, in order to compute the ID, which inccurs no further communication costs to the algorithm. This information includes
%\begin{inparaenum}[\itshape i\upshape)]
%  \item the two closest nodes and their neigbours, and 
%  \item their coresponding distances.
%\end{inparaenum}
%Using the above, it then assigns coordinates to the new-coming node, so that Euclidian distances between the node and all known hosts predict the network latency between them\cite{cox_vivaldi_2004}. These synthetic coordinates as explicitly used as the node's ID and as soon as it has been established, distances can be computed in the ID space, no longer requiring physical measurements.
%
%\paragraph{Link establishement}
%The last step of the algorithm is the interconnection among neighbours. \emph{Mithos} uses a \emph{quadrant}-based mechanism according to which each node establishes a link to the closest neighbour in each quadrant. During forwarding, the next hop is performed towards a neighbour in the same quadrant as the final destination\footnote{An implementation of this could be backed by a $d$-bit vector indexing (where $d$ is the number of dimensions) into the routing table. Then the next hop is identified by computing the difference between the current node's vector and that of the destination's.}. The problem is that after the ID assignment process, the new-coming node may not know of other neighbours in all quadrant and if even if it does, this cannot ensure that they are the nearest available. Thus, the node first identifies neighbours in all quadrants using a mechanism based on ideas similar to a perimeter walk\footnote{Used in Greedy Perimeter Stateless Routing (GPSR) protocol.} and then using parallel path processing improves the results by taking into account further geometric properties of node relationships.
%
%\paragraph{}
%% TODO review this part
%In order to avoid local minima during neighbour detection, extensive probing must be undertaken. In simulation, unfortunately, only very small-sized overlay topologies (of 200 to 1000 nodes) have been used and thus no safe conclusions can be made as for the behaviour of an extensively large, real-world p2p deployment of the scheme. 
%
%\subsection{DHT-PNS}
%
%\paragraph{}
%The work in \cite{hancong_pnsbased_2006} propose a proximity neighbour selection scheme on top of the Chord DHT. Using the Vivaldi protocol\cite{cox_vivaldi_2004} each node is assigned synthetic $2$-dimension coordinates that can be used to derive network latency using Euclidian distances in the id space and without the need of explicit probing. Then the algorithm performs the two steps described in the following paragraphs.
%
%\paragraph{Space mapping}
%The space is partitioned using a \emph{concentric circle clustering scheme} where succesive cycles of radiuses $\rho$, $2\rho$, $3\rho$ and so on, are constructed. Then the formed annuluses are divided into $2\chi-1$ \emph{sectors}, where $\chi$ denots the sequence number of the annulus starting from $\chi = 1$ for the centre cycle. It is proved in the paper, that this way each sector occupies the same area\footnote{The same area as does the center cycle.} and assuming uniform node distribution, this characteristic, favours a more load balanced clustering operation. Every sector in this $2-d$ coordinate space is mapped to a unique \emph{region} in the DHT space forming a multi-layer node identifier space. Thus, any nodes that belong to the same sector, are mapped to the same region as well, preserving their proximity relationship unveiled by the use of the Vivaldi protocol.
%
%\paragraph{Routing table optimization}
%The system described in the previous section allows any node $\alpha$ to obtain its DHT's $key_{\alpha}$ and its region's $key_r$, in a fully distributed manner, just by applying a consistent hash function. Using a $Get\left( key_r \right)$ RPC call in Chord, the node can obtain the region's master node, called \emph{Cluster Node (CN)} which is responsible for clustering the nodes belonging in the same sector or region with that of $\alpha$. On the other hand, a $Put\left( key_r, info\right)$ RPC call, registers $\alpha$ to its corresponding region and publishes its information, through the special peer CN. $\alpha$ can, additionaly, ask CN for other nodes that have previously joined the region in order to add them into its neighbour set for future routing table optimization. Even in the case when no neighbour is detected in the current region, the search is expanded to adjasent regions and towards an upper layer identifier space until one is found or the first layer reached.
%
%\subsection{Quasi-Chord}
%
%\paragraph{}
%The approach that is proposed by Sun and Zhang in \cite{sun_quasi_2008} confronts the topology mismatch problem, in the Chord DHT context, that is created by the fact that no consiteration for the underlying physical network topology is taken into account during the construction of the identifier cycle. To construct a \emph{Quasi-Chord} network three steps are needed. First, each host acquires its coordinates in the geometric space utilizing the \emph{global network position (GNP)} protocol \cite{ng_gnp_2001}. Second, using the \emph{Cantor space filling curve} the $2$-dimensional space is converted to a $1$-dimensional one, used in the last, third, step to build the Quasi-Chord circle according to the host's Cantor value. In the following paragraphs, these steps are furtherly discussed.
%
%\paragraph{GNP coordinates}
%First of all the host must be positioned in the geometric space. The algoritm models the P2P network with a well defined distance function in such a way, it can predict, with high accuracy, the distance between any two points in the space by just evaluating the output of the distance function on the coordinates of these points. This is accomplished with the \emph{global network position (GNP)} protocol which\begin{inparaenum}[\itshape i\upshape)]
%  \item creates a reference set of $N$ landmark nodes so as to minimize the error of ICMP measured distances and coordinate computed ones between them, and then
%  \item each host is able to measure its round-trip times to the $N$ Landmarks in order to compute its own coordinates.
%\end{inparaenum}
%
%\paragraph{Cantor SPF}
%After the $2$-d coordinates are set, in the next stage of the algorithn, each participating peer is assigned a Cantor value, according to the application of a \emph{space filling curve} on the coordinate space. (TODO: add figure). This results in the conversion of the $2$-dimensional space to a $1$-dimensional Cantor space that can more easily be more mapped to the Chord hierarchy.
%
%\paragraph{Quasi-Chord construction}
%As can intuitively be infered by observing the Cantor chart, close-by nodes in the physical layer, are more likely to have similar Cantor values. This attribute is exploited in order to construct a topology-aware identifier space for the Chord DHT. The cycle is constructed by sorting nodes in ascending order. Each host maintains 2 finger tables, one for clockwise and one for counter-clockwise stepping. This helps with the connectivity of the network because its not allowed to connect the first node with the last one since this will incur heavy traffic to the later\footnote{After all their Cantor values denote that they are actually the furthest of each other.}.
%
%\paragraph{}
%% TODO: doublecheck!
%The disadvantage of the algorithm is that the coordinate assignment in the first stage is backed by a not fully distributed landmark-based algorithm. Moreover the Quasi-Chord model build-up is making an indirect assumption of a maximum number of allowable hosts since it is constructed. Last but not least the doubling of the required routing information which needs to be created and maintained is an additional negative point to the efficiency and the scallability of the algorithm.
%
%\subsection{LAPTOP}
%{\sethlcolor{yellow}\hl{HA - two important issues for decentralized structured P2P:
%P2P applications. However, the design of a decentralized but structured P2P network has to
%overcome two critical issues. The first issue is the long routing latency. Several proximity
%schemes  have been proposed to avert long routing latency in current structured P2P
%networks, but they require a high-complexity procedure to periodically maintain the routing
%table (e.g. Pastry system) or they need pre-chosen landmarks to construct the overlay.
%However, the P2P system is by its very nature unstable since nodes join and leave frequently.
%For instance, the study of Gnutella shows around approximately 1200 membership changes
%per minute in a 100 000 nodes P2P system. Another proximity scheme needs some pre-chosen
%landmarks or a complete BGP routing table support. As a result, they both increase the
%difficulty of the P2P system deployment.
%The second issue is system maintenance overhead. The existing structured P2P networks allow
%nodes to keep some nearby nodes in their routing tables in order to achieve efficient routing. The
%}}
%
%\paragraph{}
%Laptop \cite{wu_laptop_2007}, introduced by Wu et al., organises the overlay into a tree based hierarchy with main focus on child-to-father relationships in order to reduce hops during message routing as well as minimize maintenance overhead. Additionally, a caching scheme is also incorporated so as to furtherly reduce routing table update cost. The authors argue that Laptop overlay network is bounded by $O\left( log_d N \right)$ hops in a balanced overlay tree, where $N$ is the number of nodes, and $d$ is the maximum degree of each node. It utilizes a geographical layout approach  and constructs a geographical layout in a self-organizing and efficient fashion, by estimating the round trip time (RTT) to a small number of nodes in the overlay network in order to make them roughly aware of their physical distances among them.
%
%\paragraph{}
%The protocol is based on four definitions.
%\begin{enumerate}
% \item The amplitude of all possible measured RTTs is devided into intervals. Each node measures its distance to its parent and is assigned a label $L_i$ where $i$ denotes the configurable RTT interval in which the measured distance falls into. A special kind of node, the root, is initially assigned the $L_1$ label and maintains (as all $L_1$ nodes do) a list of other $L_1$ nodes in the overlay.
% \item Any node can have children with level lower than theirs, except an $L_{max}$ node which can only have $L_{max}$ level children and only in the case when its parent has reached its maximum degree.
% \item Each node is assigned an address in a dotted format (e.g 1.3.4). Each octet ranges from $1$ to $d$, where $d$ is the maximum degree of the nodes. The assignment process is done by appending a unique octet to the address of its parent. Root node is assigned address 1.
% \item For any descendant node $Y$ of a node $X$, the measured distance among each other, must always be less than the lower bound of the RTT interval denoted by $X$'s label.
%\end{enumerate}
%
%\paragraph{}
%The routing scheme is similar to the IP's longest-prefix matching scheme. At each forwarding hop, any message travels up the tree until the first common uncestor of source and destination node is reached and then starts descending to arrive to its target. During tree traversals, special entries in the routing tables, called \emph{routing cache}, are maintained in order to increase routing efficiency and achieve finer load balance. Caching enables a node to forward a message to a better longest-prefix match than that of its direct ancestor making a large, quicker and more cost effective step through the overlay and toward the destination. To improve scalability, the number of children nodes and the size of the routing cache are limited.
%
%In terms of overlay maintenance, Laptop incorporates a simple \emph{heartbeat}-based technique where only the parent node is responsible for monitoring its children.
%
%At join process the newcomming node is assigned its level label as well as its address by its parent node. Additionaly it initializes its routing table (with normal and caching entries) as it traverses the overlay in search for its parents node.
%The newcommer $N$ locates the root node and the later responds with a list of $L_1$ nodes. $N$ then probes each of the $L_1$ nodes in search for the closest one. If the measured RTT to the closest $L_1$ falls into the first interval then the newcommer becomes a $L_1$ node as well. Otherwise node $N$ sets the closest $L_1$ node as its potential parent node. This potential parent becomes the actual parent if it does not have any other children. If it has, $N$ gets a list of these $L_i$ nodes and by measuring the RTT to each of them tries to spot a new potential parent in order to repeat the above process.
%
%\paragraph{}
%During a gracefull departure, the refered node checkes for children in the overlay. If it does not have any, it simply notifies its parent and leaves. If it has, it selects the child node with the lowest RTT to it in order to take its place so that the locality property is preserved.
%
%In case of an arbitrary failure, the children of the failed node are detecting their parent's absent when the later stops acknowledging their heartbeat messages. They start emmiting special messages to their grand parent node\footnote{The address of the grandparent node is stored during join process}. In case of no response from the grandparent node, children invoke the joining procedure to detect their new parent node.The parent of the failed node, aggregates the notifications from the above children nodes for a period of time, and then chooses the one with the lowest level label as the takeover node. Potential ties are broken by favouring the lower RTT. The parent of the failed node, finally informs accordingly all the children about the change in the hierarchy.
%
%\subsection{IP-based Clustering (IPBC)}
%
%\paragraph{}
%\emph{IP-based clustering} \cite{karwaczynski_ipbc_2007} is a proximity neighbour selection technique that is based on a longest IP prefix matching scheme in order to quantify the proximity among peer nodes. The relation of the IP address and the physical location of a node in the topology of the Internet, is intuitevely realized by operation the network layer of the Internet stack. For example, nodes in the same local subnetwork share the same 3 bytes of their address. Reports in \cite{freedman_iploc_2005} state that almost $97\%$ of prefixes longer than 3 bytes correspond to address assignments at a single geographic location. Moreover, observing the \emph{IANA IPv4 Address Space Registry} we can infer that, since blocks of consecutive octet prefixes are assigned to the same Regional Internet Registries (RIRs), nodes inside will be physicaly close to each other.
%
%\paragraph{}
%The implementation in this paper is based on the above observation in order to create proximity information. This information is stored in a decentralised manner, within the overlay itself, just like any other resource would have been. In order to be published in the overlay, each node is first assigned a unique identifier and subsequently generates a key by hashing a fixed-length prefix of its IP. Authors argue on the prefix length to be used, as there is a thin balance between reducing the propability of finding closer nodes by adopting a long prefix and overload nodes that are responsible for information published by many peers when chosing to hash a short one. Their verdict is for the use of a 16-bit wide prefix for real world systems deployed in an Internet scale. In either case, ID and IP is then stored in the DHT using this generated key. This way, any node, at any time (either at join time or during peer lifetime for topology adaptation) can acquire information about close-by nodes just by quering the DHT for a specific key. Moreover, the algorithm takes care of the freshness of the proximity information in two ways. First, the advertising nodes themselves periodically update their advertisements or when they voluntarily leave the overlay, they explicitly remove their data. On the other hand, in case of a failure each publication is assigned an expiration time, and thus ultimately removed by the DHT maintenance mechanisms.
%
%\subsection{CHOord considering Proximity on IPv6 (CHOP6)}
%
%\paragraph{}
%\cite{morimoto_chop6_2007} roughly estimates the proximity among nodes by exploiting the IPv6 address format as well as RTT information if this is available.The first is achieved by introducing a 64-bit ID scheme in which the least significant bit part\footnote{The exact range is a predefined system variable} is the IPv6 global routing prefix and thus enabling a longest prefix match scheme. The observation in which the protocol is based roots from the IPv6 address block assignment. Specicaly, block of /16 or /23 in size are assigned to Regional Internet Registries (RIRs) by Internet Assigned Numbers Authority (IANA). An RIR divides the signed address blockes into smaller address blocks which are furtherly assigned to Natinal Internet Registries (NIRs). That way, it is possible to estimate a node's geographical location by simply examining the upper 32-bits of its IPv6 address. Moreover, similar to Chord, CHOP6 uses a finger table, whose entries hold more than one candidate nodes. There are three cases in which a node chooses the next hop according to information it posseses.
%\begin{itemize}
% \item When no information about RTT is available candidate next hops, the sender just selects a node in the finger table entry which shares the longest prefix with the destination.
% \item There is another possibility that after some communication with other nodes, the source node should know of the RTTs to some of the nodes in finger table entry. In this case, the source node chooses the one with the smallest RTT with propability $p$. If there is a node with no measured RTT then the sender can select such a node with probability $1 - p$
% \item In this last case the source node has already communicated with all candidate node in the finger table entry. Thus, the node selects the node whose RTT is the smallest with probability $p$, the one with the second smallest with probability $q$ and so on, where $0 \leq \ldots < r < q < p < 1$ and $p+q+r+\ldots \leq 1$
%\end{itemize}
%
%\subsection{Proximity in Kademlia}
%
%\paragraph{}
%In \cite{kaune_pkad_2008}, Kaune et al., studied the Kademlia DHT, in order to build a proximity aware scheme that would work in the context of iterative lookup algorithms. The protocols discussed, focus on both the overlay and underlay tiers in the sense that in the first improves the routing performance according to a cost metric that is provided by the later. This metric is tailored to the needs of a specific aplication. For example, routing could be optimised to maximise the within-ISP traffic or reduce the lookup latencies or even avoid contacting untrustworthy subnetworks.
%
%\paragraph{}
%Two algorithms are presented for the overlay optimization. One is \emph{Proximity Neighbour Selection (PNS)} and one for \emph{Proximity Route Selection (PRS)}. The first aims at keeping peers with the least contact cost in the routing table. As Kademlia constantly learns new peers (incoming requests, through iterative lookup) no special algorithm for searching more cost effective peers is necessary and simply choose the best peers seen so far. The later (PRS) aims at choosing the best next hop during routing a message. Due to the fact that the routing in Kademlia is iterative, it is the initiator of the lookup that chooses each next hop from a set of candidate nodes. The ``vanilla'' Kedemlia always chooses the closest node with respect to the XOR metric but PRS Kademlia chooses the one with the smallest underlay metric cost. Thus, as in all such approaches, there is a trade-off between the overlay and underlay distances.
%
%An underlay metric provides information about the underlay network. The paper distingushes between three kinds of possibilities to quantify it:
%\begin{itemize}
% \item Using measurements gained by previous lookups
% \item Using measurements acquired by other peers or jointly calculated with others
% \item Using a local database to look up information
%\end{itemize}
%In the paper, for locality of traffic and subsequently for reducing the lookup latency, a clustering scheme has been implemented exploiting information stored in a geographic information database (GeoIP,MaxMind) about peers. The goal of such a metric is to constrain the largest portion of communication within the limits of a peer's ISP.
%Another incorporated approach is Vivaldi\cite{cox_vivaldi_2004}. At first all coordinates are random but as peers start to communicate they calculate RTTs as well, gradually updating their own coordinates. Vivaldi creates a system where a peer knowing the coordinates of a another, can approximate communication latency without the need of additional prompting.
%
%\subsection{Cone}
%
%\paragraph{}
%Cone\cite{wang_cone_2007} uses a two-layered identifier space. The first, named Chord-layer identifier, denoted as $Id_{Chord}$, is the same as in ``vanilla'' Chord. The second is the Cone-layer identifier, $Id_{Cone}$ which is constructed by two component identifiers. The first, known as \emph{group identifier (gid)} denotes a relevant group the node belongs to while the second, namely \emph{local identifier (lid)} indicates the local identifier within the group. The group concept, which is introduced here, is a way of dividing nodes according to a common $Id_{Chord}$ prefix.
%The structure of a Cone overlay, retains the Chord's circular topology. The differce lies on the fact that, now, two rings are created. A big ring, where nodes with the same \emph{gid} are arranged at each position. Each of these positions are a smaller ring for the particular group's \emph{lid}s. The routing is achieved in both clockwise and anti-clockwise direction in the big-ring. For this reason two routing tables are maintained, namely \emph{front} and \emph{back} finger tables, respectively. Entries in these tables, display physical network proximity with the current node. Moreover, a third table called \emph{group} table maintains information about other online peers within the current node's group in a way that entries are now close in the ID space.
%
%\paragraph{}
%Routing in Cone, comprises of the inter-group algoritm and the intra-group algorithm. First the group of nodes to which the desired key lies is detected, exploiting physical proximity information (front and back finger tables). Second during the next and last hop, the message is forwarded to the exact node that contains the desired key.
%
%Cone uses Landmark+RTTs to generate proximity information and exploits this information using proximity neighbour selection.
%
%\subsection{DynaMO}
%
%\paragraph{}
%In their work in \cite{winter_dynamo_2004}, Winter et al, try to build an overlay network, called \emph{DynaMO}, that tries to consider not only the physical proximity of peers but their mobility attributes as well. DynaMO is based on a Pastry overlay network. This was an intentional choice from the authors because Pastry's built-in locality heuristics are thoroughly analised in the literature \cite{castro_proximityp2p_2002} providing good backround against which to test and compare their results.  In order to adapt to mobile, ad-hoc networks, though, special care has been put to the maintenance of an even overlay ID distribution so that hot-spots are avoided. Pastry assignes them in a randomised fashion and then tries to consider proximity through the joining process and routing table maintenance. On the other hand, DynaMO dictates a newcoming node to gather information concerning its physical neighbourhood and uses it to assign itself an appropriate overlay ID.
%
%DynaMO tries to capitalize an observation of Pastry's routing mechanism. Each node's table consists of rows equal to the number of digits of the overlay IDs and columns equal to the ID's base. As we go down the table rows, the matching prefix between the current node's ID and the row's entries increases by one. Thus, the leaf set contains the closest nodes in the ID space. Additionally as the prefix match increases by one, the result is exponentially less candidates that can fill the tables entries as the row increases. This leads to the observation \cite{antony_pastry_2001,castro_proximityp2p_2002} that from overlay routing hop to overlay routing hop the physical distance between nodes is likely to increase leading to a dominating last routing step in terms of the overall plysical routing path distance travelled during a key lookup. Since the last routing step is usually taken from the leaf set, DynaMO focuses on making this last step as physicaly close as possible. In this context, two approaches are considered namely \emph{Random Landmarking (RLM)} and \emph{Closest Neighbour Prefix Assignment (CNPA)}.
%
%\paragraph{}
%Random Landmarking uses the overlay lookup mechanism to locate nodes responsible for a fixed set of carefully chosen\footnote{In a way that the ID space is divided into equal portions} landmark keys. This means that when a node is assigned a landmark key then it plays the role of a temporary landmark for the network. Any joing node will measure its distance to that and all other landmarks and assign to itself an ID consisting of a prefix taken from the closest landmark and a remainder that could be assigned randomnly or generated by mechanism that takes into account physical neighbourhood. The legth of the prefix can be determined as $prefix_length=|log_b k|$, where $b$ is the ID base and $k$ the number of landmarks. The above scheme results in physically close nodes, forming regions of common ID prefix that are likely to be close to each other in the ID space and thus bringing a node's leaf set closer to itself. Additionally, the advantage of dynamic landmark nodes is that the failed ones can instantly be replaced by new responsibles, that share similar physical attributes to the failed and thus preserving the balance of the network.
%
%\paragraph{}
%RLM may incur more traffic especially on landmark nodes, load that in some cases is not acceptable. CPNA on the other hand takes advantage of Pastry's specification that a new coming node is bootstraped by a physicaly close node. The new comer then assumes the ID prefix of that neighbour while the rest is generated similarly to RLM. Unfortunately, less overhead comes at the expense of being more coarse-grained.
%
%\paragraph{}
%Both algorithms, are protected against the formation of physical landmark clusters or imbalanced ID distribution\footnote{More common during the initial formation of an overlay network} , by introducing the \emph{landmark gravitation range} as a theshold over which landmark keys are reassinged (for the RLM approach) or unutilized ID prefix ranges are detected and used (for the CPNA scheme) in order to balance the distribution of regions in the overlay.
%
%\subsection{PChord}
%\emph{PChord}\cite{hong_pchord_2005} is a based on the Chord DHT which adds proximity routing into its routing mechanism. The main modification over the ``vanilla'' Chord is the inclusion of a \emph{proximity list} into its routing table so that the next hop is decided based not only by considering best progress towards the key, but on the physical proximity of the candidate nodes as well. The list is, at join time, empty but as the PChord node starts to interact with other nodes it applies a heuristic mechanism to fill up the list. Entries are dynamicaly added or removed as the network state is constantly changing. Because, it is guaranteed that at each routing step there is progress towards the target node in the ID space, PChord will result in lower hop number than Chord, for each hop is larger or at least equal in key space in PChord than in Chord. Additionaly, passing through proximity links in the underlay means reduced routing cost. Moreover, if the number of entries in the proximity list is the same as the number of network partitions, PChord prevents hops from jumping back to the same network the current node belongs to.
%
%\subsection{AChord}
%\emph{AChord}\cite{dao_achord_2006} uses IPv6's anycast functionality in order to
%\begin{itemize}
% \item releave the protocol from complex joining procedures, and
% \item achieve high accuracy of network proximity giving high routing efficiency,
%\end{itemize}
%all with a simple and lightweight mechanism that requires few changes to Chord without affecting its own advantageous characteristics. Anycast delivers a message comming from the outside of an anycast group to the physicaly closest node in that group. AChord organizes all nodes participating in the overlay network into an anycast group. Any joining node comes outside of that group and, thus, is automaticaly forwarded to the physicaly nearest node in order to bootstrap, avoiding
%\begin{inparaenum}[\itshape i\upshape)]
%  \item the need to, explicitly, maintain such nodes, and
%  \item the effort of finding a way of locating the physically nearest among them.
%\end{inparaenum}
%The ID of the new comming node is computed based on the ID of the bootstrap node and bootstrap's predecessor in a way that its ID will position itself between the formentioned.
%
%After joining, the finger table is build the same way as in the Chord protocol. Moreover, additional nodes are maintained in a structure called \emph{neighbourship table} which stores information about the closest known nodes\footnote{As a first entry, at bootstrap, the node stores information about the bootstrap node.}. The routing decision is made by using both the neighbourship and the finger table.
%
%\subsection{Chord6}
%\emph{Chord6} \cite{xiong_chord6_2005} is a Chord variant that tries to exploit the hierarchical features of IPv6 in order to create a substrate that reduces interdomain trafic and does that in a cost efficient way. Chord6 bareley modifies the original Chord protocol, only in the part of identifier definition. This, renders the aproach easily portable to other DHTs such as CAN, Pastry and Tapestry. In Chord6 the identifier contains two pares: the higher bits are obtained by hashing the  node's IPv6 address prefix of specific length, while the remaining lower bitss are the hash value of the rest of that IPv6 address. This way nodes in a domain will be mapped onto a contihuous key space on the overlay network. This guarantees that messages are routed in a way that they do not hop in and out of domains many time, thus minimizing overall routing cost. (TODO: Recheck validity) However, even though nodes in the same domain would have close identifiers, nodes in two close domains may have very different ones.
%
%%%%%%%%%%%%%%%%%%%%%%%%%%%%%%%%%%%%%%%%%%%%%%%%%%%%%%%%%%%%%%%%%%%%%%%%%%%%%%%

% TODO: Should be reviewed
\section{Conclusion}
\label{section:conclusion}

Research and development on application systems that follow the 
\p\ architectural paradigm, have flourished over the last decade.
By and large, such systems exploit the loose-coupling and self-organization 
of participating nodes to shape substrates upon which diverse applications 
can run; the latter offer services including, but not limited to, 
distributed naming, 
Internet telephony and video conferencing, 
storage and indexing, 
content and file sharing, 
web crawling and caching, 
event notification and subscribing/publishing.
These services are provided atop the 
best-effort infrastructure of today's Internet;
services have to comply with an array of key requirements and 
offer quality attributes that include node availability, robustness,
scalability, load-sharing, quality-of-service and user anonymity. 
%%AD rephrased below as above - too long sentence it was!
% and they have to effectively deal with an array of key quality attributes that
% they may need or have to offer node availability, robustness,
% scalability, load-sharing, quality-of-service and user anonymity.
%%
The difficulty to handle such characteristics effectively, 
emanates from the fact that
peers  operate at the edge of the Internet and no 
adequate attention is paid to the structure of the physical network 
during application-level network design. 
The deviation of the logical distributed overlay formation from the 
physical network yields suboptimal use of the underlying 
infrastructure and is known as the topology mismatch
problem.
%%AD rephrased what is below as above..
% The deviation of the structure of the overlay network from the
% optimal, according to the underlying physical structure, constitutes what is
% termed topology mismatch and the problem is known as the topology mismatch
% problem.

Development of \p--systems and applications
typically involve juggling a set of tradeoffs (e.g.,
choosing higher accuracy at the cost of increased system overhead).
% Computer systems design is typically about juggling a set of tradeoffs (e.g.,
% choosing higher accuracy at the cost of increased system overhead.)  
In surveying more than
a decade's worth of research efforts aimed at solving the topology mismatch
problem in both unstructured and structured \p\ networks, we find this
tussle-of-tradeoffs property to hold.  
With regards to the three criteria we used --efficiency,
overhead, and scalability-- we find that none of the proposed solutions 
is superior to the others on all three fronts.  This is not surprising.
Nonetheless, by
\begin{inparaenum}[\itshape i\upshape)]
  \item presenting an analysis of each of the solutions,
including what distinguishes it from the
others and its pros and cons and
  \item offering a pictorial
comparison of how each solution fares in regards to others as far as
efficiency, overhead, and scalability are concerned,
\end{inparaenum}
we hope to have provided \p\
developers and researchers with enough insight and perspective;
as they face specific problems, they will be readily able to 
draw the best possible design decisions for their own application
environments.
%%AD sentence was too long - this "giant" thing to much ass-kissing... no.
% when facing a specific problem, they would 
% when they face a specific problem, th 
% so that depending
% on the particular problem in hand, they are able to make a
% correct choice amongst these surveyed here or, by firmly standing on
% the shoulders of giants, propose a new solution of their own.


%%%%%%%%%%%%%%%%%%%%%%%%%%%%%%%%%%%%%%%%%%%%%%%%%%%%%%%%%%%%%%%%%%%%%%%%%%%%%%%%
\bibliographystyle{acmtrans}
\bibliography{mard-survey}
%%%%%%%%%%%%%%%%%%%%%%%%%%%%%%%%%%%%%%%%%%%%%%%%%%%%%%%%%%%%%%%%%%%%%%%%%%%%%%%%

\begin{received}
Received Month Year;
revised Month Year; accepted Month Year
\end{received}

\end{document}
